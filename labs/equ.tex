Пример задания

Вариант A-1. Линейное уравнение

Найти и вывести множество вещественных значений x, удовлетворяющих равенству
ax+b=0. Параметры a, b — любые целые числа (в том числе и нули), вводятся
пользователем с клавиатуры. Необходимо корректно разобрать все случаи.

Решение

Известно, что уравнение имеет одно решение -b/a, если a отлично от нуля.
Заметим, что при a=0 уравнение вырождается в равенство b=0, которое может быть
как истинным, так и ложным вне зависимости от x. Эти рассуждения дают нам
последовательность условий для проверки.

Поскольку в языке Си результат операции деления двух целых чисел тоже является
целым, необходимо воспользоваться операцией преобразования аргументов к
вещественному типу.

#include <stdio.h>

int main(void)
{
  int a, b;

  printf("Введите A и B через пробел: ");
  scanf("%i%i", &a, &b);

  if (a != 0)
    printf("Решение x = %f\n", -(double)b / (double)a);
  else if (b != 0)
    printf("Нет решения.\n");
  else
    printf("Верно при любом x.\n");

  return 0;
}

%%%%%%%%%%%%%%%%%%%%%%%%%%%%%%%%%%%%%%%%%%%%%%%%%%%%%%%%%%%%%%%%%%%%%%%%%%%%%%

Вариант A-2. Квадратное уравнение

Найти и вывести множество вещественных (комплексных) значений x,
удовлетворяющих равенству ax2+bx+c=0. Параметры a, b, c — любые целые числа (в
том числе и нули), вводятся пользователем с клавиатуры. Необходимо корректно
разобрать все случаи.

%%%%%%%%%%%%%%%%%%%%%%%%%%%%%%%%%%%%%%%%%%%%%%%%%%%%%%%%%%%%%%%%%%%%%%%%%%%%%%

Вариант A-3. Кубическое уравнение

Найти и вывести множество вещественных (комплексных) значений x,
удовлетворяющих равенству ax3+bx2+cx+d=0. Параметры a, b, c, d — любые целые
числа (в том числе и нули), вводятся пользователем с клавиатуры. Необходимо
корректно разобрать все случаи.

%%%%%%%%%%%%%%%%%%%%%%%%%%%%%%%%%%%%%%%%%%%%%%%%%%%%%%%%%%%%%%%%%%%%%%%%%%%%%%

Вариант A-4. Система уравнений

Найти и вывести все пары x, y удовлетворяющие системе двух равенств:
a1x+b1y=c1, a2x+b2y=c2. Параметры ak, bk, ck — любые целые числа (в том числе
и нули), вводятся пользователем с клавиатуры. Необходимо корректно разобрать
все случаи.
