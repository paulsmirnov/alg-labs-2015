%%%%%%%%%%%%%%%%%%%%%%%%%%%%%%%%%%%%%%%%%%%%%%%%%%%%%%%%%%%%%%%%%%%%%%%%%%%%%%
\zztaskgroup{SOR}{Сортировка массива}
%%%%%%%%%%%%%%%%%%%%%%%%%%%%%%%%%%%%%%%%%%%%%%%%%%%%%%%%%%%%%%%%%%%%%%%%%%%%%%

В этой задаче требуется выделить в динамической области памяти 
(см. \texttt{malloc()}, \texttt{free()}) блок под 
массив из $N$ целых чисел ($N$ задается пользователем), заполнить массив
случайными числами от $1$ до $N$ включительно и вывести его на экран. Затем 
необходимо отсортировать (упорядочить по возрастанию) массив заданным методом 
и результат тоже вывести. Алгоритм сортировки реализовать в виде отдельной функции:\zztodo{Очень хорошая задача. Только мне казалось, что вы давали в прошлом семестре ее в более веселой формулировке. Хорошо бы сделать, чтобы у них еще замерялось время работы на случайном порядке(усредняя несколько) и в худшем случае. Еще хотелось бы, чтобы у них был не один алгоритм, а чтобы сравнивали несколько, можно исходную и модификацию, можно просто разные. Задачка станет довольно непростой, зато качественной. Может дать им какой-то шаблон для измерения времени или библу дать?}

\mintinline{c}|void sort(int a[], int N);|

Подробности алгоритмов см. в дополнительной литературе.

\begin{zztask}
В рамках общего условия задачи реализовать алгоритм 
сортировки выбором (selection sort). Он заключается в нахождении минимального элемента
в массиве и обмена его местами с самым первым элементом с продолжением сортировки оставшейся
части массива (начиная уже со второго элемента).
\end{zztask}

\begin{zztask}
В рамках общего условия задачи реализовать алгоритм 
сортировки пузырьком (bubble sort). Он заключается в последовательном (от начала к концу)
сравнении пар соседних элементов и обмена их местами, если они расположены 
не по возрастанию. Массив продолжает просматриваться снова и снова до 
тех пор, пока обмены элементов местами не прекратятся.
\end{zztask}

\begin{zztask}
В рамках общего условия задачи реализовать алгоритм 
сортировки шейкером (shaker sort). Это --- модификация алгоритма сортировки пузырьком,
при которой чередуются проходы по массиву от начала к концу и от конца к началу.
\end{zztask}

\begin{zztask}
В рамках общего условия задачи реализовать алгоритм 
сортировки вставками (insertion sort). На каждом шаге алгоритма выбирается следующий 
элемент из еще неотсортированной части массива и вставляется в подходящее место в уже 
отсортированной части в начале массива (путем сдвига элементов массива к концу).
\end{zztask}

\begin{zztask}
В рамках общего условия задачи реализовать алгоритм 
сортировки слиянием (merge sort). Этот алгортим основан на принципе
<<разделяй и властвуй>> и часто реализуется в виде рекурсивной функции:
отсортировать массив можно отсортировав его первую и вторую половину 
по-отдельности, а затем слив воедино две отсортированные части.
\end{zztask}

\begin{zztask}
В рамках общего условия задачи реализовать алгоритм 
<<быстрой>> сортировки (quick sort). Как и в сортировке слиянием алгоритм
рекурсивен, но массив бьется не пополам, а разделяется на две возможно 
неравные части так, что все элементы первой меньше элементов во второй.
\end{zztask}

\begin{zztask}
В рамках общего условия задачи реализовать алгоритм 
пирамидальной сортировки (с помощью кучи, heap sort). Основная идея ---
преобразовать массив в структуру данных <<куча>>, которая позволяет мгновенно 
находить самый большой элемент. Этот элемент меняется с последним элементом 
массива, размер кучи уменьшается на один и восстанавливается нарушенная 
структура кучи. Затем все повторяется.
\end{zztask}

\begin{zztask}
В рамках общего условия задачи реализовать алгоритм 
цифровой (поразрядной) сортировки (digital sort, radix sort). Основная идея ---
отсортировать числа по последней цифре с сохранением порядка чисел с одинаковыми 
цифрами и продолжить сортировку по большей цифре. Поскольку цифр очень мало (десять), 
то их сортировка производится методом подсчета.
\end{zztask}
