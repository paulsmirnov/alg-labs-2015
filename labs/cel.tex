E. КЛЕТОЧНЫЕ АВТОМАТЫ И ИМИТАЦИОННЫЕ МОДЕЛИ

Пример задания

Общее условие

Клеточный автомат — это дискретная модель, представляющая собой бесконечную
регулярную (в данном случае квадратную) решетку из клеток, каждая из которых
может находиться в определенном состоянии (из заданного конечного набора).
Время в рассматриваемой системе тоже дискретно, состояние каждой клетки в
момент времени t есть функция от состояний ее соседей в момент времени t-1.
Вид функциональной зависимости длявсех клеток одинаков.

Для того, чтобы не иметь дела с бесконечным полем, часто рассматривают
конечное квадратное поле NxN с попарно склеенными сторонами (лево-право,
верх-низ). При этом образуется замкнутая тороидальная поверхность, по которой
можно бесконечно путешествовать в любом направлении.

Предлагается написать графическую программу, демонстрирующую развитие
(«жизнь») клеточного автомата на квадратной решетке с заданными правилами, в
зависимости от начальной конфигурации. Начальная конфигурация задается
пользователем — ему предлагается пустое поле (с клетками в нейтральном,
мертвом состоянии), на которое он может высаживать живые клетки (и удалять
непонравившиеся). Состояние клетки обычно обозначается ее цветом или простым
рисунком.

После того как начальная конфигурация задана, клавиша SPACE позволяет
переходить к следующему поколению (рассчитываемому автоматически по правилам).
В любой момент должно быть можно опять перейти в режим редактирования уже
текущей конфигурации с помощью клавиши ENTER, а потом продолжить наблюдения.
Выход из программы должен быть возможен в любое время по клавише ESC.

%%%%%%%%%%%%%%%%%%%%%%%%%%%%%%%%%%%%%%%%%%%%%%%%%%%%%%%%%%%%%%%%%%%%%%%%%%%%%%

Вариант E-1. «Жизнь» Конуэя

В рамках общего условия задачи написать программу моделирующую следующий
клеточный автомат. В некоторых клетках прямоугольной таблицы содержатся
бактерии (одна штука в одной клетке). Соседними являются клетки,
соприкасающиеся хотя бы одним углом (вершиной). Следующее поколение образуется
из предыдущего по правилам:

• бактерия выживает, если у нее ровно два или три соседа (из восьми соседних
  клеток); в противном случае она гибнет от одиночества или от перенаселения;
• бактерия рождается в пустой клетке, если на соседних клетках ровно три
  бактерии.

%%%%%%%%%%%%%%%%%%%%%%%%%%%%%%%%%%%%%%%%%%%%%%%%%%%%%%%%%%%%%%%%%%%%%%%%%%%%%%

Вариант E-2. Ресурсы

В рамках общего условия задачи написать программу моделирующую следующий
клеточный автомат. В некоторых клетках прямоугольной таблицы содержатся
бактерии (одна штука в одной клетке). В каждой клетке также присутствует
"ресурс" питания. Появившись бактерия за один ход съедает все ресурсы в восьми
соседних с ней клетках и по окончании хода умирает. В новом поколении бактерии
появятся в тех клетках, рядом с которыми были бактерии на предыдущем ходу и
рядом с которыми есть несъеденный ресурс. При обновлении поколений бактерий
происходит и обновление всех ресурсов.

%%%%%%%%%%%%%%%%%%%%%%%%%%%%%%%%%%%%%%%%%%%%%%%%%%%%%%%%%%%%%%%%%%%%%%%%%%%%%%

Вариант E-3. Стрелки

В рамках общего условия задачи написать программу моделирующую следующий
клеточный автомат. Каждая клетка прямоугольной таблицы содержит стрелку,
указывающую в одном из восьми направлений (влево и вверх, вверх, вправо и
вверх, влево, вправо, влево и вниз, вниз, вправо и вниз). В следующем
поколении направление стрелочки каждого поля меняется на направление той
соседней, на которую она указывала.

%%%%%%%%%%%%%%%%%%%%%%%%%%%%%%%%%%%%%%%%%%%%%%%%%%%%%%%%%%%%%%%%%%%%%%%%%%%%%%

Вариант E-4. «Лишай» Ван Тассела

В рамках общего условия задачи написать программу моделирующую следующий
клеточный автомат. Каждая клетка прямоугольной таблицы представляет собой
клетку кожи, которая может быть здоровой, зараженной стригущим лишаем или
невосприимчивой к инфекции. В каждый такт времени зараженная клетка может с
вероятностью 0.5 заразить каждую из соседних здоровых клеток. Через шесть
единиц времени зараженная клетка становится невосприимчивой к инфекции.
Возникший иммунитет действует в течение последующих четырех единиц времени, а
затем клетка выздоравливает.

%%%%%%%%%%%%%%%%%%%%%%%%%%%%%%%%%%%%%%%%%%%%%%%%%%%%%%%%%%%%%%%%%%%%%%%%%%%%%%

Вариант E-5. «Муравей» Лэнгтона

В рамках общего условия задачи написать программу моделирующую следующий
клеточный автомат. Клетки на плоскости раскрашены в черный и белый цвета. В
одну клетку помещается муравей, который может передвигаться в одном из четырех
направлений. Муравей двигается по следующим правилам:

• в черной клетке: повернуться направо, изменить цвет клетки и передвинуться
  вперед;
• в белой клетке: повернуться налево, изменить цвет клетки и передвинуться
  вперед.

%%%%%%%%%%%%%%%%%%%%%%%%%%%%%%%%%%%%%%%%%%%%%%%%%%%%%%%%%%%%%%%%%%%%%%%%%%%%%%

Вариант E-6. Лист

В рамках общего условия задачи написать программу моделирующую следующий
клеточный автомат. Развитие клеточного автомата начинается с единственной
исходной клетки, прилегающей своей нижней гранью к особой точке прикрепления.
В дальнейшем рост в этом направлении не происходит. На следующем шаге исходная
клетка начинает «делиться». Деление происходит в трех свободных направлениях —
вправо, вверх и влево. Далее каждая из вновь рожденных клеток также делится. В
отличие от исходной, все прочие клетки могут размножаться в любом из четырех
направлений. Однако если на какую-либо из свободных ячеек претендуют сразу
несколько вновь образующихся клеток, то они взаимно уничтожают друг друга и
ячейка остается свободной. Затравочных точек может быть и несколько,
направление первоначального роста тоже можно варьировать.

%%%%%%%%%%%%%%%%%%%%%%%%%%%%%%%%%%%%%%%%%%%%%%%%%%%%%%%%%%%%%%%%%%%%%%%%%%%%%%

Вариант E-7. «Волчий остров» Ван Тассела

В рамках общего условия задачи написать программу моделирующую следующую
ситуацию. Тороидальный остров заселен дикими кроликами, волками и волчицами.
Имеется несколько представителей каждого вида. Кролики довольно глупы: в
каждый момент времени они с одинаковой вероятностью 1/9 передвигаются в один
из восьми соседних квадратов или просто сидят неподвижно. Каждый кролик с
вероятностью 0.2 превращается в двух кроликов. Волчицы передвигаются случайным
образом до тех пор, пока в одном из соседних восьми квадратов не окажется
кролик. Если волчица и кролик оказываются в одном квадрате, волчица съедает
кролика и получает одно очко, в противном случае она теряет 0.1 очка за каждую
единицу времени. Волки и волчицы с нулевым количеством очков умирают. В
начальный момент времени все волки и волчицы имеют 1 очко. Волк ведет себя
подобно волчице до тех пор, пока в соседних квадратах не исчезнут все кролики;
в этом случае, если волчица находится в одном из восьми ближайших квадратов,
волк гонится за ней. Если волк и волчица окажутся в одном квадрате, они
производят потомство случайного пола. Проследите, как сказываются на эволюции
популяции изменение различных параметров модели.

%%%%%%%%%%%%%%%%%%%%%%%%%%%%%%%%%%%%%%%%%%%%%%%%%%%%%%%%%%%%%%%%%%%%%%%%%%%%%%

Вариант E-8. «Аква-тор» Дьюдни

В рамках общего условия задачи написать программу моделирующую следующую
ситуацию. В тороидальном аквариуме в некоторых клетках содержатся особи одного
из двух видов: рыбы и акулы. Правила существования популяций (параметры
следует подобрать самостоятельно):

• на каждом ходу рыба перемещается случайным образом в одном из восьми
  направлений (если клетка свободна), но из-за течения слева направо
  вероятность движения в правую сторону в два раза выше, чем в левую;
• акула, если видит в соседних клетках рыбу, перемещается туда и съедает ее
  (любую, если их несколько); если рыб нет, то направление движения случайно,
  течение на акул не влияет;
• если акула долго не может съесть рыбу, она погибает; время смерти является
  параметром;
• каждая особь периодически оставляет потомство а той клетке, из которой она
  переместилась; период является параметром; рыбы плодятся чаще чем акулы;
  период, через который акулы приносят потомство связан с временем голодной
  смерти (меньше на единицу).

