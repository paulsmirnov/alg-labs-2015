Пример задания

Общее условие

В следующих задачах предлагается реализовать алгоритм численного
интегрирования в виде отдельной функции, принимающей подынтегральное выражение
как параметр (в виде указателя на функцию определённого типа). Подынтегральная
функция вычисляется в равномерно распределенных точках xi, и интеграл от
функции на отрезке от a до b представляется как сумма нескольких интегралов на
отрезках [xi, xi+k], которые уже считаются по приближённой формуле.

x_i=a+ih,    h=(b-a)/n,

∫_a^b▒f(x)dx=∑_(i=0,   i÷k)^(n-k)▒∫_(x_i)^(x_(i+k))▒f(x)dx=∑_(i=0, i÷k)^(n-k)▒〖I(h,f(x_i ),…f(x_(i+k)))〗.

Указатель на исследуемую математическую функцию, границы отрезка и количество
подразбиений (или шаг разбиения) должны приниматься как параметры. Необходимо
учитывать, что некоторые методы требуют, чтобы количество точек со значениями
функции было кратно некоторому числу k.

В целях достижения максимальной производительности запрещается вычислять
значение подынтегрального выражения несколько раз в одной и той же точке
(потому что вызов f(x) может быть дорогим в плане вычислений, например, если
функция задана рядом, другим интегралом и т.п.).

В рамках общего условия задачи написать функцию, находящую приближенное
значение интеграла от f(x) на отрезке [a, b] (т.е. площади под графиком
функции) с использованием следующей формулы (метода «3/8 Симпсона»).

∫_(x_i)^(x_(i+3))▒f(x)dx≈3h/8 (f(x_i )+3f(x_(i+1) )+3f(x_(i+2) )+f(x_(i+3) ))
