%%%%%%%%%%%%%%%%%%%%%%%%%%%%%%%%%%%%%%%%%%%%%%%%%%%%%%%%%%%%%%%%%%%%%%%%%%%%%%
\zztaskgroup{LIB}{Библиотеки функций}
%%%%%%%%%%%%%%%%%%%%%%%%%%%%%%%%%%%%%%%%%%%%%%%%%%%%%%%%%%%%%%%%%%%%%%%%%%%%%%

Пример задания

Общее условие

Написать библиотеку функций с заданной функциональностью. Программа должна
состоять из двух проектов: библиотеки (например, собирается библиотека
mylib.lib из mylib.c, mylib.h) и тестовой программы (mytest.exe из mytest.c и
mylib.lib). Исходный текст библиотеки должен содержать комментарии для
автоматической генерации документации в формате html с помощью утилиты
doxygen.

Функции библиотеки в качестве префикса имен должны иметь название типа, чтобы
отличать от аналогичных функций для другого типа, например, ComplexAdd().

Краткое описание возможной функциональности для различного вида чисел:
* (create) создание по набору параметров,
* (destroy) уничтожение, в первую очередь динамической памяти,
* (from xxx) создание из определенного типа,
* (as xxx) преобразование к определенному типу,
* (parse) создание по строке, (format) запись в строковый буфер,
* (read) создание из потока, (write) запись в поток, в текстовом виде,
* (add) сложение, (sub) вычитание, (mul) умножение, (div) деление,
* (negate) противоположное, (reciprocal) обратное,
* (abs) модуль,
* (compare) сравнение (функция сравнения традиционно возвращает –1, 0 или +1 когда первый аргумент меньше, равен или больше второго соответственно).

%%%%%%%%%%%%%%%%%%%%%%%%%%%%%%%%%%%%%%%%%%%%%%%%%%%%%%%%%%%%%%%%%%%%%%%%%%%%%%

Вариант D-1. Комплексные числа

В рамках общего условия задачи написать библиотеку для работы с комплексными
числами в алгебраической форме. Ввести новый тип <<комплексное число>>
(\verb|typedef struct ... complex_t|) на основе типа double и реализовать функциональность:

* (from real), (from pair),
* (parse), (format), (read), (write), в формате (a,b),
* (add), (sub), (mul), (div), (negate), (reciprocal),
* (abs) модуль, (arg) аргумент,
* (conjugate) сопряжение,
* (re, im) получение вещественной и мнимой частей.

%%%%%%%%%%%%%%%%%%%%%%%%%%%%%%%%%%%%%%%%%%%%%%%%%%%%%%%%%%%%%%%%%%%%%%%%%%%%%%

Вариант D-2. Рациональные дроби

В рамках общего условия задачи написать библиотеку для работы с несократимыми
рациональными дробями. Ввести новый тип <<рациональная дробь>> 
(\verb|typedef struct ... rational_t|) с числителем и знаменателем типа int и реализовать
функциональность (не забывая сокращать дробь):

* (from int), (from pair), (from real),
* (parse), (format), (read), (write), в формате a/b,
* (add), (sub), (mul), (div), (negate), (reciprocal), (abs),
* (as real), (as int),
* (round), (floor), (ceil) округление до ближайшего целого, вниз или вверх,
* (compare).

%%%%%%%%%%%%%%%%%%%%%%%%%%%%%%%%%%%%%%%%%%%%%%%%%%%%%%%%%%%%%%%%%%%%%%%%%%%%%%

Вариант D-3. Вещественные числа с фиксированной точкой

В рамках общего условия задачи написать библиотеку для работы с вещественными
числами, хранящимися в формате не с плавающей, а с фиксированной точкой.
Ввести новый тип <<число с фиксированной точкой 16:16>> занимающий 32 бита
(\verb|typedef int fixed_t|), по 16 бит на целую и дробную части, и реализовать
функциональность:

* (from int), (from real),
* (parse), (format), (read), (write),
* (add), (sub), (mul), (div), (negate), (reciprocal), (abs),
* (as real), (as int),
* (round), (floor), (ceil) округление до ближайшего целого, вниз или вверх,
* (compare).

Пример для меньшего количества бит: число 12.75 = 1100.11(2) в формате <<8:8>>
будет храниться как short со значением 0000110011000000(2).

%%%%%%%%%%%%%%%%%%%%%%%%%%%%%%%%%%%%%%%%%%%%%%%%%%%%%%%%%%%%%%%%%%%%%%%%%%%%%%

Вариант D-4. Длинные целые числа

В рамках общего условия задачи написать библиотеку для работы с длинными
целыми числами. Ввести новый 18-знаковый целый тип <<очень длинный>>
(\verb|typedef struct ... verylong_t|) из двух чисел типа long, в каждом хранить по 9 десятичных
цифр, в старшем числе знак, и реализовать функциональность:

* (from long),
* (parse), (format), (read), (write),
* (add), (sub), (mul), (div), (negate), (abs),
* (as long),
* (compare).

%%%%%%%%%%%%%%%%%%%%%%%%%%%%%%%%%%%%%%%%%%%%%%%%%%%%%%%%%%%%%%%%%%%%%%%%%%%%%%

Вариант D-5. Приближенные вещественные числа

В рамках общего условия задачи написать библиотеку для работы с вещественными
интервалами (приближенно известными вещественными числами). Ввести новый тип
<<приближенный>> (\verb|typedef struct ... approx_t|) на основе типа double и реализовать
функциональность:

* (from real) создание по центру и радиусу (точности),
* (from range) создание по левой-правой границе,
* (parse), (format), (read), (write), в формате range(a,b),
* (add), (sub), (mul), (div), (negate), (abs),
* (as real), (center) центр, (radius) радиус,
* (left) левая, (right) правая границы.

%%%%%%%%%%%%%%%%%%%%%%%%%%%%%%%%%%%%%%%%%%%%%%%%%%%%%%%%%%%%%%%%%%%%%%%%%%%%%%

Вариант D-6. Элементарная нечеткая логика

В рамках общего условия задачи написать библиотеку для работы с переменными
нечеткой логики. Ввести новый тип <<нечеткий bool>> (\verb|typedef float fool_t|) и
реализовать функциональность:

* (from int), (from real),
* (parse), (format), (read), (write), в формате fuzzy(a),
* (not) логическое отрицание,
* (and) конъюнкция, (or) дизъюнкция,
* (as real) получение степени истинности, от 0 до 1,
* (round) <<округление>> до ближайшего булевского значения.

%%%%%%%%%%%%%%%%%%%%%%%%%%%%%%%%%%%%%%%%%%%%%%%%%%%%%%%%%%%%%%%%%%%%%%%%%%%%%%

Вариант D-7. Матрицы 3x3

В рамках общего условия задачи написать библиотеку для работы с матрицами.
Ввести новый тип (\verb|typedef ... matrix33_t|) <<матрица>> и реализовать
функциональность:

* (set zero) заполнение нулевой,
* (set identity) заполнение единичной,
* (parse), (format), (read), (write), в формате [[a,b,c],[d,e,f],[g,h,i]],
* (add), (sub), (mul), (mul real),
* (determinant) определитель, (trace) след,
* (inverse) обратная, (transpose) транспонированная,
* (element) макрокоманда доступа к элементу [i, j].

Примечание: в качестве типа можно использовать обычный массив из 9 элементов,
возможно, обернутый в структуру для удобства.

%%%%%%%%%%%%%%%%%%%%%%%%%%%%%%%%%%%%%%%%%%%%%%%%%%%%%%%%%%%%%%%%%%%%%%%%%%%%%%

Вариант D-8. Матрицы произвольного размера

В рамках общего условия задачи написать библиотеку для работы с матрицами
динамического размера NxM. Ввести новый тип (\verb|typedef struct ... matrix_t|)
<<матрица>> и реализовать функциональность:

* (create) создание по N и M в динамической памяти,
* (destroy) уничтожение,
* (set zero) заполнение нулевой,
* (set identity) заполнение единичной, 
* (parse), (format), (read), (write), в формате [[a,b],[c,d],[e,f]],
* (add), (sub), (mul), (mul real),
* (determinant) определитель, (trace) след,
* (inverse) обратная, (transpose) транспонированная,
* (element) макрокоманда доступа к элементу [i, j].

Примечание: в качестве типа можно использовать структуру с тремя полями,
хранящими число строк, столбцов и указатель на выделенную память.

%%%%%%%%%%%%%%%%%%%%%%%%%%%%%%%%%%%%%%%%%%%%%%%%%%%%%%%%%%%%%%%%%%%%%%%%%%%%%%

Вариант D-9. Многочлены

В рамках общего условия задачи написать библиотеку для работы с многочленами.
Ввести новый тип <<многочлен>> (\verb|typedef ... poly_t|) и реализовать
функциональность:

* (create) создание по N в динамической памяти, (destroy),
* (from) создание по N коэффициентам, с переменным числом аргументов,
* (parse), (format), (read), (write), в формате $ax^3 – bx + c$,
* (add), (sub), (mul), (mul real),
* (evaluate) вычислить в точке,
* (element) макрокоманда доступа к элементу [i].

Примечание: в качестве типа можно использовать структуру с двумя полями,
хранящими степень многочлена и указатель на выделенную память под
коэффициенты.

%%%%%%%%%%%%%%%%%%%%%%%%%%%%%%%%%%%%%%%%%%%%%%%%%%%%%%%%%%%%%%%%%%%%%%%%%%%%%%

Вариант D-11. BCD-числа

В рамках общего условия задачи написать библиотеку для работы с BCD-числами
(binary coded decimal). Ввести новый тип (typedef) <<bcd-число>> и реализовать
функциональность: инициализация целым числом, сложение, вычитание, умножение
(например, в столбик), перевод обратно в целое число и сравнение. Функция
сравнения традиционно возвращает значение –1, 0 или +1 когда первый аргумент
меньше, равен или больше второго соответственно.

Примечание: каждая цифра числа кодируется четырьмя битами и результат
склеивается вместе. Так число 125 состоит из трех цифр -0001(2), 0010(2),
0101(2) и в формате <<bcd>> будет храниться как 0...000100100101.

%%%%%%%%%%%%%%%%%%%%%%%%%%%%%%%%%%%%%%%%%%%%%%%%%%%%%%%%%%%%%%%%%%%%%%%%%%%%%%

Вариант D-12. Строки

В рамках общего условия задачи написать библиотеку для работы со строковыми
буферами, хранящими размер выделенной под строку памяти. Ввести новый тип
(typedef) <<строка>> и реализовать функциональность: инициализация с выделением
памяти и уничтожение, преобразование между новым и стандартным типом строк,
копирование, конкатенация, поиск строки и символа, смена регистра.

Примечание: новый тип может быть структурой с двумя полями, хранящими размер
буфера и указатель на него.

%%%%%%%%%%%%%%%%%%%%%%%%%%%%%%%%%%%%%%%%%%%%%%%%%%%%%%%%%%%%%%%%%%%%%%%%%%%%%%
