%%%%%%%%%%%%%%%%%%%%%%%%%%%%%%%%%%%%%%%%%%%%%%%%%%%%%%%%%%%%%%%%%%%%%%%%%%%%%%
\zztaskgroup{ARY}{Одномерные массивы}
%%%%%%%%%%%%%%%%%%%%%%%%%%%%%%%%%%%%%%%%%%%%%%%%%%%%%%%%%%%%%%%%%%%%%%%%%%%%%%

Элементы языка: простые массивы, генератор псевдослучайных чисел.

Здесь приводятся упражнения на работу с одномерными массивами и индексацию в
массив. Показаны стандартные приемы нахождения минимального и максимального
элементов массива. Также вводится понятие генератора псевдослучайных чисел,
закрепляются навыки его использования.\zztodo{Не очень понятно, что даст эта лаба, как-то тут кучно все. Все классе тут не успеть нифига,а домашка вроде как для имбецилов получается. }


%%%%%%%%%%%%%%%%%%%%%%%%%%%%%%%%%%%%%%%%%%%%%%%%%%%%%%%%%%%%%%%%%%%%%%%%%%%%%%

Задача C-1

В начало массива из 20 элементов записать N случайных натуральных чисел из
диапазона от 1 до N, воспользовавшись функцией rand(). Число N вводится с
клавиатуры. Сосчитать сумму элементов. В другой массив записать суммы пар
элементов (N сумм): первого и последнего, второго и предпоследнего, третьего и
пред-предпоследнего... Вывести на экран результирующий массив и сумму его
элементов, а также минимальный и максимальный эл-т. Повторять выполнение
программы до тех пор, пока в качестве N не введут 0.\zztodo{Зачем дублировать парные суммы?}

Иванов И.И. (1057/1): Массивы, нат.случайные числа от 1 до N.
Введите размер (или 0 для завершения): 5
Сгенерированный массив A = { 4 3 1 4 5 }, сумма = 17
Результат B = { 9 7 2 7 9 }, сумма = 34
Минимум = 2, максимум = 9
Введите размер (или 0 для завершения): 0

%%%%%%%%%%%%%%%%%%%%%%%%%%%%%%%%%%%%%%%%%%%%%%%%%%%%%%%%%%%%%%%%%%%%%%%%%%%%%%

Задача C-2

В начало массива из 20 элементов записать N случайных натуральных чисел из
диапазона от 1 до N, воспользовавшись функцией rand(). Число N вводится с
клавиатуры. Сосчитать сумму элементов. В другой массив записать суммы пар
элементов (N сумм): первого и второго, второго и третьего, ..., последнего и
первого. Вывести на экран результирующий массив и сумму его элементов, а также
минимальный и максимальный эл-т. Повторять выполнение программы до тех пор,
пока в качестве N не введут 0.

Иванов И.И. (1057/1): Массивы, нат.случайные числа от 1 до N.
Введите размер (или 0 для завершения): 5
Сгенерированный массив A = { 4 3 1 4 5 }, сумма = 17
Результат B = { 7 4 5 9 9 }, сумма = 34
Минимум = 4, максимум = 9
Введите размер (или 0 для завершения): 0

%%%%%%%%%%%%%%%%%%%%%%%%%%%%%%%%%%%%%%%%%%%%%%%%%%%%%%%%%%%%%%%%%%%%%%%%%%%%%%

Задача C-3

В начало массива из 20 элементов записать N случайных целых чисел из диапазона
от A до B, воспользовавшись функцией rand(). Числа N, A, B вводятся с
клавиатуры. Сосчитать сумму элементов. В другой массив записать суммы пар
элементов (N сумм): первого и последнего, второго и предпоследнего, третьего и
пред-предпоследнего... Вывести на экран результирующий массив и сумму его
элементов, а также минимальный и максимальный эл-т. Повторять выполнение
программы до тех пор, пока в качестве N не введут 0.

Иванов И.И. (1057/1): Массивы, целые случайные числа от A до B.
Введите размер (или 0 для завершения): 5
Введите диапазон A B через пробел: 3 7
Сгенерированный массив A = { 4 3 7 6 5 }, сумма = 25
Результат B = { 9 9 14 9 9 }, сумма = 50
Минимум = 9, максимум = 14
Введите размер (или 0 для завершения): 0

%%%%%%%%%%%%%%%%%%%%%%%%%%%%%%%%%%%%%%%%%%%%%%%%%%%%%%%%%%%%%%%%%%%%%%%%%%%%%%

Задача C-4

В начало массива из 20 элементов записать N случайных целых чисел из диапазона
от A до B, воспользовавшись функцией rand(). Числа N, A, B вводятся с
клавиатуры. Сосчитать сумму элементов. В другой массив записать суммы пар
элементов (N сумм): первого и второго, второго и третьего, ..., последнего и
первого. Вывести на экран результирующий массив и сумму его элементов, а также
минимальный и максимальный эл-т. Повторять выполнение программы до тех пор,
пока в качестве N не введут 0.

Иванов И.И. (1057/1): Массивы, целые случайные числа от A до B.
Введите размер (или 0 для завершения): 5
Введите диапазон A B через пробел: 3 7
Сгенерированный массив A = { 4 3 7 6 5 }, сумма = 25
Результат B = { 7 10 13 11 9 }, сумма = 50
Минимум = 7, максимум = 13
Введите размер (или 0 для завершения): 0

%%%%%%%%%%%%%%%%%%%%%%%%%%%%%%%%%%%%%%%%%%%%%%%%%%%%%%%%%%%%%%%%%%%%%%%%%%%%%%

Задача C-5

В начало массива из 20 элементов записать N случайных вещественных чисел из
диапазона от 0 до 1, воспользовавшись функцией rand(). Число N вводится с
клавиатуры. Сосчитать сумму элементов. В другой массив записать суммы пар
элементов (N сумм): первого и последнего, второго и предпоследнего, третьего и
пред-предпоследнего... Вывести на экран результирующий массив и сумму его
элементов, а также минимальный и максимальный эл-т. Повторять выполнение
программы до тех пор, пока в качестве N не введут 0.

Иванов И.И. (1057/1): Массивы, вещ. случайные числа от 0 до 1.
Введите размер (или 0 для завершения): 5
Сгенерированный массив A = { 0.43 0.31 0.74 0.16 0.50 }, сумма = 2.14
Результат B = { 0.93 0.47 1.48 0.47 0.93 }, сумма = 4.28
Минимум = 0.47, максимум = 1.48
Введите размер (или 0 для завершения): 0

%%%%%%%%%%%%%%%%%%%%%%%%%%%%%%%%%%%%%%%%%%%%%%%%%%%%%%%%%%%%%%%%%%%%%%%%%%%%%%

Задача C-6

В начало массива из 20 элементов записать N случайных вещественных чисел из
диапазона от 0 до 1, воспользовавшись функцией rand(). Число N вводится с
клавиатуры. Сосчитать сумму элементов. В другой массив записать суммы пар
элементов (N сумм): первого и второго, второго и третьего, ..., последнего и
первого. Вывести на экран результирующий массив и сумму его элементов, а также
минимальный и максимальный эл-т. Повторять выполнение программы до тех пор,
пока в качестве N не введут 0.

Иванов И.И. (1057/1): Массивы, вещ. случайные числа от 0 до 1.
Введите размер (или 0 для завершения): 5
Сгенерированный массив A = { 0.43 0.31 0.74 0.16 0.50 }, сумма = 2.14
Результат B = { 0.74 1.05 0.90 0.66 0.93 }, сумма = 4.28
Минимум = 0.66, максимум = 1.05
Введите размер (или 0 для завершения): 0

%%%%%%%%%%%%%%%%%%%%%%%%%%%%%%%%%%%%%%%%%%%%%%%%%%%%%%%%%%%%%%%%%%%%%%%%%%%%%%

Задача C-7

В начало массива из 20 элементов записать N случайных вещественных чисел из
диапазона от 0 до N с точностью 0.1, воспользовавшись функцией rand(). Число N
вводится с клавиатуры. Сосчитать сумму элементов. В другой массив записать
суммы пар элементов (N сумм): первого и последнего, второго и предпоследнего,
третьего и пред-предпоследнего... Вывести на экран результирующий массив и
сумму его элементов, а также минимальный и максимальный эл-т. Повторять
выполнение программы до тех пор, пока в качестве N не введут 0.

Иванов И.И. (1057/1): Массивы, вещ. случайные числа от 0 до N.
Введите размер (или 0 для завершения): 5
Сгенерированный массив A = { 3.40 0.30 4.70 1.10 4.50 }, сумма = 14
Результат B = { 7.90 1.40 9.40 1.40 7.90 }, сумма = 28
Минимум = 1.40, максимум = 9.40
Введите размер (или 0 для завершения): 0

%%%%%%%%%%%%%%%%%%%%%%%%%%%%%%%%%%%%%%%%%%%%%%%%%%%%%%%%%%%%%%%%%%%%%%%%%%%%%%

Задача C-8

В начало массива из 20 элементов записать N случайных вещественных чисел из
диапазона от 0 до N с точностью 0.1, воспользовавшись функцией rand(). Число N
вводится с клавиатуры. Сосчитать сумму элементов. В другой массив записать
суммы пар элементов (N сумм): первого и второго, второго и третьего, ...,
последнего и первого. Вывести на экран результирующий массив и сумму его
элементов, а также минимальный и максимальный эл-т. Повторять выполнение
программы до тех пор, пока в качестве N не введут 0.

Иванов И.И. (1057/1): Массивы, вещ. случайные числа от 0 до N.
Введите размер (или 0 для завершения): 5
Сгенерированный массив A = { 3.40 0.30 4.70 1.10 4.50 }, сумма = 14
Результат B = { 3.70 5.00 5.80 5.60 7.90 }, сумма = 28
Минимум = 3.70, максимум = 7.90
Введите размер (или 0 для завершения): 0

%%%%%%%%%%%%%%%%%%%%%%%%%%%%%%%%%%%%%%%%%%%%%%%%%%%%%%%%%%%%%%%%%%%%%%%%%%%%%%

Задача C-9

В начало массива из 20 элементов записать N случайных натуральных чисел из
диапазона от 1 до N, воспользовавшись функцией rand(). Число N вводится с
клавиатуры. Сосчитать сумму элементов. В другой массив записать суммы пар
элементов (N сумм): первого и последнего, второго и предпоследнего, третьего и
пред-предпоследнего... Вывести на экран результирующий массив и сумму его
элементов, а также индексы минимального и максимального эл-та (если таких
несколько, то первый минимальный и последний максимальный). Повторять
выполнение программы до тех пор, пока в качестве N не введут 0.

Иванов И.И. (1057/1): Массивы, нат.случайные числа от 1 до N.
Введите размер (или 0 для завершения): 5
Сгенерированный массив A = { 4 3 1 4 5 }, сумма = 17
Результат B = { 9 7 2 7 9 }, сумма = 34
Минимум = B[2], максимум = B[4]
Введите размер (или 0 для завершения): 0

%%%%%%%%%%%%%%%%%%%%%%%%%%%%%%%%%%%%%%%%%%%%%%%%%%%%%%%%%%%%%%%%%%%%%%%%%%%%%%

Задача C-10

В начало массива из 20 элементов записать N случайных натуральных чисел из
диапазона от 1 до N, воспользовавшись функцией rand(). Число N вводится с
клавиатуры. Сосчитать сумму элементов. В другой массив записать суммы пар
элементов (N сумм): первого и второго, второго и третьего, ..., последнего и
первого. Вывести на экран результирующий массив и сумму его элементов, а также
индексы минимального и максимального эл-та (если таких несколько, то первый
минимальный и последний максимальный). Повторять выполнение программы до тех
пор, пока в качестве N не введут 0.

Иванов И.И. (1057/1): Массивы, нат.случайные числа от 1 до N.
Введите размер (или 0 для завершения): 5
Сгенерированный массив A = { 4 3 1 4 5 }, сумма = 17
Результат B = { 7 4 5 9 9 }, сумма = 34
Минимум = B[1], максимум = B[4]
Введите размер (или 0 для завершения): 0

%%%%%%%%%%%%%%%%%%%%%%%%%%%%%%%%%%%%%%%%%%%%%%%%%%%%%%%%%%%%%%%%%%%%%%%%%%%%%%

Задача C-11

В начало массива из 20 элементов записать N случайных целых чисел из диапазона
от A до B, воспользовавшись функцией rand(). Числа N, A, B вводятся с
клавиатуры. Сосчитать сумму элементов. В другой массив записать суммы пар
элементов (N сумм): первого и последнего, второго и предпоследнего, третьего и
пред-предпоследнего... Вывести на экран результирующий массив и сумму его
элементов, а также индексы минимального и максимального эл-та (если таких
несколько, то первый минимальный и последний максимальный). Повторять
выполнение программы до тех пор, пока в качестве N не введут 0.

Иванов И.И. (1057/1): Массивы, целые случайные числа от A до B.
Введите размер (или 0 для завершения): 5
Введите диапазон A B через пробел: 3 7
Сгенерированный массив A = { 4 3 7 6 5 }, сумма = 25
Результат B = { 9 9 14 9 9 }, сумма = 50
Минимум = B[0], максимум = B[2]
Введите размер (или 0 для завершения): 0

%%%%%%%%%%%%%%%%%%%%%%%%%%%%%%%%%%%%%%%%%%%%%%%%%%%%%%%%%%%%%%%%%%%%%%%%%%%%%%

Задача C-12

В начало массива из 20 элементов записать N случайных целых чисел из диапазона
от A до B, воспользовавшись функцией rand(). Числа N, A, B вводятся с
клавиатуры. Сосчитать сумму элементов. В другой массив записать суммы пар
элементов (N сумм): первого и второго, второго и третьего, ..., последнего и
первого. Вывести на экран результирующий массив и сумму его элементов, а также
индексы минимального и максимального эл-та (если таких несколько, то первый
минимальный и последний максимальный). Повторять выполнение программы до тех
пор, пока в качестве N не введут 0.

Иванов И.И. (1057/1): Массивы, целые случайные числа от A до B.
Введите размер (или 0 для завершения): 5
Введите диапазон A B через пробел: 3 7
Сгенерированный массив A = { 4 3 7 6 5 }, сумма = 25
Результат B = { 7 10 13 11 9 }, сумма = 50
Минимум = B[0], максимум = B[2]
Введите размер (или 0 для завершения): 0

%%%%%%%%%%%%%%%%%%%%%%%%%%%%%%%%%%%%%%%%%%%%%%%%%%%%%%%%%%%%%%%%%%%%%%%%%%%%%%

Задача C-13

В начало массива из 20 элементов записать N случайных вещественных чисел из
диапазона от 0 до 1, воспользовавшись функцией rand(). Число N вводится с
клавиатуры. Сосчитать сумму элементов. В другой массив записать суммы пар
элементов (N сумм): первого и последнего, второго и предпоследнего, третьего и
пред-предпоследнего... Вывести на экран результирующий массив и сумму его
элементов, а также индексы минимального и максимального эл-та (если таких
несколько, то первый минимальный и последний максимальный). Повторять
выполнение программы до тех пор, пока в качестве N не введут 0.

Иванов И.И. (1057/1): Массивы, вещ. случайные числа от 0 до 1.
Введите размер (или 0 для завершения): 5
Сгенерированный массив A = { 0.43 0.31 0.74 0.16 0.50 }, сумма = 2.14
Результат B = { 0.93 0.47 1.48 0.47 0.93 }, сумма = 4.28
Минимум = B[1], максимум = B[2]
Введите размер (или 0 для завершения): 0

%%%%%%%%%%%%%%%%%%%%%%%%%%%%%%%%%%%%%%%%%%%%%%%%%%%%%%%%%%%%%%%%%%%%%%%%%%%%%%

Задача C-14

В начало массива из 20 элементов записать N случайных вещественных чисел из
диапазона от 0 до 1, воспользовавшись функцией rand(). Число N вводится с
клавиатуры. Сосчитать сумму элементов. В другой массив записать суммы пар
элементов (N сумм): первого и второго, второго и третьего, ..., последнего и
первого. Вывести на экран результирующий массив и сумму его элементов, а также
индексы минимального и максимального эл-та (если таких несколько, то первый
минимальный и последний максимальный). Повторять выполнение программы до тех
пор, пока в качестве N не введут 0.

Иванов И.И. (1057/1): Массивы, вещ. случайные числа от 0 до 1.
Введите размер (или 0 для завершения): 5
Сгенерированный массив A = { 0.43 0.31 0.74 0.16 0.50 }, сумма = 2.14
Результат B = { 0.74 1.05 0.90 0.66 0.93 }, сумма = 4.28
Минимум = B[3], максимум = B[1]
Введите размер (или 0 для завершения): 0

%%%%%%%%%%%%%%%%%%%%%%%%%%%%%%%%%%%%%%%%%%%%%%%%%%%%%%%%%%%%%%%%%%%%%%%%%%%%%%

Задача C-15

В начало массива из 20 элементов записать N случайных вещественных чисел из
диапазона от 0 до N с точностью 0.1, воспользовавшись функцией rand(). Число N
вводится с клавиатуры. Сосчитать сумму элементов. В другой массив записать
суммы пар элементов (N сумм): первого и последнего, второго и предпоследнего,
третьего и пред-предпоследнего... Вывести на экран результирующий массив и
сумму его элементов, а также индексы минимального и максимального эл-та (если
таких несколько, то первый минимальный и последний максимальный). Повторять
выполнение программы до тех пор, пока в качестве N не введут 0.

Иванов И.И. (1057/1): Массивы, вещ. случайные числа от 0 до N.
Введите размер (или 0 для завершения): 5
Сгенерированный массив A = { 3.40 0.30 4.70 1.10 4.50 }, сумма = 14
Результат B = { 7.90 1.40 9.40 1.40 7.90 }, сумма = 28
Минимум = B[1], максимум = B[2]
Введите размер (или 0 для завершения): 0

%%%%%%%%%%%%%%%%%%%%%%%%%%%%%%%%%%%%%%%%%%%%%%%%%%%%%%%%%%%%%%%%%%%%%%%%%%%%%%

Задача C-16

В начало массива из 20 элементов записать N случайных вещественных чисел из
диапазона от 0 до N с точностью 0.1, воспользовавшись функцией rand(). Число N
вводится с клавиатуры. Сосчитать сумму элементов. В другой массив записать
суммы пар элементов (N сумм): первого и второго, второго и третьего, ...,
последнего и первого. Вывести на экран результирующий массив и сумму его
элементов, а также индексы минимального и максимального эл-та (если таких
несколько, то первый минимальный и последний максимальный). Повторять
выполнение программы до тех пор, пока в качестве N не введут 0.

Иванов И.И. (1057/1): Массивы, вещ. случайные числа от 0 до N.
Введите размер (или 0 для завершения): 5
Сгенерированный массив A = { 3.40 0.30 4.70 1.10 4.50 }, сумма = 14
Результат B = { 3.70 5.00 5.80 5.60 7.90 }, сумма = 28
Минимум = B[0], максимум = B[4]
Введите размер (или 0 для завершения): 0

%%%%%%%%%%%%%%%%%%%%%%%%%%%%%%%%%%%%%%%%%%%%%%%%%%%%%%%%%%%%%%%%%%%%%%%%%%%%%%

Задача C-17

В начало массива из 20 элементов записать N случайных натуральных чисел из
диапазона от 1 до N, воспользовавшись функцией rand(). Число N вводится с
клавиатуры. Найти индексы минимального и максимального эл-та (если таких
несколько, то первый минимальный и последний максимальный). В другой массив
записать элементы, лежащие между найденными индексами включительно. Вывести на
экран результирующий массив и сумму его элементов. Повторять выполнение
программы до тех пор, пока в качестве N не введут 0.

Иванов И.И. (1057/1): Массивы, нат.случайные числа от 1 до N.
Введите размер (или 0 для завершения): 5
Сгенерированный массив A = { 4 3 1 4 5 }
Минимум = A[2], максимум = A[4]
Результат B = { 1 4 5 }, сумма = 10
Введите размер (или 0 для завершения): 0

%%%%%%%%%%%%%%%%%%%%%%%%%%%%%%%%%%%%%%%%%%%%%%%%%%%%%%%%%%%%%%%%%%%%%%%%%%%%%%

Задача C-18

В начало массива из 20 элементов записать N случайных целых чисел из диапазона
от A до B, воспользовавшись функцией rand(). Числа N, A, B вводятся с
клавиатуры. Найти индексы минимального и максимального эл-та (если таких
несколько, то первый минимальный и последний максимальный). В другой массив
записать элементы, лежащие между найденными индексами включительно. Вывести на
экран результирующий массив и сумму его элементов. Повторять выполнение
программы до тех пор, пока в качестве N не введут 0.

Иванов И.И. (1057/1): Массивы, целые случайные числа от A до B.
Введите размер (или 0 для завершения): 5
Введите диапазон A B через пробел: 3 7
Сгенерированный массив A = { 4 3 7 4 5 }
Минимум = A[1], максимум = A[2]
Результат B = { 3 7 }, сумма = 10
Введите размер (или 0 для завершения): 0

%%%%%%%%%%%%%%%%%%%%%%%%%%%%%%%%%%%%%%%%%%%%%%%%%%%%%%%%%%%%%%%%%%%%%%%%%%%%%%

Задача C-19

В начало массива из 20 элементов записать N случайных вещественных чисел из
диапазона от 0 до 1, воспользовавшись функцией rand(). Число N вводится с
клавиатуры. Найти индексы минимального и максимального эл-та (если таких
несколько, то первый минимальный и последний максимальный). В другой массив
записать элементы, лежащие между найденными индексами включительно. Вывести на
экран результирующий массив и сумму его элементов. Повторять выполнение
программы до тех пор, пока в качестве N не введут 0.

Иванов И.И. (1057/1): Массивы, вещ. случайные числа от 0 до 1.
Введите размер (или 0 для завершения): 5
Сгенерированный массив A = { 0.42 0.11 0.29 0.94 0.75 }
Минимум = A[1], максимум = A[3]
Результат B = { 0.11 0.29 0.94 }, сумма = 1.34
Введите размер (или 0 для завершения): 0

%%%%%%%%%%%%%%%%%%%%%%%%%%%%%%%%%%%%%%%%%%%%%%%%%%%%%%%%%%%%%%%%%%%%%%%%%%%%%%

Задача C-20

В начало массива из 20 элементов записать N случайных вещественных чисел из
диапазона от 0 до N с точностью 0.1, воспользовавшись функцией rand(). Число N
вводится с клавиатуры. Найти индексы минимального и максимального эл-та (если
таких несколько, то первый минимальный и последний максимальный). В другой
массив записать элементы, лежащие между найденными индексами включительно.
Вывести на экран результирующий массив и сумму его элементов. Повторять
выполнение программы до тех пор, пока в качестве N не введут 0.

Иванов И.И. (1057/1): Массивы, вещ. случайные числа от 0 до N.
Введите размер (или 0 для завершения): 5
Сгенерированный массив A = { 0.40 0.90 0.20 0.10 0.70 }
Минимум = A[3], максимум = A[1]
Результат B = { 0.90 0.20 0.10 }, сумма = 1.20
Введите размер (или 0 для завершения): 0

%%%%%%%%%%%%%%%%%%%%%%%%%%%%%%%%%%%%%%%%%%%%%%%%%%%%%%%%%%%%%%%%%%%%%%%%%%%%%%

Задача C-21

В начало массива из 20 элементов записать N случайных вещественных чисел из
диапазона от 0 до N с точностью 0.1, воспользовавшись функцией rand(). Число N
вводится с клавиатуры. Найти индексы минимального и максимального эл-та (если
таких несколько, то первый минимальный и последний максимальный). В другой
массив записать элементы, лежащие между найденными индексами включительно,
начиная с минимального индекса (может оказаться в обратном порядке). Вывести
на экран результирующий массив и сумму его элементов. Повторять выполнение
программы до тех пор, пока в качестве N не введут 0.

Иванов И.И. (1057/1): Массивы, вещ. случайные числа от 0 до N.
Введите размер (или 0 для завершения): 5
Сгенерированный массив A = { 0.40 0.90 0.20 0.10 0.70 }
Минимум = A[3], максимум = A[1]
Результат B = { 0.10 0.20 0.90 }, сумма = 1.20
Введите размер (или 0 для завершения): 0

%%%%%%%%%%%%%%%%%%%%%%%%%%%%%%%%%%%%%%%%%%%%%%%%%%%%%%%%%%%%%%%%%%%%%%%%%%%%%%

