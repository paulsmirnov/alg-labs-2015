%%%%%%%%%%%%%%%%%%%%%%%%%%%%%%%%%%%%%%%%%%%%%%%%%%%%%%%%%%%%%%%%%%%%%%%%%%%%%%
\zztaskgroup{BIN}{Бинарные файлы}
%%%%%%%%%%%%%%%%%%%%%%%%%%%%%%%%%%%%%%%%%%%%%%%%%%%%%%%%%%%%%%%%%%%%%%%%%%%%%%
\zztodo{Причесать и во второй семестр за милую душу.}
Эти задачки на работу с файлами как с бинарными данными: fopen() в режимах
“rb” / ”wb”, чтение и запись с помощью fread() / fwrite(). Дополнительные
функции fseek(), ftell().

%%%%%%%%%%%%%%%%%%%%%%%%%%%%%%%%%%%%%%%%%%%%%%%%%%%%%%%%%%%%%%%%%%%%%%%%%%%%%%

IX-1 Степень компрессии BMP/TIFF/…

Зная внутреннюю структуру файла, хранящего сжатое изображение в формате
BMP/TIFF/… (без знания алгоритма сжатия), определить степень компрессии
изображения как отношение полного размера файла к размеру памяти, необходимой
для хранения несжатого изображения (ширина * высота * <<байт на точку>> +
заголовок).

%%%%%%%%%%%%%%%%%%%%%%%%%%%%%%%%%%%%%%%%%%%%%%%%%%%%%%%%%%%%%%%%%%%%%%%%%%%%%%

IX-2 Гистограмма из BMP/TIFF/…

Зная внутреннюю структуру файла, хранящего изображение в формате BMP/TIFF/…
(только непакованное изображение), проанализировать содержимое и собрать
статистику: сколько пикселов изображения имеют яркость N (N=0..255) и
построить гистограмму (столбчатую диаграмму зависимости количества пикселов от
яркости). Яркость вычислять по формуле (R+G+B)/3, где R,G,B – красная, зеленая
и синяя компонента цвета пиксела.

%%%%%%%%%%%%%%%%%%%%%%%%%%%%%%%%%%%%%%%%%%%%%%%%%%%%%%%%%%%%%%%%%%%%%%%%%%%%%%

IX-3 Deinterlace BMP/TIFF/…

Зная внутреннюю структуру файла, хранящего изображение в формате BMP/TIFF/…
(только непакованное изображение), изменить изображение следующим образом:
собрать все четные строки изображения (0, 2, 4...) в верхней половине
изображения, а нечетные — в нижней. Новый файл под тем же самым именем должен
быть записать вместо исходного.

%%%%%%%%%%%%%%%%%%%%%%%%%%%%%%%%%%%%%%%%%%%%%%%%%%%%%%%%%%%%%%%%%%%%%%%%%%%%%%

IX-4 Interlace BMP/TIFF/…

Зная внутреннюю структуру файла, хранящего изображение в формате BMP/TIFF/…
(только непакованное изображение), изменить изображение следующим образом:
перемешать верхнюю и нижнюю половины изображения, перемежая строчки через
одну. Четные строки конечного изображения (0, 2, 4...) брать в верхней
половине изображения, а нечетные — в нижней. Новый файл под тем же самым
именем должен быть записать вместо исходного.

%%%%%%%%%%%%%%%%%%%%%%%%%%%%%%%%%%%%%%%%%%%%%%%%%%%%%%%%%%%%%%%%%%%%%%%%%%%%%%

IX-5 Rotate BMP/TIFF/…

Зная внутреннюю структуру файла, хранящего изображение в формате BMP/TIFF/…
(только непакованное изображение), повернуть изображение на 90 градусов против
часовой стрелки. Новый файл под тем же самым именем должен быть записать
вместо исходного.

%%%%%%%%%%%%%%%%%%%%%%%%%%%%%%%%%%%%%%%%%%%%%%%%%%%%%%%%%%%%%%%%%%%%%%%%%%%%%%

IX-6 Список файлов в ZIP/RAR/…

Зная внутренню структуру архива ZIP/RAR/… (без необходимости компрессии /
декомпрессии), вывести список файлов, находящихся в нем. Если задан ключ /m,
изменить в архиве регистр букв в именах файлов (заглавные на строчные и
наоборот).

%%%%%%%%%%%%%%%%%%%%%%%%%%%%%%%%%%%%%%%%%%%%%%%%%%%%%%%%%%%%%%%%%%%%%%%%%%%%%%

IX-7 Сохранение файлов в ZIP/RAR/…

Зная внутренню структуру архива ZIP/RAR/… (без необходимости компрессии /
декомпрессии), создать архив содержащий файлы, переданные как параметры.
Использовать степень компрессии 0 (store), т.е. не сжимать данные.

%%%%%%%%%%%%%%%%%%%%%%%%%%%%%%%%%%%%%%%%%%%%%%%%%%%%%%%%%%%%%%%%%%%%%%%%%%%%%%
