%%%%%%%%%%%%%%%%%%%%%%%%%%%%%%%%%%%%%%%%%%%%%%%%%%%%%%%%%%%%%%%%%%%%%%%%%%%%%%
\chapter{Задачи}
%%%%%%%%%%%%%%%%%%%%%%%%%%%%%%%%%%%%%%%%%%%%%%%%%%%%%%%%%%%%%%%%%%%%%%%%%%%%%%

Бла-бла-бла о задачах в целом\zztodo{Написать что-нибудь умное}.

Более-менее сложные пограммы имеет смысл писать по частям, реализуя требуемую
функциональность постепенно. Необходимо примерно спланировать решение всей
задачи, понять, из решения каких основных подзадач небольшого размера оно
состоит, и упорядочить их по шагам таким образом, чтобы после каждого шага
иметь осмысленный, легко проверяемый результат, приближающий вас к решению
основной задачи.

В предлагаемых задачах разбиение на такие шаги уже выполнено. Старайтесь,
по-возможности, придерживаться указанного порядка.

%%%%%%%%%%%%%%%%%%%%%%%%%%%%%%%%%%%%%%%%%%%%%%%%%%%%%%%%%%%%%%%%%%%%%%%%%%%%%%
\zztaskgroup{XPR}{Вычисление выражений и взаимодействие с пользователем}
%%%%%%%%%%%%%%%%%%%%%%%%%%%%%%%%%%%%%%%%%%%%%%%%%%%%%%%%%%%%%%%%%%%%%%%%%%%%%%

В следующих задачах требуется написать программу, автоматизирующую вычисления
по заданной формуле. Формула может включать в себя несколько заранее известных
констант и заранее неизвестных переменных (параметров), которые должен ввести
в программу пользователь после соответствующего приглашения. Запуская
программу несколько раз, вводя разные значения параметров, пользователь сможет
получать соответствующее этим параметрам решение задачи.

Приложение должно быть дружелюбным к пользователю (user friendly), то есть
вести с ним разумный диалог. Пользователь в каждый момент времени должен
знать, чего ожидает от него программа. Программа должна производить проверку
корректности исходных данных, вводимых пользователем.
Обычно вводимые величины имеют минимум одно ограничение --- масса, длина,
скорость должны быть неотрицательны.
В случае недопустимых данных программа должна выдавать осмысленное сообщение
и повторно запрашивать соответствующее значение. Запросы должны повторяться
до тех пор пока пользователь не введет корректное значение. Так должно происходить
для каждого вводимого параметра.

Примеры диалога программы и пользователя\zztodo{АТ. Задача ясно сформулирована, простая, является хорошим "разгоночным" упражнением. Может стоит сделать наружный бесконечный цикл? Здесь формулировка гораздо подробнее, чем на классной(все константы есть и тп), но стоит ли ее делать домашкой?}:

\begin{zzoutput}
  Задание \thezztaskgroup-1: Диаметр шара заданной массы и плотности
  Введите массу (кг): \zzuser{-1.5}
  Ошибка ввода, масса должна быть больше нуля!
  Введите массу (кг): \zzuser{0}
  Ошибка ввода, масса должна быть больше нуля!
  Введите массу (кг): \zzuser{1.5}
  Введите плотность (кг/м3): \zzuser{0}
  Ошибка ввода, плотность должна быть больше нуля!
  Введите плотность (кг/м3): \zzuser{19320}
  Диаметр получившегося шара (м): d = 0.052929
\end{zzoutput}
 
Рекомендуется проверять программу на реальных значениях параметров, при
которых ответ на задачу заранее известен или может быть легко проверен на
адекватность. В тестировании могут пригодиться некоторые табличные данные.
Для сведения:
%
\begin{itemize}
%
\item плотность~$\rho$ золота, железа, льда и пробки ---
19320, 7870, 916 и 240~кг/м\textsuperscript{3} соответственно;
%
\item ускорение свободного падения $g$ примерно равно 
9.8~м/с\textsuperscript{2};
%
\item человек привык измерять углы в градусах, а компьютер ---
в радианах, это уже учтено в предлагаемых ниже формулах;
полный круг $360^{\circ}$ равен $2\pi$ радиан ($\pi = 3.14159265358979323846\dots$);
%
\item человек привык измерять проценты в диапазоне от 0 до 100\%,
а компьютер -- в долях, от 0 до 1, что тоже учитывают формулы.
\end{itemize}


%%%%%%%%%%%%%%%%%%%%%%%%%%%%%%%%%%%%%%%%%%%%%%%%%%%%%%%%%%%%%%%%%%%%%%%%%%%%%%
\bigskip
%%%%%%%%%%%%%%%%%%%%%%%%%%%%%%%%%%%%%%%%%%%%%%%%%%%%%%%%%%%%%%%%%%%%%%%%%%%%%%


\begin{zztask} \zztodo{Тест примечаний к условиям задач}
Какого диаметра получится шар массы $m$, изготовленный из материала\zztodo{во даёт!} с
плотностью~$\rho$?
%
\[
d = 2\cdot \sqrt[3]{\frac{3m}{4\pi\rho}}
\]
\end{zztask}

%%%%%%%%%%%%%%%%%%%%%%%%%%%%%%%%%%%%%%%%%%%%%%%%%%%%%%%%%%%%%%%%%%%%%%%%%%%%%%

\begin{zztask}
Какой массы получится шар диаметра $d$, изготовленный из материала с
плотностью~$\rho$?
%
\[
m = \frac{1}{6}\pi\rho d^3
\]
\end{zztask}

%%%%%%%%%%%%%%%%%%%%%%%%%%%%%%%%%%%%%%%%%%%%%%%%%%%%%%%%%%%%%%%%%%%%%%%%%%%%%%

\begin{zztask}
Какого диаметра получится стержень массы $m$ и длины $l$, изготовленный из
материала с плотностью~$\rho$?
%
\[
d = 2\cdot \sqrt{\frac{m}{\pi\rho l}}
\]
\end{zztask}

%%%%%%%%%%%%%%%%%%%%%%%%%%%%%%%%%%%%%%%%%%%%%%%%%%%%%%%%%%%%%%%%%%%%%%%%%%%%%%

\begin{zztask}
Какой длины получится стержень массы $m$ диаметром $d$, изготовленный из
материала с плотностью~$\rho$?
%
\[
l = \frac{4m}{\pi\rho d^2}
\]
\end{zztask}

%%%%%%%%%%%%%%%%%%%%%%%%%%%%%%%%%%%%%%%%%%%%%%%%%%%%%%%%%%%%%%%%%%%%%%%%%%%%%%

\begin{zztask}
Какой массы получится стержень диаметром $d$ длиной $l$, изготовленный из
материала с плотностью~$\rho$?
%
\[
m = \frac{1}{4}\pi\rho d^2 l
\]
\end{zztask}

%%%%%%%%%%%%%%%%%%%%%%%%%%%%%%%%%%%%%%%%%%%%%%%%%%%%%%%%%%%%%%%%%%%%%%%%%%%%%%

\begin{zztask}
Какой массы получится пирамида с длиной ребра $a$, изготовленная из
материала с плотностью~$\rho$?
%
\[
m = \frac{\sqrt2}{12}\rho a^3
\]
\end{zztask}

%%%%%%%%%%%%%%%%%%%%%%%%%%%%%%%%%%%%%%%%%%%%%%%%%%%%%%%%%%%%%%%%%%%%%%%%%%%%%%

\begin{zztask}
Какой высоты получится пирамида массы $m$, изготовленная из
материала с плотностью~$\rho$?
%
\[
h = \frac{\sqrt6}{3} \sqrt[3]{\frac{6\sqrt2 m}{\rho}}
\]
\end{zztask}

%%%%%%%%%%%%%%%%%%%%%%%%%%%%%%%%%%%%%%%%%%%%%%%%%%%%%%%%%%%%%%%%%%%%%%%%%%%%%%

\begin{zztask}
Как далеко улетит тело, если его бросить под углом $\alpha$ к горизонту
со скоростью $v$?
%
\[
l = \frac{v^2}{g}\sin\frac{\alpha\pi}{90}
\]
\end{zztask}

%%%%%%%%%%%%%%%%%%%%%%%%%%%%%%%%%%%%%%%%%%%%%%%%%%%%%%%%%%%%%%%%%%%%%%%%%%%%%%

\begin{zztask}
Под каким углом к горизонту нужно бросить тело, чтобы оно улетело на
расстояние $l$ со скоростью $v$?
%
\[
\alpha = \frac{90}{\pi}\arcsin\frac{lg}{v^2}
\]
\end{zztask}

%%%%%%%%%%%%%%%%%%%%%%%%%%%%%%%%%%%%%%%%%%%%%%%%%%%%%%%%%%%%%%%%%%%%%%%%%%%%%%

\begin{zztask}
С какой скоростью нужно бросить тело под углом $\alpha$ к горизонту,
чтобы оно улетело на расстояние $l$?
%
\[
v = \sqrt{\frac{lg}{\sin(\alpha\pi/90)}}
\]
\end{zztask}

%%%%%%%%%%%%%%%%%%%%%%%%%%%%%%%%%%%%%%%%%%%%%%%%%%%%%%%%%%%%%%%%%%%%%%%%%%%%%%

\begin{zztask}
Какой доход за год принесет сумма $m$ на счёте в банке при годовой
процентной ставке $p$, если вклад подразумевает ежемесячную
капитализацию процентов?
%
\[
i = m \left(1 + \frac{p}{1200}\right)^{12} - m
\]
\end{zztask}

%%%%%%%%%%%%%%%%%%%%%%%%%%%%%%%%%%%%%%%%%%%%%%%%%%%%%%%%%%%%%%%%%%%%%%%%%%%%%%

\begin{zztask}
Сколько денег надо положить на счет в банк, чтобы при ежемесячной
капитализации процентов получить годовой доход $i$ при процентной
ставке $p$?
%
\[
m = \frac{i}{\left(1 + p/1200\right)^{12} - 1}
\]
\end{zztask}

%%%%%%%%%%%%%%%%%%%%%%%%%%%%%%%%%%%%%%%%%%%%%%%%%%%%%%%%%%%%%%%%%%%%%%%%%%%%%%

\begin{zztask}
Какой должна быть ежемесячная выплата по кредиту, если сумма $m$ берётся 
под процент $p$ на $k$ месяцев?
%
\[
a = \frac{mp}{1200}\left(1 + \frac{1}{(1 + p/1200)^k - 1}\right)
\]
\end{zztask}

%%%%%%%%%%%%%%%%%%%%%%%%%%%%%%%%%%%%%%%%%%%%%%%%%%%%%%%%%%%%%%%%%%%%%%%%%%%%%%

\begin{zztask}
Какой будет суммарная переплата по кредиту, если сумма $m$ берётся
под процент $p$ на $k$ месяцев?
%
\[
i = \frac{mkp}{1200}\left(1 + \frac{1}{(1 + p/1200)^k - 1}\right) - m
\]
\end{zztask}

%%%%%%%%%%%%%%%%%%%%%%%%%%%%%%%%%%%%%%%%%%%%%%%%%%%%%%%%%%%%%%%%%%%%%%%%%%%%%%

\begin{zztask}
На какую максимальную сумму кредита под процент $p$ на $k$ месяцев
можно расчитывать, если есть возможность отдавать ежемесячно сумму $a$?
%
\[
m = \frac{a}{p/1200}\left(1 - \frac{1}{(1 + p/1200)^k}\right)
\]
\end{zztask}

%%%%%%%%%%%%%%%%%%%%%%%%%%%%%%%%%%%%%%%%%%%%%%%%%%%%%%%%%%%%%%%%%%%%%%%%%%%%%%


%%%%%%%%%%%%%%%%%%%%%%%%%%%%%%%%%%%%%%%%%%%%%%%%%%%%%%%%%%%%%%%%%%%%%%%%%%%%%%
\zztaskgroup{SEQ}{Последовательность}
%%%%%%%%%%%%%%%%%%%%%%%%%%%%%%%%%%%%%%%%%%%%%%%%%%%%%%%%%%%%%%%%%%%%%%%%%%%%%%

В следующих задачах требуется написать программу, находящую сумму первых $n$
чисел последовательности, строящейся по определенному правилу ($n$ вводится в программу
пользователем). Сумму надо посчитать двумя способами, складывая числа в цикле и
используя замкнутую формулу, выведенную из математических соображений, если это возможно. % \zztodo{А почему написано "заданную формулу", а сами формулы закомменчены?}
Результаты сравнить. На экран также вывести
суммируемые числа, разделенные знаками <<плюс>> (или <<минус>>, если число
отрицательное).

Примечание: в некоторых задачах формулу вывести сложно или невозможно, это
указано в их условии. % \zztodo{Что изучается в этой лабе? Циклы? Вместо чего ее можно давать? вместо вычисления выражения для сильных или как? Довольно простая задачка, может ли на классную пойти?}

Примеры диалога программы и пользователя:

\begin{zzoutput}
  Задание \thezztaskgroup-1: Суммирование N натуральных чисел
  Введите N: \zzuser{5}
  Сумма равна 1 + 2 + 3 + 4 + 5 = 15 (и 15 по формуле)
\end{zzoutput}


%%%%%%%%%%%%%%%%%%%%%%%%%%%%%%%%%%%%%%%%%%%%%%%%%%%%%%%%%%%%%%%%%%%%%%%%%%%%%%
\bigskip
%%%%%%%%%%%%%%%%%%%%%%%%%%%%%%%%%%%%%%%%%%%%%%%%%%%%%%%%%%%%%%%%%%%%%%%%%%%%%%


\begin{zztask}[Натуральные числа]
В рамках общего условия задачи найти сумму первых $n$ натуральных чисел
(первые числа: $1$, $2$, $3$, $4$, $5$\dots).
Формула: $s = n(n+1)/2$.
\end{zztask}

%%%%%%%%%%%%%%%%%%%%%%%%%%%%%%%%%%%%%%%%%%%%%%%%%%%%%%%%%%%%%%%%%%%%%%%%%%%%%%

\begin{zztask}[Нечётные числа]
В рамках общего условия задачи найти сумму первых $n$ нечётных натуральных 
чисел (первые числа: $1$, $3$, $5$, $7$, $9$\dots).
Формула: $s = n^2$.
\end{zztask}

%%%%%%%%%%%%%%%%%%%%%%%%%%%%%%%%%%%%%%%%%%%%%%%%%%%%%%%%%%%%%%%%%%%%%%%%%%%%%%

\begin{zztask}[Степени двойки]
В рамках общего условия задачи найти сумму первых $n$ чисел, являющихся
степенью двойки. Примечание: $k$-ая степень двойки получается из предыдущей
умножением на 2 (первые числа: $1$, $1\cdot2=2$, $2\cdot2=4$, $4\cdot2=8$, 
$16$\dots).
Формула: $s = 2^{n+1}-1$.
\end{zztask}

%%%%%%%%%%%%%%%%%%%%%%%%%%%%%%%%%%%%%%%%%%%%%%%%%%%%%%%%%%%%%%%%%%%%%%%%%%%%%%

\begin{zztask}[Треугольные числа]
В рамках общего условия задачи найти сумму первых $n$ треугольных чисел.
Примечание: $k$-ое треугольное число получается из предыдущего прибавлением
к нему $k$ (первые числа: $1$, $1+2=3$, $3+3=6$, $6+4=10$, $15$, $21$\dots).
Формула: $s = n(n+1)(n+2)/6$.
\end{zztask}

%%%%%%%%%%%%%%%%%%%%%%%%%%%%%%%%%%%%%%%%%%%%%%%%%%%%%%%%%%%%%%%%%%%%%%%%%%%%%%

\begin{zztask}[Прямоугольные числа]
В рамках общего условия задачи найти сумму первых $n$ прямоугольных чисел.
Примечание: прямоугольные числа --- это такие натуральные числа, которые
являются произведением двух последовательных натуральных чисел
(первые числа: $1\cdot2=2$, $2\cdot3=6$, $3\cdot4=12$, $20$, $30$, $42$\dots).
Формула: $s = n(n+1)(n+2)/3$.
\end{zztask}

%%%%%%%%%%%%%%%%%%%%%%%%%%%%%%%%%%%%%%%%%%%%%%%%%%%%%%%%%%%%%%%%%%%%%%%%%%%%%%

\begin{zztask}[Шестиугольные числа]
В рамках общего условия задачи найти сумму первых $n$ шестиугольных чисел.
Примечание: $k$-ое шестиугольное число получается из предыдущего прибавлением
к нему $4k-3$ (первые числа: $1$, $1+5=6$, $6+9=15$, $15+13=28$, $45$, $66$\dots).
Формула: $s = n(n+1)(4n-1)/6$.
\end{zztask}

%%%%%%%%%%%%%%%%%%%%%%%%%%%%%%%%%%%%%%%%%%%%%%%%%%%%%%%%%%%%%%%%%%%%%%%%%%%%%%

\begin{zztask}[Числа Фибоначчи]
В рамках общего условия задачи найти сумму первых $n$ чисел Фибоначчи.
Примечание: первые
два числа в последовательности чисел Фибоначчи это 0 и 1, а каждое следующее
считается как сумма двух предыдущих: $0$, $1$, $0+1=1$, $1+1=2$, $1+2=3$, $2+3=5$,
$8$, $13$\dots
Шаг подсчета по формуле опустить ввиду нетривиальности.
\end{zztask}

%%%%%%%%%%%%%%%%%%%%%%%%%%%%%%%%%%%%%%%%%%%%%%%%%%%%%%%%%%%%%%%%%%%%%%%%%%%%%%

\begin{zztask}[Числа анти-Фибоначчи]
В рамках общего условия задачи найти сумму первых $n$ чисел анти-Фибоначчи.
Примечание: первые
два числа в последовательности чисел анти-Фибоначчи это 1 и 0, а каждое
следующее считается как разность двух предыдущих: $1$, $0$, $1-0=1$, $0-1=-1$,
$1-(-1)=2$, $-3$, $5$\dots
Шаг подсчета по формуле опустить ввиду нетривиальности.
\end{zztask}

%%%%%%%%%%%%%%%%%%%%%%%%%%%%%%%%%%%%%%%%%%%%%%%%%%%%%%%%%%%%%%%%%%%%%%%%%%%%%%

\begin{zztask}[$^\star$Автоморфные числа]
В рамках общего условия задачи найти сумму первых $n$ автоморфных чисел.
Примечание:
автоморфные числа --- это такие натуральные числа, квадрат которых
оканчивается на само число (первые числа: $1$, $5$, $6$, $25$, $76$\dots).
Шаг подсчета по формуле опустить ввиду нетривиальности.  
\end{zztask}

%%%%%%%%%%%%%%%%%%%%%%%%%%%%%%%%%%%%%%%%%%%%%%%%%%%%%%%%%%%%%%%%%%%%%%%%%%%%%%

\begin{zztask}[$^\star$Простые числа]
В рамках общего условия задачи найти сумму первых $n$ простых чисел.
Примечание: простые
числа --- это такие натуральные числа, которые делятся нацело только на 1 и
на себя (первые числа: $1$, $3$, $5$, $7$, $11$\dots).
Шаг подсчета по формуле опустить ввиду нетривиальности.
\end{zztask}

%%%%%%%%%%%%%%%%%%%%%%%%%%%%%%%%%%%%%%%%%%%%%%%%%%%%%%%%%%%%%%%%%%%%%%%%%%%%%%

\begin{zztask}[$^\star$Нечётные числа в диапазоне]
В рамках общего условия задачи найти сумму всех нечётных целых чисел,
лежащих между вещественными границами $a$ и $b$ включительно. Границы вводятся
пользователем с клавиатуры, не обязательно в порядке возрастания.
\end{zztask}

%%%%%%%%%%%%%%%%%%%%%%%%%%%%%%%%%%%%%%%%%%%%%%%%%%%%%%%%%%%%%%%%%%%%%%%%%%%%%%


%%%%%%%%%%%%%%%%%%%%%%%%%%%%%%%%%%%%%%%%%%%%%%%%%%%%%%%%%%%%%%%%%%%%%%%%%%%%%%
\zztaskgroup{SER}{Сумма бесконечного ряда}
%%%%%%%%%%%%%%%%%%%%%%%%%%%%%%%%%%%%%%%%%%%%%%%%%%%%%%%%%%%%%%%%%%%%%%%%%%%%%%

В следующих задачах требуется написать программу, выводящую таблицу из $M$
значений некоторой функции $f(x)$ на промежутке от $a$ до $b$ (равномерно распределённых, в том числе и в
граничных точках). Все параметры ($a$, $b$, $M$) задаются пользователем,
функция в каждой задаче своя, причём заданная как сумма бесконечного ряда.
%
\[
  f(x)=S(x)=\sum_{k=0}^\infty u_k(x)=u_0(x)+u_1(x)+\cdots+u_k(x)+\cdots
\]
%
Для сравнения $f(x)$ также задана аналитически, через элементарные функции, и
условие на $x$, при котором это верно. Требуется вывести таблицу из $M$ строк
(по одной строке на значение $x \in [a,b]$), содержащих значение $x$, сумму
$S_n(x)$, точное значением $f(x)$, разницу $\Delta = |S_n(x) - f(x)|$,
количество потребовавшихся членов суммы $n$ и значение
$\varepsilon(S_n)=u_{n+1}$ (описание параметра $\varepsilon$ см. ниже).


%%%%%%%%%%%%%%%%%%%%%%%%%%%%%%%%%%%%%%%%%%%%%%%%%%%%%%%%%%%%%%%%%%%%%%%%%%%%%%
\zzsectionCOMMENTS
%%%%%%%%%%%%%%%%%%%%%%%%%%%%%%%%%%%%%%%%%%%%%%%%%%%%%%%%%%%%%%%%%%%%%%%%%%%%%%


\paragraph{Машинный эпсилон}
В данной задаче бесконечное число слагаемых складывается в течение бесконечного
времени, что делает задачу нерешаемой в такой формулировке за конечное время. 
Поэтому будем вычислять только частичную сумму $S_n$, где $n$ выбирается так,
чтобы получить максимально возможную для типа \texttt{double} точность
суммирования. Для этого будем использовать существование ``машинного
$\varepsilon$'' $\forall y$, то есть такого числа $\varepsilon(y) > 0$, которое при
прибавлении к $y$ не увеличивает его: $y + \varepsilon(y) = y$.
%
\[
S_n(x)=\sum_{k=0}^n u_k(x),\qquad S_n + u_{n+1} = S_n
\]

\paragraph{Вычисление значений слагаемых}
Поскольку слагаемые содержат возведения в степень и факториалы, которые не
могут быть вычислены в лоб с нужной точностью для достаточно больших $n$, а так же в целях увеличения производительности, необходимо составить формулу для вычисления следующего члена суммы с использованием значения предыдущего:
%
\[
u_k(x) = u_{k-1}(x)\cdot v_k(x)
\]
%
В таком случае отпадает необходимость в реализации собственных функции. Кроме того, вычисление в лоб может привести к переполнению, ввиду того, что промежуточные значения могут быть слишком велики (поразмышляйте, например, о вычислении значения слагаемого $\frac{n!}{(n-1)!}$ при $n = 70$).

\paragraph{Форматирование таблицы}
Итоговая таблица должна представлять собой не просто несколько столбцов числен, а аккуратную таблицу с границами, нарисованную с помощью символов.
Вертикальные линии должны обозначать границы каждого столбца, рекомендуется рисовать их с помощью символа ``|''. Горизонтальные линии обозначают границы таблицы и отделяют заголовок от остальных строк, горизонтальные линии нужно рисовать с помощью символа ``-'', а соединительные элементы - символом ``+''. Пример таблицы:
\begin{minted}{c}
+-----------+--------------+------------+-...
|     x     |     f(x)     |    s(x)    |
+-----------+--------------+------------+-...
|      0.10 |      0.00000 |    0.00000 |
|      0.11 |    -10.00001 |  -10.00001 |
|       ... |          ... |        ... |
+-----------+--------------+------------+-...
\end{minted}

\paragraph{Форматирование столбцов}
Для того, чтобы ширина таблицы имела фиксированную величину, необходимо
пользоваться дополнительным модификаторами при использовании функции
\texttt{printf()} (см., например,~\cite{cppref}). Эти модификаторы указываются
после знака $\%$ и перед конструкцией, обозначающей тип параметра, рассмотрим
необходимые нам модификаторы:
%
\begin{itemize}
	\item \textbf{Ширина}. Обозначает минимальное количество символов, которое будет занимать выводимое значение. Если длина значения меньше данной величины, оно будет выровнено по правому краю (для выравнивания по левому краю, используйте символ ``-'', например, \verb|%-8i|). При превышении указанной ширины, значение обрезаться не будет. Например, для вывода числа типа \texttt{double} с шириной поля $10$, необходимо написать \verb|%10lf|.
	\item \textbf{Точность}. Для вывода вещественных чисел может быть полезным использование модификатора точности вывода. Он указывается после модификатора ширины, через точку. Например, конструкция \verb|%10.5f| будет выводить вещественные числа в поле шириной $10$ и с точностью $5$ знаков после запятой.
	\item \textbf{Экспоненциальный формат}. Зачастую, для вывода очень больших или очень маленьких числе с плавающей точкой, вместо модификатора \verb|f| используют \verb|e|, например, при использовании этого модификатора для вывода числа $392.65$, вывод будет следующим: \texttt{3.9265e+2}. Рекомендуется использовать этот модификатор для вывода значения $\varepsilon$ и $\Delta$.
\end{itemize}


\begin{comment}
\begin{itemize}
	\item Здесь нужно оставить описание того, как сделать пошаговое вычисление суммы ряда.
	
	Непонятно только, что тут еще писать, как именно реализовывать работу? В целом тут нужно написать план в стиле:
	\begin{enumerate}
		\item Определитесь с первым членом последовательности
		\item Найдите, как выражается через $x$ и $i$ $v_{k+1}/v_k$. 
		\item Напишите обычный цикл, который вычисляет сумму  первых N членов последовательности и выведите на экран ее и референсное значение функции.
		\item Замените цикл на нормальный.
		 
	\end{enumerate}
	\item Описать, как должен выглядеть заголовок и одна строка таблицы (на живом примере). Описать особенности задания формата в функции printf.
	\item Напишите прототип функции построения таблицы. /*Можно вставить сам прототип*/ Напишите код, выводящий заголовок.
	Затем, напишите цикл, который выводит равномерные значения $x$ от $a$ до $b$. 
	\item Вставьте в тело цикла код вывода одной строчки таблицы.
\end{itemize}
\end{comment}
%%%%%%%%%%%%%%%%%%%%%%%%%%%%%%%%%%%%%%%%%%%%%%%%%%%%%%%%%%%%%%%%%%%%%%%%%%%%%%
\zzsectionPLAN
%%%%%%%%%%%%%%%%%%%%%%%%%%%%%%%%%%%%%%%%%%%%%%%%%%%%%%%%%%%%%%%%%%%%%%%%%%%%%%

\begin{enumerate}
\item Сначала необходимо реализовать функциональность для вычисления слагаемых нашего ряда. Для этого требуется определить, какое значение имеет первое слагаемое и каково отношение между двумя последовательными слагаемыми.
\item После этого можно реализовать вычисление частичных сумм. Для этого оформляем цикл, для начала с фиксированным числом итераций (например, 1000), перед циклом инициализируем значения частичной суммы и слагаемого, после этого в цикле обновляем эти переменные. Для теста можно выводить значения, получаемые на каждом шаге и значения функции, используемой для проверки.\zztodo{Стоит ли рекомендовать вынести вычисление суммы ряда в отдельную функцию, куда дельта и эпсилон передаются по указателю?}
\item Далее необходимо ввести механизм остановки, цикла, подсчет $\varepsilon$ и $\Delta$.
\item На данном этапе подготовлены все данные для вывода таблицы, можно приступать к нему. Комментарии по форматированию можно найти выше. 
\end{enumerate}


%%%%%%%%%%%%%%%%%%%%%%%%%%%%%%%%%%%%%%%%%%%%%%%%%%%%%%%%%%%%%%%%%%%%%%%%%%%%%%
\zzsectionVARIATIONS
%%%%%%%%%%%%%%%%%%%%%%%%%%%%%%%%%%%%%%%%%%%%%%%%%%%%%%%%%%%%%%%%%%%%%%%%%%%%%%


\begin{zztask}
В рамках общего условия задачи вывести таблицу значений функции, заданной рядом:
\[ % 100: "sin z", where z = x/2
	S(x)= \sum_{n=1}^\infty (-1)^{n+1} \frac{x^{2n-1}}{2^{2n-1}(2n-1)!};\quad
	f(x)= \sin \frac{x}{2},
	\quad 0 \leq x \leq 2;
\]
\end{zztask}

%%%%%%%%%%%%%%%%%%%%%%%%%%%%%%%%%%%%%%%%%%%%%%%%%%%%%%%%%%%%%%%%%%%%%%%%%%%%%%

\begin{zztask}
В рамках общего условия задачи вывести таблицу значений функции, заданной рядом:
\[ % 106: "cos z", where z = 2x
  S(x)= 1 + \sum_{n=1}^\infty (-1)^n \frac{2^{2n}}{(2n)!} x^{2n},\quad
  f(x)= \cos 2x,
  \quad |x| \leq 1;
\]
\end{zztask}

%%%%%%%%%%%%%%%%%%%%%%%%%%%%%%%%%%%%%%%%%%%%%%%%%%%%%%%%%%%%%%%%%%%%%%%%%%%%%%

\begin{zztask}
В рамках общего условия задачи вывести таблицу значений функции, заданной рядом:
\[ % 101: "sin^2 x = (1 - cos 2x)/2"
  S(x)= \sum_{n=1}^\infty (-1)^{n-1} \frac{2^{2n-1}}{(2n)!} x^{2n};\quad
  f(x)= \sin^2 x,
  \quad |x| \leq 1;
\]
\end{zztask}

%%%%%%%%%%%%%%%%%%%%%%%%%%%%%%%%%%%%%%%%%%%%%%%%%%%%%%%%%%%%%%%%%%%%%%%%%%%%%%

\begin{zztask}
В рамках общего условия задачи вывести таблицу значений функции, заданной рядом:
\[ % 104: "cos^2 x = (1 + cos 2x)/2"
  S(x)= 1 + \frac{1}{2} \sum_{n=1}^\infty (-1)^n \frac{(2x)^{2n}}{(2n)!};\quad
  f(x)= \cos^2 x,
  \quad |x| \leq 1;
\]
\end{zztask}

%%%%%%%%%%%%%%%%%%%%%%%%%%%%%%%%%%%%%%%%%%%%%%%%%%%%%%%%%%%%%%%%%%%%%%%%%%%%%%

\begin{zztask}
В рамках общего условия задачи вывести таблицу значений функции, заданной рядом:
\[
  S(x)= 2 \sum_{n=0}^\infty \frac{(x-1)^{2n+1}}{(2n+1)(x+1)^{2n+1}};\quad
  f(x)= \ln x,
  \quad 0 < x \leq 1;
\]
\end{zztask}

%%%%%%%%%%%%%%%%%%%%%%%%%%%%%%%%%%%%%%%%%%%%%%%%%%%%%%%%%%%%%%%%%%%%%%%%%%%%%%

\begin{zztask}
В рамках общего условия задачи вывести таблицу значений функции, заданной рядом:
\[
  S(x)= \sum_{n=1}^\infty (-1)^{n+1}\frac{(x-3)^{n-1}}{3^n};\quad
  f(x)= \frac{1}{x},
  \quad 3 < x \leq 4;
\]
\end{zztask}

%%%%%%%%%%%%%%%%%%%%%%%%%%%%%%%%%%%%%%%%%%%%%%%%%%%%%%%%%%%%%%%%%%%%%%%%%%%%%%

\begin{zztask}
В рамках общего условия задачи вывести таблицу значений функции, заданной рядом:
\[
  S(x)= x + \sum_{n=2}^\infty (-1)^{n-1}\frac{2^{n-1}}{(n-1)!} x^n;\quad
  f(x)= x/e^{2x},
  \quad |x| \leq 1;
\]
\end{zztask}

%%%%%%%%%%%%%%%%%%%%%%%%%%%%%%%%%%%%%%%%%%%%%%%%%%%%%%%%%%%%%%%%%%%%%%%%%%%%%%

\begin{zztask}
В рамках общего условия задачи вывести таблицу значений функции, заданной рядом:
\[
  S(x)= \frac{1}{2} \sum_{n=1}^\infty \frac{\big((n-1)!\big)^2}{(2n)!} (2x)^{2n};\quad
  f(x)= \arcsin^2x,
  \quad 0 \leq x \leq \sqrt2/2;
\]
\end{zztask}

%%%%%%%%%%%%%%%%%%%%%%%%%%%%%%%%%%%%%%%%%%%%%%%%%%%%%%%%%%%%%%%%%%%%%%%%%%%%%%

%%%%%%%%%%%%%%%%%%%%%%%%%%%%%%%%%%%%%%%%%%%%%%%%%%%%%%%%%%%%%%%%%%%%%%%%%%%%%%
\zztaskgroup{STR}{Обработка строк}
%%%%%%%%%%%%%%%%%%%%%%%%%%%%%%%%%%%%%%%%%%%%%%%%%%%%%%%%%%%%%%%%%%%%%%%%%%%%%%

В следующих задачах требуется написать указанную в варианте функцию для
обработки строк, \textbf{отвечающую заданному прототипу}. Требуется использовать
эту функцию в тестовой программе, которая должна в цикле читать строки с
клавиатуры и выдавать ответ на экран (желательно точно под исходной строкой)
до тех пор, \textbf{пока пользователь не введет пустую строку} (строку нулевой длины).
В этом задании можно считать, что все вводимые строки ограниченной
длины~($<100$) и использовать буфер (массив) фиксированного размера и
функцию чтения строк \texttt{gets(s)}.

Буквами договоримся считать прописные и строчные латинские буквы 
A--Z и a--z, словами~--- последовательность букв и цифр 0--9. Все остальные
символы в строке будем считать \textbf{разделителями} слов.

В учебных целях при решении задач данного раздела функций стандартной
библиотеки, упрощающих работу со строками и символами (напр. из 
\texttt{<string.h>}), следует \textbf{избегать}, реализуя необходимую
функциональность самостоятельно.\zztodo{К этой нет вопросов. и примерчики и полный пакет.}

Примеры диалога программы и пользователя:

\begin{zzoutput}
  Задание \thezztaskgroup-1: Разворот строк
  Введите строку: \zzuser{Hello, world!}
  Результат     : !dlrow ,olleH
  Введите строку: \zzuser{int main(void);}
  Результат     : ;)diov(niam tni
  Введите строку: \zzuser{ }
\end{zzoutput}


%%%%%%%%%%%%%%%%%%%%%%%%%%%%%%%%%%%%%%%%%%%%%%%%%%%%%%%%%%%%%%%%%%%%%%%%%%%%%%
\bigskip
%%%%%%%%%%%%%%%%%%%%%%%%%%%%%%%%%%%%%%%%%%%%%%%%%%%%%%%%%%%%%%%%%%%%%%%%%%%%%%


\begin{zztask}[Разворот строк]
В рамках общего условия задачи написать функцию, которая в заданном буфере
разворачивает строку задом наперед.
Например, из
<<\texttt{The good and the EVIL ones.}>>
должно получиться
<<\texttt{.seno LIVE eht dna doog ehT}>>.

Прототип: \mintinline{c}|void Reverse(char str[]);|
\end{zztask}

%%%%%%%%%%%%%%%%%%%%%%%%%%%%%%%%%%%%%%%%%%%%%%%%%%%%%%%%%%%%%%%%%%%%%%%%%%%%%%

\begin{zztask}[Разворот слов]
В рамках общего условия задачи написать функцию, которая в заданном буфере
разворачивает каждое слово строки задом наперед. Разделители при этом не
меняются.
Например, из
<<\texttt{The good and the EVIL ones.}>>
должно получиться
<<\texttt{ehT doog dna eht LIVE seno.}>>.

Прототип: \mintinline{c}|void ReverseWords(char str[]);|
\end{zztask}

%%%%%%%%%%%%%%%%%%%%%%%%%%%%%%%%%%%%%%%%%%%%%%%%%%%%%%%%%%%%%%%%%%%%%%%%%%%%%%

\begin{zztask}[Верхний регистр]
В рамках общего условия задачи написать функцию, которая в заданном буфере
все буквы заменяет на заглавные.
Например, из
<<\texttt{The good and the EVIL ones.}>>
должно получиться
<<\texttt{THE GOOD AND THE EVIL ONES.}>>.

Прототип: \mintinline{c}|void UpperCase(char str[]);|
\end{zztask}

%%%%%%%%%%%%%%%%%%%%%%%%%%%%%%%%%%%%%%%%%%%%%%%%%%%%%%%%%%%%%%%%%%%%%%%%%%%%%%

\begin{zztask}[Нижний регистр]
В рамках общего условия задачи написать функцию, которая в заданном буфере
все буквы заменяет на строчные.
Например, из
<<\texttt{The good and the EVIL ones.}>>
должно получиться
<<\texttt{the good and the evil ones.}>>.

Прототип: \mintinline{c}|void LowerCase(char str[]);|
\end{zztask}

%%%%%%%%%%%%%%%%%%%%%%%%%%%%%%%%%%%%%%%%%%%%%%%%%%%%%%%%%%%%%%%%%%%%%%%%%%%%%%

\begin{zztask}[Смена регистра]
В рамках общего условия задачи написать функцию, которая в заданном буфере
все заглавные буквы заменяет на строчные, а строчные~--- на заглавные.
Например, из
<<\texttt{The good and the EVIL ones.}>>
должно получиться
<<\texttt{tHE GOOD AND THE evil ONES.}>>.

Прототип: \mintinline{c}|void SwapCase(char str[]);|
\end{zztask}

%%%%%%%%%%%%%%%%%%%%%%%%%%%%%%%%%%%%%%%%%%%%%%%%%%%%%%%%%%%%%%%%%%%%%%%%%%%%%%

\begin{zztask}[Упрощенный заголовок]
В рамках общего условия задачи написать функцию, которая в заданном буфере
все первые буквы слов заменяет на заглавные, а остальные~--- на строчные.
Например, из
<<\texttt{The good and the EVIL ones.}>>
должно получиться
<<\texttt{The Good And The Evil Ones.}>>.

Прототип: \mintinline{c}|void TitleCase(char str[]);|
\end{zztask}

%%%%%%%%%%%%%%%%%%%%%%%%%%%%%%%%%%%%%%%%%%%%%%%%%%%%%%%%%%%%%%%%%%%%%%%%%%%%%%

\begin{zztask}[Заголовок]
В рамках общего условия задачи написать функцию, которая в заданном буфере
все первые буквы длинных слов ($>3$) заменяет на заглавные, а остальные~--- на
строчные. Короткое слово должно начинаться с заглавной только если это первое
слово строки.
Например, из
<<\texttt{The good and the EVIL ones.}>>
должно получиться
<<\texttt{The Good and the Evil Ones.}>>.

Прототип: \mintinline{c}|void TitleCase(char str[]);|
\end{zztask}

%%%%%%%%%%%%%%%%%%%%%%%%%%%%%%%%%%%%%%%%%%%%%%%%%%%%%%%%%%%%%%%%%%%%%%%%%%%%%%

\begin{zztask}[Удаление разделителей]
В рамках общего условия задачи написать функцию, которая в заданном буфере
удаляет разделители, прижимая слова друг к другу.
Например, из
<<\texttt{The good and the EVIL ones.}>>
должно получиться\linebreak
<<\texttt{ThegoodandtheEVILones}>>.

Прототип: \mintinline{c}|void RemoveSeparators(char str[]);|
\end{zztask}

%%%%%%%%%%%%%%%%%%%%%%%%%%%%%%%%%%%%%%%%%%%%%%%%%%%%%%%%%%%%%%%%%%%%%%%%%%%%%%

\begin{zztask}[Удаление подстрок]
В рамках общего условия задачи написать функцию, которая в заданном буфере
находит и удаляет все подстроки <<\texttt{main}>> (не обязательно слова).
Необходимо обойтись без дополнительного буфера.

Прототип: \mintinline{c}|void RemoveMain(char str[]);|
\end{zztask}

%%%%%%%%%%%%%%%%%%%%%%%%%%%%%%%%%%%%%%%%%%%%%%%%%%%%%%%%%%%%%%%%%%%%%%%%%%%%%%

\begin{zztask}[Удаление слов]
В рамках общего условия задачи написать функцию, которая в заданном буфере
находит и удаляет все слова <<\texttt{main}>> (только слова).
Необходимо обойтись без дополнительного буфера.

Прототип: \mintinline{c}|void RemoveMainWord(char str[]);|
\end{zztask}

%%%%%%%%%%%%%%%%%%%%%%%%%%%%%%%%%%%%%%%%%%%%%%%%%%%%%%%%%%%%%%%%%%%%%%%%%%%%%%

\begin{zztask}[Замена слов]
В рамках общего условия задачи написать функцию, которая в заданном буфере
находит и заменяет все слова <<\texttt{hello}>> на <<\texttt{bye}>> (только слова).
Необходимо обойтись без дополнительного буфера.

Прототип: \mintinline{c}|void ReplaceHelloWord(char str[]);|
\end{zztask}

%%%%%%%%%%%%%%%%%%%%%%%%%%%%%%%%%%%%%%%%%%%%%%%%%%%%%%%%%%%%%%%%%%%%%%%%%%%%%%

\begin{zztask}[Длинные слова]
В рамках общего условия задачи написать функцию, которая в заданном буфере
находит первое из слов максимальной длины и оставляет в буфере только его,
сдвигая к началу.
Например, из
<<\texttt{The good and the EVIL ones.}>>
должно получиться
<<\texttt{good}>>.

Прототип: \mintinline{c}|void FindLongWords(char str[]);|
\end{zztask}

%%%%%%%%%%%%%%%%%%%%%%%%%%%%%%%%%%%%%%%%%%%%%%%%%%%%%%%%%%%%%%%%%%%%%%%%%%%%%%

\begin{zztask}[Забой длинных слов]
В рамках общего условия задачи написать функцию, которая в заданном буфере
находит все слова максимальной длины и забивает их звездочками.
Например, из
<<\texttt{The good and the EVIL ones.}>>
должно получиться
<<\texttt{The **** and the **** ****.}>>.

Прототип: \mintinline{c}|void HideLongWords(char str[]);|
\end{zztask}

%%%%%%%%%%%%%%%%%%%%%%%%%%%%%%%%%%%%%%%%%%%%%%%%%%%%%%%%%%%%%%%%%%%%%%%%%%%%%%

\begin{zztask}[Забой слов-палиндромов]
В рамках общего условия задачи написать функцию, которая в заданном буфере
находит все слова, являющиеся палиндромами (игнорируя регистр букв), и
забивает их звездочками.

Прототип: \mintinline{c}|void HidePalindromes(char str[]);|
\end{zztask}

%%%%%%%%%%%%%%%%%%%%%%%%%%%%%%%%%%%%%%%%%%%%%%%%%%%%%%%%%%%%%%%%%%%%%%%%%%%%%%

\begin{zztask}[Подмена слов]
В рамках общего условия задачи написать функцию, которая в заданном буфере
находит все слова максимальной длины и заменяет их все на первое из
найденных.
Необходимо обойтись без дополнительного буфера.
Например, из
<<\texttt{The good and the EVIL ones.}>>
должно получиться
<<\texttt{The good and the good good.}>>.

Прототип: \mintinline{c}|void SubstituteLongWords(char str[]);|
\end{zztask}

%%%%%%%%%%%%%%%%%%%%%%%%%%%%%%%%%%%%%%%%%%%%%%%%%%%%%%%%%%%%%%%%%%%%%%%%%%%%%%

\begin{zztask}[Обмен слов]
В рамках общего условия задачи написать функцию, которая в заданном буфере
меняет местами соседние пары слов, сдвигая разделители при необходимости, то
есть из <<\texttt{a+=bc*d}>> должно получиться <<\texttt{bc+=a*d}>>.
Необходимо обойтись без дополнительного буфера.

Прототип: \mintinline{c}|void SwapWords(char str[]);|
\end{zztask}

%%%%%%%%%%%%%%%%%%%%%%%%%%%%%%%%%%%%%%%%%%%%%%%%%%%%%%%%%%%%%%%%%%%%%%%%%%%%%%

\begin{zztask}[Циклическая перестановка слов]
В рамках общего условия задачи написать функцию, которая в заданном буфере
циклически переставляет слова влево, сдвигая разделители при необходимости,
то есть из <<\texttt{a+=bc*d}>> должно получиться <<\texttt{bc+=d*a}>>.
Необходимо обойтись без дополнительного буфера.

Прототип: \mintinline{c}|void RotateWords(char str[]);|
\end{zztask}

%%%%%%%%%%%%%%%%%%%%%%%%%%%%%%%%%%%%%%%%%%%%%%%%%%%%%%%%%%%%%%%%%%%%%%%%%%%%%%

\begin{zztask}[Удаление комментариев]
В рамках общего условия задачи написать функцию, которая в заданном буфере
удаляет все между соответствующими парами \texttt{/*} и \texttt{*/}.
Вложенные комментарии поддерживать не надо, то есть от 
<<\texttt{a/*b/*c*/d*/e}>> останется <<\texttt{ad*/e}>>. 
Необходимо обойтись без дополнительного буфера.

Прототип: \mintinline{c}|void RemoveComments(char str[]);|
\end{zztask}

%%%%%%%%%%%%%%%%%%%%%%%%%%%%%%%%%%%%%%%%%%%%%%%%%%%%%%%%%%%%%%%%%%%%%%%%%%%%%%

\begin{zztask}[Удаление вложенных комментариев]
В рамках общего условия задачи написать функцию, которая в заданном буфере
удаляет все между соответствующими парами \texttt{/*} и \texttt{*/}.
Надо поддержать вложенные комментарии, то есть от 
<<\texttt{a/*b/*c*/d*/e}>> останется <<\texttt{ae}>>. 
Необходимо обойтись без дополнительного буфера.

Прототип: \mintinline{c}|void RemoveNestedComments(char str[]);|
\end{zztask}

%%%%%%%%%%%%%%%%%%%%%%%%%%%%%%%%%%%%%%%%%%%%%%%%%%%%%%%%%%%%%%%%%%%%%%%%%%%%%%

%%%%%%%%%%%%%%%%%%%%%%%%%%%%%%%%%%%%%%%%%%%%%%%%%%%%%%%%%%%%%%%%%%%%%%%%%%%%%%
\zztaskgroup{BIT}{Манипуляции с битами}
%%%%%%%%%%%%%%%%%%%%%%%%%%%%%%%%%%%%%%%%%%%%%%%%%%%%%%%%%%%%%%%%%%%%%%%%%%%%%%

В следующих задачах предлагается написать ту или иную функцию для
преобразования целых чисел, оптимально используя битовые операции языка Си.
При разработке алгоритма и написании функции не следует заранее делать никаких
предположений о размере целых чисел в битах, пользуясь при необходимости
операцией \texttt{sizeof()}.

Для того чтобы обобщить свой код, абстрагироваться от конкретного целого типа
(который в зависимости от вашего выбора, архитектуры системы, компилятора
может иметь разный размер в битах), \textbf{введём свой собственный тип} с
именем \texttt{Integer}. Определим его как синоним одного из имеющихся
беззнаковых типов языка Си при помощи ключевого слова \texttt{typedef}:

\medskip
\begin{minted}{c}
// typedef unsigned char  Integer;
// typedef unsigned short Integer;
   typedef unsigned int   Integer;
// typedef unsigned long  Integer;
\end{minted}
\medskip

Не только написанная функция, но и вся программа должна быть целиком
универсальной относительно конкретного определения типа \texttt{Integer}.
Сразу следует предостеречь от стандартной ошибки: проверок вида
\texttt{if (sizeof(Integer) == ...)} в коде быть не должно.

Требуется использовать эту функцию в тестовой программе, которая должна читать
с клавиатуры числа и давать ответ до тех пор, пока пользователь не введёт ноль
(перед выходом из программы \textbf{для нуля ответ тоже должен выдаться}).
Введённое значение и полученный результат необходимо выводить на экран
\textbf{в трёх системах счисления}: десятичной, двоичной и шестнадцатеричной,
полностью указывая \textbf{даже ведущие нулевые цифры} в последних двух
системах, дополняя число до максимально возможного количества цифр.

Прототип функции: \texttt{Integer Process(Integer n);}


%%%%%%%%%%%%%%%%%%%%%%%%%%%%%%%%%%%%%%%%%%%%%%%%%%%%%%%%%%%%%%%%%%%%%%%%%%%%%%
\bigskip
%%%%%%%%%%%%%%%%%%%%%%%%%%%%%%%%%%%%%%%%%%%%%%%%%%%%%%%%%%%%%%%%%%%%%%%%%%%%%%


\begin{zztask}[Реверс битов]
В рамках общего условия задачи написать функцию, которая по заданному
числу типа \texttt{Integer} возвращает другое число, в котором переставлены местами
самый младший бит с самым старшим, второй сверху со вторым снизу и т.д.
Пример работы для 8-битного типа \texttt{char}:
\begin{zzoutput}
  Задание \thezztask: Реверс битов (8 бит)
  Введите число: \zzuser{163}
  Вы ввели  :  163 = 0xA3 = 10100011
  Результат :  197 = 0xC5 = 11000101
  Введите число: \zzuser{ }
\end{zzoutput}
\end{zztask}

%%%%%%%%%%%%%%%%%%%%%%%%%%%%%%%%%%%%%%%%%%%%%%%%%%%%%%%%%%%%%%%%%%%%%%%%%%%%%%

\begin{zztask}[Реверс пар]
В рамках общего условия задачи написать функцию, которая по заданному числу
типа \texttt{Integer} возвращает другое число, в котором переставлены
местами пары битов: самая младшая пара с самой старшей, вторая
сверху со второй снизу и т.д.
Пример работы для 8-битного типа \texttt{char}:
\begin{zzoutput}
  Задание \thezztask: Реверс пар (8 бит)
  Введите число: \zzuser{163}
  Вы ввели  :  163 = 0xA3 = 10100011
  Результат :  202 = 0xСA = 11001010
  Введите число: \zzuser{ }
\end{zzoutput}
\end{zztask}

%%%%%%%%%%%%%%%%%%%%%%%%%%%%%%%%%%%%%%%%%%%%%%%%%%%%%%%%%%%%%%%%%%%%%%%%%%%%%%

\begin{zztask}[Реверс четверок]
В рамках общего условия задачи написать функцию, которая по заданному числу
типа \texttt{Integer} возвращает другое число, в котором переставлены
местами четверки битов: самая младшая четверка с самой старшей, вторая
сверху со второй снизу и т.д.
Пример работы для 8-битного типа \texttt{char}:
\begin{zzoutput}
  Задание \thezztask: Реверс четверок (8 бит)
  Введите число: \zzuser{163}
  Вы ввели  :  163 = 0xA3 = 10100011
  Результат :   58 = 0x3A = 00111010
  Введите число: \zzuser{ }
\end{zzoutput}
\end{zztask}

%%%%%%%%%%%%%%%%%%%%%%%%%%%%%%%%%%%%%%%%%%%%%%%%%%%%%%%%%%%%%%%%%%%%%%%%%%%%%%

\begin{zztask}[Реверс байтов]
В рамках общего условия задачи написать функцию, которая по заданному числу
типа \texttt{Integer} возвращает другое число, в котором переставлены
местами байты: самый младший байт с самым старшим, второй сверху со вторым
снизу и т.д.
Пример работы для 16-битного типа \texttt{short}:
\begin{zzoutput}
  Задание \thezztask: Реверс байтов (16 бит)
  Введите число: \zzuser{41906}
  Вы ввели  : 41906 = 0xA3B2 = 10100011 10110010
  Результат : 45731 = 0xB2A3 = 10110010 10100011
  Введите число: \zzuser{ }
\end{zzoutput}
\end{zztask}

%%%%%%%%%%%%%%%%%%%%%%%%%%%%%%%%%%%%%%%%%%%%%%%%%%%%%%%%%%%%%%%%%%%%%%%%%%%%%%

\begin{zztask}[Обмен битов]
В рамках общего условия задачи написать функцию, которая по заданному числу
типа \texttt{Integer} возвращает другое число, в котором переставлены
местами соседние биты: первый со вторым, третий с четвертым и т.д.
Пример работы для 8-битного типа \texttt{char}:
\begin{zzoutput}
  Задание \thezztask: Обмен битов (8 бит)
  Введите число: \zzuser{163}
  Вы ввели  :  163 = 0xA3 = 10100011
  Результат :   83 = 0x53 = 01010011
  Введите число: \zzuser{ }
\end{zzoutput}
\end{zztask}

%%%%%%%%%%%%%%%%%%%%%%%%%%%%%%%%%%%%%%%%%%%%%%%%%%%%%%%%%%%%%%%%%%%%%%%%%%%%%%

\begin{zztask}[Реверс битов в четверках]
В рамках общего условия задачи написать функцию, которая по заданному числу
типа \texttt{Integer} возвращает другое число, в котором переставлены
местами биты в четверках: самый младший бит четверки с самым старшим битом
той же четверки, второй сверху со вторым снизу и т.д.
Пример работы для 8-битного типа \texttt{char}:
\begin{zzoutput}
  Задание \thezztask: Реверс битов в четверках (8 бит)
  Введите число: \zzuser{163}
  Вы ввели  :  163 = 0xA3 = 10100011
  Результат :   92 = 0x5С = 01011100
  Введите число: \zzuser{ }
\end{zzoutput}
\end{zztask}

%%%%%%%%%%%%%%%%%%%%%%%%%%%%%%%%%%%%%%%%%%%%%%%%%%%%%%%%%%%%%%%%%%%%%%%%%%%%%%

\begin{zztask}[Реверс битов в байтах]
В рамках общего условия задачи написать функцию, которая по заданному числу
типа \texttt{Integer} возвращает другое число, в котором переставлены
местами биты в байтах: самый младший бит байта с самым старшим битом того
же байта, второй сверху со вторым снизу и т.д.
Пример работы для 16-битного типа \texttt{short}:
\begin{zzoutput}
  Задание \thezztask: Реверс битов в байтах (16 бит)
  Введите число: \zzuser{23456}
  Вы ввели  : 23456 = 0x5BA0 = 01011011 10100000
  Результат : 55813 = 0xDA05 = 11011010 00000101
  Введите число: \zzuser{ }
\end{zzoutput}
\end{zztask}

%%%%%%%%%%%%%%%%%%%%%%%%%%%%%%%%%%%%%%%%%%%%%%%%%%%%%%%%%%%%%%%%%%%%%%%%%%%%%%

\begin{zztask}[Циклический сдвиг влево]
В рамках общего условия задачи написать функцию, которая по заданному числу
типа \texttt{Integer} и целому числу возвращает результат циклического
сдвига первого числа влево на количество бит, определяемое вторым числом.
Пример работы для 16-битного типа \texttt{short}:
\begin{zzoutput}
  Задание \thezztask: Циклический сдвиг влево (16 бит)
  Введите числа: \zzuser{23456 3}
  Вы ввели  : 23456 = 0x5BA0 = 01011011 10100000
  Результат : 56578 = 0xDD02 = 11011101 00000010
  Введите числа: \zzuser{ }
\end{zzoutput}
\end{zztask}

%%%%%%%%%%%%%%%%%%%%%%%%%%%%%%%%%%%%%%%%%%%%%%%%%%%%%%%%%%%%%%%%%%%%%%%%%%%%%%

\begin{zztask}[Циклический сдвиг вправо]
В рамках общего условия задачи написать функцию, которая по заданному числу
типа \texttt{Integer} и целому числу возвращает результат циклического
сдвига первого числа вправо на количество бит, определяемое вторым числом.
Пример работы для 16-битного типа \texttt{short}:
\begin{zzoutput}
  Задание \thezztask: Циклический сдвиг вправо (16 бит)
  Введите числа: \zzuser{23456 3}
  Вы ввели  : 23456 = 0x5BA0 = 01011011 10100000
  Результат :  2932 = 0x0B74 = 00001011 01110100
  Введите числа: \zzuser{ }
\end{zzoutput}
\end{zztask}

%%%%%%%%%%%%%%%%%%%%%%%%%%%%%%%%%%%%%%%%%%%%%%%%%%%%%%%%%%%%%%%%%%%%%%%%%%%%%%

\begin{zztask}[Циклический сдвиг влево внутри байтов]
В рамках общего условия задачи написать функцию, которая по заданному числу
типа \texttt{Integer} и целому числу возвращает результат, в котором каждый байт
циклически сдвинут влево на количество бит, определяемое вторым числом.
Пример работы для 16-битного типа \texttt{short}:
\begin{zzoutput}
  Задание \thezztask: Циклический сдвиг влево внутри байтов (16 бит)
  Введите числа: \zzuser{23456 3}
  Вы ввели  : 23456 = 0x5BA0 = 01011011 10100000
  Результат : 55813 = 0xDA05 = 11011010 00000101
  Введите числа: \zzuser{ }
\end{zzoutput}
\end{zztask}

%%%%%%%%%%%%%%%%%%%%%%%%%%%%%%%%%%%%%%%%%%%%%%%%%%%%%%%%%%%%%%%%%%%%%%%%%%%%%%

\begin{zztask}[Циклический сдвиг вправо внутри байтов]
В рамках общего условия задачи написать функцию, которая по заданному числу
типа \texttt{Integer} и целому числу возвращает результат, в котором каждый байт
циклически сдвинут вправо на количество бит, определяемое вторым числом.
Пример работы для 16-битного типа \texttt{short}:
\begin{zzoutput}
  Задание \thezztask: Циклический сдвиг вправо внутри байтов (16 бит)
  Введите числа: \zzuser{23456 3}
  Вы ввели  : 23456 = 0x5BA0 = 01011011 10100000
  Результат : 27412 = 0x6B14 = 01101011 00010100
  Введите числа: \zzuser{ }
\end{zzoutput}
\end{zztask}

%%%%%%%%%%%%%%%%%%%%%%%%%%%%%%%%%%%%%%%%%%%%%%%%%%%%%%%%%%%%%%%%%%%%%%%%%%%%%%


%%%%%%%%%%%%%%%%%%%%%%%%%%%%%%%%%%%%%%%%%%%%%%%%%%%%%%%%%%%%%%%%%%%%%%%%%%%%%%
\zztaskgroup{DST}{Динамические строки}
%%%%%%%%%%%%%%%%%%%%%%%%%%%%%%%%%%%%%%%%%%%%%%%%%%%%%%%%%%%%%%%%%%%%%%%%%%%%%%

В задаче предлагается написать указанную в варианте функцию для обработки строк,
отвечающую заданному прототипу (при этом в самой функции исходные строки-аргументы
запрещается менять). Требуется использовать эту функцию в тестовой программе,
которая должна в цикле читать строки с клавиатуры и выдавать ответ на экран до
тех пор, пока пользователь не введет пустую строку (строку нулевой длины). 
Заданную функцию можно и \textbf{нужно} разбивать на более мелкие и понятные
подфункции, обеспечивая читаемость сложного алгоритма.

В этом задании нельзя предполагать, что все вводимые строки имеют длину меньше
заранее оговоренной константы (пользователь всегда может ввести строку длиннее,
чем вы предполагаете). Для чтения строки и для результата надо использовать
динамический буфер и функции динамического распределения памяти (написать отдельную
функцию чтения строки заранее неизвестной длины \texttt{char* ReadLine(void)}).
Для чтения нельзя использовать опасную функцию \texttt{gets(s)},
как альтернатива есть \texttt{fgets()} или (более медленное) посимвольное чтение
через \texttt{getchar()}.

В некоторых задачах <<строки>> языка Си
могут содержать один или несколько символов \verb|'\n'|, то есть несколько
логческих строк текста (абзац). Такие абзацы необходимо читать, склеивая
прочитанные логические строки, пока пользователь не обозначит пустую логическую
строку (нулевой длины), написав функцию \texttt{char* ReadParagraph(void)}).

Буквами договоримся считать прописные и строчные латинские буквы 
A--Z и a--z, словами --- последовательность букв и цифр 0--9. Все остальные
символы в строке будем считать разделителями слов. Подстрокой будем называть
произвольную часть строки.

В учебных целях при решении задач данного раздела функций стандартной
библиотеки, упрощающих работу со строками и символами (напр. из 
\texttt{<string.h>}), следует избегать, реализуя необходимую
функциональность самостоятельно.

Примеры диалога программы и пользователя:

\begin{zzoutput}
  Задание \thezztaskgroup-1: Выбор слов
  Введите строку: \zzuser{Hello, world!}
  Результат     : Hello, world
  Введите строку: \zzuser{int main(int argc, char* argv[]);}
  Результат     : int, main, int, argc, char, argv
  Введите строку: \zzuser{ }
\end{zzoutput}


%%%%%%%%%%%%%%%%%%%%%%%%%%%%%%%%%%%%%%%%%%%%%%%%%%%%%%%%%%%%%%%%%%%%%%%%%%%%%%
\bigskip
%%%%%%%%%%%%%%%%%%%%%%%%%%%%%%%%%%%%%%%%%%%%%%%%%%%%%%%%%%%%%%%%%%%%%%%%%%%%%%


\begin{zztask}[Выбор слов]
В рамках общего условия задачи написать функцию, которая по заданной строке
создает в динамической памяти другую, содержащую слова исходной,
разделенные запятой и пробелом.

Прототип: \mintinline{c}|char* ExtractWords(char const* str);|
\end{zztask}

\begin{zztask}[Выбор уникальных слов]
В рамках общего условия задачи написать функцию, которая по заданной строке
создает в динамической памяти другую, содержащую слова исходной без
повторений, разделенные запятой и пробелом.

Прототип: \mintinline{c}|char* UniqueWords(char const* str);|
\end{zztask}

\begin{zztask}[Выбор букв]
В рамках общего условия задачи написать функцию, которая по заданной строке
создает в динамической памяти другую, содержащую буквы исходной (в том же
порядке), разделенные запятой и пробелом.

Прототип: \mintinline{c}|char* ExtractLetters(char const* str);|
\end{zztask}

\begin{zztask}[Выбор уникальных букв]
В рамках общего условия задачи написать функцию, которая по заданной строке
создает в динамической памяти другую, содержащую буквы исходной без
повторений (в том же порядке), разделенные запятой и пробелом.

Прототип: \mintinline{c}|char* UniqueLetters(char const* str);|
\end{zztask}

\begin{zztask}[Поиск зеркальных подстрок]
В рамках общего условия задачи написать функцию, которая по заданной строке
создает в динамической памяти другую, содержащую список без повторений всех
подстрок длиной больше одного символа, зеркальное отражение которых также
содержится в строке. Все разделители при этом следует игнорировать.

Прототип: \mintinline{c}|char* FindMirror(char const* str);|
\end{zztask}

\begin{zztask}[Упрощённый поиск по маске]
В рамках общего условия задачи написать функцию, которая по заданной строке
и строковой же <<маске>> создает в динамической памяти другую, содержащую список
всех слов исходной строки (через запятую), удовлетворяющих маске. Маска может
содержать буквы, цифры и ровно один знак \verb|'*'|, обозначающий совпадение с любой
последовательностью букв, в том числе и пустой (\verb|"c*p"| даёт совпадение с
\textit{cp, cap, clip, creep\dots}).

Прототип: \mintinline{c}|char* FindMaskWords(char const* str, char const* mask);|
\end{zztask}

\begin{zztask}[Поиск по маске]
В рамках общего условия задачи написать функцию, которая по заданной строке
и строковой же <<маске>> создает в динамической памяти другую, содержащую список
всех слов исходной строки (через запятую), удовлетворяющих маске. Маска может
содержать буквы, цифры и знаки \verb|'*'|, обозначающие совпадение с любой
последовательностью букв, в том числе и пустой (\verb|"c*p*"| даёт
\textit{cp, cap, clip, couple, champion\dots}).

Прототип: \mintinline{c}|char* FindMaskWords(char const* str, char const* mask);|
\end{zztask}

\begin{zztask}[Поиск цепочек]
В рамках общего условия задачи написать функцию, которая по заданной строке
создает в динамической памяти другую, содержащую список новых слов (через
запятую), образованных по следующему правилу. Если одно слово исходной строки
заканчивается, а какое-то другое начинается с одинакового сочетания букв
(длины $> 1$), то эта пара образует новое слово результирующей строки
($\textit{table} + \textit{length} \rightarrow \textit{tablength}$).

Прототип: \mintinline{c}|char* FindChains(char const* str);|
\end{zztask}

\begin{zztask}[Школоло строки] \textbf{ПЛОХО ОПРЕДЕЛЁННАЯ ЗАДАЧА}
В рамках общего условия задачи написать функцию, которая по заданной строке
создает в динамической памяти другую, содержащую <<транслитерированную>>
строчку. В этой строке оригинальные русские буквы должны быть заменены на
школоло-последовательности ASCII символов, приблизительно похожих.

Пример:
\verb|AJIEIIIA| (<<Алёша>>), \verb|HAPyIIIuTEJIb 3AKOHA|
(<<нарушитель закона>>), \verb!>|<uBOTHOE! (<<животное>>).

Прототип: \mintinline{c}|char* ConvertShkololo(char const* str);|
\end{zztask}

\begin{zztask}[Транслитерация строк]
В рамках общего условия задачи написать функцию, которая по заданной строке
создает в динамической памяти другую, содержащую транслитерированную
строчку в соответствие со стандартами МИД РФ (\url{www.mid.ru}).
В этой строке оригинальные русские буквы должны быть заменены на
последовательности английских букв.

Пример:
\texttt{Alyosha} (<<Алёша>>), \texttt{narushitel{'} zakona}
(<<нарушитель закона>>), \texttt{zhivotnoye} (<<животное>>).

Прототип: \mintinline{c}|char* ConvertRussian(char const* str);|
\end{zztask}

\begin{zztask}[Перенос текста]
В рамках общего условия задачи написать функцию, которая по заданной строковой
переменной (возможно, содержащей несколько строк текста, разделенных символом
перевода строки \verb|'\n'|) создает в динамической памяти другую, содержащую
исходный текст, отформатированный в один абзац так, чтобы длина каждой
строки не превышала заданной ширины (вводится в начале тестовой программы).
Несколько подряд идущих пробелов следует заменять одним пробелом.
Короткие строки надо дополнять словами со следующей строки, а слишком
длинные слова переносить целиком на следующую.

Прототип: \mintinline{c}|char* FormatText(char const* str, int width);|
\end{zztask}

\begin{zztask}[Выравнивание текста]
В рамках общего условия задачи написать функцию, которая по заданной строковой
переменной (возможно, содержащей несколько строк текста, разделенных символом
перевода строки \verb|'\n'|) создает в динамической памяти другую, содержащую
исходный текст, отформатированный в один абзац так, чтобы длина каждой
строки была равна заданной ширине (вводится в начале тестовой программы).
Делать это надо за счет изменения числа подряд идущих пробелов. Слова
разбивать нельзя. Короткие строки надо дополнять словами со следующей
строки, а слишком длинные слова переносить целиком на следующую.

Прототип: \mintinline{c}|char* JustifyText(char const* str, int width);|
\end{zztask}


%%%%%%%%%%%%%%%%%%%%%%%%%%%%%%%%%%%%%%%%%%%%%%%%%%%%%%%%%%%%%%%%%%%%%%%%%%%%%%
\chapter{В печать}
%%%%%%%%%%%%%%%%%%%%%%%%%%%%%%%%%%%%%%%%%%%%%%%%%%%%%%%%%%%%%%%%%%%%%%%%%%%%%%

Следующие задачи хочется по-быстрому привести в божеский вид и включить в
книгу. Они либо используются сейчас в обучении, либо их стоит туда (в)вернуть.

%%%%%%%%%%%%%%%%%%%%%%%%%%%%%%%%%%%%%%%%%%%%%%%%%%%%%%%%%%%%%%%%%%%%%%%%%%%%%%
\zztaskgroup{INT}{Указатели на функцию в численных методах}
%%%%%%%%%%%%%%%%%%%%%%%%%%%%%%%%%%%%%%%%%%%%%%%%%%%%%%%%%%%%%%%%%%%%%%%%%%%%%%
\begin{itemize}
    \item Здесь можно не мудрствовать и позаимствовать текст из книжки красно-белой
\end{itemize}
В следующих задачах предлагается реализовать алгоритм численного 
интегрирования в виде отдельной функции, принимающей подынтегральное выражение
как параметр (в виде указателя на функцию определённого типа). Подынтегральная 
функция вычисляется в равномерно распределенных точках $x_i$, и интеграл от 
функции на отрезке от $a$ до $b$ представляется как сумма нескольких интегралов 
на отрезках $[x_i,\, x_{i+k}]$, которые уже считаются по приближённой формуле.

\[ x_i = a + i h,\qquad h = (b-a)/n, \]

\[ 
\int\limits_a^b f(x)\,dx = 
\sum_{{i=0} \above 0pt {i\div k}}^{n-k} \int\limits_{x_i}^{x_{i+k}} f(x)\,dx \approx
\sum_{{i=0} \above 0pt {i\div k}}^{n-k} I(h, f(x_i),\dots, f(x_{i+k}))
\]

Указатель на исследуемую математическую функцию, границы отрезка и количество
подразбиений (или шаг разбиения) должны приниматься как параметры. Необходимо 
учитывать, что некоторые методы требуют, чтобы количество точек со значениями 
функции было кратно некоторому числу $k$.\zztodo{Есть ли смысл в таких методах передавать, например, n, а работать уже с $4n$ или в этом случае подмена понятий произойдет?}

В целях достижения максимальной производительности \textbf{запрещается} вычислять
значение подынтегрального выражения несколько раз в одной и той же точке (потому 
что вызов $f(x)$ может быть дорогим в плане вычислений, например, если функция 
задана рядом, другим интегралом и т.п.).\zztodo{Хорошая лаба, я ее помню в матлабе дал, позаимствовав у тебя. Может ее как раз давать в классе? Она проще рядов. Единственный нюанс тут - указатель на функцию, но его можно дать как должное. + тут еще есть сложность с валидацией, может им либу дать какую-нибудь, заодно научатся их подключать?}

%%%%%%%%%%%%%%%%%%%%%%%%%%%%%%%%%%%%%%%%%%%%%%%%%%%%%%%%%%%%%%%%%%%%%%%%%%%%%%

\begin{zztask}
В рамках общего условия задачи написать функцию находящую приближенное значение 
интеграла от $f(x)$ на отрезке $[a, b]$ (т.е. площади под графиком функции)
с использованием метода левых прямоугольников. 
\[
\int\limits_{x_i}^{x_{i+1}} f(x)\,dx \approx h f(x_i)
\]
\end{zztask}

%%%%%%%%%%%%%%%%%%%%%%%%%%%%%%%%%%%%%%%%%%%%%%%%%%%%%%%%%%%%%%%%%%%%%%%%%%%%%%

\begin{zztask}
В рамках общего условия задачи написать функцию находящую приближенное значение 
интеграла от $f(x)$ на отрезке $[a, b]$ (т.е. площади под графиком функции)
с использованием метода трапеций. 
\[
\int\limits_{x_i}^{x_{i+1}} f(x)\,dx \approx \frac{h}{2} \left(f(x_i)+f(x_{i+1})\right)
\]
\end{zztask}

%%%%%%%%%%%%%%%%%%%%%%%%%%%%%%%%%%%%%%%%%%%%%%%%%%%%%%%%%%%%%%%%%%%%%%%%%%%%%%

\begin{zztask}
В рамках общего условия задачи написать функцию находящую приближенное значение 
интеграла от $f(x)$ на отрезке $[a, b]$ (т.е. площади под графиком функции)
с использованием метода Симпсона (парабол).
\[
\int\limits_{x_i}^{x_{i+2}} f(x)\,dx \approx 
\frac{h}{3} \left(f(x_i) + 4 f(x_{i+1}) + f(x_{i+2})\right)
\]
\end{zztask}

%%%%%%%%%%%%%%%%%%%%%%%%%%%%%%%%%%%%%%%%%%%%%%%%%%%%%%%%%%%%%%%%%%%%%%%%%%%%%%

\begin{zztask}
В рамках общего условия задачи написать функцию находящую приближенное значение 
интеграла от $f(x)$ на отрезке $[a, b]$ (т.е. площади под графиком функции)
с использованием метода $3/8$ Симпсона.
\[
\int\limits_{x_i}^{x_{i+3}} f(x)\,dx \approx 
\frac{3h}{8} \left(f(x_i) + 3 f(x_{i+1}) + 3 f(x_{i+2}) + f(x_{i+3})\right)
\]
\end{zztask}

%%%%%%%%%%%%%%%%%%%%%%%%%%%%%%%%%%%%%%%%%%%%%%%%%%%%%%%%%%%%%%%%%%%%%%%%%%%%%%

\begin{zztask}
В рамках общего условия задачи написать функцию находящую приближенное значение 
интеграла от $f(x)$ на отрезке $[a, b]$ (т.е. площади под графиком функции)
с использованием метода Буля.
\[
\int\limits_{x_i}^{x_{i+4}} f(x)\,dx \approx 
\frac{2h}{45} \left(7 f(x_i) + 32 f(x_{i+1}) + 12 f(x_{i+2}) + 32 f(x_{i+3}) + 7 f(x_{i+4})\right)
\]
\end{zztask}

%%%%%%%%%%%%%%%%%%%%%%%%%%%%%%%%%%%%%%%%%%%%%%%%%%%%%%%%%%%%%%%%%%%%%%%%%%%%%%

\begin{zztask}
В рамках общего условия задачи написать функцию находящую приближенное значение 
интеграла от $f(x)$ на отрезке $[a, b]$ (т.е. площади под графиком функции)
с использованием следующей приближенной формулы (открытого типа). 
\[
\int\limits_{x_i}^{x_{i+2}} f(x)\,dx \approx 2h f(x_{i+1})
\]
\end{zztask}

%%%%%%%%%%%%%%%%%%%%%%%%%%%%%%%%%%%%%%%%%%%%%%%%%%%%%%%%%%%%%%%%%%%%%%%%%%%%%%

\begin{zztask}
В рамках общего условия задачи написать функцию находящую приближенное значение 
интеграла от $f(x)$ на отрезке $[a, b]$ (т.е. площади под графиком функции)
с использованием следующей приближенной формулы (открытого типа). 
\[
\int\limits_{x_i}^{x_{i+3}} f(x)\,dx \approx \frac{3h}{2} \left( f(x_{i+1}) + f(x_{i+2}) \right)
\]
\end{zztask}

%%%%%%%%%%%%%%%%%%%%%%%%%%%%%%%%%%%%%%%%%%%%%%%%%%%%%%%%%%%%%%%%%%%%%%%%%%%%%%

\begin{zztask}
В рамках общего условия задачи написать функцию находящую приближенное значение 
интеграла от $f(x)$ на отрезке $[a, b]$ (т.е. площади под графиком функции)
с использованием следующей приближенной формулы (открытого типа). 
\[
\int\limits_{x_i}^{x_{i+4}} f(x)\,dx \approx \frac{4h}{3} \left( 2f(x_{i+1}) - f(x_{i+2}) + 2f(x_{i+3}) \right)
\]
\end{zztask}

%%%%%%%%%%%%%%%%%%%%%%%%%%%%%%%%%%%%%%%%%%%%%%%%%%%%%%%%%%%%%%%%%%%%%%%%%%%%%%

\begin{zztask}
В рамках общего условия задачи написать функцию находящую приближенное значение 
интеграла от $f(x)$ на отрезке $[a, b]$ (т.е. площади под графиком функции)
с использованием следующей приближенной формулы (открытого типа). 
\[
\int\limits_{x_i}^{x_{i+5}} f(x)\,dx \approx 
\frac{5h}{24} \left( 11 f(x_{i+1}) + f(x_{i+2}) + f(x_{i+3}) + 11f(x_{i+4}) \right)
\]
\end{zztask}

%%%%%%%%%%%%%%%%%%%%%%%%%%%%%%%%%%%%%%%%%%%%%%%%%%%%%%%%%%%%%%%%%%%%%%%%%%%%%%

%%%%%%%%%%%%%%%%%%%%%%%%%%%%%%%%%%%%%%%%%%%%%%%%%%%%%%%%%%%%%%%%%%%%%%%%%%%%%%
\zztaskgroup{BAR}{Штрих-код}
%%%%%%%%%%%%%%%%%%%%%%%%%%%%%%%%%%%%%%%%%%%%%%%%%%%%%%%%%%%%%%%%%%%%%%%%%%%%%%

См. вордовский файл (примеры). Текст условия надо писать.
Для общего условия: контрольные цифры они должны считать самостоятельно.

Два возможных варианта вывода: в графическом режиме и в текстовом режиме с
использованием псевдографики (половинных символов для увеличения горизонтального
разрешения). Возможно даже потребовать поддержку обоих, это вставляет мозги на
предмет аккуратного планирования --- кодирование отдельно от отображения.

Раньше в эту же кучу, под предлогом кодирования, я сваливал также азбуку Морзе и
шрифт Брайля, но там мне такое выдают\dots Еще хуже, чем для штрих-кодов исходники.
Хотя, исходно идея была неплохая.

\begin{zztask}[EAN-8]
В рамках общего условия задачи\dots
\end{zztask}

\begin{zztask}[EAN-13]
В рамках общего условия задачи\dots
\end{zztask}

\begin{zztask}[UPC-A]
В рамках общего условия задачи\dots
\end{zztask}

\begin{zztask}[UPC-E]
В рамках общего условия задачи\dots
\end{zztask}

\begin{zztask}[CODE-39]
В рамках общего условия задачи\dots
\end{zztask}

\begin{zztask}[CODE-93]
В рамках общего условия задачи\dots
\end{zztask}

\begin{zztask}[CODE-128]
В рамках общего условия задачи\dots
\end{zztask}

\begin{zztask}[CODABAR]
В рамках общего условия задачи\dots
\end{zztask}

\begin{zztask}[ITF]
В рамках общего условия задачи\dots
\end{zztask}

\begin{zztask}[QR-CODE]
В рамках общего условия задачи\dots
\end{zztask}

\begin{zztask}[Data Matrix]
В рамках общего условия задачи\dots
\end{zztask}

%%%%%%%%%%%%%%%%%%%%%%%%%%%%%%%%%%%%%%%%%%%%%%%%%%%%%%%%%%%%%%%%%%%%%%%%%%%%%%
\zztaskgroup{LIB}{Библиотеки функций}
%%%%%%%%%%%%%%%%%%%%%%%%%%%%%%%%%%%%%%%%%%%%%%%%%%%%%%%%%%%%%%%%%%%%%%%%%%%%%%

В следующих задачах требуется написать библиотеку функций с заданным
интерфейсом для работы с определённым типом данных и программу,
демонстрирующую возможности библиотеки. Таким образом, решение должно состоять
из двух проектов:
%
\begin{enumerate}
\item Библиотека \texttt{xxx.lib}, которая собирается из \texttt{xxx.c}
      и \texttt{xxx.h}.
\item Консольное приложение \texttt{xxxtest.exe}, которое собирается из
      \texttt{xxxtest.c} с использованием \texttt{xxx.h} и \texttt{xxx.lib}.
\end{enumerate}

Исходный текст библиотеки (заголовочный файл) должен содержать комментарии для
каждого экспортируемого объекта (функции, типа, константы), для автоматической
генерации документации в формате HTML/CHM с помощью утилиты Doxygen.%\zztodo{Задача ничего, но только они все разнородные и вроде у них мейн не фиксирован, может заставить их еще какие-то штуки писать типа юнит-тестов? }

Все функции библиотеки традиционно в качестве префикса имён должны иметь
название типа, чтобы отличать от аналогичных функций для другого типа,
например, для типа \texttt{xxx\_t} функции \texttt{XxxCreate()},
\texttt{XxxAdd()} и т.п.


%%%%%%%%%%%%%%%%%%%%%%%%%%%%%%%%%%%%%%%%%%%%%%%%%%%%%%%%%%%%%%%%%%%%%%%%%%%%%%
\zzsectionCOMMENTS
%%%%%%%%%%%%%%%%%%%%%%%%%%%%%%%%%%%%%%%%%%%%%%%%%%%%%%%%%%%%%%%%%%%%%%%%%%%%%%


\paragraph{Интерфейс}
Краткое описание возможной функциональности для различного вида чисел или
аналогичных объектов:
%
\begin{itemize}
%
\item[--] \texttt{XxxCreate}, \texttt{XxxDestroy}: создание по набору параметров, с выделением динамической памяти при необходимости, уничтожение, в первую очередь динамической памяти;
%
\item[--] \texttt{XxxFromYyy}, \texttt{XxxAsYyy}: создание из определенного типа, преобразование к определенному типу;
%
\item[--] \texttt{XxxParse}, \texttt{XxxFormat}: создание по строке, запись в строковый буфер;
%
\item[--] \texttt{XxxRead}, \texttt{XxxWrite}: создание из потока, запись в поток, в текстовом виде;
%
\item[--] \texttt{XxxAdd}, \texttt{XxxSub}, \texttt{XxxMul}, \texttt{XxxDiv}: сложение, вычитание, умножение, деление;
%
\item[--] \texttt{XxxNegate}, \texttt{XxxReciprocal}, \texttt{XxxInverse}: противоположное, обратное;
%
\item[--] \texttt{XxxCompare}: сравнение (функция сравнения традиционно принимает два указателя и возвращает –1, 0 или +1 когда первый аргумент меньше, равен или больше второго соответственно);
%
\end{itemize}


%%%%%%%%%%%%%%%%%%%%%%%%%%%%%%%%%%%%%%%%%%%%%%%%%%%%%%%%%%%%%%%%%%%%%%%%%%%%%%
\zzsectionPLAN
%%%%%%%%%%%%%%%%%%%%%%%%%%%%%%%%%%%%%%%%%%%%%%%%%%%%%%%%%%%%%%%%%%%%%%%%%%%%%%


Предлагаем вам следующие шаги решения.

\begin{enumerate}
\item Объявите\dots
%
\end{enumerate}



%%%%%%%%%%%%%%%%%%%%%%%%%%%%%%%%%%%%%%%%%%%%%%%%%%%%%%%%%%%%%%%%%%%%%%%%%%%%%%
\zzsectionVARIATIONS
%%%%%%%%%%%%%%%%%%%%%%%%%%%%%%%%%%%%%%%%%%%%%%%%%%%%%%%%%%%%%%%%%%%%%%%%%%%%%%


\begin{zztask}[Комплексные числа]
В рамках общего условия задачи ввести новый тип <<комплексное число>>
(\texttt{complex\_t}) на основе значений \texttt{re}, \texttt{im} типа
\texttt{double} и реализовать функциональность:
%
\begin{itemize}
\item Create;
\item Parse, Format, Read, Write, в формате $(a, b)$;
\item Add, Sub, Mul, Div, Negate, Reciprocal;
\item Abs, Arg: модуль, аргумент;
\item Conjugate: сопряжение;
\end{itemize}
\end{zztask}

%%%%%%%%%%%%%%%%%%%%%%%%%%%%%%%%%%%%%%%%%%%%%%%%%%%%%%%%%%%%%%%%%%%%%%%%%%%%%%

\begin{zztask}[Рациональные дроби]
В рамках общего условия задачи ввести новый тип <<несократимая рациональная
дробь>> (\texttt{rational\_t}) на основе значений \texttt{num}, \texttt{den}
типа \texttt{int} и реализовать функциональность (не забывая сокращать дробь):
%
\begin{itemize}
\item Create; \textit{FromDouble}${}^{(\ast)}$, AsDouble, AsInt;
\item Parse, Format, Read, Write, в формате $a\backslash b$;
\item Add, Sub, Mul, Div, Negate, Reciprocal, Abs;
\item Round, Floor, Ceil: округления;
\item Compare;
\end{itemize}
\end{zztask}

%%%%%%%%%%%%%%%%%%%%%%%%%%%%%%%%%%%%%%%%%%%%%%%%%%%%%%%%%%%%%%%%%%%%%%%%%%%%%%

\begin{zztask}[Вещественные числа с фиксированной точкой]
В рамках общего условия задачи ввести новый тип <<с фиксированной точкой>>
(\texttt{fixed\_t}) на основе типа \texttt{int} (напр., 32 бита как 16:16) и
реализовать функциональность:
%
\begin{itemize}
\item FromInt, FromDouble, AsInt, AsDouble;
\item Parse, Format, Read, Write;
\item Add, Sub, Mul, Div, Negate, Reciprocal, Abs;
\item Round, Floor, Ceil: округления;
\item Compare;
\end{itemize}
\end{zztask}

%%%%%%%%%%%%%%%%%%%%%%%%%%%%%%%%%%%%%%%%%%%%%%%%%%%%%%%%%%%%%%%%%%%%%%%%%%%%%%

\begin{zztask}[Длинные целые числа]
В рамках общего условия задачи ввести новый 18-значный тип <<очень длинный>>
(\texttt{verylong\_t}) на основе двух значений \texttt{hi}, \texttt{lo} типа
\texttt{long} (в каждом хранить по 9 десятичных цифр, в старшем числе знак) и
реализовать функциональность:
%
\begin{itemize}
\item FromLong, AsLong;
\item Parse, Format, Read, Write;
\item Add, Sub, Mul, Div, Negate, Abs;
\item Compare;
\end{itemize}
\end{zztask}

%%%%%%%%%%%%%%%%%%%%%%%%%%%%%%%%%%%%%%%%%%%%%%%%%%%%%%%%%%%%%%%%%%%%%%%%%%%%%%

\begin{zztask}[Приближённые вещественные числа]
В рамках общего условия задачи ввести новый тип <<приближённое число>>
(\texttt{approx\_t}) на основе двух значений \texttt{lo}, \texttt{hi} типа
\texttt{double} и реализовать функциональность:
%
\begin{itemize}
\item FromDouble: по центру и радиусу; FromRange, AsDouble, GetRadius;
\item Parse, Format, Read, Write, в формате $[a, b]$;
\item Add, Sub, Mul, Div, Negate, Abs;
\item Compare;
\end{itemize}
\end{zztask}

%%%%%%%%%%%%%%%%%%%%%%%%%%%%%%%%%%%%%%%%%%%%%%%%%%%%%%%%%%%%%%%%%%%%%%%%%%%%%%

\begin{zztask}[Нечёткая логика]
В рамках общего условия задачи ввести новый тип <<нечёткий bool>>
(\texttt{fool\_t}) на основе значения типа \texttt{double} и реализовать
функциональность:
%
\begin{itemize}
\item FromInt, FromDouble, AsInt, AsDouble;
\item Parse, Format, Read, Write, в формате $\mathit{fool}(a)$;
\item And, Or, Not: конъюнкция, дизъюнкция, отрицание;
\item Compare;
\end{itemize}
\end{zztask}

%%%%%%%%%%%%%%%%%%%%%%%%%%%%%%%%%%%%%%%%%%%%%%%%%%%%%%%%%%%%%%%%%%%%%%%%%%%%%%

\begin{zztask}[Матрицы $2\times 2$]
В рамках общего условия задачи ввести новый тип <<матрица $2\times 2$>>
(\texttt{matrix22\_t}) на основе значений \texttt{a}, \texttt{b}, \texttt{c},
\texttt{d} типа \texttt{double} и реализовать функциональность:
%
\begin{itemize}
\item SetZero, SetIdentity: заполнение как нулевой и единичной;
\item Parse, Format, Read, Write, в формате $\{\{a,b,c,d\}\}$;
\item Add, Sub, Mul, MulDouble;
\item Determinant, Trace, Inverse, Transpose;
\item ELEMENT: макрокоманда доступа к элементу $A_{ij}$;
\end{itemize}
\end{zztask}

%%%%%%%%%%%%%%%%%%%%%%%%%%%%%%%%%%%%%%%%%%%%%%%%%%%%%%%%%%%%%%%%%%%%%%%%%%%%%%

\begin{zztask}[Матрицы произвольного размера]
В рамках общего условия задачи ввести новый тип <<матрица>>
(\texttt{matrix\_t}) на основе значений типа \texttt{double} и реализовать
функциональность:
%
\begin{itemize}
\item Create, Destroy: в динамической памяти;
\item SetZero, SetIdentity: заполнение как нулевой и единичной;
\item Parse, Format, Read, Write, в формате $\{\{a,b\},\{c,d\}\}$;
\item Add, Sub, Mul, MulDouble;
\item \textit{Determinant}${}^{(\ast)}$, Trace, \textit{Inverse}${}^{(\ast)}$, Transpose;
\item ELEMENT: макрокоманда доступа к элементу $A_{ij}$;
\end{itemize}
%
Примечание: в качестве типа рекомендуется использовать структуру с тремя
полями, хранящими число строк, столбцов и указатель на линейную память под
элементы.
\end{zztask}

%%%%%%%%%%%%%%%%%%%%%%%%%%%%%%%%%%%%%%%%%%%%%%%%%%%%%%%%%%%%%%%%%%%%%%%%%%%%%%

\begin{zztask}[Многочлены]
В рамках общего условия задачи ввести новый тип <<многочлен>>
(\texttt{poly\_t}) на основе значений типа \texttt{int} и реализовать
функциональность:
%
\begin{itemize}
\item Create, Destroy: по массиву, в динамической памяти;
\item From: по N коэффициентам, с переменным числом аргументов;
\item Parse, Format, Read, Write, в формате $2x^3 - 5x + 7$;
\item Add, Sub, Mul, MulInt; Evaluate: вычислить в точке;
\item ELEMENT: макрокоманда доступа к элементу $A_{i}$;
\end{itemize}
%
Примечание: в качестве типа можно использовать структуру с двумя полями,
хранящими степень многочлена и указатель на выделенную память под
коэффициенты.
\end{zztask}

%%%%%%%%%%%%%%%%%%%%%%%%%%%%%%%%%%%%%%%%%%%%%%%%%%%%%%%%%%%%%%%%%%%%%%%%%%%%%%

%%%%%%%%%%%%%%%%%%%%%%%%%%%%%%%%%%%%%%%%%%%%%%%%%%%%%%%%%%%%%%%%%%%%%%%%%%%%%%
\zztaskgroup{SOR}{Сортировка массива}
%%%%%%%%%%%%%%%%%%%%%%%%%%%%%%%%%%%%%%%%%%%%%%%%%%%%%%%%%%%%%%%%%%%%%%%%%%%%%%

Задача сортировки (упорядочивания) массива решается далеко не единственным способом. Известно много
базовых схем, и в каждой из них возможны вариации, которые отличаются не только вычислительной
сложностью, но и просто временем работы~--- константами, зависящими от количества сравнений и
копирований элементов.

В данной работе требуется запрограммировать некоторый набор алгоритмов сортировки (вариаций) и
произвести их наглядное сравнение. Для этого необходимо измерить среднее время работы каждого
алгоритма на случайных массивах определённой длины. Из этих времён составить таблицу, показывающую
зависимость среднего времени работы каждого алгоритма от размера массива, а по таблице построить
графики в Excel. Первый столбец таблицы должен содержать размер массива, остальные~--- время работы,
по столбцу на алгоритм, столбцы должны быть озаглавлены. Таблица должна быть сохранена в обычный
текстовый файл формата CSV (Comma Separated Values), который в то же время легко открывается в
Excel.

Для соблюдения принципа модульности все сортировки должны быть оформлены в виде отдельной
библиотеки, а замер времени с созданием отчёта~--- в виде консольного приложения-профайлера. Чтобы
обеспечить лучшую взаимозаменяемость библиотек, в этой задаче будем использовать динамические
библиотеки (DLL на Windows, SO на Linux) с динамическим связыванием (действующие как <<плагины>>).
При таком подходе ваше приложение-профайлер будет работать с любой другой библиотекой сортировок и
наоборот.

Тестовое приложение не только занимается сбором статистики, но и проверяет правильность работы
вызываемых алгоритмов (в отладочной конфигурации). Для этого необходимо после каждой сортировки
сравнивать результат с эталонным, полученным при помощи библиотечной функции \texttt{qsort()}.

Ваша задача не просто реализовать заданные алгоритмы, но постараться сделать это наиболее быстрым
образом, оптимизируя вычисления. Возможно, стоит поискать и другие более быстрые модификации
алгоритмов, не указанные в задании.


%%%%%%%%%%%%%%%%%%%%%%%%%%%%%%%%%%%%%%%%%%%%%%%%%%%%%%%%%%%%%%%%%%%%%%%%%%%%%%
\zzsectionCOMMENTS
%%%%%%%%%%%%%%%%%%%%%%%%%%%%%%%%%%%%%%%%%%%%%%%%%%%%%%%%%%%%%%%%%%%%%%%%%%%%%%


\paragraph{Общность}
В библиотеке все алгоритмы должны быть реализованы в виде \texttt{static}-функций с одинаковым,
фиксированным для всех прототипом, соответствующим приведённому указателю на функцию:
%
\begin{minted}{c}
void (*sort_func_t)(int array[], int size);
\end{minted}

Динамическая библиотека должна экспортировать только одну функцию:
%
\begin{minted}{c}
__declspec(dllexport) sort_info_t* GetSortList(int* count);
\end{minted}
%
которая возвращает указатель на первый элемент массива и количество элементов в нём.
Каждый элемент этого массива соответствует одному реализованному алгоритму сортировки:

\begin{minted}{c}
// Описание одной вариации сортировки
typedef struct
{
  sort_func_t sort;             // Функция сортировки
  sort_family_t family;         // Базовое семейство
  sort_complexity_t complexity; // Вычислительная сложность
  char const* name;             // Имя, описывающее вариацию
} sort_info_t;

// Семейства сортировок
typedef enum
{
  // (Не определено)
  SORT_NA = -1,
  // Квадратичные
  SORT_SELECTION, SORT_INSERTION, SORT_BUBBLE,
  // Субквадратичные
  SORT_SHELL,
  // Квазилинейные
  SORT_COMB, SORT_MERGE, SORT_QUICK, SORT_HEAP,
  // Линейные
  SORT_COUNT, SORT_RADIX, SORT_BUCKET,
} sort_family_t;

// Вычислительная сложность
typedef enum
{
  SORT_QUADRATIC,
  SORT_SUBQUADRATIC,
  SORT_QUASILINEAR,
  SORT_LINEAR
} sort_complexity_t;
\end{minted}

В тестовом приложении вы будете использовать только поля \texttt{sort} и \texttt{name}, остальное
будет учитываться преподавателем при проверке.

\paragraph{Локализация}
Формат CSV в оригинальном своём исполнении, как это следует из названия, содержит ячейки таблицы,
разделённые запятыми. В русских настройках Excel запятая используется для разделения целой и дробной
части вещественных чисел, поэтому элементы строки таблицы разделяются точкой с запятой. Грамотно
воспользуйтесь \texttt{locale.h}, чтобы сохранить в файл нужные разделители.

\paragraph{Усреднение}
Обратите внимание, что нас интересует среднее время, т.е. необходимо усреднять по нескольким
случайным массивам одного размера (напр., \texttt{REPEAT\_COUNT} равный 5, 10, или 100). В то же
время для сравнения разных сортировок необходимо использовать одинаковые массивы. Вспомните, что
функция \texttt{rand()} генерирует последовательность псевдослучайных чисел, которая может быть
воспроизведена сколько угодно раз. Для сброса генератора в начало последовательности используется
функция \texttt{srand(r0)}.

\paragraph{Детализация}
Чтобы понять зависимость времени работы алгоритма от размера массива и получить плавный график,
нельзя ограничиваться одним, двумя, пятью значениями $N$. Размер массива следует менять от
\texttt{MIN\_SIZE} до \texttt{MAX\_SIZE} с шагом \texttt{SIZE\_STEP}, значения параметров подобрать
в зависимости от интересующего вас диапазона и от скорости работы алгоритма (например, от 1 до 100
с шагом 1, от 10 до 10000 с шагом 10 и т.п.).

\paragraph{Профилирование}
Мы хотим получить как можно более точное время работы алгоритма, поэтому при измерении времени
следует использовать подходящие функции стандартной библиотеки и операционной системы. При
недостаточной точности необходимо прибегнуть к дополнительным способам уменьшения погрешности
(повторное выполнение алгоритма). Разумеется, лишние отладочные проверки тоже не должны мешать
правильным измерениям, поэтому профилирование принято проводить не в отладочной, а в финальной
конфигурации проекта, с включенными оптимизациями компилятора.


%%%%%%%%%%%%%%%%%%%%%%%%%%%%%%%%%%%%%%%%%%%%%%%%%%%%%%%%%%%%%%%%%%%%%%%%%%%%%%
\zzsectionPLAN
%%%%%%%%%%%%%%%%%%%%%%%%%%%%%%%%%%%%%%%%%%%%%%%%%%%%%%%%%%%%%%%%%%%%%%%%%%%%%%


Как полагается в разработке программного обеспечения, инструментарий определяется вариантами его
использования. Следовательно, начинать разработку следует с написания небольшого тестового
приложения.

\begin{enumerate}
\item Для начала создайте два проекта: один будет
собираться в динамическую библиотеку, другой --- в исполняемый файл. Проставьте зависимость второго
проекта от первого, чтобы изменения в библиотеке вели к её своевременной перекомпиляции во время
запуска тестового приложения.
%
\item Далее, в библиотеке добавьте заголовочный файл нужного формата и файл с исходным кодом, в
котором реализуйте функцию \texttt{GetSortList()} и добавьте одну функцию сортировки, имеющую
соответствующий прототип. Эта функция, например, может содержать вызов функции \texttt{qsort()}.
%
\item Далее, попробуйте загрузить библиотеку с помощью функции \texttt{LoadLibrary()} и после этого
воспользоваться функцией \texttt{GetProcAddress()} для доступа к адресу функции
\texttt{GetSortList()}. Убедитесь, что возвращаются корректные значения, функция вызывается и
возвращает то, что ожидается.
%
\item После этого, можно реализовать верификацию алгоритмов и измерение времени с учетом
комментариев выше. Начать рекомендуется с кода генерации случайных массивов. Затем стоит реализовать
последовательный вызов алгоритмов сортировки, полученных из \texttt{GetSortList()}, контролируя, что
в одном измерении алгоритмы запускаются на одних и тех же данных. Далее можно добавить проверку
того, что функция действительно сортирует введенный массив и завершить всё реализацией подсчета
времени и вывода таблицы.
%
\item В итоге, после того, как вы реализуете и протестируете весь сценарий использования, можно
будет очень легко добавлять новые алгоритмы, убеждаться в их корректности и производить замеры
времени.
\end{enumerate}


%%%%%%%%%%%%%%%%%%%%%%%%%%%%%%%%%%%%%%%%%%%%%%%%%%%%%%%%%%%%%%%%%%%%%%%%%%%%%%
\zzsectionVARIATIONS
%%%%%%%%%%%%%%%%%%%%%%%%%%%%%%%%%%%%%%%%%%%%%%%%%%%%%%%%%%%%%%%%%%%%%%%%%%%%%%


\begin{zztask}[Выбором и пузырьком]
В рамках общего условия задачи реализовать алгоритмы сортировки:
\begin{enumerate}
\item Выбором (Selection sort)
  \begin{itemize}[--]
  \item со сдвигом (стабильная),
  \item с обменом,
  \item[$\star$] квадратичный выбор ($\sqrt N$ блоков по $\sqrt N$ элементов).
  \end{itemize}
\item Пузырьком (Bubble sort)
  \begin{itemize}[--]
  \item просто $N$ раз по $N$,
  \item остановка каждый раз на 1 раньше,
  \item остановка если не было обменов,
  \item шейкером (Cocktail sort),
  \item расчёской / гребнем (Comb sort) с фактором $2$ и $1.3$.
  \end{itemize}
\end{enumerate}
%
Рекомендованные источники:
Вирт~\cite[\S2.2.2--2.2.3]{wirth2016algoritmy},
Кнут~\cite[\S5.2.2--5.2.3]{knuth2014iskusstvo},
Макконнелл~\cite[\S3.2]{mcconnel2009analyz}.
\end{zztask}

%%%%%%%%%%%%%%%%%%%%%%%%%%%%%%%%%%%%%%%%%%%%%%%%%%%%%%%%%%%%%%%%%%%%%%%%%%%%%%

\begin{zztask}[Вставками]
В рамках общего условия задачи реализовать алгоритмы сортировки вставками (Insertion sort):
\begin{itemize}[--]
\item простые вставки с поиском слева направо,
\item простые вставки с поиском справа налево,
\item бинарные вставки,
\item попарные вставки с поиском сразу двух элементов,
\item[$\star$] двухпутевые вставки с дополнительной памятью.
\end{itemize}
%
Рекомендованные источники:
Вирт~\cite[\S2.2.1]{wirth2016algoritmy},
Кормен~\cite[\S1.1]{kormen2016algoritmy},
Кнут~\cite[\S5.2.1]{knuth2014iskusstvo},
Макконнелл~\cite[\S3.1]{mcconnel2009analyz}.
\end{zztask}

%%%%%%%%%%%%%%%%%%%%%%%%%%%%%%%%%%%%%%%%%%%%%%%%%%%%%%%%%%%%%%%%%%%%%%%%%%%%%%

\begin{zztask}[Шелла]
В рамках общего условия задачи реализовать алгоритмы сортировки Шелла (Shellsort) с убывающими шагами вида:
\begin{itemize}[--]
\item $\Delta_k = \left\lfloor\Delta_{k-1}/2\right\rfloor;\quad\Delta_0=N$,
\item $\Delta_k = \left\lfloor5\Delta_{k-1}/11\right\rfloor;\quad\Delta_0=N$,
\item $\Delta_k \in \{2^n-1\}_{n>0}$,
\item $\Delta_k \in \{4^n+3*2^{n-1}+1\}_{n>0}\cup{1}$,
\item $\Delta_k \in \{2^n \cdot 3^m\}_{n,m>0}$,
\item эмпирическая $\dots,701,301,132,57,23,10,4,1$ начинающаяся как $\Delta_{k-1} = 2.25\Delta_{k}$
\end{itemize}
%
Рекомендованные источники:
Вирт~\cite[\S2.3.1]{wirth2016algoritmy},
Кнут~\cite[\S5.2.1]{knuth2014iskusstvo},
Макконнелл~\cite[\S3.3]{mcconnel2009analyz}.
\end{zztask}

%%%%%%%%%%%%%%%%%%%%%%%%%%%%%%%%%%%%%%%%%%%%%%%%%%%%%%%%%%%%%%%%%%%%%%%%%%%%%%

\begin{zztask}[Слиянием]
В рамках общего условия задачи реализовать алгоритмы сортировки слиянием (Merge sort):
\begin{itemize}[--]
\item сверху вниз,
\item снизу вверх,
\item естественная,
\item разбиение на три части,
\item[$\star$] без дополнительной памяти.
\end{itemize}
%
Рекомендованные источники:
Вирт~\cite[\S2.4.1--2.4.3]{wirth2016algoritmy},
Кормен~\cite[\S1.3]{kormen2016algoritmy},
Кнут~\cite[\S5.2.4]{knuth2014iskusstvo},
Макконнелл~\cite[\S3.6]{mcconnel2009analyz}.
\end{zztask}

%%%%%%%%%%%%%%%%%%%%%%%%%%%%%%%%%%%%%%%%%%%%%%%%%%%%%%%%%%%%%%%%%%%%%%%%%%%%%%

\begin{zztask}[Быстрая и кучей]
В рамках общего условия задачи реализовать алгоритмы сортировки:
\begin{enumerate}
\item Быстрая (Quicksort)
  \begin{itemize}[--]
  \item перекидывая слева направо (Ломуто),
  \item классическое разбиение с двух концов (Хоара),
  \item при маленьком размере просто останавливаться, потом целиком вставками,
  \item выбор pivot случайный,
  \item выбор pivot как медианы из трех.
  \end{itemize}
\item Пирамидальная / кучей (Heapsort)
  \begin{itemize}[--]
  \item просеивание сверху вниз,
  \item только с одним сравнением в узле (<<снизу вверх>>).
  \end{itemize}
\end{enumerate}
%
Рекомендованные источники:
Вирт~\cite[\S2.3.2--2.3.3]{wirth2016algoritmy},
Кормен~\cite[\S7.1--7.5, \S8.1--8.4]{kormen2016algoritmy},
Кнут~\cite[\S5.2.2--5.2.3]{knuth2014iskusstvo},
Макконнелл~\cite[\S3.5, \S3.7]{mcconnel2009analyz}.
\end{zztask}

%%%%%%%%%%%%%%%%%%%%%%%%%%%%%%%%%%%%%%%%%%%%%%%%%%%%%%%%%%%%%%%%%%%%%%%%%%%%%%

\begin{zztask}[Линейные]
В рамках общего условия задачи реализовать алгоритмы сортировки:
\begin{enumerate}
\item Цифровая (Radix sort)
  \begin{itemize}[--]
  \item десятичная система,
  \item двоичная система,
  \item байтовая вариация (по основанию 256).
  \end{itemize}
\item Черпаком (Bucket sort).
\end{enumerate}
%
Рекомендованные источники:
Кормен~\cite[\S9.2--9.4]{kormen2016algoritmy},
Кнут~\cite[\S5.2.5]{knuth2014iskusstvo},
Макконнелл~\cite[\S3.4]{mcconnel2009analyz}.
\end{zztask}


%%%%%%%%%%%%%%%%%%%%%%%%%%%%%%%%%%%%%%%%%%%%%%%%%%%%%%%%%%%%%%%%%%%%%%%%%%%%%%
%%%%%%%%%%%%%%%%%%%%%%%%%%%%%%%%%%%%%%%%%%%%%%%%%%%%%%%%%%%%%%%%%%%%%%%%%%%%%%
%%%%%%%%%%%%%%%%%%%%%%%%%%%%%%%%%%%%%%%%%%%%%%%%%%%%%%%%%%%%%%%%%%%%%%%%%%%%%%

\endinput

Комментарии:
%
\begin{itemize}
\item 
\item 
\item Для справедливого сравнения при фиксированном размере различные алгоритмы следует применять к массивам с одинаковым содержимым.
\item Во избежание копирования кода организовать цикл по сортировкам с использованием массива указателей на функции (или структур, их содержащих).
\end{itemize}


%%%%%%%%%%%%%%%%%%%%%%%%%%%%%%%%%%%%%%%%%%%%%%%%%%%%%%%%%%%%%%%%%%%%%%%%%%%%%%
\zztaskgroup{CUR}{Самоподобные (фрактальные) кривые}
%%%%%%%%%%%%%%%%%%%%%%%%%%%%%%%%%%%%%%%%%%%%%%%%%%%%%%%%%%%%%%%%%%%%%%%%%%%%%%

%Графические программы пишутся в среде Borland C с использованием BGI-графики.
%Режим обычный (автоопределяемый), 640х480 (VGAHI), 16 цветов. При желании можно 
%использовать режим 640х350 (VGAMED), позволяющий иметь 2 видеостраницы.

Написать программу, строящую N-ное приближение (итерацию) одной из 
нижеперечисленных рекурентно заданных фрактальных кривых. После запуска 
программа позволяет управляемым с клавиатуры курсором (который рисуется 
самостоятельно) выбрать начальную и конечную точки кривой и отображает нулевую 
итерацию. После этого стрелки вправо/влево на клавиатуре позволяют 
увеличивать/уменьшать номер итерации, и кривая перерисовывается.

Рекурентно заданые кривые должны быть реализованы в виде рекурсивной (вызывающей 
саму себя) функции.

%%%%%%%%%%%%%%%%%%%%%%%%%%%%%%%%%%%%%%%%%%%%%%%%%%%%%%%%%%%%%%%%%%%%%%%%%%%%%%

\begin{zztask}[Koch Curve]
В рамках общего условия задачи построить кривую Коха. База для построения кривой
(нулевое приближение) представляет собой отрезок. Переход от приближения
$n$ к приближению $(n+1)$ осуществляется заменой каждого отрезка фигуры на четыре
(см. рис. при $n=1$). Все отрезки (включая пустой промежуток) имеют одинаковую 
длину. Угол между отрезками составляет $60^\circ$.
\par\endinput
\begin{pspicture}(-0.5,-0.5)(1.5,1.5)
\psgrid
\rput(0.5,-0.25){$n=0$}
\rput(0,0.5){
\psline[linewidth=0.06cm, arrows=o-o]
  (0, 0)
  (1.000000, 0.000000)
}
\end{pspicture}
%
\hskip0.1in
%
\begin{pspicture}(-0.5,-0.5)(1.5,1.5)
\psgrid
\rput(0.5,-0.25){$n=1$}
\rput(0,0.5){
\psline[linewidth=0.06cm, arrows=o-o, showpoints=true]
  (0, 0)
  (0.333333, 0.000000)
  (0.500000, 0.288675)
  (0.666667, 0.000000)
  (1.000000, 0.000000)
}
\end{pspicture}
%
\hskip0.1in
%
\begin{pspicture}(-0.5,-0.5)(1.5,1.5)
\psgrid
\rput(0.5,-0.25){$n=2$}
\rput(0,0.5){
\psline
  (0, 0)
  (0.111111, 0.000000)
  (0.166667, 0.096225)
  (0.222222, -0.000000)
  (0.333333, -0.000000)
  (0.388889, 0.096225)
  (0.333333, 0.192450)
  (0.444444, 0.192450)
  (0.500000, 0.288675)
  (0.555556, 0.192450)
  (0.666667, 0.192450)
  (0.611111, 0.096225)
  (0.666667, 0.000000)
  (0.777778, 0.000000)
  (0.833333, 0.096225)
  (0.888889, 0.000000)
  (1.000000, 0.000000)
}
\end{pspicture}
%
\hskip0.1in
%
\begin{pspicture}(-0.5,-0.5)(1.5,1.5)
\psgrid
\rput(0.5,-0.25){$n=3$}
\rput(0,0.5){
\psline
  (0, 0)
  (0.037037, 0.000000)
  (0.055556, 0.032075)
  (0.074074, -0.000000)
  (0.111111, -0.000000)
  (0.129630, 0.032075)
  (0.111111, 0.064150)
  (0.148148, 0.064150)
  (0.166667, 0.096225)
  (0.185185, 0.064150)
  (0.222222, 0.064150)
  (0.203704, 0.032075)
  (0.222222, -0.000000)
  (0.259259, -0.000000)
  (0.277778, 0.032075)
  (0.296296, -0.000000)
  (0.333333, -0.000000)
  (0.351852, 0.032075)
  (0.333333, 0.064150)
  (0.370370, 0.064150)
  (0.388889, 0.096225)
  (0.370370, 0.128300)
  (0.333333, 0.128300)
  (0.351852, 0.160375)
  (0.333333, 0.192450)
  (0.370370, 0.192450)
  (0.388889, 0.224525)
  (0.407407, 0.192450)
  (0.444444, 0.192450)
  (0.462963, 0.224525)
  (0.444444, 0.256600)
  (0.481481, 0.256600)
  (0.500000, 0.288675)
  (0.518518, 0.256600)
  (0.555555, 0.256600)
  (0.537037, 0.224525)
  (0.555555, 0.192450)
  (0.592592, 0.192450)
  (0.611111, 0.224525)
  (0.629629, 0.192450)
  (0.666666, 0.192450)
  (0.648148, 0.160375)
  (0.666666, 0.128300)
  (0.629629, 0.128300)
  (0.611111, 0.096225)
  (0.629629, 0.064150)
  (0.666666, 0.064150)
  (0.648148, 0.032075)
  (0.666666, -0.000000)
  (0.703703, -0.000000)
  (0.722222, 0.032075)
  (0.740740, -0.000000)
  (0.777777, -0.000000)
  (0.796296, 0.032075)
  (0.777777, 0.064150)
  (0.814815, 0.064150)
  (0.833333, 0.096225)
  (0.851852, 0.064150)
  (0.888889, 0.064150)
  (0.870370, 0.032075)
  (0.888889, -0.000000)
  (0.925926, -0.000000)
  (0.944444, 0.032075)
  (0.962963, -0.000000)
  (1.000000, -0.000000)
}
\end{pspicture}
%
\hskip0.1in \dots
%

\end{zztask}

%%%%%%%%%%%%%%%%%%%%%%%%%%%%%%%%%%%%%%%%%%%%%%%%%%%%%%%%%%%%%%%%%%%%%%%%%%%%%%

\begin{zztask}[Koch Snow{f}lake]
В рамках общего условия задачи построить снежинку Коха, составленную из
трех одинаковых кривых Коха. База для построения кривой (нулевое приближение) 
представляет собой равносторонний треугольник. Переход от приближения
$n$ к приближению $(n+1)$ осуществляется заменой каждого отрезка фигуры на четыре
(см. рис. при $n=1$, выделено жирным). Все отрезки (включая пустой промежуток)
имеют одинаковую длину. Угол между отрезками \mbox{составляет $60^\circ$}.
\par\endinput
\begin{pspicture}(-0.5,-0.5)(1.5,1.5)
\psgrid
\rput(0.5,-0.25){$n=0$}
\rput(0,0.21){
\psline
  (0.500000, 0.866025)
  (1.000000, 0.000000)
  (-0.000000, 0.000000)
\psline[linewidth=0.06cm, arrows=o-o]
  (0, 0)
  (0.500000, 0.866025)
}
\end{pspicture}
%
\hskip0.1in
%
\begin{pspicture}(-0.5,-0.5)(1.5,1.5)
\psgrid
\rput(0.5,-0.25){$n=1$}
\rput(0,0.21){
\psline
  (0.500000, 0.866025)
  (0.666667, 0.577350)
  (1.000000, 0.577350)
  (0.833333, 0.288675)
  (1.000000, 0.000000)
  (0.666667, 0.000000)
  (0.500000, -0.288675)
  (0.333333, 0.000000)
  (0.000000, 0.000000)
\psline[linewidth=0.06cm, arrows=o-o, showpoints=true]
  (0, 0)
  (0.166667, 0.288675)
  (-0.000000, 0.577350)
  (0.333333, 0.577350)
  (0.500000, 0.866025)
}
\end{pspicture}
%
\hskip0.1in
%
\begin{pspicture}(-0.5,-0.5)(1.5,1.5)
\psgrid
\rput(0.5,-0.25){$n=2$}
\rput(0,0.21){
\psline
  (0, 0)
  (0.055556, 0.096225)
  (-0.000000, 0.192450)
  (0.111111, 0.192450)
  (0.166667, 0.288675)
  (0.111111, 0.384900)
  (-0.000000, 0.384900)
  (0.055556, 0.481125)
  (-0.000000, 0.577350)
  (0.111111, 0.577350)
  (0.166667, 0.673575)
  (0.222222, 0.577350)
  (0.333333, 0.577350)
  (0.388889, 0.673575)
  (0.333333, 0.769800)
  (0.444444, 0.769800)
  (0.500000, 0.866025)
  (0.555556, 0.769800)
  (0.666667, 0.769800)
  (0.611111, 0.673575)
  (0.666667, 0.577350)
  (0.777778, 0.577350)
  (0.833333, 0.673575)
  (0.888889, 0.577350)
  (1.000000, 0.577350)
  (0.944444, 0.481125)
  (1.000000, 0.384900)
  (0.888889, 0.384900)
  (0.833333, 0.288675)
  (0.888889, 0.192450)
  (1.000000, 0.192450)
  (0.944444, 0.096225)
  (1.000000, 0.000000)
  (0.888889, 0.000000)
  (0.833333, -0.096225)
  (0.777778, 0.000000)
  (0.666667, 0.000000)
  (0.611111, -0.096225)
  (0.666667, -0.192450)
  (0.555556, -0.192450)
  (0.500000, -0.288675)
  (0.444444, -0.192450)
  (0.333333, -0.192450)
  (0.388889, -0.096225)
  (0.333333, 0.000000)
  (0.222222, 0.000000)
  (0.166667, -0.096225)
  (0.111111, 0.000000)
  (-0.000000, 0.000000)
}
\end{pspicture}
%
\hskip0.1in
%
\begin{pspicture}(-0.5,-0.5)(1.5,1.5)
\psgrid
\rput(0.5,-0.25){$n=3$}
\rput(0,0.21){
\psline
  (0, 0)
  (0.018519, 0.032075)
  (-0.000000, 0.064150)
  (0.037037, 0.064150)
  (0.055556, 0.096225)
  (0.037037, 0.128300)
  (-0.000000, 0.128300)
  (0.018519, 0.160375)
  (-0.000000, 0.192450)
  (0.037037, 0.192450)
  (0.055556, 0.224525)
  (0.074074, 0.192450)
  (0.111111, 0.192450)
  (0.129630, 0.224525)
  (0.111111, 0.256600)
  (0.148148, 0.256600)
  (0.166667, 0.288675)
  (0.148148, 0.320750)
  (0.111111, 0.320750)
  (0.129630, 0.352825)
  (0.111111, 0.384900)
  (0.074074, 0.384900)
  (0.055556, 0.352825)
  (0.037037, 0.384900)
  (-0.000000, 0.384900)
  (0.018519, 0.416975)
  (-0.000000, 0.449050)
  (0.037037, 0.449050)
  (0.055556, 0.481125)
  (0.037037, 0.513200)
  (-0.000000, 0.513200)
  (0.018519, 0.545275)
  (-0.000000, 0.577350)
  (0.037037, 0.577350)
  (0.055556, 0.609425)
  (0.074074, 0.577350)
  (0.111111, 0.577350)
  (0.129630, 0.609425)
  (0.111111, 0.641500)
  (0.148148, 0.641500)
  (0.166667, 0.673575)
  (0.185185, 0.641500)
  (0.222222, 0.641500)
  (0.203704, 0.609425)
  (0.222222, 0.577350)
  (0.259259, 0.577350)
  (0.277778, 0.609425)
  (0.296296, 0.577350)
  (0.333333, 0.577350)
  (0.351852, 0.609425)
  (0.333333, 0.641500)
  (0.370370, 0.641500)
  (0.388889, 0.673575)
  (0.370370, 0.705650)
  (0.333333, 0.705650)
  (0.351852, 0.737725)
  (0.333333, 0.769800)
  (0.370370, 0.769800)
  (0.388889, 0.801875)
  (0.407407, 0.769800)
  (0.444444, 0.769800)
  (0.462963, 0.801875)
  (0.444444, 0.833950)
  (0.481481, 0.833950)
  (0.500000, 0.866025)
  (0.518518, 0.833950)
  (0.555555, 0.833950)
  (0.537037, 0.801875)
  (0.555555, 0.769800)
  (0.592592, 0.769800)
  (0.611111, 0.801875)
  (0.629629, 0.769800)
  (0.666666, 0.769800)
  (0.648148, 0.737725)
  (0.666666, 0.705650)
  (0.629629, 0.705650)
  (0.611111, 0.673575)
  (0.629629, 0.641500)
  (0.666666, 0.641500)
  (0.648148, 0.609425)
  (0.666666, 0.577350)
  (0.703703, 0.577350)
  (0.722222, 0.609425)
  (0.740740, 0.577350)
  (0.777777, 0.577350)
  (0.796296, 0.609425)
  (0.777777, 0.641500)
  (0.814815, 0.641500)
  (0.833333, 0.673575)
  (0.851852, 0.641500)
  (0.888889, 0.641500)
  (0.870370, 0.609425)
  (0.888889, 0.577350)
  (0.925926, 0.577350)
  (0.944444, 0.609425)
  (0.962963, 0.577350)
  (1.000000, 0.577350)
  (0.981481, 0.545275)
  (1.000000, 0.513200)
  (0.962963, 0.513200)
  (0.944444, 0.481125)
  (0.962963, 0.449050)
  (1.000000, 0.449050)
  (0.981481, 0.416975)
  (1.000000, 0.384900)
  (0.962963, 0.384900)
  (0.944444, 0.352825)
  (0.925926, 0.384900)
  (0.888889, 0.384900)
  (0.870370, 0.352825)
  (0.888889, 0.320750)
  (0.851852, 0.320750)
  (0.833333, 0.288675)
  (0.851852, 0.256600)
  (0.888889, 0.256600)
  (0.870370, 0.224525)
  (0.888889, 0.192450)
  (0.925926, 0.192450)
  (0.944444, 0.224525)
  (0.962963, 0.192450)
  (1.000000, 0.192450)
  (0.981481, 0.160375)
  (1.000000, 0.128300)
  (0.962963, 0.128300)
  (0.944444, 0.096225)
  (0.962963, 0.064150)
  (1.000000, 0.064150)
  (0.981481, 0.032075)
  (1.000000, -0.000000)
  (0.962963, 0.000000)
  (0.944444, -0.032075)
  (0.925926, 0.000000)
  (0.888889, 0.000000)
  (0.870370, -0.032075)
  (0.888889, -0.064150)
  (0.851852, -0.064150)
  (0.833333, -0.096225)
  (0.814815, -0.064150)
  (0.777777, -0.064150)
  (0.796296, -0.032075)
  (0.777777, 0.000000)
  (0.740740, 0.000000)
  (0.722222, -0.032075)
  (0.703703, 0.000000)
  (0.666666, 0.000000)
  (0.648148, -0.032075)
  (0.666666, -0.064150)
  (0.629629, -0.064150)
  (0.611111, -0.096225)
  (0.629629, -0.128300)
  (0.666666, -0.128300)
  (0.648148, -0.160375)
  (0.666666, -0.192450)
  (0.629629, -0.192450)
  (0.611111, -0.224525)
  (0.592592, -0.192450)
  (0.555555, -0.192450)
  (0.537037, -0.224525)
  (0.555555, -0.256600)
  (0.518518, -0.256600)
  (0.500000, -0.288675)
  (0.481481, -0.256600)
  (0.444444, -0.256600)
  (0.462963, -0.224525)
  (0.444444, -0.192450)
  (0.407407, -0.192450)
  (0.388889, -0.224525)
  (0.370370, -0.192450)
  (0.333333, -0.192450)
  (0.351852, -0.160375)
  (0.333333, -0.128300)
  (0.370370, -0.128300)
  (0.388889, -0.096225)
  (0.370370, -0.064150)
  (0.333333, -0.064150)
  (0.351852, -0.032075)
  (0.333333, 0.000000)
  (0.296296, 0.000000)
  (0.277778, -0.032075)
  (0.259259, 0.000000)
  (0.222222, 0.000000)
  (0.203704, -0.032075)
  (0.222222, -0.064150)
  (0.185185, -0.064150)
  (0.166666, -0.096225)
  (0.148148, -0.064150)
  (0.111111, -0.064150)
  (0.129629, -0.032075)
  (0.111111, 0.000000)
  (0.074074, 0.000000)
  (0.055555, -0.032075)
  (0.037037, 0.000000)
  (-0.000000, 0.000000)
}
\end{pspicture}
%
\hskip0.1in\dots
%

\end{zztask}

%%%%%%%%%%%%%%%%%%%%%%%%%%%%%%%%%%%%%%%%%%%%%%%%%%%%%%%%%%%%%%%%%%%%%%%%%%%%%%

\begin{zztask}[Harter-Heighway Dragon]
В рамках общего условия задачи построить кривую дракона Хейуэя. База для 
построения кривой (нулевое приближение) представляет собой отрезок. Переход 
от приближения $n$ к приближению $(n+1)$ осуществляется заменой каждого 
отрезка фигуры на два (см. рис. при $n=1$). Все отрезки 
имеют одинаковую длину, угол между отрезками составляет $90^\circ$.
Выгиб происходит чередуясь, то в одну, то в другую сторону (см. рис. при $n=2$).
\par\endinput
\begin{pspicture}(-0.5,-0.5)(1.5,1.5)
\psgrid
\rput(0.5,-0.25){$n=0$}
\rput(0,0.5){
\psline[linewidth=0.06cm, arrows=o-o]
  (0, 0)
  (1.000000, 0.000000)
}
\end{pspicture}
%
\hskip0.1in
%
\begin{pspicture}(-0.5,-0.5)(1.5,1.5)
\psgrid
\rput(0.5,-0.25){$n=1$}
\rput(0,0.5){
\psline[linewidth=0.01cm, linestyle=dashed]
  (0, 0)
  (1.000000, 0.000000)
\psline[linewidth=0.06cm, arrows=o-o, showpoints=true]
  (0, 0)
  (0.500000, 0.500000)
  (1.000000, -0.000000)
}
\end{pspicture}
%
\hskip0.1in
%
\begin{pspicture}(-0.5,-0.5)(1.5,1.5)
\psgrid
\rput(0.5,-0.25){$n=2$}
\rput(0,0.5){
\psline[linewidth=0.01cm, linestyle=dashed]
  (0, 0)
  (0.500000, 0.500000)
  (1.000000, -0.000000)
\psline[linewidth=0.06cm, arrows=o-o, showpoints=true]
  (0, 0)
  (-0.000000, 0.500000)
  (0.500000, 0.500000)
\psline[linewidth=0.06cm, arrows=o-o, showpoints=true]
  (0.500000, 0.500000)
  (0.500000, 0.000000)
  (1.000000, 0.000000)
}
\end{pspicture}
%
\hskip0.1in
%
\begin{pspicture}(-0.5,-0.5)(1.5,1.5)
\psgrid
\rput(0.5,-0.25){$n=3$}
\rput(0,0.5){
\psline
  (0, 0)
  (-0.250000, 0.250000)
  (0.000000, 0.500000)
  (0.250000, 0.250000)
  (0.500000, 0.500000)
  (0.750000, 0.250000)
  (0.500000, -0.000000)
  (0.750000, -0.250000)
  (1.000000, -0.000000)
}
\end{pspicture}
%
\hskip0.1in \dots \hskip0.1in
%
\begin{pspicture}(-0.5,-0.5)(1.5,1.5)
\psgrid
\rput(0.5,-0.25){$n=9$}
\rput(0,0.5){
\psline
  (0, 0)
  (0.031250, 0.031250)
  (0.062500, -0.000000)
  (0.031250, -0.031250)
  (0.062500, -0.062500)
  (0.031250, -0.093750)
  (0.000000, -0.062500)
  (-0.031250, -0.093750)
  (-0.000000, -0.125000)
  (-0.031250, -0.156250)
  (-0.062500, -0.125000)
  (-0.031250, -0.093750)
  (-0.062500, -0.062500)
  (-0.093750, -0.093750)
  (-0.125000, -0.062500)
  (-0.156250, -0.093750)
  (-0.125000, -0.125000)
  (-0.156250, -0.156250)
  (-0.187500, -0.125000)
  (-0.156250, -0.093750)
  (-0.187500, -0.062500)
  (-0.156250, -0.031250)
  (-0.125000, -0.062500)
  (-0.093750, -0.031250)
  (-0.125000, -0.000000)
  (-0.156250, -0.031250)
  (-0.187500, 0.000000)
  (-0.156250, 0.031250)
  (-0.187500, 0.062500)
  (-0.218750, 0.031250)
  (-0.250000, 0.062500)
  (-0.281250, 0.031250)
  (-0.250000, 0.000000)
  (-0.281250, -0.031250)
  (-0.312500, 0.000000)
  (-0.281250, 0.031250)
  (-0.312500, 0.062500)
  (-0.281250, 0.093750)
  (-0.250000, 0.062500)
  (-0.218750, 0.093750)
  (-0.250000, 0.125000)
  (-0.218750, 0.156250)
  (-0.187500, 0.125000)
  (-0.218750, 0.093750)
  (-0.187500, 0.062500)
  (-0.156250, 0.093750)
  (-0.125000, 0.062500)
  (-0.093750, 0.093750)
  (-0.125000, 0.125000)
  (-0.156250, 0.093750)
  (-0.187500, 0.125000)
  (-0.156250, 0.156250)
  (-0.187500, 0.187500)
  (-0.156250, 0.218750)
  (-0.125000, 0.187500)
  (-0.093750, 0.218750)
  (-0.125000, 0.250000)
  (-0.156250, 0.218750)
  (-0.187500, 0.250000)
  (-0.156250, 0.281250)
  (-0.187500, 0.312500)
  (-0.218750, 0.281250)
  (-0.250000, 0.312500)
  (-0.281250, 0.281250)
  (-0.250000, 0.250000)
  (-0.281250, 0.218750)
  (-0.312500, 0.250000)
  (-0.281250, 0.281250)
  (-0.312500, 0.312500)
  (-0.281250, 0.343750)
  (-0.250000, 0.312500)
  (-0.218750, 0.343750)
  (-0.250000, 0.375000)
  (-0.218750, 0.406250)
  (-0.187500, 0.375000)
  (-0.218750, 0.343750)
  (-0.187500, 0.312500)
  (-0.156250, 0.343750)
  (-0.125000, 0.312500)
  (-0.093750, 0.343750)
  (-0.125000, 0.375000)
  (-0.093750, 0.406250)
  (-0.062500, 0.375000)
  (-0.093750, 0.343750)
  (-0.062500, 0.312500)
  (-0.093750, 0.281250)
  (-0.125000, 0.312500)
  (-0.156250, 0.281250)
  (-0.125000, 0.250000)
  (-0.093750, 0.281250)
  (-0.062500, 0.250000)
  (-0.093750, 0.218750)
  (-0.062500, 0.187500)
  (-0.031250, 0.218750)
  (-0.000000, 0.187500)
  (0.031250, 0.218750)
  (0.000000, 0.250000)
  (-0.031250, 0.218750)
  (-0.062500, 0.250000)
  (-0.031250, 0.281250)
  (-0.062500, 0.312500)
  (-0.031250, 0.343750)
  (0.000000, 0.312500)
  (0.031250, 0.343750)
  (0.000000, 0.375000)
  (0.031250, 0.406250)
  (0.062500, 0.375000)
  (0.031250, 0.343750)
  (0.062500, 0.312500)
  (0.093750, 0.343750)
  (0.125000, 0.312500)
  (0.156250, 0.343750)
  (0.125000, 0.375000)
  (0.093750, 0.343750)
  (0.062500, 0.375000)
  (0.093750, 0.406250)
  (0.062500, 0.437500)
  (0.093750, 0.468750)
  (0.125000, 0.437500)
  (0.156250, 0.468750)
  (0.125000, 0.500000)
  (0.093750, 0.468750)
  (0.062500, 0.500000)
  (0.093750, 0.531250)
  (0.062500, 0.562500)
  (0.031250, 0.531250)
  (0.000000, 0.562500)
  (-0.031250, 0.531250)
  (0.000000, 0.500000)
  (-0.031250, 0.468750)
  (-0.062500, 0.500000)
  (-0.031250, 0.531250)
  (-0.062500, 0.562500)
  (-0.031250, 0.593750)
  (0.000000, 0.562500)
  (0.031250, 0.593750)
  (0.000000, 0.625000)
  (0.031250, 0.656250)
  (0.062500, 0.625000)
  (0.031250, 0.593750)
  (0.062500, 0.562500)
  (0.093750, 0.593750)
  (0.125000, 0.562500)
  (0.156250, 0.593750)
  (0.125000, 0.625000)
  (0.156250, 0.656250)
  (0.187500, 0.625000)
  (0.156250, 0.593750)
  (0.187500, 0.562500)
  (0.156250, 0.531250)
  (0.125000, 0.562500)
  (0.093750, 0.531250)
  (0.125000, 0.500000)
  (0.156250, 0.531250)
  (0.187500, 0.500000)
  (0.156250, 0.468750)
  (0.187500, 0.437500)
  (0.218750, 0.468750)
  (0.250000, 0.437500)
  (0.281250, 0.468750)
  (0.250000, 0.500000)
  (0.281250, 0.531250)
  (0.312500, 0.500000)
  (0.281250, 0.468750)
  (0.312500, 0.437500)
  (0.281250, 0.406250)
  (0.250000, 0.437500)
  (0.218750, 0.406250)
  (0.250000, 0.375000)
  (0.218750, 0.343750)
  (0.187500, 0.375000)
  (0.218750, 0.406250)
  (0.187500, 0.437500)
  (0.156250, 0.406250)
  (0.125000, 0.437500)
  (0.093750, 0.406250)
  (0.125000, 0.375000)
  (0.156250, 0.406250)
  (0.187500, 0.375000)
  (0.156250, 0.343750)
  (0.187500, 0.312500)
  (0.156250, 0.281250)
  (0.125000, 0.312500)
  (0.093750, 0.281250)
  (0.125000, 0.250000)
  (0.156250, 0.281250)
  (0.187500, 0.250000)
  (0.156250, 0.218750)
  (0.187500, 0.187500)
  (0.218750, 0.218750)
  (0.250000, 0.187500)
  (0.281250, 0.218750)
  (0.250000, 0.250000)
  (0.218750, 0.218750)
  (0.187500, 0.250000)
  (0.218750, 0.281250)
  (0.187500, 0.312500)
  (0.218750, 0.343750)
  (0.250000, 0.312500)
  (0.281250, 0.343750)
  (0.250000, 0.375000)
  (0.281250, 0.406250)
  (0.312500, 0.375000)
  (0.281250, 0.343750)
  (0.312500, 0.312500)
  (0.343750, 0.343750)
  (0.375000, 0.312500)
  (0.406250, 0.343750)
  (0.375000, 0.375000)
  (0.406250, 0.406250)
  (0.437500, 0.375000)
  (0.406250, 0.343750)
  (0.437500, 0.312500)
  (0.406250, 0.281250)
  (0.375000, 0.312500)
  (0.343750, 0.281250)
  (0.375000, 0.250000)
  (0.406250, 0.281250)
  (0.437500, 0.250000)
  (0.406250, 0.218750)
  (0.437500, 0.187500)
  (0.468750, 0.218750)
  (0.500000, 0.187500)
  (0.531250, 0.218750)
  (0.500000, 0.250000)
  (0.468750, 0.218750)
  (0.437500, 0.250000)
  (0.468750, 0.281250)
  (0.437500, 0.312500)
  (0.468750, 0.343750)
  (0.500000, 0.312500)
  (0.531250, 0.343750)
  (0.500000, 0.375000)
  (0.531250, 0.406250)
  (0.562500, 0.375000)
  (0.531250, 0.343750)
  (0.562500, 0.312500)
  (0.593750, 0.343750)
  (0.625000, 0.312500)
  (0.656250, 0.343750)
  (0.625000, 0.375000)
  (0.593750, 0.343750)
  (0.562500, 0.375000)
  (0.593750, 0.406250)
  (0.562500, 0.437500)
  (0.593750, 0.468750)
  (0.625000, 0.437500)
  (0.656250, 0.468750)
  (0.625000, 0.500000)
  (0.593750, 0.468750)
  (0.562500, 0.500000)
  (0.593750, 0.531250)
  (0.562500, 0.562500)
  (0.531250, 0.531250)
  (0.500000, 0.562500)
  (0.468750, 0.531250)
  (0.500000, 0.500000)
  (0.468750, 0.468750)
  (0.437500, 0.500000)
  (0.468750, 0.531250)
  (0.437500, 0.562500)
  (0.468750, 0.593750)
  (0.500000, 0.562500)
  (0.531250, 0.593750)
  (0.500000, 0.625000)
  (0.531250, 0.656250)
  (0.562500, 0.625000)
  (0.531250, 0.593750)
  (0.562500, 0.562500)
  (0.593750, 0.593750)
  (0.625000, 0.562500)
  (0.656250, 0.593750)
  (0.625000, 0.625000)
  (0.656250, 0.656250)
  (0.687500, 0.625000)
  (0.656250, 0.593750)
  (0.687500, 0.562500)
  (0.656250, 0.531250)
  (0.625000, 0.562500)
  (0.593750, 0.531250)
  (0.625000, 0.500000)
  (0.656250, 0.531250)
  (0.687500, 0.500000)
  (0.656250, 0.468750)
  (0.687500, 0.437500)
  (0.718750, 0.468750)
  (0.750000, 0.437500)
  (0.781250, 0.468750)
  (0.750000, 0.500000)
  (0.781250, 0.531250)
  (0.812500, 0.500000)
  (0.781250, 0.468750)
  (0.812500, 0.437500)
  (0.781250, 0.406250)
  (0.750000, 0.437500)
  (0.718750, 0.406250)
  (0.750000, 0.375000)
  (0.718750, 0.343750)
  (0.687500, 0.375000)
  (0.718750, 0.406250)
  (0.687500, 0.437500)
  (0.656250, 0.406250)
  (0.625000, 0.437500)
  (0.593750, 0.406250)
  (0.625000, 0.375000)
  (0.656250, 0.406250)
  (0.687500, 0.375000)
  (0.656250, 0.343750)
  (0.687500, 0.312500)
  (0.656250, 0.281250)
  (0.625000, 0.312500)
  (0.593750, 0.281250)
  (0.625000, 0.250000)
  (0.656250, 0.281250)
  (0.687500, 0.250000)
  (0.656250, 0.218750)
  (0.687500, 0.187500)
  (0.718750, 0.218750)
  (0.750000, 0.187500)
  (0.781250, 0.218750)
  (0.750000, 0.250000)
  (0.781250, 0.281250)
  (0.812500, 0.250000)
  (0.781250, 0.218750)
  (0.812500, 0.187500)
  (0.781250, 0.156251)
  (0.750000, 0.187501)
  (0.718750, 0.156251)
  (0.750000, 0.125001)
  (0.718750, 0.093751)
  (0.687500, 0.125001)
  (0.718750, 0.156251)
  (0.687500, 0.187501)
  (0.656250, 0.156251)
  (0.625000, 0.187501)
  (0.593750, 0.156251)
  (0.625000, 0.125001)
  (0.593750, 0.093751)
  (0.562500, 0.125001)
  (0.593750, 0.156251)
  (0.562500, 0.187501)
  (0.593750, 0.218751)
  (0.625000, 0.187501)
  (0.656250, 0.218751)
  (0.625000, 0.250001)
  (0.593750, 0.218751)
  (0.562500, 0.250001)
  (0.593750, 0.281251)
  (0.562500, 0.312501)
  (0.531250, 0.281251)
  (0.500000, 0.312501)
  (0.468750, 0.281251)
  (0.500000, 0.250001)
  (0.531250, 0.281251)
  (0.562500, 0.250001)
  (0.531250, 0.218751)
  (0.562500, 0.187501)
  (0.531250, 0.156251)
  (0.500000, 0.187501)
  (0.468750, 0.156251)
  (0.500000, 0.125001)
  (0.468750, 0.093751)
  (0.437500, 0.125001)
  (0.468750, 0.156251)
  (0.437500, 0.187501)
  (0.406250, 0.156251)
  (0.375000, 0.187501)
  (0.343750, 0.156251)
  (0.375000, 0.125001)
  (0.406250, 0.156251)
  (0.437500, 0.125001)
  (0.406250, 0.093751)
  (0.437500, 0.062501)
  (0.406250, 0.031251)
  (0.375000, 0.062501)
  (0.343750, 0.031251)
  (0.375000, 0.000001)
  (0.406250, 0.031251)
  (0.437500, 0.000001)
  (0.406250, -0.031249)
  (0.437500, -0.062499)
  (0.468750, -0.031249)
  (0.500000, -0.062499)
  (0.531250, -0.031249)
  (0.500000, 0.000001)
  (0.468750, -0.031249)
  (0.437500, 0.000001)
  (0.468750, 0.031251)
  (0.437500, 0.062501)
  (0.468750, 0.093751)
  (0.500000, 0.062501)
  (0.531250, 0.093751)
  (0.500000, 0.125001)
  (0.531250, 0.156251)
  (0.562500, 0.125001)
  (0.531250, 0.093751)
  (0.562500, 0.062501)
  (0.593750, 0.093751)
  (0.625000, 0.062501)
  (0.656250, 0.093751)
  (0.625000, 0.125001)
  (0.656250, 0.156251)
  (0.687500, 0.125001)
  (0.656250, 0.093751)
  (0.687500, 0.062501)
  (0.656250, 0.031251)
  (0.625000, 0.062501)
  (0.593750, 0.031251)
  (0.625000, 0.000001)
  (0.656250, 0.031251)
  (0.687500, 0.000001)
  (0.656250, -0.031249)
  (0.687500, -0.062499)
  (0.718750, -0.031249)
  (0.750000, -0.062499)
  (0.781250, -0.031249)
  (0.750000, 0.000001)
  (0.781250, 0.031251)
  (0.812500, 0.000001)
  (0.781250, -0.031249)
  (0.812500, -0.062499)
  (0.781250, -0.093749)
  (0.750000, -0.062499)
  (0.718750, -0.093749)
  (0.750000, -0.124999)
  (0.718750, -0.156249)
  (0.687500, -0.124999)
  (0.718750, -0.093749)
  (0.687500, -0.062499)
  (0.656250, -0.093749)
  (0.625000, -0.062499)
  (0.593750, -0.093749)
  (0.625000, -0.124999)
  (0.656250, -0.093749)
  (0.687500, -0.124999)
  (0.656250, -0.156249)
  (0.687500, -0.187499)
  (0.656250, -0.218749)
  (0.625000, -0.187499)
  (0.593750, -0.218749)
  (0.625000, -0.249999)
  (0.656250, -0.218749)
  (0.687500, -0.249999)
  (0.656250, -0.281249)
  (0.687500, -0.312499)
  (0.718750, -0.281249)
  (0.750000, -0.312499)
  (0.781250, -0.281249)
  (0.750000, -0.249999)
  (0.718750, -0.281249)
  (0.687500, -0.249999)
  (0.718750, -0.218749)
  (0.687500, -0.187499)
  (0.718750, -0.156249)
  (0.750000, -0.187499)
  (0.781250, -0.156249)
  (0.750000, -0.124999)
  (0.781250, -0.093749)
  (0.812500, -0.124999)
  (0.781250, -0.156249)
  (0.812500, -0.187499)
  (0.843750, -0.156249)
  (0.875000, -0.187499)
  (0.906250, -0.156249)
  (0.875000, -0.124999)
  (0.906250, -0.093749)
  (0.937500, -0.124999)
  (0.906250, -0.156249)
  (0.937500, -0.187499)
  (0.906250, -0.218749)
  (0.875000, -0.187499)
  (0.843750, -0.218749)
  (0.875000, -0.249999)
  (0.906250, -0.218749)
  (0.937500, -0.249999)
  (0.906250, -0.281249)
  (0.937500, -0.312499)
  (0.968750, -0.281249)
  (1.000000, -0.312499)
  (1.031250, -0.281249)
  (1.000000, -0.249999)
  (0.968750, -0.281249)
  (0.937500, -0.249999)
  (0.968750, -0.218749)
  (0.937500, -0.187499)
  (0.968750, -0.156249)
  (1.000000, -0.187499)
  (1.031250, -0.156249)
  (1.000000, -0.124999)
  (1.031250, -0.093749)
  (1.062500, -0.124999)
  (1.031250, -0.156249)
  (1.062500, -0.187499)
  (1.093750, -0.156249)
  (1.125000, -0.187499)
  (1.156250, -0.156249)
  (1.125000, -0.124999)
  (1.093750, -0.156249)
  (1.062500, -0.124999)
  (1.093750, -0.093749)
  (1.062500, -0.062499)
  (1.093750, -0.031249)
  (1.125000, -0.062499)
  (1.156250, -0.031249)
  (1.125000, 0.000001)
  (1.093750, -0.031249)
  (1.062500, 0.000001)
  (1.093750, 0.031251)
  (1.062500, 0.062501)
  (1.031250, 0.031251)
  (1.000000, 0.062501)
  (0.968750, 0.031251)
  (1.000000, 0.000001)
}
\end{pspicture}
%
\hskip0.1in \dots
%

\end{zztask}

%\clearpage

%%%%%%%%%%%%%%%%%%%%%%%%%%%%%%%%%%%%%%%%%%%%%%%%%%%%%%%%%%%%%%%%%%%%%%%%%%%%%%

\begin{zztask}[Knuth's Terdragon]
В рамках общего условия задачи построить кривую дракона Кнута. База для 
построения кривой (нулевое приближение) представляет собой отрезок. Переход 
от приближения $n$ к приближению $(n+1)$ осуществляется заменой каждого 
отрезка фигуры на три (см. рис. при $n=1$). Все отрезки 
имеют одинаковую длину, угол между отрезками составляет $60^\circ$.
\par\endinput
\begin{pspicture}(-0.5,-0.5)(1.5,1.5)
\psgrid
\rput(0.5,-0.25){$n=0$}
\rput(0,0.5){
\psline[linewidth=0.06cm, arrows=o-o]
  (0, 0)
  (1.000000, 0.000000)
}
\end{pspicture}
%
\hskip0.1in
%
\begin{pspicture}(-0.5,-0.5)(1.5,1.5)
\psgrid
\rput(0.5,-0.25){$n=1$}
\rput(0,0.5){
\psline[linewidth=0.01cm, linestyle=dashed]
  (0, 0)
  (1.000000, 0.000000)
\psline[linewidth=0.06cm, arrows=o-o, showpoints=true]
  (0, 0)
  (0.500000, 0.288675)
  (0.500000, -0.288675)
  (1.000000, 0.000000)
}
\end{pspicture}
%
\hskip0.1in
%
\begin{pspicture}(-0.5,-0.5)(1.5,1.5)
\psgrid
\rput(0.5,-0.25){$n=2$}
\rput(0,0.5){
\psline[linewidth=0.01cm, linestyle=dashed]
  (0, 0)
  (0.500000, 0.288675)
  (0.500000, -0.288675)
  (1.000000, 0.000000)
\psline
  (0, 0)
  (0.166667, 0.288675)
  (0.333333, 0.000000)
  (0.500000, 0.288675)
  (0.666667, 0.000000)
  (0.333333, -0.000000)
  (0.500000, -0.288675)
  (0.666667, -0.000000)
  (0.833333, -0.288675)
  (1.000000, -0.000000)
\psline[linewidth=0.06cm, arrows=o-o, showpoints=true]
  (0, 0)
  (0.166667, 0.288675)
  (0.333333, 0.000000)
  (0.500000, 0.288675)
}
\end{pspicture}
%
\hskip0.1in
%
\begin{pspicture}(-0.5,-0.5)(1.5,1.5)
\psgrid
\rput(0.5,-0.25){$n=3$}
\rput(0,0.5){
\psline
  (0, 0)
  (-0.000000, 0.192450)
  (0.166667, 0.096225)
  (0.166667, 0.288675)
  (0.333333, 0.192450)
  (0.166667, 0.096225)
  (0.333333, -0.000000)
  (0.333333, 0.192450)
  (0.500000, 0.096225)
  (0.500000, 0.288675)
  (0.666667, 0.192450)
  (0.500000, 0.096225)
  (0.666667, -0.000000)
  (0.500000, -0.096225)
  (0.500000, 0.096225)
  (0.333333, -0.000000)
  (0.500000, -0.096225)
  (0.333333, -0.192450)
  (0.500000, -0.288675)
  (0.500000, -0.096225)
  (0.666666, -0.192450)
  (0.666666, -0.000000)
  (0.833333, -0.096225)
  (0.666666, -0.192450)
  (0.833333, -0.288675)
  (0.833333, -0.096225)
  (1.000000, -0.192450)
  (1.000000, -0.000000)
}
\end{pspicture}
%
\hskip0.1in\dots\hskip0.1in
%
\begin{pspicture}(-0.5,-0.5)(1.5,1.5)
\psgrid
\rput(0.5,-0.25){$n=5$}
\rput(0,0.5){
\psline
  (0, 0)
  (-0.055556, 0.032075)
  (0.000000, 0.064150)
  (-0.055556, 0.096225)
  (0.000000, 0.128300)
  (-0.000000, 0.064150)
  (0.055556, 0.096225)
  (-0.000000, 0.128300)
  (0.055556, 0.160375)
  (-0.000000, 0.192450)
  (0.055556, 0.224525)
  (0.055556, 0.160375)
  (0.111111, 0.192450)
  (0.111111, 0.128300)
  (0.055556, 0.160375)
  (0.055556, 0.096225)
  (0.111111, 0.128300)
  (0.111111, 0.064150)
  (0.166667, 0.096225)
  (0.111111, 0.128300)
  (0.166667, 0.160375)
  (0.111111, 0.192450)
  (0.166667, 0.224525)
  (0.166667, 0.160375)
  (0.222222, 0.192450)
  (0.166667, 0.224525)
  (0.222222, 0.256600)
  (0.166667, 0.288675)
  (0.222222, 0.320750)
  (0.222222, 0.256600)
  (0.277778, 0.288675)
  (0.277778, 0.224525)
  (0.222222, 0.256600)
  (0.222222, 0.192450)
  (0.277778, 0.224525)
  (0.277778, 0.160375)
  (0.333333, 0.192450)
  (0.333333, 0.128300)
  (0.277778, 0.160375)
  (0.277778, 0.096225)
  (0.222222, 0.128300)
  (0.277778, 0.160375)
  (0.222222, 0.192450)
  (0.222222, 0.128300)
  (0.166667, 0.160375)
  (0.166667, 0.096225)
  (0.222222, 0.128300)
  (0.222222, 0.064150)
  (0.277778, 0.096225)
  (0.277778, 0.032075)
  (0.222222, 0.064150)
  (0.222222, 0.000000)
  (0.277778, 0.032075)
  (0.277778, -0.032075)
  (0.333333, 0.000000)
  (0.277778, 0.032075)
  (0.333333, 0.064150)
  (0.277778, 0.096225)
  (0.333333, 0.128300)
  (0.333333, 0.064150)
  (0.388889, 0.096225)
  (0.333333, 0.128300)
  (0.388889, 0.160375)
  (0.333333, 0.192450)
  (0.388889, 0.224525)
  (0.388889, 0.160375)
  (0.444444, 0.192450)
  (0.444444, 0.128300)
  (0.388889, 0.160375)
  (0.388889, 0.096225)
  (0.444444, 0.128300)
  (0.444444, 0.064150)
  (0.500000, 0.096225)
  (0.444444, 0.128300)
  (0.500000, 0.160375)
  (0.444444, 0.192450)
  (0.500000, 0.224525)
  (0.500000, 0.160375)
  (0.555555, 0.192450)
  (0.500000, 0.224525)
  (0.555555, 0.256600)
  (0.500000, 0.288675)
  (0.555555, 0.320750)
  (0.555555, 0.256600)
  (0.611111, 0.288675)
  (0.611111, 0.224525)
  (0.555555, 0.256600)
  (0.555555, 0.192450)
  (0.611111, 0.224525)
  (0.611111, 0.160375)
  (0.666667, 0.192450)
  (0.666667, 0.128300)
  (0.611111, 0.160375)
  (0.611111, 0.096225)
  (0.555555, 0.128300)
  (0.611111, 0.160375)
  (0.555555, 0.192450)
  (0.555555, 0.128300)
  (0.500000, 0.160375)
  (0.500000, 0.096225)
  (0.555555, 0.128300)
  (0.555555, 0.064150)
  (0.611111, 0.096225)
  (0.611111, 0.032075)
  (0.555555, 0.064150)
  (0.555555, 0.000000)
  (0.611111, 0.032075)
  (0.611111, -0.032075)
  (0.666667, 0.000000)
  (0.666667, -0.064150)
  (0.611111, -0.032075)
  (0.611111, -0.096225)
  (0.555555, -0.064150)
  (0.611111, -0.032075)
  (0.555555, 0.000000)
  (0.555555, -0.064150)
  (0.500000, -0.032075)
  (0.500000, -0.096225)
  (0.444444, -0.064150)
  (0.500000, -0.032075)
  (0.444444, 0.000000)
  (0.500000, 0.032075)
  (0.500000, -0.032075)
  (0.555555, 0.000000)
  (0.500000, 0.032075)
  (0.555555, 0.064150)
  (0.500000, 0.096225)
  (0.500000, 0.032075)
  (0.444444, 0.064150)
  (0.444444, 0.000000)
  (0.388889, 0.032075)
  (0.444444, 0.064150)
  (0.388889, 0.096225)
  (0.388889, 0.032075)
  (0.333333, 0.064150)
  (0.333333, 0.000000)
  (0.388889, 0.032075)
  (0.388889, -0.032075)
  (0.444444, 0.000000)
  (0.444444, -0.064150)
  (0.388889, -0.032075)
  (0.388889, -0.096225)
  (0.444444, -0.064150)
  (0.444444, -0.128300)
  (0.500000, -0.096225)
  (0.500000, -0.160375)
  (0.444444, -0.128300)
  (0.444444, -0.192450)
  (0.388889, -0.160375)
  (0.444444, -0.128300)
  (0.388889, -0.096225)
  (0.388889, -0.160375)
  (0.333333, -0.128300)
  (0.333333, -0.192450)
  (0.388889, -0.160375)
  (0.388889, -0.224525)
  (0.444444, -0.192450)
  (0.444444, -0.256600)
  (0.388889, -0.224525)
  (0.388889, -0.288675)
  (0.444444, -0.256600)
  (0.444444, -0.320750)
  (0.500000, -0.288675)
  (0.444444, -0.256600)
  (0.500000, -0.224525)
  (0.444444, -0.192450)
  (0.500000, -0.160375)
  (0.500000, -0.224525)
  (0.555555, -0.192450)
  (0.500000, -0.160375)
  (0.555555, -0.128300)
  (0.500000, -0.096225)
  (0.555555, -0.064150)
  (0.555555, -0.128300)
  (0.611111, -0.096225)
  (0.611111, -0.160375)
  (0.555555, -0.128300)
  (0.555555, -0.192450)
  (0.611111, -0.160375)
  (0.611111, -0.224525)
  (0.666666, -0.192450)
  (0.611111, -0.160375)
  (0.666666, -0.128300)
  (0.611111, -0.096225)
  (0.666666, -0.064150)
  (0.666666, -0.128300)
  (0.722222, -0.096225)
  (0.666666, -0.064150)
  (0.722222, -0.032075)
  (0.666666, 0.000000)
  (0.722222, 0.032075)
  (0.722222, -0.032075)
  (0.777778, 0.000000)
  (0.777778, -0.064150)
  (0.722222, -0.032075)
  (0.722222, -0.096225)
  (0.777778, -0.064150)
  (0.777778, -0.128300)
  (0.833333, -0.096225)
  (0.833333, -0.160375)
  (0.777778, -0.128300)
  (0.777778, -0.192450)
  (0.722222, -0.160375)
  (0.777778, -0.128300)
  (0.722222, -0.096225)
  (0.722222, -0.160375)
  (0.666666, -0.128300)
  (0.666666, -0.192450)
  (0.722222, -0.160375)
  (0.722222, -0.224525)
  (0.777778, -0.192450)
  (0.777778, -0.256600)
  (0.722222, -0.224525)
  (0.722222, -0.288675)
  (0.777778, -0.256600)
  (0.777778, -0.320750)
  (0.833333, -0.288675)
  (0.777778, -0.256600)
  (0.833333, -0.224525)
  (0.777778, -0.192450)
  (0.833333, -0.160375)
  (0.833333, -0.224525)
  (0.888889, -0.192450)
  (0.833333, -0.160375)
  (0.888889, -0.128300)
  (0.833333, -0.096225)
  (0.888889, -0.064150)
  (0.888889, -0.128300)
  (0.944444, -0.096225)
  (0.944444, -0.160375)
  (0.888889, -0.128300)
  (0.888889, -0.192450)
  (0.944444, -0.160375)
  (0.944444, -0.224525)
  (1.000000, -0.192450)
  (0.944444, -0.160375)
  (1.000000, -0.128300)
  (0.944444, -0.096225)
  (1.000000, -0.064150)
  (1.000000, -0.128300)
  (1.055555, -0.096225)
  (1.000000, -0.064150)
  (1.055555, -0.032075)
  (1.000000, 0.000000)
}
\end{pspicture}
%
\hskip0.1in\dots
%
\end{zztask}

%%%%%%%%%%%%%%%%%%%%%%%%%%%%%%%%%%%%%%%%%%%%%%%%%%%%%%%%%%%%%%%%%%%%%%%%%%%%%%

\begin{zztask}[L\'evy C Сurve]
В рамках общего условия задачи построить кривую Леви. База для 
построения кривой (нулевое приближение) представляет собой отрезок. Переход 
от приближения $n$ к приближению $(n+1)$ осуществляется заменой каждого 
отрезка фигуры на два (см. рис. при $n=1$). Все отрезки 
имеют одинаковую длину, угол между отрезками составляет $90^\circ$.
\par\endinput
\begin{pspicture}(-0.5,-0.5)(1.5,1.5)
\psgrid
\rput(0.5,-0.25){$n=0$}
\rput(0,0.5){
\psline[linewidth=0.06cm, arrows=o-o]
  (0, 0)
  (1.000000, 0.000000)
}
\end{pspicture}
%
\hskip0.1in
%
\begin{pspicture}(-0.5,-0.5)(1.5,1.5)
\psgrid
\rput(0.5,-0.25){$n=1$}
\rput(0,0.5){
\psline[linewidth=0.01cm, linestyle=dashed]
  (0, 0)
  (1.000000, 0.000000)
\psline[linewidth=0.06cm, arrows=o-o, showpoints=true]
  (0, 0)
  (0.500000, 0.500000)
  (1.000000, -0.000000)
}
\end{pspicture}
%
\hskip0.1in
%
\begin{pspicture}(-0.5,-0.5)(1.5,1.5)
\psgrid
\rput(0.5,-0.25){$n=2$}
\rput(0,0.5){
\psline[linewidth=0.01cm, linestyle=dashed]
  (0, 0)
  (0.500000, 0.500000)
  (1.000000, -0.000000)
\psline[linewidth=0.06cm, arrows=o-o, showpoints=true]
  (0, 0)
  (-0.000000, 0.500000)
  (0.500000, 0.500000)
\psline[linewidth=0.06cm, arrows=o-o, showpoints=true]
  (0.500000, 0.500000)
  (1.000000, 0.500000)
  (1.000000, 0.000000)
}
\end{pspicture}
%
\hskip0.1in
%
\begin{pspicture}(-0.5,-0.5)(1.5,1.5)
\psgrid
\rput(0.5,-0.25){$n=3$}
\rput(0,0.5){
\psline
  (0, 0)
  (-0.250000, 0.250000)
  (0.000000, 0.500000)
  (0.250000, 0.750000)
  (0.500000, 0.500000)
  (0.750000, 0.750000)
  (1.000000, 0.500000)
  (1.250000, 0.250000)
  (1.000000, -0.000000)
}
\end{pspicture}
%
\hskip0.1in \dots \hskip0.1in
%
\begin{pspicture}(-0.5,-0.5)(1.5,1.5)
\psgrid
\rput(0.5,-0.25){$n=8$}
\rput(0,0.5){
\psline
  (0, 0)
  (0.062500, -0.000000)
  (0.062500, -0.062500)
  (0.062500, -0.125000)
  (0.000000, -0.125000)
  (0.000000, -0.187500)
  (-0.062500, -0.187500)
  (-0.125000, -0.187500)
  (-0.125000, -0.125000)
  (-0.125000, -0.187500)
  (-0.187500, -0.187500)
  (-0.250000, -0.187500)
  (-0.250000, -0.125000)
  (-0.312500, -0.125000)
  (-0.312500, -0.062500)
  (-0.312500, 0.000000)
  (-0.250000, 0.000000)
  (-0.250000, -0.062500)
  (-0.312500, -0.062500)
  (-0.375000, -0.062500)
  (-0.375000, 0.000000)
  (-0.437500, 0.000000)
  (-0.437500, 0.062500)
  (-0.437500, 0.125000)
  (-0.375000, 0.125000)
  (-0.437500, 0.125000)
  (-0.437500, 0.187500)
  (-0.437500, 0.250000)
  (-0.375000, 0.250000)
  (-0.375000, 0.312500)
  (-0.312500, 0.312500)
  (-0.250000, 0.312500)
  (-0.250000, 0.250000)
  (-0.250000, 0.187500)
  (-0.312500, 0.187500)
  (-0.375000, 0.187500)
  (-0.375000, 0.250000)
  (-0.437500, 0.250000)
  (-0.437500, 0.312500)
  (-0.437500, 0.375000)
  (-0.375000, 0.375000)
  (-0.437500, 0.375000)
  (-0.437500, 0.437500)
  (-0.437500, 0.500000)
  (-0.375000, 0.500000)
  (-0.375000, 0.562500)
  (-0.312500, 0.562500)
  (-0.250000, 0.562500)
  (-0.250000, 0.500000)
  (-0.312500, 0.500000)
  (-0.312500, 0.562500)
  (-0.312499, 0.625000)
  (-0.250000, 0.625000)
  (-0.249999, 0.687500)
  (-0.187500, 0.687500)
  (-0.125000, 0.687500)
  (-0.125000, 0.625000)
  (-0.125000, 0.687500)
  (-0.062500, 0.687500)
  (0.000000, 0.687500)
  (0.000000, 0.625000)
  (0.062500, 0.625000)
  (0.062500, 0.562500)
  (0.062500, 0.500000)
  (0.000000, 0.500000)
  (0.000000, 0.437500)
  (-0.062500, 0.437500)
  (-0.125000, 0.437500)
  (-0.125000, 0.500000)
  (-0.187500, 0.500000)
  (-0.187500, 0.562500)
  (-0.187500, 0.625000)
  (-0.125000, 0.625000)
  (-0.187500, 0.625000)
  (-0.187500, 0.687500)
  (-0.187499, 0.750000)
  (-0.125000, 0.750000)
  (-0.124999, 0.812500)
  (-0.062499, 0.812500)
  (0.000001, 0.812500)
  (0.000000, 0.750000)
  (-0.062500, 0.750000)
  (-0.062499, 0.812500)
  (-0.062499, 0.875000)
  (0.000001, 0.875000)
  (0.000001, 0.937500)
  (0.062501, 0.937500)
  (0.125001, 0.937500)
  (0.125000, 0.875000)
  (0.125001, 0.937500)
  (0.187500, 0.937500)
  (0.250000, 0.937500)
  (0.250000, 0.875000)
  (0.312500, 0.875000)
  (0.312500, 0.812500)
  (0.312500, 0.750000)
  (0.250000, 0.750000)
  (0.187500, 0.750000)
  (0.187500, 0.812500)
  (0.187500, 0.875000)
  (0.250000, 0.875000)
  (0.250000, 0.937500)
  (0.312500, 0.937500)
  (0.375000, 0.937500)
  (0.375000, 0.875000)
  (0.375000, 0.937500)
  (0.437500, 0.937500)
  (0.500000, 0.937500)
  (0.500000, 0.875000)
  (0.562500, 0.875000)
  (0.562500, 0.812500)
  (0.562500, 0.750000)
  (0.500000, 0.750000)
  (0.500000, 0.812500)
  (0.562500, 0.812500)
  (0.625000, 0.812500)
  (0.625000, 0.750000)
  (0.687500, 0.750000)
  (0.687500, 0.687500)
  (0.687500, 0.625000)
  (0.625000, 0.625000)
  (0.687500, 0.625000)
  (0.687500, 0.562500)
  (0.687500, 0.500000)
  (0.625000, 0.500000)
  (0.625000, 0.437500)
  (0.562500, 0.437500)
  (0.500000, 0.437500)
  (0.500000, 0.500000)
  (0.500000, 0.437500)
  (0.437500, 0.437500)
  (0.375000, 0.437500)
  (0.375001, 0.500000)
  (0.312501, 0.500000)
  (0.312501, 0.562500)
  (0.312501, 0.625000)
  (0.375001, 0.625000)
  (0.312501, 0.625000)
  (0.312501, 0.687500)
  (0.312501, 0.750000)
  (0.375001, 0.750000)
  (0.375001, 0.812500)
  (0.437501, 0.812500)
  (0.500001, 0.812500)
  (0.500001, 0.750000)
  (0.437501, 0.750000)
  (0.437501, 0.812500)
  (0.437501, 0.875000)
  (0.500001, 0.875000)
  (0.500001, 0.937500)
  (0.562501, 0.937500)
  (0.625001, 0.937500)
  (0.625001, 0.875000)
  (0.625001, 0.937500)
  (0.687501, 0.937500)
  (0.750001, 0.937500)
  (0.750001, 0.875000)
  (0.812501, 0.875000)
  (0.812501, 0.812500)
  (0.812501, 0.750000)
  (0.750001, 0.750000)
  (0.687501, 0.750000)
  (0.687501, 0.812500)
  (0.687501, 0.875000)
  (0.750001, 0.875000)
  (0.750001, 0.937500)
  (0.812501, 0.937500)
  (0.875001, 0.937500)
  (0.875001, 0.875000)
  (0.875001, 0.937500)
  (0.937501, 0.937500)
  (1.000001, 0.937500)
  (1.000001, 0.875000)
  (1.062501, 0.875000)
  (1.062501, 0.812500)
  (1.062501, 0.750000)
  (1.000001, 0.750000)
  (1.000001, 0.812500)
  (1.062501, 0.812500)
  (1.125001, 0.812500)
  (1.125001, 0.750000)
  (1.187501, 0.750000)
  (1.187501, 0.687500)
  (1.187501, 0.625000)
  (1.125001, 0.625000)
  (1.187501, 0.625000)
  (1.187501, 0.562500)
  (1.187501, 0.500000)
  (1.125001, 0.500000)
  (1.125001, 0.437500)
  (1.062501, 0.437501)
  (1.000001, 0.437501)
  (1.000001, 0.500001)
  (0.937501, 0.500001)
  (0.937501, 0.562501)
  (0.937501, 0.625001)
  (1.000001, 0.625001)
  (1.000001, 0.687501)
  (1.062501, 0.687501)
  (1.125001, 0.687501)
  (1.125001, 0.625001)
  (1.125001, 0.687501)
  (1.187501, 0.687501)
  (1.250001, 0.687501)
  (1.250001, 0.625001)
  (1.312501, 0.625001)
  (1.312501, 0.562501)
  (1.312501, 0.500001)
  (1.250001, 0.500001)
  (1.250001, 0.562501)
  (1.312501, 0.562501)
  (1.375001, 0.562501)
  (1.375001, 0.500001)
  (1.437501, 0.500001)
  (1.437501, 0.437501)
  (1.437501, 0.375001)
  (1.375001, 0.375001)
  (1.437501, 0.375001)
  (1.437501, 0.312501)
  (1.437501, 0.250001)
  (1.375001, 0.250001)
  (1.375001, 0.187501)
  (1.312501, 0.187501)
  (1.250001, 0.187501)
  (1.250001, 0.250001)
  (1.250001, 0.312501)
  (1.312501, 0.312501)
  (1.375001, 0.312501)
  (1.375001, 0.250001)
  (1.437501, 0.250001)
  (1.437501, 0.187501)
  (1.437501, 0.125001)
  (1.375001, 0.125001)
  (1.437501, 0.125001)
  (1.437501, 0.062501)
  (1.437501, 0.000001)
  (1.375001, 0.000001)
  (1.375001, -0.062499)
  (1.312501, -0.062499)
  (1.250001, -0.062499)
  (1.250001, 0.000001)
  (1.312501, 0.000001)
  (1.312501, -0.062499)
  (1.312501, -0.124999)
  (1.250001, -0.124999)
  (1.250001, -0.187499)
  (1.187501, -0.187499)
  (1.125001, -0.187499)
  (1.125001, -0.124999)
  (1.125001, -0.187499)
  (1.062501, -0.187499)
  (1.000001, -0.187499)
  (1.000001, -0.124999)
  (0.937501, -0.124999)
  (0.937501, -0.062499)
  (0.937501, 0.000001)
  (1.000001, 0.000001)
}
\end{pspicture}
%
\hskip0.1in \dots
%

\end{zztask}

%%%%%%%%%%%%%%%%%%%%%%%%%%%%%%%%%%%%%%%%%%%%%%%%%%%%%%%%%%%%%%%%%%%%%%%%%%%%%%

\begin{zztask}[Sierpi\'nski Arrowhead Curve]
В рамках общего условия задачи построить стреловидную кривую Серпинского.
База для построения кривой (нулевое приближение) представляет собой отрезок.
Переход от приближения $n$ к приближению $(n+1)$ осуществляется заменой каждого 
отрезка фигуры на три (см. рис. при $n=1$). 
Выгиб происходит чередуясь, то в одну, то в другую сторону (см. рис. при $n=2$).
Все отрезки 
имеют одинаковую длину, угол между отрезками составляет $120^\circ$. Вся фигура
как бы ``вписывается'' в равносторонний треугольник (заполняя треугольник 
Серпинского).
\par\endinput
\begin{pspicture}(-0.5,-0.5)(1.5,1.5)
\psgrid
\rput(0.5,-0.25){$n=0$}
\rput(0,0){
\psline[linewidth=0.06cm, arrows=o-o]
  (0, 0)
  (1.000000, 0.000000)
}
\end{pspicture}
%
\hskip0.1in
%
\begin{pspicture}(-0.5,-0.5)(1.5,1.5)
\psgrid
\rput(0.5,-0.25){$n=1$}
\rput(0,0){
\psline[linewidth=0.01cm, linestyle=dashed]
  (0, 0)
  (1.000000, 0.000000)
\psline[linewidth=0.06cm, arrows=o-o, showpoints=true]
  (0, 0)
  (0.250000, 0.433013)
  (0.750000, 0.433013)
  (1.000000, 0.000000)
}
\end{pspicture}
%
\hskip0.1in
%
\begin{pspicture}(-0.5,-0.5)(1.5,1.5)
\psgrid
\rput(0.5,-0.25){$n=2$}
\rput(0,0){
\psline[linewidth=0.01cm, linestyle=dashed]
  (0, 0)
  (0.250000, 0.433013)
  (0.750000, 0.433013)
  (1.000000, 0.000000)
\psline[linewidth=0.06cm, arrows=o-, showpoints=true]
  (0, 0)
  (0.250000, 0.000000)
  (0.375000, 0.216506)
  (0.250000, 0.433013)
\psline[linewidth=0.06cm, arrows=o-, showpoints=true]
  (0.250000, 0.433013)
  (0.375000, 0.649519)
  (0.625000, 0.649519)
  (0.750000, 0.433013)
\psline[linewidth=0.06cm, arrows=o-o, showpoints=true]
  (0.750000, 0.433013)
  (0.625000, 0.216506)
  (0.750000, -0.000000)
  (1.000000, -0.000000)
}
\end{pspicture}
%
\hskip0.1in
%
\begin{pspicture}(-0.5,-0.5)(1.5,1.5)
\psgrid
\rput(0.5,-0.25){$n=3$}
\rput(0,0){
\psline
  (0, 0)
  (0.062500, 0.108253)
  (0.187500, 0.108253)
  (0.250000, 0.000000)
  (0.375000, 0.000000)
  (0.437500, 0.108253)
  (0.375000, 0.216506)
  (0.250000, 0.216506)
  (0.187500, 0.324760)
  (0.250000, 0.433013)
  (0.375000, 0.433013)
  (0.437500, 0.541266)
  (0.375000, 0.649519)
  (0.437500, 0.757772)
  (0.562500, 0.757772)
  (0.625000, 0.649519)
  (0.562500, 0.541266)
  (0.625000, 0.433013)
  (0.750000, 0.433013)
  (0.812500, 0.324759)
  (0.750000, 0.216506)
  (0.625000, 0.216506)
  (0.562500, 0.108253)
  (0.625000, -0.000000)
  (0.750000, -0.000000)
  (0.812500, 0.108253)
  (0.937500, 0.108253)
  (1.000000, -0.000000)
}
\end{pspicture}
%
\hskip0.1in\dots\hskip0.1in
%
\begin{pspicture}(-0.5,-0.5)(1.5,1.5)
\psgrid
\rput(0.5,-0.25){$n=6$}
\rput(0,0){
\psline
  (0, 0)
  (0.015625, 0.000000)
  (0.023438, 0.013532)
  (0.015625, 0.027063)
  (0.023437, 0.040595)
  (0.039063, 0.040595)
  (0.046875, 0.027063)
  (0.039063, 0.013532)
  (0.046875, -0.000000)
  (0.062500, -0.000000)
  (0.070313, 0.013532)
  (0.085938, 0.013532)
  (0.093750, -0.000000)
  (0.109375, -0.000000)
  (0.117188, 0.013532)
  (0.109375, 0.027063)
  (0.093750, 0.027063)
  (0.085938, 0.040595)
  (0.093750, 0.054127)
  (0.085938, 0.067658)
  (0.070313, 0.067658)
  (0.062500, 0.054127)
  (0.046875, 0.054127)
  (0.039063, 0.067658)
  (0.046875, 0.081190)
  (0.062500, 0.081190)
  (0.070313, 0.094722)
  (0.062500, 0.108253)
  (0.070313, 0.121785)
  (0.085938, 0.121785)
  (0.093750, 0.108253)
  (0.109375, 0.108253)
  (0.117188, 0.121785)
  (0.109375, 0.135316)
  (0.093750, 0.135316)
  (0.085938, 0.148848)
  (0.093750, 0.162380)
  (0.109375, 0.162380)
  (0.117188, 0.175911)
  (0.109375, 0.189443)
  (0.117188, 0.202975)
  (0.132813, 0.202975)
  (0.140625, 0.189443)
  (0.132813, 0.175911)
  (0.140625, 0.162380)
  (0.156250, 0.162380)
  (0.164063, 0.148848)
  (0.156250, 0.135316)
  (0.140625, 0.135316)
  (0.132813, 0.121785)
  (0.140625, 0.108253)
  (0.156250, 0.108253)
  (0.164063, 0.121785)
  (0.179688, 0.121785)
  (0.187500, 0.108253)
  (0.179688, 0.094722)
  (0.187500, 0.081190)
  (0.203125, 0.081190)
  (0.210938, 0.067658)
  (0.203125, 0.054127)
  (0.187500, 0.054127)
  (0.179688, 0.067658)
  (0.164063, 0.067658)
  (0.156250, 0.054127)
  (0.164063, 0.040595)
  (0.156250, 0.027063)
  (0.140625, 0.027063)
  (0.132813, 0.013532)
  (0.140625, -0.000000)
  (0.156250, -0.000000)
  (0.164063, 0.013532)
  (0.179688, 0.013532)
  (0.187500, -0.000000)
  (0.203125, -0.000000)
  (0.210938, 0.013532)
  (0.203125, 0.027063)
  (0.210938, 0.040595)
  (0.226563, 0.040595)
  (0.234375, 0.027063)
  (0.226563, 0.013532)
  (0.234375, -0.000000)
  (0.250000, -0.000000)
  (0.257813, 0.013532)
  (0.273438, 0.013532)
  (0.281250, -0.000000)
  (0.296875, -0.000000)
  (0.304688, 0.013532)
  (0.296875, 0.027063)
  (0.281250, 0.027063)
  (0.273438, 0.040595)
  (0.281250, 0.054127)
  (0.296875, 0.054127)
  (0.304688, 0.067658)
  (0.296875, 0.081190)
  (0.304688, 0.094722)
  (0.320313, 0.094722)
  (0.328125, 0.081190)
  (0.320313, 0.067658)
  (0.328125, 0.054127)
  (0.343750, 0.054127)
  (0.351563, 0.040595)
  (0.343750, 0.027063)
  (0.328125, 0.027063)
  (0.320313, 0.013532)
  (0.328125, -0.000000)
  (0.343750, -0.000000)
  (0.351563, 0.013532)
  (0.367188, 0.013532)
  (0.375000, -0.000000)
  (0.390625, -0.000000)
  (0.398438, 0.013532)
  (0.390625, 0.027063)
  (0.398438, 0.040595)
  (0.414063, 0.040595)
  (0.421875, 0.027063)
  (0.414063, 0.013532)
  (0.421875, -0.000000)
  (0.437500, -0.000000)
  (0.445313, 0.013532)
  (0.460938, 0.013532)
  (0.468750, -0.000000)
  (0.484375, -0.000000)
  (0.492188, 0.013532)
  (0.484375, 0.027063)
  (0.468750, 0.027063)
  (0.460938, 0.040595)
  (0.468750, 0.054127)
  (0.460938, 0.067658)
  (0.445313, 0.067658)
  (0.437500, 0.054127)
  (0.421875, 0.054127)
  (0.414063, 0.067658)
  (0.421875, 0.081190)
  (0.437500, 0.081190)
  (0.445313, 0.094722)
  (0.437500, 0.108253)
  (0.421875, 0.108253)
  (0.414063, 0.121785)
  (0.421875, 0.135316)
  (0.414063, 0.148848)
  (0.398438, 0.148848)
  (0.390625, 0.135316)
  (0.398438, 0.121785)
  (0.390625, 0.108253)
  (0.375000, 0.108253)
  (0.367188, 0.121785)
  (0.351563, 0.121785)
  (0.343750, 0.108253)
  (0.328125, 0.108253)
  (0.320313, 0.121785)
  (0.328125, 0.135316)
  (0.343750, 0.135316)
  (0.351563, 0.148848)
  (0.343750, 0.162380)
  (0.351563, 0.175911)
  (0.367188, 0.175911)
  (0.375000, 0.162380)
  (0.390625, 0.162380)
  (0.398438, 0.175911)
  (0.390625, 0.189443)
  (0.375000, 0.189443)
  (0.367188, 0.202975)
  (0.375000, 0.216506)
  (0.367188, 0.230038)
  (0.351563, 0.230038)
  (0.343750, 0.216506)
  (0.328125, 0.216506)
  (0.320313, 0.230038)
  (0.328125, 0.243570)
  (0.343750, 0.243570)
  (0.351563, 0.257101)
  (0.343750, 0.270633)
  (0.328125, 0.270633)
  (0.320313, 0.284165)
  (0.328125, 0.297696)
  (0.320313, 0.311228)
  (0.304688, 0.311228)
  (0.296875, 0.297696)
  (0.304688, 0.284165)
  (0.296875, 0.270633)
  (0.281250, 0.270633)
  (0.273438, 0.257101)
  (0.281250, 0.243570)
  (0.296875, 0.243570)
  (0.304688, 0.230038)
  (0.296875, 0.216506)
  (0.281250, 0.216506)
  (0.273438, 0.230038)
  (0.257813, 0.230038)
  (0.250000, 0.216506)
  (0.234375, 0.216506)
  (0.226563, 0.230038)
  (0.234375, 0.243570)
  (0.226563, 0.257101)
  (0.210938, 0.257101)
  (0.203125, 0.243570)
  (0.210938, 0.230038)
  (0.203125, 0.216506)
  (0.187500, 0.216506)
  (0.179688, 0.230038)
  (0.164063, 0.230038)
  (0.156250, 0.216506)
  (0.140625, 0.216506)
  (0.132813, 0.230038)
  (0.140625, 0.243569)
  (0.156250, 0.243569)
  (0.164063, 0.257101)
  (0.156250, 0.270633)
  (0.164063, 0.284164)
  (0.179688, 0.284164)
  (0.187500, 0.270633)
  (0.203125, 0.270633)
  (0.210938, 0.284164)
  (0.203125, 0.297696)
  (0.187500, 0.297696)
  (0.179688, 0.311228)
  (0.187500, 0.324759)
  (0.203125, 0.324759)
  (0.210938, 0.338291)
  (0.203125, 0.351823)
  (0.210938, 0.365354)
  (0.226563, 0.365354)
  (0.234375, 0.351823)
  (0.226563, 0.338291)
  (0.234375, 0.324759)
  (0.250000, 0.324759)
  (0.257813, 0.338291)
  (0.273438, 0.338291)
  (0.281250, 0.324759)
  (0.296875, 0.324759)
  (0.304688, 0.338291)
  (0.296875, 0.351823)
  (0.281250, 0.351823)
  (0.273438, 0.365354)
  (0.281250, 0.378886)
  (0.273438, 0.392418)
  (0.257813, 0.392418)
  (0.250000, 0.378886)
  (0.234375, 0.378886)
  (0.226563, 0.392418)
  (0.234375, 0.405949)
  (0.250000, 0.405949)
  (0.257813, 0.419481)
  (0.250000, 0.433013)
  (0.257813, 0.446544)
  (0.273438, 0.446544)
  (0.281250, 0.433013)
  (0.296875, 0.433013)
  (0.304688, 0.446544)
  (0.296875, 0.460076)
  (0.281250, 0.460076)
  (0.273438, 0.473608)
  (0.281250, 0.487139)
  (0.296875, 0.487139)
  (0.304688, 0.500671)
  (0.296875, 0.514202)
  (0.304688, 0.527734)
  (0.320313, 0.527734)
  (0.328125, 0.514202)
  (0.320313, 0.500671)
  (0.328125, 0.487139)
  (0.343750, 0.487139)
  (0.351563, 0.473608)
  (0.343750, 0.460076)
  (0.328125, 0.460076)
  (0.320313, 0.446544)
  (0.328125, 0.433013)
  (0.343750, 0.433013)
  (0.351563, 0.446544)
  (0.367188, 0.446544)
  (0.375000, 0.433013)
  (0.390625, 0.433013)
  (0.398438, 0.446544)
  (0.390625, 0.460076)
  (0.398438, 0.473608)
  (0.414063, 0.473608)
  (0.421875, 0.460076)
  (0.414063, 0.446544)
  (0.421875, 0.433013)
  (0.437500, 0.433013)
  (0.445313, 0.446544)
  (0.460938, 0.446544)
  (0.468750, 0.433013)
  (0.484375, 0.433013)
  (0.492188, 0.446544)
  (0.484375, 0.460076)
  (0.468750, 0.460076)
  (0.460938, 0.473608)
  (0.468750, 0.487139)
  (0.460938, 0.500671)
  (0.445313, 0.500671)
  (0.437500, 0.487139)
  (0.421875, 0.487139)
  (0.414063, 0.500671)
  (0.421875, 0.514202)
  (0.437500, 0.514202)
  (0.445313, 0.527734)
  (0.437500, 0.541266)
  (0.421875, 0.541266)
  (0.414063, 0.554797)
  (0.421875, 0.568329)
  (0.414063, 0.581861)
  (0.398438, 0.581861)
  (0.390625, 0.568329)
  (0.398438, 0.554797)
  (0.390625, 0.541266)
  (0.375000, 0.541266)
  (0.367188, 0.554797)
  (0.351563, 0.554797)
  (0.343750, 0.541266)
  (0.328125, 0.541266)
  (0.320313, 0.554797)
  (0.328125, 0.568329)
  (0.343750, 0.568329)
  (0.351563, 0.581861)
  (0.343750, 0.595392)
  (0.351563, 0.608924)
  (0.367188, 0.608924)
  (0.375000, 0.595392)
  (0.390625, 0.595392)
  (0.398438, 0.608924)
  (0.390625, 0.622455)
  (0.375000, 0.622455)
  (0.367188, 0.635987)
  (0.375000, 0.649519)
  (0.390625, 0.649519)
  (0.398438, 0.663050)
  (0.390625, 0.676582)
  (0.398438, 0.690114)
  (0.414063, 0.690114)
  (0.421875, 0.676582)
  (0.414063, 0.663050)
  (0.421875, 0.649519)
  (0.437500, 0.649519)
  (0.445313, 0.663050)
  (0.460938, 0.663050)
  (0.468750, 0.649519)
  (0.484375, 0.649519)
  (0.492188, 0.663050)
  (0.484375, 0.676582)
  (0.468750, 0.676582)
  (0.460938, 0.690114)
  (0.468750, 0.703645)
  (0.460938, 0.717177)
  (0.445313, 0.717177)
  (0.437500, 0.703645)
  (0.421875, 0.703645)
  (0.414063, 0.717177)
  (0.421875, 0.730709)
  (0.437500, 0.730708)
  (0.445313, 0.744240)
  (0.437500, 0.757772)
  (0.445313, 0.771303)
  (0.460938, 0.771303)
  (0.468750, 0.757772)
  (0.484375, 0.757772)
  (0.492188, 0.771303)
  (0.484375, 0.784835)
  (0.468750, 0.784835)
  (0.460938, 0.798367)
  (0.468750, 0.811898)
  (0.484375, 0.811898)
  (0.492188, 0.825430)
  (0.484375, 0.838962)
  (0.492188, 0.852493)
  (0.507813, 0.852493)
  (0.515625, 0.838961)
  (0.507813, 0.825430)
  (0.515625, 0.811898)
  (0.531250, 0.811898)
  (0.539063, 0.798367)
  (0.531250, 0.784835)
  (0.515625, 0.784835)
  (0.507813, 0.771303)
  (0.515625, 0.757772)
  (0.531250, 0.757772)
  (0.539063, 0.771303)
  (0.554688, 0.771303)
  (0.562500, 0.757772)
  (0.554688, 0.744240)
  (0.562500, 0.730708)
  (0.578125, 0.730708)
  (0.585938, 0.717177)
  (0.578125, 0.703645)
  (0.562500, 0.703645)
  (0.554688, 0.717177)
  (0.539063, 0.717177)
  (0.531250, 0.703645)
  (0.539063, 0.690114)
  (0.531250, 0.676582)
  (0.515625, 0.676582)
  (0.507813, 0.663050)
  (0.515625, 0.649519)
  (0.531250, 0.649519)
  (0.539063, 0.663050)
  (0.554688, 0.663050)
  (0.562500, 0.649519)
  (0.578125, 0.649519)
  (0.585938, 0.663050)
  (0.578125, 0.676582)
  (0.585938, 0.690114)
  (0.601563, 0.690114)
  (0.609375, 0.676582)
  (0.601563, 0.663050)
  (0.609375, 0.649519)
  (0.625000, 0.649519)
  (0.632813, 0.635987)
  (0.625000, 0.622455)
  (0.609375, 0.622455)
  (0.601563, 0.608924)
  (0.609375, 0.595392)
  (0.625000, 0.595392)
  (0.632813, 0.608924)
  (0.648438, 0.608924)
  (0.656250, 0.595392)
  (0.648438, 0.581861)
  (0.656250, 0.568329)
  (0.671875, 0.568329)
  (0.679688, 0.554797)
  (0.671875, 0.541266)
  (0.656250, 0.541266)
  (0.648438, 0.554797)
  (0.632813, 0.554797)
  (0.625000, 0.541266)
  (0.609375, 0.541266)
  (0.601563, 0.554797)
  (0.609375, 0.568329)
  (0.601563, 0.581861)
  (0.585938, 0.581861)
  (0.578125, 0.568329)
  (0.585938, 0.554797)
  (0.578125, 0.541266)
  (0.562500, 0.541266)
  (0.554688, 0.527734)
  (0.562500, 0.514202)
  (0.578125, 0.514202)
  (0.585938, 0.500671)
  (0.578125, 0.487139)
  (0.562500, 0.487139)
  (0.554688, 0.500671)
  (0.539063, 0.500671)
  (0.531250, 0.487139)
  (0.539063, 0.473608)
  (0.531250, 0.460076)
  (0.515625, 0.460076)
  (0.507813, 0.446544)
  (0.515625, 0.433013)
  (0.531250, 0.433013)
  (0.539063, 0.446544)
  (0.554688, 0.446544)
  (0.562500, 0.433013)
  (0.578125, 0.433013)
  (0.585938, 0.446544)
  (0.578125, 0.460076)
  (0.585938, 0.473608)
  (0.601563, 0.473608)
  (0.609375, 0.460076)
  (0.601563, 0.446544)
  (0.609375, 0.433013)
  (0.625000, 0.433013)
  (0.632813, 0.446544)
  (0.648438, 0.446544)
  (0.656250, 0.433013)
  (0.671875, 0.433013)
  (0.679688, 0.446544)
  (0.671875, 0.460076)
  (0.656250, 0.460076)
  (0.648438, 0.473608)
  (0.656250, 0.487139)
  (0.671875, 0.487139)
  (0.679688, 0.500671)
  (0.671875, 0.514202)
  (0.679688, 0.527734)
  (0.695313, 0.527734)
  (0.703125, 0.514202)
  (0.695313, 0.500671)
  (0.703125, 0.487139)
  (0.718750, 0.487139)
  (0.726563, 0.473608)
  (0.718750, 0.460076)
  (0.703125, 0.460076)
  (0.695313, 0.446544)
  (0.703125, 0.433013)
  (0.718750, 0.433013)
  (0.726563, 0.446544)
  (0.742188, 0.446544)
  (0.750000, 0.433013)
  (0.742188, 0.419481)
  (0.750000, 0.405949)
  (0.765625, 0.405949)
  (0.773438, 0.392418)
  (0.765625, 0.378886)
  (0.750000, 0.378886)
  (0.742188, 0.392418)
  (0.726563, 0.392418)
  (0.718750, 0.378886)
  (0.726563, 0.365354)
  (0.718750, 0.351823)
  (0.703125, 0.351823)
  (0.695313, 0.338291)
  (0.703125, 0.324759)
  (0.718750, 0.324759)
  (0.726563, 0.338291)
  (0.742188, 0.338291)
  (0.750000, 0.324759)
  (0.765625, 0.324759)
  (0.773438, 0.338291)
  (0.765625, 0.351823)
  (0.773438, 0.365354)
  (0.789063, 0.365354)
  (0.796875, 0.351823)
  (0.789063, 0.338291)
  (0.796875, 0.324759)
  (0.812500, 0.324759)
  (0.820313, 0.311228)
  (0.812500, 0.297696)
  (0.796875, 0.297696)
  (0.789063, 0.284164)
  (0.796875, 0.270633)
  (0.812500, 0.270633)
  (0.820313, 0.284164)
  (0.835938, 0.284164)
  (0.843750, 0.270633)
  (0.835938, 0.257101)
  (0.843750, 0.243569)
  (0.859375, 0.243569)
  (0.867188, 0.230038)
  (0.859375, 0.216506)
  (0.843750, 0.216506)
  (0.835938, 0.230038)
  (0.820313, 0.230038)
  (0.812500, 0.216506)
  (0.796875, 0.216506)
  (0.789063, 0.230038)
  (0.796875, 0.243569)
  (0.789063, 0.257101)
  (0.773438, 0.257101)
  (0.765625, 0.243569)
  (0.773438, 0.230038)
  (0.765625, 0.216506)
  (0.750000, 0.216506)
  (0.742188, 0.230038)
  (0.726563, 0.230038)
  (0.718750, 0.216506)
  (0.703125, 0.216506)
  (0.695313, 0.230038)
  (0.703125, 0.243570)
  (0.718750, 0.243570)
  (0.726563, 0.257101)
  (0.718750, 0.270633)
  (0.703125, 0.270633)
  (0.695313, 0.284164)
  (0.703125, 0.297696)
  (0.695313, 0.311228)
  (0.679688, 0.311228)
  (0.671875, 0.297696)
  (0.679688, 0.284164)
  (0.671875, 0.270633)
  (0.656250, 0.270633)
  (0.648438, 0.257101)
  (0.656250, 0.243570)
  (0.671875, 0.243570)
  (0.679688, 0.230038)
  (0.671875, 0.216506)
  (0.656250, 0.216506)
  (0.648438, 0.230038)
  (0.632813, 0.230038)
  (0.625000, 0.216506)
  (0.632813, 0.202975)
  (0.625000, 0.189443)
  (0.609375, 0.189443)
  (0.601563, 0.175911)
  (0.609375, 0.162380)
  (0.625000, 0.162380)
  (0.632813, 0.175911)
  (0.648438, 0.175911)
  (0.656250, 0.162380)
  (0.648438, 0.148848)
  (0.656250, 0.135316)
  (0.671875, 0.135316)
  (0.679688, 0.121785)
  (0.671875, 0.108253)
  (0.656250, 0.108253)
  (0.648438, 0.121785)
  (0.632813, 0.121785)
  (0.625000, 0.108253)
  (0.609375, 0.108253)
  (0.601563, 0.121785)
  (0.609375, 0.135316)
  (0.601563, 0.148848)
  (0.585938, 0.148848)
  (0.578125, 0.135316)
  (0.585938, 0.121785)
  (0.578125, 0.108253)
  (0.562500, 0.108253)
  (0.554688, 0.094722)
  (0.562500, 0.081190)
  (0.578125, 0.081190)
  (0.585938, 0.067658)
  (0.578125, 0.054127)
  (0.562500, 0.054127)
  (0.554688, 0.067658)
  (0.539063, 0.067658)
  (0.531250, 0.054127)
  (0.539063, 0.040595)
  (0.531250, 0.027063)
  (0.515625, 0.027063)
  (0.507813, 0.013532)
  (0.515625, -0.000000)
  (0.531250, -0.000000)
  (0.539063, 0.013532)
  (0.554688, 0.013532)
  (0.562500, -0.000000)
  (0.578125, -0.000000)
  (0.585938, 0.013532)
  (0.578125, 0.027063)
  (0.585938, 0.040595)
  (0.601563, 0.040595)
  (0.609375, 0.027063)
  (0.601563, 0.013532)
  (0.609375, -0.000000)
  (0.625000, -0.000000)
  (0.632813, 0.013532)
  (0.648438, 0.013532)
  (0.656250, -0.000000)
  (0.671875, -0.000000)
  (0.679688, 0.013532)
  (0.671875, 0.027063)
  (0.656250, 0.027063)
  (0.648438, 0.040595)
  (0.656250, 0.054127)
  (0.671875, 0.054127)
  (0.679688, 0.067658)
  (0.671875, 0.081190)
  (0.679688, 0.094722)
  (0.695313, 0.094722)
  (0.703125, 0.081190)
  (0.695313, 0.067658)
  (0.703125, 0.054127)
  (0.718750, 0.054127)
  (0.726563, 0.040595)
  (0.718750, 0.027063)
  (0.703125, 0.027063)
  (0.695313, 0.013532)
  (0.703125, -0.000000)
  (0.718750, -0.000000)
  (0.726563, 0.013532)
  (0.742188, 0.013532)
  (0.750000, -0.000000)
  (0.765625, -0.000000)
  (0.773438, 0.013532)
  (0.765625, 0.027063)
  (0.773438, 0.040595)
  (0.789063, 0.040595)
  (0.796875, 0.027063)
  (0.789063, 0.013532)
  (0.796875, -0.000000)
  (0.812500, -0.000000)
  (0.820313, 0.013532)
  (0.835938, 0.013532)
  (0.843750, -0.000000)
  (0.859375, -0.000000)
  (0.867188, 0.013532)
  (0.859375, 0.027063)
  (0.843750, 0.027063)
  (0.835938, 0.040595)
  (0.843750, 0.054127)
  (0.835938, 0.067658)
  (0.820313, 0.067658)
  (0.812500, 0.054127)
  (0.796875, 0.054127)
  (0.789063, 0.067658)
  (0.796875, 0.081190)
  (0.812500, 0.081190)
  (0.820313, 0.094722)
  (0.812500, 0.108253)
  (0.820313, 0.121785)
  (0.835938, 0.121785)
  (0.843750, 0.108253)
  (0.859375, 0.108253)
  (0.867188, 0.121785)
  (0.859375, 0.135316)
  (0.843750, 0.135316)
  (0.835938, 0.148848)
  (0.843750, 0.162380)
  (0.859375, 0.162380)
  (0.867188, 0.175911)
  (0.859375, 0.189443)
  (0.867188, 0.202975)
  (0.882813, 0.202975)
  (0.890625, 0.189443)
  (0.882813, 0.175911)
  (0.890625, 0.162380)
  (0.906250, 0.162380)
  (0.914063, 0.148848)
  (0.906250, 0.135316)
  (0.890625, 0.135316)
  (0.882813, 0.121785)
  (0.890625, 0.108253)
  (0.906250, 0.108253)
  (0.914063, 0.121785)
  (0.929688, 0.121785)
  (0.937500, 0.108253)
  (0.929688, 0.094722)
  (0.937500, 0.081190)
  (0.953125, 0.081190)
  (0.960938, 0.067658)
  (0.953125, 0.054127)
  (0.937500, 0.054127)
  (0.929688, 0.067658)
  (0.914063, 0.067658)
  (0.906250, 0.054127)
  (0.914063, 0.040595)
  (0.906250, 0.027063)
  (0.890625, 0.027063)
  (0.882813, 0.013532)
  (0.890625, -0.000000)
  (0.906250, -0.000000)
  (0.914063, 0.013532)
  (0.929688, 0.013532)
  (0.937500, -0.000000)
  (0.953125, -0.000000)
  (0.960938, 0.013532)
  (0.953125, 0.027063)
  (0.960938, 0.040595)
  (0.976563, 0.040595)
  (0.984375, 0.027063)
  (0.976563, 0.013532)
  (0.984375, -0.000000)
  (1.000000, -0.000000)
}
\end{pspicture}
%
\hskip0.1in\dots
%

\end{zztask}

%%%%%%%%%%%%%%%%%%%%%%%%%%%%%%%%%%%%%%%%%%%%%%%%%%%%%%%%%%%%%%%%%%%%%%%%%%%%%%

\begin{zztask}[Minkowski Sausage]
В рамках общего условия задачи построить колбасу Минковского.
База для построения кривой (нулевое приближение) представляет собой отрезок.
Переход от приближения $n$ к приближению $(n+1)$ осуществляется заменой каждого 
отрезка фигуры на восемь (см. рис. при $n=1$). 
Все отрезки имеют одинаковую длину, угол между отрезками составляет $90^\circ$.
\par\input{labs/cur/cur-pic-mink}
\end{zztask}

%%%%%%%%%%%%%%%%%%%%%%%%%%%%%%%%%%%%%%%%%%%%%%%%%%%%%%%%%%%%%%%%%%%%%%%%%%%%%%

\begin{zztask}[Quadratic Koch Island]
В рамках общего условия задачи построить остров Коха. База для построения кривой 
(нулевое приближение) представляет собой квадрат.
Переход от приближения $n$ к приближению $(n+1)$ осуществляется заменой каждого 
отрезка фигуры на три (см. рис. при $n=1$). 
Все отрезки имеют одинаковую длину, угол между отрезками составляет $90^\circ$.
\par\endinput
\begin{pspicture}(-0.5,-0.5)(1.5,1.5)
\psgrid
\rput(0.0,-0.25){$n=0$}
\rput(0,0){
\psline
  (-0.000000, 1.000000)
  (1.000000, 1.000000)
  (1.000000, 0.000000)
  (-0.000000, 0.000000)
\psline[linewidth=0.06cm, arrows=o-o]
  (0, 0)
  (-0.000000, 1.000000)
}
\end{pspicture}
%
\hskip0.1in
%
\begin{pspicture}(-0.5,-0.5)(1.5,1.5)
\psgrid
\rput(0.0,-0.25){$n=1$}
\rput(0,0){
\psline
  (0.000000, 1.000000)
  (0.400000, 1.200000)
  (0.600000, 0.800000)
  (1.000000, 1.000000)
  (1.200000, 0.600000)
  (0.800000, 0.400000)
  (1.000000, -0.000000)
  (0.600000, -0.200000)
  (0.400000, 0.200000)
  (0.000000, -0.000000)
\psline[linewidth=0.06cm, arrows=o-o, showpoints=true]
  (0, 0)
  (-0.200000, 0.400000)
  (0.200000, 0.600000)
  (0.000000, 1.000000)
}
\end{pspicture}
%
\hskip0.1in
%
\begin{pspicture}(-0.5,-0.5)(1.5,1.5)
\psgrid
\rput(0.0,-0.25){$n=2$}
\rput(0,0){
\psline
  (0, 0)
  (-0.160000, 0.120000)
  (-0.040000, 0.280000)
  (-0.200000, 0.400000)
  (-0.080000, 0.560000)
  (0.080000, 0.440000)
  (0.200000, 0.600000)
  (0.040000, 0.720000)
  (0.160000, 0.880000)
  (-0.000000, 1.000000)
  (0.120000, 1.160000)
  (0.280000, 1.040000)
  (0.400000, 1.200000)
  (0.560000, 1.080000)
  (0.440000, 0.920000)
  (0.600000, 0.800000)
  (0.720000, 0.960000)
  (0.880000, 0.840000)
  (1.000000, 1.000000)
  (1.160000, 0.880000)
  (1.040000, 0.720000)
  (1.200000, 0.600000)
  (1.080000, 0.440000)
  (0.920000, 0.560000)
  (0.800000, 0.400000)
  (0.960000, 0.280000)
  (0.840000, 0.120000)
  (1.000000, 0.000000)
  (0.880000, -0.160000)
  (0.720000, -0.040000)
  (0.600000, -0.200000)
  (0.440000, -0.080000)
  (0.560000, 0.080000)
  (0.400000, 0.200000)
  (0.280000, 0.040000)
  (0.120000, 0.160000)
  (0.000000, 0.000000)
}
\end{pspicture}
%
\hskip0.1in
%
\begin{pspicture}(-0.5,-0.5)(1.5,1.5)
\psgrid
\rput(0.0,-0.25){$n=3$}
\rput(0,0){
\psline
  (0, 0)
  (-0.088000, 0.016000)
  (-0.072000, 0.104000)
  (-0.160000, 0.120000)
  (-0.144000, 0.208000)
  (-0.056000, 0.192000)
  (-0.040000, 0.280000)
  (-0.128000, 0.296000)
  (-0.112000, 0.384000)
  (-0.200000, 0.400000)
  (-0.184000, 0.488000)
  (-0.096000, 0.472000)
  (-0.080000, 0.560000)
  (0.008000, 0.544000)
  (-0.008000, 0.456000)
  (0.080000, 0.440000)
  (0.096000, 0.528000)
  (0.184000, 0.512000)
  (0.200000, 0.600000)
  (0.112000, 0.616000)
  (0.128000, 0.704000)
  (0.040000, 0.720000)
  (0.056000, 0.808000)
  (0.144000, 0.792000)
  (0.160000, 0.880000)
  (0.072000, 0.896000)
  (0.088000, 0.984000)
  (0.000000, 1.000000)
  (0.016000, 1.088000)
  (0.104000, 1.072000)
  (0.120000, 1.160000)
  (0.208000, 1.144000)
  (0.192000, 1.056000)
  (0.280000, 1.040000)
  (0.296000, 1.128000)
  (0.384000, 1.112000)
  (0.400000, 1.200000)
  (0.488000, 1.184000)
  (0.472000, 1.096000)
  (0.560000, 1.080000)
  (0.544000, 0.992000)
  (0.456000, 1.008000)
  (0.440000, 0.920000)
  (0.528000, 0.904000)
  (0.512000, 0.816000)
  (0.600000, 0.800000)
  (0.616000, 0.888000)
  (0.704000, 0.872000)
  (0.720000, 0.960000)
  (0.808000, 0.944000)
  (0.792000, 0.856000)
  (0.880000, 0.840000)
  (0.896000, 0.927999)
  (0.984000, 0.911999)
  (1.000000, 0.999999)
  (1.088000, 0.983999)
  (1.072000, 0.895999)
  (1.160000, 0.879999)
  (1.144000, 0.791999)
  (1.056000, 0.807999)
  (1.040000, 0.719999)
  (1.128000, 0.703999)
  (1.112000, 0.615999)
  (1.200000, 0.599999)
  (1.184000, 0.511999)
  (1.096000, 0.527999)
  (1.080000, 0.439999)
  (0.992000, 0.455999)
  (1.008000, 0.543999)
  (0.920000, 0.559999)
  (0.904000, 0.471999)
  (0.816000, 0.487999)
  (0.800000, 0.400000)
  (0.888000, 0.383999)
  (0.872000, 0.296000)
  (0.960000, 0.279999)
  (0.944000, 0.192000)
  (0.856000, 0.208000)
  (0.840000, 0.120000)
  (0.928000, 0.104000)
  (0.912000, 0.016000)
  (1.000000, -0.000001)
  (0.984000, -0.088000)
  (0.896000, -0.072000)
  (0.880000, -0.160000)
  (0.792000, -0.144000)
  (0.808000, -0.056000)
  (0.720000, -0.040000)
  (0.704000, -0.128000)
  (0.616000, -0.112000)
  (0.599999, -0.200000)
  (0.511999, -0.184000)
  (0.528000, -0.096000)
  (0.440000, -0.080000)
  (0.456000, 0.008000)
  (0.544000, -0.008000)
  (0.560000, 0.080000)
  (0.472000, 0.096000)
  (0.488000, 0.184000)
  (0.400000, 0.200000)
  (0.384000, 0.112000)
  (0.296000, 0.128000)
  (0.280000, 0.040000)
  (0.192000, 0.056000)
  (0.208000, 0.144000)
  (0.120000, 0.160000)
  (0.104000, 0.072000)
  (0.016000, 0.088000)
  (-0.000000, -0.000000)
}
\end{pspicture}
%
\hskip0.1in\dots
%

\end{zztask}

%%%%%%%%%%%%%%%%%%%%%%%%%%%%%%%%%%%%%%%%%%%%%%%%%%%%%%%%%%%%%%%%%%%%%%%%%%%%%%

\begin{zztask}[Minkowski Island]
В рамках общего условия задачи построить остров Минковского, составленный из
четырех одинаковых колбас Минковского. База для построения кривой 
(нулевое приближение) представляет собой квадрат.
Переход от приближения $n$ к приближению $(n+1)$ осуществляется заменой каждого 
отрезка фигуры на восемь (см. рис. при $n=1$). 
Все отрезки имеют одинаковую длину, угол между отрезками составляет $90^\circ$.
\par\input{labs/cur/cur-pic-kochis}
\end{zztask}

%%%%%%%%%%%%%%%%%%%%%%%%%%%%%%%%%%%%%%%%%%%%%%%%%%%%%%%%%%%%%%%%%%%%%%%%%%%%%%

\begin{zztask}[T-Square Curve]
В рамках общего условия задачи построить кривую T-Square.
База для построения кривой (нулевое приближение) представляет собой квадрат.
Переход от приближения $n$ к приближению $(n+1)$ осуществляется заменой каждого 
угла квадрата на квадрат без одной четверти (см. рис. при $n=1$). 
Угол (заменяемая часть) захватывает $1/4$ от каждой стороны.
\par\endinput
\begin{pspicture}(-0.5,-0.5)(1.5,1.5)
\psgrid
\rput(0.5,0.5){$n=0$}
\rput(0,0.5){
\psline
  (0, 0)
  (-0.000000, 0.500000)
  (0.500000, 0.500000)
  (1.000000, 0.500000)
  (1.000000, 0.000000)
  (1.000000, -0.500000)
  (0.500000, -0.500000)
  (0.000000, -0.500000)
  (0.000000, 0.000000)
\psline[linewidth=0.06cm, arrows=o-o]
  (0,0.25)
  (0,0.5)
  (0.25,0.5)
}
\end{pspicture}
%
\hskip0.1in
%
\begin{pspicture}(-0.5,-0.5)(1.5,1.5)
\psgrid
\rput(0.5,0.5){$n=1$}
\rput(0,0.5){
\psline
  (0, 0)
  (-0.000000, 0.250000)
  (-0.250000, 0.250000)
  (-0.250000, 0.500000)
  (-0.250000, 0.750000)
  (0.000000, 0.750000)
  (0.250000, 0.750000)
  (0.250000, 0.500000)
  (0.500000, 0.500000)
  (0.750000, 0.500000)
  (0.750000, 0.750000)
  (1.000000, 0.750000)
  (1.250000, 0.750000)
  (1.250000, 0.500000)
  (1.250000, 0.250000)
  (1.000000, 0.250000)
  (1.000000, -0.000000)
  (1.000000, -0.250000)
  (1.250000, -0.250000)
  (1.250000, -0.500000)
  (1.250000, -0.750000)
  (1.000000, -0.750000)
  (0.750000, -0.750000)
  (0.750000, -0.500000)
  (0.500000, -0.500000)
  (0.250000, -0.500000)
  (0.250000, -0.750000)
  (0.000000, -0.750000)
  (-0.250000, -0.750000)
  (-0.250000, -0.500000)
  (-0.250000, -0.250000)
  (0.000000, -0.250000)
  (0.000000, -0.000000)
\psline[linewidth=0.06cm, arrows=o-o]
  (0,0.25)
  (-0.25,0.25)
  (-0.25,0.75)
  (0.25,0.75)
  (0.25,0.5)
}
\end{pspicture}
%
\hskip0.1in
%
\begin{pspicture}(-0.5,-0.5)(1.5,1.5)
\psgrid
\rput(0.5,0.5){$n=2$}
\rput(0,0.5){
\psline
  (0, 0)
  (-0.000000, 0.125000)
  (-0.000000, 0.250000)
  (-0.125000, 0.250000)
  (-0.125000, 0.125000)
  (-0.250000, 0.125000)
  (-0.375000, 0.125000)
  (-0.375000, 0.250000)
  (-0.375000, 0.375000)
  (-0.250000, 0.375000)
  (-0.250000, 0.500000)
  (-0.250000, 0.625000)
  (-0.375000, 0.625000)
  (-0.375000, 0.750000)
  (-0.375000, 0.875000)
  (-0.250000, 0.875000)
  (-0.125000, 0.875000)
  (-0.125000, 0.750000)
  (0.000000, 0.750000)
  (0.125000, 0.750000)
  (0.125000, 0.875000)
  (0.250000, 0.875000)
  (0.375000, 0.875000)
  (0.375000, 0.750000)
  (0.375000, 0.625000)
  (0.250000, 0.625000)
  (0.250000, 0.500000)
  (0.375000, 0.500000)
  (0.500000, 0.500000)
  (0.625000, 0.500000)
  (0.750000, 0.500000)
  (0.750000, 0.625000)
  (0.625000, 0.625000)
  (0.625000, 0.750000)
  (0.625000, 0.875000)
  (0.750000, 0.875000)
  (0.875000, 0.875000)
  (0.875000, 0.750000)
  (1.000000, 0.750000)
  (1.125000, 0.750000)
  (1.125000, 0.875000)
  (1.250000, 0.875000)
  (1.375000, 0.875000)
  (1.375000, 0.750000)
  (1.375000, 0.625000)
  (1.250000, 0.625000)
  (1.250000, 0.500000)
  (1.250000, 0.375000)
  (1.375000, 0.375000)
  (1.375000, 0.250000)
  (1.375000, 0.125000)
  (1.250000, 0.125000)
  (1.125000, 0.125000)
  (1.125000, 0.250000)
  (1.000000, 0.250000)
  (1.000000, 0.125000)
  (1.000000, -0.000000)
  (1.000000, -0.125000)
  (1.000000, -0.250000)
  (1.125000, -0.250000)
  (1.125000, -0.125000)
  (1.250000, -0.125000)
  (1.375000, -0.125000)
  (1.375000, -0.250000)
  (1.375000, -0.375000)
  (1.250000, -0.375000)
  (1.250000, -0.500000)
  (1.250000, -0.625000)
  (1.375000, -0.625000)
  (1.375000, -0.750000)
  (1.375000, -0.875000)
  (1.250000, -0.875000)
  (1.125000, -0.875000)
  (1.125000, -0.750000)
  (1.000000, -0.750000)
  (0.875000, -0.750000)
  (0.875000, -0.875000)
  (0.750000, -0.875000)
  (0.625000, -0.875000)
  (0.625000, -0.750000)
  (0.625000, -0.625000)
  (0.750000, -0.625000)
  (0.750000, -0.500000)
  (0.625000, -0.500000)
  (0.500000, -0.500000)
  (0.375000, -0.500000)
  (0.250000, -0.500000)
  (0.250000, -0.625000)
  (0.375000, -0.625000)
  (0.375000, -0.750000)
  (0.375000, -0.875000)
  (0.250000, -0.875000)
  (0.125000, -0.875000)
  (0.125000, -0.750000)
  (0.000000, -0.750000)
  (-0.125000, -0.750000)
  (-0.125000, -0.875000)
  (-0.250000, -0.874999)
  (-0.375000, -0.874999)
  (-0.375000, -0.749999)
  (-0.375000, -0.624999)
  (-0.250000, -0.624999)
  (-0.250000, -0.499999)
  (-0.249999, -0.374999)
  (-0.374999, -0.374999)
  (-0.374999, -0.249999)
  (-0.374999, -0.124999)
  (-0.249999, -0.125000)
  (-0.124999, -0.125000)
  (-0.124999, -0.250000)
  (0.000001, -0.250000)
  (0.000001, -0.125000)
  (0.000001, 0.000000)
}
\end{pspicture}
%
\hskip0.1in
%
\begin{pspicture}(-0.5,-0.5)(1.5,1.5)
\psgrid
\rput(0.5,0.5){$n=3$}
\rput(0,0.5){
\psline
  (0, 0)
  (-0.000000, 0.062500)
  (-0.000000, 0.125000)
  (-0.000000, 0.187500)
  (-0.000000, 0.250000)
  (-0.062500, 0.250000)
  (-0.125000, 0.250000)
  (-0.125000, 0.187500)
  (-0.062500, 0.187500)
  (-0.062500, 0.125000)
  (-0.062500, 0.062500)
  (-0.125000, 0.062500)
  (-0.187500, 0.062500)
  (-0.187500, 0.125000)
  (-0.250000, 0.125000)
  (-0.312500, 0.125000)
  (-0.312500, 0.062500)
  (-0.375000, 0.062500)
  (-0.437500, 0.062500)
  (-0.437500, 0.125000)
  (-0.437500, 0.187500)
  (-0.375000, 0.187500)
  (-0.375000, 0.250000)
  (-0.375000, 0.312500)
  (-0.437500, 0.312500)
  (-0.437500, 0.375000)
  (-0.437500, 0.437500)
  (-0.375000, 0.437500)
  (-0.312500, 0.437500)
  (-0.312500, 0.375000)
  (-0.250000, 0.375000)
  (-0.250000, 0.437500)
  (-0.250000, 0.500000)
  (-0.250000, 0.562500)
  (-0.250000, 0.625000)
  (-0.312500, 0.625000)
  (-0.312500, 0.562500)
  (-0.375000, 0.562500)
  (-0.437500, 0.562500)
  (-0.437500, 0.625000)
  (-0.437500, 0.687500)
  (-0.375000, 0.687500)
  (-0.375000, 0.750000)
  (-0.375000, 0.812500)
  (-0.437500, 0.812500)
  (-0.437500, 0.875000)
  (-0.437500, 0.937500)
  (-0.375000, 0.937500)
  (-0.312500, 0.937500)
  (-0.312500, 0.875000)
  (-0.250000, 0.875000)
  (-0.187500, 0.875000)
  (-0.187500, 0.937500)
  (-0.125000, 0.937500)
  (-0.062500, 0.937500)
  (-0.062500, 0.875000)
  (-0.062500, 0.812500)
  (-0.125000, 0.812500)
  (-0.125000, 0.750000)
  (-0.062500, 0.750000)
  (-0.000000, 0.750000)
  (0.062500, 0.750000)
  (0.125000, 0.750000)
  (0.125000, 0.812500)
  (0.062500, 0.812500)
  (0.062500, 0.875000)
  (0.062500, 0.937500)
  (0.125000, 0.937500)
  (0.187500, 0.937500)
  (0.187500, 0.875000)
  (0.250000, 0.875000)
  (0.312500, 0.875000)
  (0.312500, 0.937500)
  (0.375000, 0.937500)
  (0.437500, 0.937500)
  (0.437500, 0.875000)
  (0.437500, 0.812500)
  (0.375000, 0.812500)
  (0.375000, 0.750000)
  (0.375000, 0.687500)
  (0.437500, 0.687500)
  (0.437500, 0.625000)
  (0.437500, 0.562500)
  (0.375000, 0.562500)
  (0.312500, 0.562500)
  (0.312500, 0.625000)
  (0.250000, 0.625000)
  (0.250000, 0.562500)
  (0.250000, 0.500000)
  (0.312500, 0.500000)
  (0.375000, 0.500000)
  (0.437500, 0.500000)
  (0.500000, 0.500000)
  (0.562500, 0.500000)
  (0.625000, 0.500000)
  (0.687500, 0.500000)
  (0.750000, 0.500000)
  (0.750000, 0.562500)
  (0.750000, 0.625000)
  (0.687500, 0.625000)
  (0.687500, 0.562500)
  (0.625000, 0.562500)
  (0.562500, 0.562500)
  (0.562500, 0.625000)
  (0.562500, 0.687500)
  (0.625000, 0.687500)
  (0.625000, 0.750000)
  (0.625000, 0.812500)
  (0.562500, 0.812500)
  (0.562500, 0.875000)
  (0.562500, 0.937500)
  (0.625000, 0.937500)
  (0.687500, 0.937500)
  (0.687500, 0.875000)
  (0.750000, 0.875000)
  (0.812500, 0.875000)
  (0.812500, 0.937500)
  (0.875000, 0.937500)
  (0.937500, 0.937500)
  (0.937500, 0.875000)
  (0.937500, 0.812500)
  (0.875000, 0.812500)
  (0.875000, 0.750000)
  (0.937500, 0.750000)
  (1.000000, 0.750000)
  (1.062500, 0.750000)
  (1.125000, 0.750000)
  (1.125000, 0.812500)
  (1.062500, 0.812500)
  (1.062500, 0.875000)
  (1.062500, 0.937500)
  (1.125000, 0.937500)
  (1.187500, 0.937500)
  (1.187500, 0.875000)
  (1.250000, 0.875000)
  (1.312500, 0.875000)
  (1.312500, 0.937500)
  (1.375000, 0.937500)
  (1.437500, 0.937500)
  (1.437500, 0.875000)
  (1.437500, 0.812500)
  (1.375000, 0.812500)
  (1.375000, 0.750000)
  (1.375000, 0.687500)
  (1.437500, 0.687500)
  (1.437500, 0.625000)
  (1.437500, 0.562500)
  (1.375000, 0.562500)
  (1.312500, 0.562500)
  (1.312500, 0.625000)
  (1.250000, 0.625000)
  (1.250000, 0.562500)
  (1.250000, 0.500000)
  (1.250000, 0.437500)
  (1.250000, 0.375000)
  (1.312500, 0.375000)
  (1.312500, 0.437500)
  (1.375000, 0.437500)
  (1.437500, 0.437500)
  (1.437500, 0.375000)
  (1.437500, 0.312500)
  (1.375000, 0.312500)
  (1.375000, 0.250000)
  (1.375000, 0.187500)
  (1.437500, 0.187500)
  (1.437500, 0.125000)
  (1.437500, 0.062500)
  (1.375000, 0.062500)
  (1.312500, 0.062500)
  (1.312500, 0.125000)
  (1.250000, 0.125000)
  (1.187500, 0.125000)
  (1.187500, 0.062500)
  (1.125000, 0.062500)
  (1.062500, 0.062500)
  (1.062500, 0.125000)
  (1.062500, 0.187500)
  (1.125000, 0.187500)
  (1.125000, 0.250000)
  (1.062500, 0.250000)
  (1.000000, 0.250000)
  (1.000000, 0.187500)
  (1.000000, 0.125000)
  (1.000000, 0.062500)
  (1.000000, 0.000000)
  (1.000000, -0.062500)
  (1.000000, -0.125000)
  (1.000000, -0.187500)
  (1.000000, -0.250000)
  (1.062500, -0.250000)
  (1.125000, -0.250000)
  (1.125000, -0.187500)
  (1.062500, -0.187500)
  (1.062500, -0.125000)
  (1.062500, -0.062500)
  (1.125000, -0.062500)
  (1.187500, -0.062500)
  (1.187500, -0.125000)
  (1.250000, -0.125000)
  (1.312500, -0.125000)
  (1.312500, -0.062500)
  (1.375000, -0.062500)
  (1.437500, -0.062500)
  (1.437500, -0.125000)
  (1.437500, -0.187500)
  (1.375000, -0.187500)
  (1.375000, -0.250000)
  (1.375000, -0.312500)
  (1.437500, -0.312500)
  (1.437500, -0.375000)
  (1.437500, -0.437500)
  (1.375000, -0.437500)
  (1.312500, -0.437500)
  (1.312500, -0.375000)
  (1.250000, -0.375000)
  (1.250000, -0.437500)
  (1.250000, -0.500000)
  (1.250000, -0.562500)
  (1.250000, -0.625000)
  (1.312500, -0.625000)
  (1.312500, -0.562500)
  (1.375000, -0.562500)
  (1.437500, -0.562500)
  (1.437500, -0.625000)
  (1.437500, -0.687500)
  (1.375000, -0.687500)
  (1.375000, -0.750000)
  (1.375000, -0.812500)
  (1.437500, -0.812500)
  (1.437500, -0.875000)
  (1.437500, -0.937500)
  (1.375000, -0.937500)
  (1.312500, -0.937500)
  (1.312500, -0.875000)
  (1.250000, -0.875000)
  (1.187500, -0.875000)
  (1.187500, -0.937500)
  (1.125000, -0.937500)
  (1.062500, -0.937500)
  (1.062500, -0.875000)
  (1.062500, -0.812500)
  (1.125000, -0.812500)
  (1.125000, -0.750000)
  (1.062500, -0.750000)
  (1.000000, -0.750000)
  (0.937500, -0.750000)
  (0.875000, -0.750000)
  (0.875000, -0.812500)
  (0.937500, -0.812500)
  (0.937500, -0.875000)
  (0.937500, -0.937500)
  (0.875000, -0.937500)
  (0.812500, -0.937500)
  (0.812500, -0.875000)
  (0.750000, -0.875000)
  (0.687500, -0.875000)
  (0.687500, -0.937500)
  (0.625000, -0.937500)
  (0.562500, -0.937500)
  (0.562500, -0.875000)
  (0.562500, -0.812500)
  (0.625000, -0.812500)
  (0.625000, -0.750000)
  (0.625000, -0.687500)
  (0.562500, -0.687500)
  (0.562500, -0.625000)
  (0.562500, -0.562500)
  (0.625000, -0.562500)
  (0.687500, -0.562500)
  (0.687500, -0.625000)
  (0.750000, -0.625000)
  (0.750000, -0.562500)
  (0.750000, -0.500000)
  (0.687500, -0.500000)
  (0.625000, -0.500000)
  (0.562500, -0.500000)
  (0.500000, -0.500000)
  (0.437500, -0.500000)
  (0.375000, -0.500000)
  (0.312500, -0.500000)
  (0.250000, -0.500000)
  (0.250000, -0.562500)
  (0.250000, -0.625000)
  (0.312500, -0.625000)
  (0.312500, -0.562500)
  (0.375000, -0.562500)
  (0.437500, -0.562500)
  (0.437500, -0.625000)
  (0.437500, -0.687500)
  (0.375000, -0.687500)
  (0.375000, -0.750000)
  (0.375000, -0.812500)
  (0.437500, -0.812500)
  (0.437500, -0.875000)
  (0.437500, -0.937500)
  (0.375000, -0.937500)
  (0.312500, -0.937500)
  (0.312500, -0.875000)
  (0.250000, -0.875000)
  (0.187500, -0.875000)
  (0.187500, -0.937500)
  (0.125000, -0.937500)
  (0.062500, -0.937500)
  (0.062500, -0.875000)
  (0.062500, -0.812500)
  (0.125000, -0.812500)
  (0.125000, -0.750000)
  (0.062500, -0.750000)
  (0.000000, -0.750000)
  (-0.062500, -0.750000)
  (-0.125000, -0.750000)
  (-0.125000, -0.812500)
  (-0.062500, -0.812500)
  (-0.062500, -0.875000)
  (-0.062500, -0.937500)
  (-0.125000, -0.937500)
  (-0.187500, -0.937500)
  (-0.187500, -0.875000)
  (-0.250000, -0.875000)
  (-0.312500, -0.875000)
  (-0.312500, -0.937500)
  (-0.375000, -0.937500)
  (-0.437500, -0.937500)
  (-0.437500, -0.875000)
  (-0.437500, -0.812500)
  (-0.375000, -0.812500)
  (-0.375000, -0.750000)
  (-0.375000, -0.687500)
  (-0.437500, -0.687500)
  (-0.437500, -0.625000)
  (-0.437500, -0.562500)
  (-0.375000, -0.562500)
  (-0.312500, -0.562500)
  (-0.312500, -0.625000)
  (-0.250000, -0.625000)
  (-0.250000, -0.562500)
  (-0.250000, -0.500000)
  (-0.250000, -0.437500)
  (-0.250000, -0.375000)
  (-0.312500, -0.375000)
  (-0.312500, -0.437500)
  (-0.375000, -0.437500)
  (-0.437500, -0.437500)
  (-0.437500, -0.375000)
  (-0.437500, -0.312500)
  (-0.375000, -0.312500)
  (-0.375000, -0.250000)
  (-0.375000, -0.187500)
  (-0.437500, -0.187500)
  (-0.437500, -0.125000)
  (-0.437500, -0.062500)
  (-0.375000, -0.062500)
  (-0.312500, -0.062500)
  (-0.312500, -0.125000)
  (-0.250000, -0.125000)
  (-0.187500, -0.125000)
  (-0.187500, -0.062500)
  (-0.125000, -0.062500)
  (-0.062500, -0.062501)
  (-0.062500, -0.125001)
  (-0.062500, -0.187501)
  (-0.125000, -0.187500)
  (-0.125000, -0.250000)
  (-0.062500, -0.250000)
  (0.000000, -0.250001)
  (0.000000, -0.187501)
  (0.000000, -0.125001)
  (0.000000, -0.062501)
  (0.000000, -0.000001)
}
\end{pspicture}
%
\hskip0.1in\dots
%

\end{zztask}

%%%%%%%%%%%%%%%%%%%%%%%%%%%%%%%%%%%%%%%%%%%%%%%%%%%%%%%%%%%%%%%%%%%%%%%%%%%%%%

\begin{zztask}[Cross-Stitch Curve]
В рамках общего условия задачи построить ``вышивку крестиком''.
База для построения кривой (нулевое приближение) представляет собой квадрат.
Переход от приближения $n$ к приближению $(n+1)$ осуществляется заменой каждого 
отрезка фигуры на пять (см. рис. при $n=1$). 
Все отрезки имеют одинаковую длину, угол между отрезками составляет $90^\circ$.
\par\endinput
\begin{pspicture}(-0.5,-0.5)(1.5,1.5)
\psgrid
\rput(0.5,0.5){$n=0$}
\rput(0,0){
\psline
  (0, 0)
  (-0.000000, 1.000000)
  (1.000000, 1.000000)
  (1.000000, 0.000000)
  (0, 0)
\psline[linewidth=0.06cm, arrows=o-o]
  (0, 0)
  (-0.000000, 1.000000)
}
\end{pspicture}
%
\hskip0.1in
%
\begin{pspicture}(-0.5,-0.5)(1.5,1.5)
\psgrid
\rput(0.5,0.5){$n=1$}
\rput(0,0){
\psline
  (0, 0)
  (-0.000000, 0.333333)
  (-0.333333, 0.333333)
  (-0.333333, 0.666667)
  (0.000000, 0.666667)
  (-0.000000, 1.000000)
  (0.333333, 1.000000)
  (0.333333, 1.333333)
  (0.666667, 1.333333)
  (0.666667, 1.000000)
  (1.000000, 1.000000)
  (1.000000, 0.666667)
  (1.333333, 0.666667)
  (1.333333, 0.333333)
  (1.000000, 0.333333)
  (1.000000, -0.000000)
  (0.666667, -0.000000)
  (0.666667, -0.333333)
  (0.333333, -0.333333)
  (0.333333, 0.000000)
  (0.000000, -0.000000)
\psline[linewidth=0.06cm, arrows=o-o, showpoints=true]
  (0, 0)
  (-0.000000, 0.333333)
  (-0.333333, 0.333333)
  (-0.333333, 0.666667)
  (0.000000, 0.666667)
  (-0.000000, 1.000000)
}
\end{pspicture}
%
\hskip0.1in
%
\begin{pspicture}(-0.5,-0.5)(1.5,1.5)
\psgrid
\rput(0.5,0.5){$n=2$}
\rput(0,0){
\psline
  (0, 0)
  (-0.000000, 0.111111)
  (-0.111111, 0.111111)
  (-0.111111, 0.222222)
  (-0.000000, 0.222222)
  (-0.000000, 0.333333)
  (-0.111111, 0.333333)
  (-0.111111, 0.222222)
  (-0.222222, 0.222222)
  (-0.222222, 0.333333)
  (-0.333333, 0.333333)
  (-0.333333, 0.444444)
  (-0.444444, 0.444444)
  (-0.444444, 0.555556)
  (-0.333333, 0.555556)
  (-0.333333, 0.666667)
  (-0.222222, 0.666667)
  (-0.222222, 0.777778)
  (-0.111111, 0.777778)
  (-0.111111, 0.666667)
  (0.000000, 0.666667)
  (0.000000, 0.777778)
  (-0.111111, 0.777778)
  (-0.111111, 0.888889)
  (0.000000, 0.888889)
  (0.000000, 1.000000)
  (0.111111, 1.000000)
  (0.111111, 1.111111)
  (0.222222, 1.111111)
  (0.222222, 1.000000)
  (0.333333, 1.000000)
  (0.333334, 1.111111)
  (0.222222, 1.111111)
  (0.222222, 1.222222)
  (0.333334, 1.222222)
  (0.333334, 1.333334)
  (0.444445, 1.333334)
  (0.444445, 1.444445)
  (0.555556, 1.444445)
  (0.555556, 1.333334)
  (0.666667, 1.333334)
  (0.666667, 1.222222)
  (0.777778, 1.222222)
  (0.777778, 1.111111)
  (0.666667, 1.111111)
  (0.666667, 1.000000)
  (0.777778, 1.000000)
  (0.777778, 1.111111)
  (0.888889, 1.111111)
  (0.888889, 1.000000)
  (1.000000, 1.000000)
  (1.000000, 0.888889)
  (1.111111, 0.888889)
  (1.111111, 0.777778)
  (1.000000, 0.777778)
  (1.000000, 0.666667)
  (1.111111, 0.666667)
  (1.111111, 0.777778)
  (1.222222, 0.777778)
  (1.222222, 0.666667)
  (1.333333, 0.666667)
  (1.333333, 0.555556)
  (1.444445, 0.555556)
  (1.444445, 0.444445)
  (1.333333, 0.444445)
  (1.333333, 0.333334)
  (1.222222, 0.333334)
  (1.222222, 0.222222)
  (1.111111, 0.222223)
  (1.111111, 0.333334)
  (1.000000, 0.333334)
  (1.000000, 0.222223)
  (1.111111, 0.222223)
  (1.111111, 0.111111)
  (1.000000, 0.111111)
  (1.000000, 0.000000)
  (0.888889, 0.000000)
  (0.888889, -0.111111)
  (0.777778, -0.111111)
  (0.777778, 0.000000)
  (0.666667, 0.000000)
  (0.666667, -0.111111)
  (0.777778, -0.111111)
  (0.777778, -0.222222)
  (0.666666, -0.222222)
  (0.666666, -0.333333)
  (0.555555, -0.333333)
  (0.555555, -0.444444)
  (0.444444, -0.444444)
  (0.444444, -0.333333)
  (0.333333, -0.333333)
  (0.333333, -0.222222)
  (0.222222, -0.222222)
  (0.222222, -0.111111)
  (0.333333, -0.111111)
  (0.333333, 0.000000)
  (0.222222, 0.000001)
  (0.222222, -0.111111)
  (0.111111, -0.111111)
  (0.111111, 0.000001)
  (-0.000000, 0.000001)
}
\end{pspicture}
%
\hskip0.1in
%
\begin{pspicture}(-0.5,-0.5)(1.5,1.5)
\psgrid
\rput(0.5,0.5){$n=3$}
\rput(0,0){
\psline
  (0, 0)
  (-0.000000, 0.037037)
  (-0.037037, 0.037037)
  (-0.037037, 0.074074)
  (0.000000, 0.074074)
  (-0.000000, 0.111111)
  (-0.037037, 0.111111)
  (-0.037037, 0.074074)
  (-0.074074, 0.074074)
  (-0.074074, 0.111111)
  (-0.111111, 0.111111)
  (-0.111111, 0.148148)
  (-0.148148, 0.148148)
  (-0.148148, 0.185185)
  (-0.111111, 0.185185)
  (-0.111111, 0.222222)
  (-0.074074, 0.222222)
  (-0.074074, 0.259259)
  (-0.037037, 0.259259)
  (-0.037037, 0.222222)
  (0.000000, 0.222222)
  (0.000000, 0.259259)
  (-0.037037, 0.259259)
  (-0.037037, 0.296296)
  (0.000000, 0.296296)
  (0.000000, 0.333333)
  (-0.037037, 0.333333)
  (-0.037037, 0.296296)
  (-0.074074, 0.296296)
  (-0.074074, 0.333333)
  (-0.111111, 0.333333)
  (-0.111111, 0.296296)
  (-0.074074, 0.296296)
  (-0.074074, 0.259259)
  (-0.111111, 0.259259)
  (-0.111111, 0.222222)
  (-0.148148, 0.222222)
  (-0.148148, 0.185185)
  (-0.185185, 0.185185)
  (-0.185185, 0.222222)
  (-0.222222, 0.222222)
  (-0.222222, 0.259259)
  (-0.259259, 0.259259)
  (-0.259259, 0.296296)
  (-0.222222, 0.296296)
  (-0.222222, 0.333333)
  (-0.259259, 0.333333)
  (-0.259259, 0.296296)
  (-0.296296, 0.296296)
  (-0.296296, 0.333333)
  (-0.333333, 0.333333)
  (-0.333333, 0.370370)
  (-0.370370, 0.370370)
  (-0.370370, 0.407407)
  (-0.333333, 0.407407)
  (-0.333333, 0.444444)
  (-0.370370, 0.444444)
  (-0.370370, 0.407407)
  (-0.407407, 0.407407)
  (-0.407407, 0.444444)
  (-0.444444, 0.444444)
  (-0.444444, 0.481481)
  (-0.481482, 0.481481)
  (-0.481482, 0.518519)
  (-0.444444, 0.518518)
  (-0.444444, 0.555556)
  (-0.407407, 0.555556)
  (-0.407407, 0.592593)
  (-0.370370, 0.592593)
  (-0.370370, 0.555555)
  (-0.333333, 0.555556)
  (-0.333333, 0.592593)
  (-0.370370, 0.592593)
  (-0.370370, 0.629630)
  (-0.333333, 0.629630)
  (-0.333333, 0.666667)
  (-0.296296, 0.666667)
  (-0.296296, 0.703704)
  (-0.259259, 0.703704)
  (-0.259259, 0.666667)
  (-0.222222, 0.666667)
  (-0.222222, 0.703704)
  (-0.259259, 0.703704)
  (-0.259259, 0.740741)
  (-0.222222, 0.740741)
  (-0.222222, 0.777778)
  (-0.185185, 0.777778)
  (-0.185185, 0.814815)
  (-0.148148, 0.814815)
  (-0.148148, 0.777778)
  (-0.111111, 0.777778)
  (-0.111111, 0.740741)
  (-0.074074, 0.740741)
  (-0.074074, 0.703704)
  (-0.111111, 0.703704)
  (-0.111111, 0.666667)
  (-0.074074, 0.666667)
  (-0.074074, 0.703704)
  (-0.037037, 0.703704)
  (-0.037037, 0.666667)
  (-0.000000, 0.666667)
  (0.000000, 0.703704)
  (-0.037037, 0.703704)
  (-0.037037, 0.740741)
  (0.000000, 0.740741)
  (0.000000, 0.777778)
  (-0.037037, 0.777778)
  (-0.037037, 0.740741)
  (-0.074074, 0.740741)
  (-0.074074, 0.777778)
  (-0.111111, 0.777778)
  (-0.111111, 0.814815)
  (-0.148148, 0.814815)
  (-0.148148, 0.851852)
  (-0.111111, 0.851852)
  (-0.111111, 0.888889)
  (-0.074074, 0.888889)
  (-0.074074, 0.925926)
  (-0.037037, 0.925926)
  (-0.037037, 0.888889)
  (0.000000, 0.888889)
  (0.000000, 0.925926)
  (-0.037037, 0.925926)
  (-0.037037, 0.962963)
  (0.000000, 0.962963)
  (0.000000, 1.000000)
  (0.037037, 1.000000)
  (0.037037, 1.037037)
  (0.074074, 1.037037)
  (0.074074, 1.000000)
  (0.111111, 1.000000)
  (0.111111, 1.037037)
  (0.074074, 1.037037)
  (0.074074, 1.074074)
  (0.111111, 1.074074)
  (0.111111, 1.111111)
  (0.148148, 1.111111)
  (0.148148, 1.148148)
  (0.185185, 1.148148)
  (0.185185, 1.111111)
  (0.222222, 1.111111)
  (0.222222, 1.074074)
  (0.259259, 1.074074)
  (0.259259, 1.037037)
  (0.222222, 1.037037)
  (0.222222, 1.000000)
  (0.259259, 1.000000)
  (0.259259, 1.037037)
  (0.296296, 1.037037)
  (0.296296, 1.000000)
  (0.333333, 1.000000)
  (0.333333, 1.037037)
  (0.296296, 1.037037)
  (0.296296, 1.074074)
  (0.333333, 1.074074)
  (0.333333, 1.111111)
  (0.296296, 1.111111)
  (0.296296, 1.074074)
  (0.259259, 1.074074)
  (0.259259, 1.111111)
  (0.222222, 1.111111)
  (0.222222, 1.148148)
  (0.185185, 1.148148)
  (0.185185, 1.185185)
  (0.222222, 1.185185)
  (0.222222, 1.222222)
  (0.259259, 1.222222)
  (0.259259, 1.259259)
  (0.296296, 1.259259)
  (0.296296, 1.222222)
  (0.333333, 1.222222)
  (0.333333, 1.259259)
  (0.296296, 1.259259)
  (0.296296, 1.296296)
  (0.333333, 1.296296)
  (0.333333, 1.333333)
  (0.370370, 1.333333)
  (0.370370, 1.370370)
  (0.407407, 1.370370)
  (0.407407, 1.333333)
  (0.444445, 1.333333)
  (0.444445, 1.370370)
  (0.407407, 1.370370)
  (0.407407, 1.407407)
  (0.444445, 1.407407)
  (0.444445, 1.444444)
  (0.481482, 1.444444)
  (0.481482, 1.481481)
  (0.518519, 1.481481)
  (0.518519, 1.444444)
  (0.555556, 1.444444)
  (0.555556, 1.407407)
  (0.592593, 1.407407)
  (0.592593, 1.370370)
  (0.555556, 1.370370)
  (0.555556, 1.333333)
  (0.592593, 1.333333)
  (0.592593, 1.370370)
  (0.629630, 1.370370)
  (0.629630, 1.333333)
  (0.666667, 1.333333)
  (0.666667, 1.296296)
  (0.703704, 1.296296)
  (0.703704, 1.259259)
  (0.666667, 1.259259)
  (0.666667, 1.222222)
  (0.703704, 1.222222)
  (0.703704, 1.259259)
  (0.740741, 1.259259)
  (0.740741, 1.222222)
  (0.777778, 1.222222)
  (0.777778, 1.185185)
  (0.814815, 1.185185)
  (0.814815, 1.148148)
  (0.777778, 1.148148)
  (0.777778, 1.111111)
  (0.740741, 1.111111)
  (0.740741, 1.074074)
  (0.703704, 1.074074)
  (0.703704, 1.111111)
  (0.666667, 1.111111)
  (0.666667, 1.074074)
  (0.703704, 1.074074)
  (0.703704, 1.037037)
  (0.666667, 1.037037)
  (0.666667, 1.000000)
  (0.703704, 1.000000)
  (0.703704, 1.037037)
  (0.740741, 1.037037)
  (0.740741, 1.000000)
  (0.777778, 1.000000)
  (0.777778, 1.037037)
  (0.740741, 1.037037)
  (0.740741, 1.074074)
  (0.777778, 1.074074)
  (0.777778, 1.111111)
  (0.814815, 1.111111)
  (0.814815, 1.148148)
  (0.851852, 1.148148)
  (0.851852, 1.111111)
  (0.888889, 1.111111)
  (0.888889, 1.074074)
  (0.925926, 1.074074)
  (0.925926, 1.037037)
  (0.888889, 1.037037)
  (0.888889, 1.000000)
  (0.925926, 1.000000)
  (0.925926, 1.037037)
  (0.962963, 1.037037)
  (0.962963, 1.000000)
  (1.000000, 1.000000)
  (1.000000, 0.962963)
  (1.037037, 0.962963)
  (1.037037, 0.925926)
  (1.000000, 0.925926)
  (1.000000, 0.888889)
  (1.037037, 0.888889)
  (1.037037, 0.925926)
  (1.074074, 0.925926)
  (1.074074, 0.888889)
  (1.111111, 0.888889)
  (1.111111, 0.851852)
  (1.148148, 0.851852)
  (1.148148, 0.814815)
  (1.111111, 0.814815)
  (1.111111, 0.777778)
  (1.074074, 0.777778)
  (1.074074, 0.740741)
  (1.037037, 0.740741)
  (1.037037, 0.777778)
  (1.000000, 0.777778)
  (1.000000, 0.740741)
  (1.037037, 0.740741)
  (1.037037, 0.703704)
  (1.000000, 0.703704)
  (1.000000, 0.666667)
  (1.037037, 0.666667)
  (1.037037, 0.703704)
  (1.074074, 0.703704)
  (1.074074, 0.666667)
  (1.111111, 0.666667)
  (1.111111, 0.703704)
  (1.074074, 0.703704)
  (1.074074, 0.740741)
  (1.111111, 0.740741)
  (1.111111, 0.777778)
  (1.148148, 0.777778)
  (1.148148, 0.814815)
  (1.185185, 0.814815)
  (1.185185, 0.777778)
  (1.222222, 0.777778)
  (1.222222, 0.740741)
  (1.259259, 0.740741)
  (1.259259, 0.703704)
  (1.222222, 0.703704)
  (1.222222, 0.666667)
  (1.259259, 0.666667)
  (1.259259, 0.703704)
  (1.296296, 0.703704)
  (1.296296, 0.666667)
  (1.333333, 0.666667)
  (1.333333, 0.629630)
  (1.370370, 0.629630)
  (1.370370, 0.592593)
  (1.333333, 0.592593)
  (1.333333, 0.555556)
  (1.370370, 0.555556)
  (1.370370, 0.592593)
  (1.407407, 0.592593)
  (1.407407, 0.555556)
  (1.444444, 0.555556)
  (1.444444, 0.518519)
  (1.481481, 0.518519)
  (1.481481, 0.481482)
  (1.444444, 0.481482)
  (1.444444, 0.444444)
  (1.407407, 0.444444)
  (1.407407, 0.407407)
  (1.370370, 0.407407)
  (1.370370, 0.444444)
  (1.333333, 0.444444)
  (1.333333, 0.407407)
  (1.370370, 0.407407)
  (1.370370, 0.370370)
  (1.333333, 0.370370)
  (1.333333, 0.333333)
  (1.296296, 0.333333)
  (1.296296, 0.296296)
  (1.259259, 0.296296)
  (1.259259, 0.333333)
  (1.222222, 0.333333)
  (1.222222, 0.296296)
  (1.259259, 0.296296)
  (1.259259, 0.259259)
  (1.222222, 0.259259)
  (1.222222, 0.222222)
  (1.185185, 0.222222)
  (1.185185, 0.185185)
  (1.148148, 0.185185)
  (1.148148, 0.222222)
  (1.111111, 0.222222)
  (1.111111, 0.259259)
  (1.074074, 0.259259)
  (1.074074, 0.296296)
  (1.111111, 0.296296)
  (1.111111, 0.333333)
  (1.074074, 0.333333)
  (1.074074, 0.296296)
  (1.037037, 0.296296)
  (1.037037, 0.333333)
  (1.000000, 0.333333)
  (1.000000, 0.296296)
  (1.037037, 0.296296)
  (1.037037, 0.259259)
  (1.000000, 0.259259)
  (1.000000, 0.222222)
  (1.037037, 0.222222)
  (1.037037, 0.259259)
  (1.074074, 0.259259)
  (1.074074, 0.222222)
  (1.111111, 0.222222)
  (1.111111, 0.185185)
  (1.148148, 0.185185)
  (1.148148, 0.148148)
  (1.111111, 0.148148)
  (1.111111, 0.111111)
  (1.074074, 0.111111)
  (1.074074, 0.074074)
  (1.037037, 0.074074)
  (1.037037, 0.111111)
  (1.000000, 0.111111)
  (1.000000, 0.074074)
  (1.037037, 0.074074)
  (1.037037, 0.037037)
  (1.000000, 0.037037)
  (1.000000, -0.000000)
  (0.962963, -0.000000)
  (0.962963, -0.037037)
  (0.925926, -0.037037)
  (0.925926, -0.000000)
  (0.888889, 0.000000)
  (0.888889, -0.037037)
  (0.925926, -0.037037)
  (0.925926, -0.074074)
  (0.888889, -0.074074)
  (0.888889, -0.111111)
  (0.851852, -0.111111)
  (0.851852, -0.148148)
  (0.814815, -0.148148)
  (0.814815, -0.111111)
  (0.777778, -0.111111)
  (0.777778, -0.074074)
  (0.740741, -0.074074)
  (0.740741, -0.037037)
  (0.777778, -0.037037)
  (0.777778, 0.000000)
  (0.740741, 0.000000)
  (0.740741, -0.037037)
  (0.703704, -0.037037)
  (0.703704, 0.000000)
  (0.666667, 0.000000)
  (0.666667, -0.037037)
  (0.703704, -0.037037)
  (0.703704, -0.074074)
  (0.666667, -0.074074)
  (0.666667, -0.111111)
  (0.703704, -0.111111)
  (0.703704, -0.074074)
  (0.740741, -0.074074)
  (0.740741, -0.111111)
  (0.777778, -0.111111)
  (0.777778, -0.148148)
  (0.814815, -0.148148)
  (0.814815, -0.185185)
  (0.777778, -0.185185)
  (0.777778, -0.222222)
  (0.740741, -0.222222)
  (0.740741, -0.259259)
  (0.703704, -0.259259)
  (0.703704, -0.222222)
  (0.666667, -0.222222)
  (0.666667, -0.259259)
  (0.703704, -0.259259)
  (0.703704, -0.296296)
  (0.666667, -0.296296)
  (0.666667, -0.333333)
  (0.629630, -0.333333)
  (0.629630, -0.370370)
  (0.592593, -0.370370)
  (0.592593, -0.333333)
  (0.555556, -0.333333)
  (0.555556, -0.370370)
  (0.592593, -0.370370)
  (0.592593, -0.407407)
  (0.555556, -0.407407)
  (0.555556, -0.444444)
  (0.518519, -0.444444)
  (0.518519, -0.481481)
  (0.481482, -0.481481)
  (0.481482, -0.444444)
  (0.444445, -0.444444)
  (0.444445, -0.407407)
  (0.407408, -0.407407)
  (0.407408, -0.370370)
  (0.444445, -0.370370)
  (0.444445, -0.333333)
  (0.407408, -0.333333)
  (0.407408, -0.370370)
  (0.370371, -0.370370)
  (0.370371, -0.333333)
  (0.333334, -0.333333)
  (0.333334, -0.296296)
  (0.296297, -0.296296)
  (0.296297, -0.259259)
  (0.333334, -0.259259)
  (0.333334, -0.222222)
  (0.296297, -0.222222)
  (0.296297, -0.259259)
  (0.259260, -0.259259)
  (0.259260, -0.222222)
  (0.222223, -0.222222)
  (0.222223, -0.185185)
  (0.185186, -0.185185)
  (0.185186, -0.148148)
  (0.222223, -0.148148)
  (0.222223, -0.111111)
  (0.259260, -0.111111)
  (0.259260, -0.074074)
  (0.296297, -0.074074)
  (0.296297, -0.111111)
  (0.333334, -0.111111)
  (0.333334, -0.074074)
  (0.296297, -0.074074)
  (0.296297, -0.037037)
  (0.333334, -0.037037)
  (0.333334, 0.000000)
  (0.296297, 0.000000)
  (0.296297, -0.037037)
  (0.259260, -0.037037)
  (0.259260, 0.000000)
  (0.222223, 0.000000)
  (0.222223, -0.037037)
  (0.259260, -0.037037)
  (0.259260, -0.074074)
  (0.222223, -0.074074)
  (0.222223, -0.111111)
  (0.185186, -0.111111)
  (0.185186, -0.148148)
  (0.148149, -0.148148)
  (0.148149, -0.111111)
  (0.111112, -0.111111)
  (0.111112, -0.074074)
  (0.074075, -0.074074)
  (0.074075, -0.037037)
  (0.111112, -0.037037)
  (0.111112, 0.000000)
  (0.074075, 0.000000)
  (0.074075, -0.037037)
  (0.037038, -0.037037)
  (0.037038, 0.000000)
  (0.000001, 0.000000)
}
\end{pspicture}
%
\hskip0.1in\dots
%

\end{zztask}

%%%%%%%%%%%%%%%%%%%%%%%%%%%%%%%%%%%%%%%%%%%%%%%%%%%%%%%%%%%%%%%%%%%%%%%%%%%%%%

\begin{zztask}[Anti-Cross-Stitch Curve]
В рамках общего условия задачи построить анти-``вышивку крестиком''.
База для построения кривой (нулевое приближение) представляет собой квадрат.
Переход от приближения $n$ к приближению $(n+1)$ осуществляется заменой каждого 
отрезка фигуры на пять (см. рис. при $n=1$). 
Все отрезки имеют одинаковую длину, угол между отрезками составляет $90^\circ$.
\par\endinput
\begin{pspicture}(-0.5,-0.5)(1.5,1.5)
\psgrid
\rput(0.5,-0.25){$n=0$}
\rput(0,0){
\psline
  (0, 0)
  (-0.000000, 1.000000)
  (1.000000, 1.000000)
  (1.000000, 0.000000)
  (-0.000000, 0.000000)
\psline[linewidth=0.06cm, arrows=o-o]
  (0, 0)
  (-0.000000, 1.000000)
}
\end{pspicture}
%
\hskip0.1in
%
\begin{pspicture}(-0.5,-0.5)(1.5,1.5)
\psgrid
\rput(0.5,-0.25){$n=1$}
\rput(0,0){
\psline
  (0, 0)
  (-0.000000, 0.333333)
  (0.333333, 0.333333)
  (0.333333, 0.666667)
  (0.000000, 0.666667)
  (-0.000000, 1.000000)
  (0.333333, 1.000000)
  (0.333333, 0.666667)
  (0.666667, 0.666667)
  (0.666667, 1.000000)
  (1.000000, 1.000000)
  (1.000000, 0.666667)
  (0.666667, 0.666667)
  (0.666667, 0.333333)
  (1.000000, 0.333333)
  (1.000000, 0.000000)
  (0.666667, 0.000000)
  (0.666667, 0.333333)
  (0.333333, 0.333333)
  (0.333333, -0.000000)
  (0.000000, -0.000000)
\psline[linewidth=0.06cm, arrows=o-o, showpoints=true]
  (0, 0)
  (-0.000000, 0.333333)
  (0.333333, 0.333333)
  (0.333333, 0.666667)
  (0.000000, 0.666667)
  (-0.000000, 1.000000)
}
\end{pspicture}
%
\hskip0.1in
%
\begin{pspicture}(-0.5,-0.5)(1.5,1.5)
\psgrid
\rput(0.5,-0.25){$n=2$}
\rput(0,0){
\psline
  (0, 0)
  (-0.000000, 0.111111)
  (0.111111, 0.111111)
  (0.111111, 0.222222)
  (-0.000000, 0.222222)
  (-0.000000, 0.333333)
  (0.111111, 0.333333)
  (0.111111, 0.222222)
  (0.222222, 0.222222)
  (0.222222, 0.333333)
  (0.333333, 0.333333)
  (0.333333, 0.444444)
  (0.444444, 0.444444)
  (0.444444, 0.555556)
  (0.333333, 0.555556)
  (0.333333, 0.666667)
  (0.222222, 0.666667)
  (0.222222, 0.777778)
  (0.111111, 0.777778)
  (0.111111, 0.666667)
  (-0.000000, 0.666667)
  (-0.000000, 0.777778)
  (0.111111, 0.777778)
  (0.111111, 0.888889)
  (0.000000, 0.888889)
  (0.000000, 1.000000)
  (0.111111, 1.000000)
  (0.111111, 0.888889)
  (0.222222, 0.888889)
  (0.222222, 1.000000)
  (0.333333, 1.000000)
  (0.333333, 0.888889)
  (0.222222, 0.888889)
  (0.222222, 0.777778)
  (0.333333, 0.777778)
  (0.333333, 0.666667)
  (0.444444, 0.666667)
  (0.444444, 0.555556)
  (0.555555, 0.555556)
  (0.555555, 0.666667)
  (0.666667, 0.666667)
  (0.666667, 0.777778)
  (0.777778, 0.777778)
  (0.777778, 0.888889)
  (0.666667, 0.888889)
  (0.666667, 1.000000)
  (0.777778, 1.000000)
  (0.777778, 0.888889)
  (0.888889, 0.888889)
  (0.888889, 1.000000)
  (1.000000, 1.000000)
  (1.000000, 0.888889)
  (0.888889, 0.888889)
  (0.888889, 0.777778)
  (1.000000, 0.777778)
  (1.000000, 0.666667)
  (0.888889, 0.666667)
  (0.888889, 0.777778)
  (0.777777, 0.777778)
  (0.777777, 0.666667)
  (0.666666, 0.666667)
  (0.666666, 0.555556)
  (0.555555, 0.555556)
  (0.555555, 0.444445)
  (0.666666, 0.444445)
  (0.666666, 0.333334)
  (0.777777, 0.333334)
  (0.777777, 0.222223)
  (0.888888, 0.222223)
  (0.888888, 0.333334)
  (0.999999, 0.333334)
  (0.999999, 0.222223)
  (0.888888, 0.222223)
  (0.888888, 0.111111)
  (0.999999, 0.111111)
  (0.999999, 0.000000)
  (0.888888, 0.000000)
  (0.888888, 0.111111)
  (0.777777, 0.111112)
  (0.777777, 0.000000)
  (0.666666, 0.000000)
  (0.666666, 0.111112)
  (0.777777, 0.111111)
  (0.777777, 0.222223)
  (0.666666, 0.222223)
  (0.666666, 0.333334)
  (0.555555, 0.333334)
  (0.555555, 0.444445)
  (0.444444, 0.444445)
  (0.444444, 0.333334)
  (0.333333, 0.333334)
  (0.333333, 0.222223)
  (0.222222, 0.222223)
  (0.222222, 0.111112)
  (0.333333, 0.111112)
  (0.333333, 0.000000)
  (0.222222, 0.000000)
  (0.222222, 0.111112)
  (0.111111, 0.111112)
  (0.111111, 0.000001)
  (-0.000001, 0.000001)
}
\end{pspicture}
%
\hskip0.1in
%
\begin{pspicture}(-0.5,-0.5)(1.5,1.5)
\psgrid
\rput(0.5,-0.25){$n=3$}
\rput(0,0){
\psline
  (0, 0)
  (-0.000000, 0.037037)
  (0.037037, 0.037037)
  (0.037037, 0.074074)
  (0.000000, 0.074074)
  (-0.000000, 0.111111)
  (0.037037, 0.111111)
  (0.037037, 0.074074)
  (0.074074, 0.074074)
  (0.074074, 0.111111)
  (0.111111, 0.111111)
  (0.111111, 0.148148)
  (0.148148, 0.148148)
  (0.148148, 0.185185)
  (0.111111, 0.185185)
  (0.111111, 0.222222)
  (0.074074, 0.222222)
  (0.074074, 0.259259)
  (0.037037, 0.259259)
  (0.037037, 0.222222)
  (0.000000, 0.222222)
  (0.000000, 0.259259)
  (0.037037, 0.259259)
  (0.037037, 0.296296)
  (0.000000, 0.296296)
  (0.000000, 0.333333)
  (0.037037, 0.333333)
  (0.037037, 0.296296)
  (0.074074, 0.296296)
  (0.074074, 0.333333)
  (0.111111, 0.333333)
  (0.111111, 0.296296)
  (0.074074, 0.296296)
  (0.074074, 0.259259)
  (0.111111, 0.259259)
  (0.111111, 0.222222)
  (0.148148, 0.222222)
  (0.148148, 0.185185)
  (0.185185, 0.185185)
  (0.185185, 0.222222)
  (0.222222, 0.222222)
  (0.222222, 0.259259)
  (0.259259, 0.259259)
  (0.259259, 0.296296)
  (0.222222, 0.296296)
  (0.222222, 0.333333)
  (0.259259, 0.333333)
  (0.259259, 0.296296)
  (0.296296, 0.296296)
  (0.296296, 0.333333)
  (0.333333, 0.333333)
  (0.333333, 0.370370)
  (0.370370, 0.370370)
  (0.370370, 0.407407)
  (0.333333, 0.407407)
  (0.333333, 0.444444)
  (0.370370, 0.444444)
  (0.370370, 0.407407)
  (0.407407, 0.407407)
  (0.407407, 0.444444)
  (0.444445, 0.444444)
  (0.444445, 0.481481)
  (0.481482, 0.481481)
  (0.481482, 0.518519)
  (0.444445, 0.518519)
  (0.444445, 0.555556)
  (0.407407, 0.555556)
  (0.407407, 0.592593)
  (0.370370, 0.592593)
  (0.370370, 0.555555)
  (0.333333, 0.555556)
  (0.333333, 0.592593)
  (0.370370, 0.592593)
  (0.370370, 0.629630)
  (0.333333, 0.629630)
  (0.333333, 0.666667)
  (0.296296, 0.666667)
  (0.296296, 0.703704)
  (0.259259, 0.703704)
  (0.259259, 0.666667)
  (0.222222, 0.666667)
  (0.222222, 0.703704)
  (0.259259, 0.703704)
  (0.259259, 0.740741)
  (0.222222, 0.740741)
  (0.222222, 0.777778)
  (0.185185, 0.777778)
  (0.185185, 0.814815)
  (0.148148, 0.814815)
  (0.148148, 0.777778)
  (0.111111, 0.777778)
  (0.111111, 0.740741)
  (0.074074, 0.740741)
  (0.074074, 0.703704)
  (0.111111, 0.703704)
  (0.111111, 0.666667)
  (0.074074, 0.666667)
  (0.074074, 0.703704)
  (0.037037, 0.703704)
  (0.037037, 0.666667)
  (0.000000, 0.666667)
  (0.000000, 0.703704)
  (0.037037, 0.703704)
  (0.037037, 0.740741)
  (0.000000, 0.740741)
  (0.000000, 0.777778)
  (0.037037, 0.777778)
  (0.037037, 0.740741)
  (0.074074, 0.740741)
  (0.074074, 0.777778)
  (0.111111, 0.777778)
  (0.111111, 0.814815)
  (0.148148, 0.814815)
  (0.148148, 0.851852)
  (0.111111, 0.851852)
  (0.111111, 0.888889)
  (0.074074, 0.888889)
  (0.074074, 0.925926)
  (0.037037, 0.925926)
  (0.037037, 0.888889)
  (0.000000, 0.888889)
  (0.000000, 0.925926)
  (0.037037, 0.925926)
  (0.037037, 0.962963)
  (0.000000, 0.962963)
  (0.000000, 1.000000)
  (0.037037, 1.000000)
  (0.037037, 0.962963)
  (0.074074, 0.962963)
  (0.074074, 1.000000)
  (0.111111, 1.000000)
  (0.111111, 0.962963)
  (0.074074, 0.962963)
  (0.074074, 0.925926)
  (0.111111, 0.925926)
  (0.111111, 0.888889)
  (0.148148, 0.888889)
  (0.148148, 0.851852)
  (0.185185, 0.851852)
  (0.185185, 0.888889)
  (0.222222, 0.888889)
  (0.222222, 0.925926)
  (0.259259, 0.925926)
  (0.259259, 0.962963)
  (0.222222, 0.962963)
  (0.222222, 1.000000)
  (0.259259, 1.000000)
  (0.259259, 0.962963)
  (0.296296, 0.962963)
  (0.296296, 1.000000)
  (0.333333, 1.000000)
  (0.333333, 0.962963)
  (0.296296, 0.962963)
  (0.296296, 0.925926)
  (0.333333, 0.925926)
  (0.333333, 0.888889)
  (0.296296, 0.888889)
  (0.296296, 0.925926)
  (0.259259, 0.925926)
  (0.259259, 0.888889)
  (0.222222, 0.888889)
  (0.222222, 0.851852)
  (0.185185, 0.851852)
  (0.185185, 0.814815)
  (0.222222, 0.814815)
  (0.222222, 0.777778)
  (0.259259, 0.777778)
  (0.259259, 0.740741)
  (0.296296, 0.740741)
  (0.296296, 0.777778)
  (0.333333, 0.777778)
  (0.333333, 0.740741)
  (0.296296, 0.740741)
  (0.296296, 0.703704)
  (0.333333, 0.703704)
  (0.333333, 0.666667)
  (0.370370, 0.666667)
  (0.370370, 0.629630)
  (0.407407, 0.629630)
  (0.407407, 0.666667)
  (0.444445, 0.666667)
  (0.444445, 0.629630)
  (0.407407, 0.629630)
  (0.407407, 0.592593)
  (0.444445, 0.592593)
  (0.444445, 0.555555)
  (0.481482, 0.555556)
  (0.481482, 0.518518)
  (0.518519, 0.518518)
  (0.518519, 0.555556)
  (0.555556, 0.555556)
  (0.555556, 0.592593)
  (0.592593, 0.592593)
  (0.592593, 0.629630)
  (0.555556, 0.629630)
  (0.555556, 0.666667)
  (0.592593, 0.666667)
  (0.592593, 0.629630)
  (0.629630, 0.629630)
  (0.629630, 0.666667)
  (0.666667, 0.666667)
  (0.666667, 0.703704)
  (0.703704, 0.703704)
  (0.703704, 0.740741)
  (0.666667, 0.740741)
  (0.666667, 0.777778)
  (0.703704, 0.777778)
  (0.703704, 0.740741)
  (0.740741, 0.740741)
  (0.740741, 0.777778)
  (0.777778, 0.777778)
  (0.777778, 0.814815)
  (0.814815, 0.814815)
  (0.814815, 0.851852)
  (0.777778, 0.851852)
  (0.777778, 0.888889)
  (0.740741, 0.888889)
  (0.740741, 0.925926)
  (0.703704, 0.925926)
  (0.703704, 0.888889)
  (0.666667, 0.888889)
  (0.666667, 0.925926)
  (0.703704, 0.925926)
  (0.703704, 0.962963)
  (0.666667, 0.962963)
  (0.666667, 1.000000)
  (0.703704, 1.000000)
  (0.703704, 0.962963)
  (0.740741, 0.962963)
  (0.740741, 1.000000)
  (0.777778, 1.000000)
  (0.777778, 0.962963)
  (0.740741, 0.962963)
  (0.740741, 0.925926)
  (0.777778, 0.925926)
  (0.777778, 0.888889)
  (0.814815, 0.888889)
  (0.814815, 0.851852)
  (0.851852, 0.851852)
  (0.851852, 0.888889)
  (0.888889, 0.888889)
  (0.888889, 0.925926)
  (0.925926, 0.925926)
  (0.925926, 0.962963)
  (0.888889, 0.962963)
  (0.888889, 1.000000)
  (0.925926, 1.000000)
  (0.925926, 0.962963)
  (0.962963, 0.962963)
  (0.962963, 1.000000)
  (1.000000, 1.000000)
  (1.000000, 0.962963)
  (0.962963, 0.962963)
  (0.962963, 0.925926)
  (1.000000, 0.925926)
  (1.000000, 0.888889)
  (0.962963, 0.888889)
  (0.962963, 0.925926)
  (0.925926, 0.925926)
  (0.925926, 0.888889)
  (0.888889, 0.888889)
  (0.888889, 0.851852)
  (0.851852, 0.851852)
  (0.851852, 0.814815)
  (0.888889, 0.814815)
  (0.888889, 0.777778)
  (0.925926, 0.777778)
  (0.925926, 0.740741)
  (0.962963, 0.740741)
  (0.962963, 0.777778)
  (1.000000, 0.777778)
  (1.000000, 0.740741)
  (0.962963, 0.740741)
  (0.962963, 0.703704)
  (1.000000, 0.703704)
  (1.000000, 0.666667)
  (0.962963, 0.666667)
  (0.962963, 0.703704)
  (0.925926, 0.703704)
  (0.925926, 0.666667)
  (0.888889, 0.666667)
  (0.888889, 0.703704)
  (0.925926, 0.703704)
  (0.925926, 0.740741)
  (0.888889, 0.740741)
  (0.888889, 0.777778)
  (0.851852, 0.777778)
  (0.851852, 0.814815)
  (0.814815, 0.814815)
  (0.814815, 0.777778)
  (0.777778, 0.777778)
  (0.777778, 0.740741)
  (0.740741, 0.740741)
  (0.740741, 0.703704)
  (0.777778, 0.703704)
  (0.777778, 0.666667)
  (0.740741, 0.666667)
  (0.740741, 0.703704)
  (0.703704, 0.703704)
  (0.703704, 0.666667)
  (0.666667, 0.666667)
  (0.666667, 0.629630)
  (0.629630, 0.629630)
  (0.629630, 0.592593)
  (0.666667, 0.592593)
  (0.666667, 0.555555)
  (0.629630, 0.555556)
  (0.629630, 0.592593)
  (0.592593, 0.592593)
  (0.592593, 0.555555)
  (0.555556, 0.555556)
  (0.555556, 0.518518)
  (0.518519, 0.518519)
  (0.518519, 0.481481)
  (0.555556, 0.481481)
  (0.555556, 0.444444)
  (0.592593, 0.444444)
  (0.592593, 0.407407)
  (0.629630, 0.407407)
  (0.629630, 0.444444)
  (0.666667, 0.444444)
  (0.666667, 0.407407)
  (0.629630, 0.407407)
  (0.629630, 0.370370)
  (0.666667, 0.370370)
  (0.666667, 0.333333)
  (0.703704, 0.333333)
  (0.703704, 0.296296)
  (0.740741, 0.296296)
  (0.740741, 0.333333)
  (0.777778, 0.333333)
  (0.777778, 0.296296)
  (0.740741, 0.296296)
  (0.740741, 0.259259)
  (0.777778, 0.259259)
  (0.777778, 0.222222)
  (0.814815, 0.222222)
  (0.814815, 0.185185)
  (0.851852, 0.185185)
  (0.851852, 0.222222)
  (0.888889, 0.222222)
  (0.888889, 0.259259)
  (0.925926, 0.259259)
  (0.925926, 0.296296)
  (0.888889, 0.296296)
  (0.888889, 0.333333)
  (0.925926, 0.333333)
  (0.925926, 0.296296)
  (0.962963, 0.296296)
  (0.962963, 0.333333)
  (1.000000, 0.333333)
  (1.000000, 0.296296)
  (0.962963, 0.296296)
  (0.962963, 0.259259)
  (1.000000, 0.259259)
  (1.000000, 0.222222)
  (0.962963, 0.222222)
  (0.962963, 0.259259)
  (0.925926, 0.259259)
  (0.925926, 0.222222)
  (0.888889, 0.222222)
  (0.888889, 0.185185)
  (0.851852, 0.185185)
  (0.851852, 0.148148)
  (0.888889, 0.148148)
  (0.888889, 0.111111)
  (0.925926, 0.111111)
  (0.925926, 0.074074)
  (0.962963, 0.074074)
  (0.962963, 0.111111)
  (1.000000, 0.111111)
  (1.000000, 0.074074)
  (0.962963, 0.074074)
  (0.962963, 0.037037)
  (1.000000, 0.037037)
  (1.000000, -0.000000)
  (0.962963, -0.000000)
  (0.962963, 0.037037)
  (0.925926, 0.037037)
  (0.925926, -0.000000)
  (0.888889, -0.000000)
  (0.888889, 0.037037)
  (0.925926, 0.037037)
  (0.925926, 0.074074)
  (0.888889, 0.074074)
  (0.888889, 0.111111)
  (0.851852, 0.111111)
  (0.851852, 0.148148)
  (0.814815, 0.148148)
  (0.814815, 0.111111)
  (0.777778, 0.111111)
  (0.777778, 0.074074)
  (0.740741, 0.074074)
  (0.740741, 0.037037)
  (0.777778, 0.037037)
  (0.777778, -0.000000)
  (0.740741, -0.000000)
  (0.740741, 0.037037)
  (0.703704, 0.037037)
  (0.703704, -0.000000)
  (0.666667, -0.000000)
  (0.666667, 0.037037)
  (0.703704, 0.037037)
  (0.703704, 0.074074)
  (0.666667, 0.074074)
  (0.666667, 0.111111)
  (0.703704, 0.111111)
  (0.703704, 0.074074)
  (0.740741, 0.074074)
  (0.740741, 0.111111)
  (0.777778, 0.111111)
  (0.777778, 0.148148)
  (0.814815, 0.148148)
  (0.814815, 0.185185)
  (0.777778, 0.185185)
  (0.777778, 0.222222)
  (0.740741, 0.222222)
  (0.740741, 0.259259)
  (0.703704, 0.259259)
  (0.703704, 0.222222)
  (0.666667, 0.222222)
  (0.666667, 0.259259)
  (0.703704, 0.259259)
  (0.703704, 0.296296)
  (0.666667, 0.296296)
  (0.666667, 0.333333)
  (0.629630, 0.333333)
  (0.629630, 0.370370)
  (0.592593, 0.370370)
  (0.592593, 0.333333)
  (0.555556, 0.333333)
  (0.555556, 0.370370)
  (0.592593, 0.370370)
  (0.592593, 0.407407)
  (0.555556, 0.407407)
  (0.555556, 0.444444)
  (0.518519, 0.444444)
  (0.518519, 0.481481)
  (0.481481, 0.481481)
  (0.481481, 0.444444)
  (0.444444, 0.444444)
  (0.444444, 0.407407)
  (0.407407, 0.407407)
  (0.407407, 0.370370)
  (0.444444, 0.370370)
  (0.444444, 0.333333)
  (0.407407, 0.333333)
  (0.407407, 0.370370)
  (0.370370, 0.370370)
  (0.370370, 0.333333)
  (0.333333, 0.333333)
  (0.333333, 0.296296)
  (0.296296, 0.296296)
  (0.296296, 0.259259)
  (0.333333, 0.259259)
  (0.333333, 0.222222)
  (0.296296, 0.222222)
  (0.296296, 0.259259)
  (0.259259, 0.259259)
  (0.259259, 0.222222)
  (0.222222, 0.222222)
  (0.222222, 0.185185)
  (0.185185, 0.185185)
  (0.185185, 0.148148)
  (0.222222, 0.148148)
  (0.222222, 0.111111)
  (0.259259, 0.111111)
  (0.259259, 0.074074)
  (0.296296, 0.074074)
  (0.296296, 0.111111)
  (0.333333, 0.111111)
  (0.333333, 0.074074)
  (0.296296, 0.074074)
  (0.296296, 0.037037)
  (0.333333, 0.037037)
  (0.333333, -0.000000)
  (0.296296, -0.000000)
  (0.296296, 0.037037)
  (0.259259, 0.037037)
  (0.259259, -0.000000)
  (0.222222, -0.000000)
  (0.222222, 0.037037)
  (0.259259, 0.037037)
  (0.259259, 0.074074)
  (0.222222, 0.074074)
  (0.222222, 0.111111)
  (0.185185, 0.111111)
  (0.185185, 0.148148)
  (0.148148, 0.148148)
  (0.148148, 0.111111)
  (0.111111, 0.111111)
  (0.111111, 0.074074)
  (0.074074, 0.074074)
  (0.074074, 0.037037)
  (0.111111, 0.037037)
  (0.111111, -0.000000)
  (0.074074, -0.000000)
  (0.074074, 0.037037)
  (0.037037, 0.037037)
  (0.037037, -0.000000)
  (-0.000000, -0.000000)
}
\end{pspicture}
%
\hskip0.1in\dots
%

\end{zztask}

%%%%%%%%%%%%%%%%%%%%%%%%%%%%%%%%%%%%%%%%%%%%%%%%%%%%%%%%%%%%%%%%%%%%%%%%%%%%%%

\begin{zztask}[Fudge{f}lake]
В рамках общего условия задачи построить кривую снежинку. База для построения
кривой (нулевое приближение) представляет собой равносторонний треугольник.
Переход от приближения
$n$ к приближению $(n+1)$ осуществляется заменой каждого отрезка фигуры на два
(см. рис. при $n=1$, выделено жирным). Все отрезки
имеют одинаковую длину, угол между отрезками \mbox{составляет $120^\circ$}.
В дальнейшем, выгиб происходит чередуясь, то в одну, то в другую сторону 
(см. рис. при $n=2$).
\par\endinput
\begin{pspicture}(-0.5,-0.5)(1.5,1.5)
\psgrid
\rput(0.5,-0.25){$n=0$}
\rput(0,0.21){
\psline
  (0, 0)
  (1.000000, 0.000000)
  (0.500000, 0.866025)
  (-0.000000, 0.000000)
\psline[linewidth=0.06cm, arrows=o-o]
  (0.500000, 0.866025)
  (-0.000000, 0.000000)
}
\end{pspicture}
%
\hskip0.1in
%
\begin{pspicture}(-0.5,-0.5)(1.5,1.5)
\psgrid
\rput(0.5,-0.25){$n=1$}
\rput(0,0.21){
\psline[linewidth=0.01cm, linestyle=dashed]
  (0, 0)
  (1.000000, 0.000000)
  (0.500000, 0.866025)
  (-0.000000, 0.000000)
\psline
  (0, 0)
  (0.500000, -0.288675)
  (1.000000, 0.000000)
  (1.000000, 0.577350)
  (0.500000, 0.866025)
  (-0.000000, 0.577350)
  (-0.000000, 0.000000)
\psline[linewidth=0.06cm, arrows=o-o, showpoints=true]
  (0.500000, 0.866025)
  (-0.000000, 0.577350)
  (-0.000000, 0.000000)
}
\end{pspicture}
%
\hskip0.1in
%
\begin{pspicture}(-0.5,-0.5)(1.5,1.5)
\psgrid
\rput(0.5,-0.25){$n=2$}
\rput(0,0.21){
\psline[linewidth=0.01cm, linestyle=dashed]
  (0, 0)
  (0.500000, -0.288675)
  (1.000000, 0.000000)
  (1.000000, 0.577350)
  (0.500000, 0.866025)
  (-0.000000, 0.577350)
  (-0.000000, 0.000000)
\psline
  (0, 0)
  (0.333333, 0.000000)
  (0.500000, -0.288675)
  (0.833333, -0.288675)
  (1.000000, -0.000000)
  (0.833333, 0.288675)
  (1.000000, 0.577350)
  (0.833333, 0.866025)
  (0.500000, 0.866025)
  (0.333333, 0.577350)
  (-0.000000, 0.577350)
  (-0.166667, 0.288675)
  (-0.000001, 0.000000)
\psline[linewidth=0.06cm, arrows=o-o, showpoints=true]
  (0.500000, 0.866025)
  (0.333333, 0.577350)
  (-0.000000, 0.577350)
\psline[linewidth=0.06cm, arrows=o-o, showpoints=true]
  (-0.000000, 0.577350)
  (-0.166667, 0.288675)
  (-0.000001, 0.000000)
}
\end{pspicture}
%
\hskip0.1in
%
\begin{pspicture}(-0.5,-0.5)(1.5,1.5)
\psgrid
\rput(0.5,-0.25){$n=3$}
\rput(0,0.21){
\psline
  (0, 0)
  (0.166667, 0.096225)
  (0.333333, -0.000000)
  (0.333333, -0.192450)
  (0.500000, -0.288675)
  (0.666667, -0.192450)
  (0.833333, -0.288675)
  (1.000000, -0.192450)
  (1.000000, -0.000000)
  (0.833333, 0.096225)
  (0.833333, 0.288675)
  (1.000000, 0.384900)
  (1.000000, 0.577350)
  (0.833333, 0.673575)
  (0.833333, 0.866025)
  (0.666667, 0.962250)
  (0.500000, 0.866025)
  (0.500000, 0.673575)
  (0.333333, 0.577350)
  (0.166667, 0.673575)
  (-0.000000, 0.577350)
  (-0.000000, 0.384900)
  (-0.166667, 0.288675)
  (-0.166667, 0.096225)
  (-0.000000, -0.000000)
}
\end{pspicture}
%
\hskip0.1in\dots\hskip0.1in
%
\begin{pspicture}(-0.5,-0.5)(1.5,1.5)
\psgrid
\rput(0.5,-0.25){$n=8$}
\rput(0,0.21){
\psline
  (0, 0)
  (-0.012346, 0.000000)
  (-0.018519, 0.010692)
  (-0.012346, 0.021383)
  (-0.018519, 0.032075)
  (-0.012346, 0.042767)
  (-0.000000, 0.042767)
  (0.006173, 0.053458)
  (-0.000000, 0.064150)
  (0.006173, 0.074842)
  (0.018519, 0.074842)
  (0.024691, 0.064150)
  (0.037037, 0.064150)
  (0.043210, 0.074842)
  (0.055556, 0.074842)
  (0.061728, 0.085533)
  (0.055556, 0.096225)
  (0.061728, 0.106917)
  (0.074074, 0.106917)
  (0.080247, 0.096225)
  (0.092593, 0.096225)
  (0.098765, 0.085533)
  (0.092593, 0.074842)
  (0.098765, 0.064150)
  (0.111111, 0.064150)
  (0.117284, 0.074842)
  (0.129630, 0.074842)
  (0.135802, 0.064150)
  (0.148148, 0.064150)
  (0.154321, 0.074842)
  (0.166667, 0.074842)
  (0.172839, 0.085533)
  (0.166667, 0.096225)
  (0.172839, 0.106917)
  (0.185185, 0.106917)
  (0.191358, 0.096225)
  (0.203704, 0.096225)
  (0.209876, 0.085533)
  (0.203704, 0.074842)
  (0.209876, 0.064150)
  (0.222222, 0.064150)
  (0.228395, 0.053458)
  (0.222222, 0.042767)
  (0.209876, 0.042767)
  (0.203704, 0.032075)
  (0.209876, 0.021383)
  (0.203704, 0.010692)
  (0.209876, -0.000000)
  (0.222222, -0.000000)
  (0.228395, 0.010692)
  (0.240741, 0.010692)
  (0.246913, -0.000000)
  (0.259259, -0.000000)
  (0.265432, -0.010692)
  (0.259259, -0.021383)
  (0.265432, -0.032075)
  (0.277778, -0.032075)
  (0.283951, -0.021383)
  (0.296296, -0.021383)
  (0.302469, -0.032075)
  (0.314815, -0.032075)
  (0.320988, -0.021383)
  (0.333333, -0.021383)
  (0.339506, -0.010692)
  (0.333333, 0.000000)
  (0.339506, 0.010692)
  (0.351852, 0.010692)
  (0.358025, 0.000000)
  (0.370370, 0.000000)
  (0.376543, -0.010692)
  (0.370370, -0.021383)
  (0.376543, -0.032075)
  (0.388889, -0.032075)
  (0.395062, -0.042767)
  (0.388889, -0.053458)
  (0.376543, -0.053458)
  (0.370370, -0.064150)
  (0.376543, -0.074842)
  (0.370370, -0.085533)
  (0.376543, -0.096225)
  (0.388889, -0.096225)
  (0.395062, -0.106917)
  (0.388889, -0.117608)
  (0.376543, -0.117608)
  (0.370370, -0.128300)
  (0.358025, -0.128300)
  (0.351852, -0.117608)
  (0.339506, -0.117608)
  (0.333333, -0.128300)
  (0.339506, -0.138992)
  (0.333333, -0.149683)
  (0.320988, -0.149683)
  (0.314815, -0.160375)
  (0.320988, -0.171067)
  (0.314815, -0.181758)
  (0.320988, -0.192450)
  (0.333333, -0.192450)
  (0.339506, -0.181758)
  (0.351852, -0.181758)
  (0.358025, -0.192450)
  (0.370370, -0.192450)
  (0.376543, -0.203142)
  (0.370370, -0.213833)
  (0.376543, -0.224525)
  (0.388889, -0.224525)
  (0.395062, -0.235217)
  (0.388889, -0.245908)
  (0.376543, -0.245908)
  (0.370370, -0.256600)
  (0.376543, -0.267292)
  (0.370370, -0.277983)
  (0.376543, -0.288675)
  (0.388889, -0.288675)
  (0.395062, -0.277983)
  (0.407407, -0.277983)
  (0.413580, -0.288675)
  (0.425926, -0.288675)
  (0.432099, -0.299367)
  (0.425926, -0.310058)
  (0.432099, -0.320750)
  (0.444444, -0.320750)
  (0.450617, -0.310058)
  (0.462963, -0.310058)
  (0.469136, -0.320750)
  (0.481481, -0.320750)
  (0.487654, -0.310058)
  (0.500000, -0.310058)
  (0.506173, -0.299367)
  (0.500000, -0.288675)
  (0.487654, -0.288675)
  (0.481481, -0.277983)
  (0.487654, -0.267292)
  (0.481481, -0.256600)
  (0.487654, -0.245908)
  (0.500000, -0.245908)
  (0.506173, -0.235217)
  (0.500000, -0.224525)
  (0.506173, -0.213833)
  (0.518518, -0.213833)
  (0.524691, -0.224525)
  (0.537037, -0.224525)
  (0.543210, -0.213833)
  (0.555555, -0.213833)
  (0.561728, -0.203142)
  (0.555555, -0.192450)
  (0.561728, -0.181758)
  (0.574074, -0.181758)
  (0.580247, -0.192450)
  (0.592592, -0.192450)
  (0.598765, -0.203142)
  (0.592592, -0.213833)
  (0.598765, -0.224525)
  (0.611111, -0.224525)
  (0.617284, -0.213833)
  (0.629629, -0.213833)
  (0.635802, -0.224525)
  (0.648148, -0.224525)
  (0.654321, -0.213833)
  (0.666666, -0.213833)
  (0.672839, -0.203142)
  (0.666666, -0.192450)
  (0.672839, -0.181758)
  (0.685185, -0.181758)
  (0.691358, -0.192450)
  (0.703703, -0.192450)
  (0.709876, -0.203142)
  (0.703703, -0.213833)
  (0.709876, -0.224525)
  (0.722222, -0.224525)
  (0.728395, -0.235217)
  (0.722222, -0.245908)
  (0.709876, -0.245908)
  (0.703703, -0.256600)
  (0.709876, -0.267292)
  (0.703703, -0.277983)
  (0.709876, -0.288675)
  (0.722222, -0.288675)
  (0.728395, -0.277983)
  (0.740740, -0.277983)
  (0.746913, -0.288675)
  (0.759259, -0.288675)
  (0.765432, -0.299367)
  (0.759259, -0.310058)
  (0.765432, -0.320750)
  (0.777777, -0.320750)
  (0.783950, -0.310058)
  (0.796296, -0.310058)
  (0.802469, -0.320750)
  (0.814814, -0.320750)
  (0.820987, -0.310058)
  (0.833333, -0.310058)
  (0.839506, -0.299367)
  (0.833333, -0.288675)
  (0.820987, -0.288675)
  (0.814814, -0.277983)
  (0.820987, -0.267292)
  (0.814814, -0.256600)
  (0.820987, -0.245908)
  (0.833333, -0.245908)
  (0.839506, -0.235217)
  (0.833333, -0.224525)
  (0.839506, -0.213833)
  (0.851851, -0.213833)
  (0.858024, -0.224525)
  (0.870370, -0.224525)
  (0.876543, -0.213833)
  (0.888888, -0.213833)
  (0.895061, -0.203142)
  (0.888888, -0.192450)
  (0.895061, -0.181758)
  (0.907407, -0.181758)
  (0.913580, -0.192450)
  (0.925925, -0.192450)
  (0.932098, -0.203142)
  (0.925925, -0.213833)
  (0.932098, -0.224525)
  (0.944444, -0.224525)
  (0.950617, -0.213833)
  (0.962962, -0.213833)
  (0.969135, -0.224525)
  (0.981481, -0.224525)
  (0.987654, -0.213833)
  (0.999999, -0.213833)
  (1.006172, -0.203142)
  (0.999999, -0.192450)
  (0.987654, -0.192450)
  (0.981481, -0.181758)
  (0.987654, -0.171067)
  (0.981481, -0.160375)
  (0.987654, -0.149683)
  (0.999999, -0.149683)
  (1.006172, -0.138992)
  (0.999999, -0.128300)
  (1.006172, -0.117608)
  (1.018518, -0.117608)
  (1.024691, -0.128300)
  (1.037037, -0.128300)
  (1.043209, -0.117608)
  (1.055555, -0.117608)
  (1.061728, -0.106917)
  (1.055555, -0.096225)
  (1.043209, -0.096225)
  (1.037037, -0.085533)
  (1.043209, -0.074842)
  (1.037037, -0.064150)
  (1.043209, -0.053458)
  (1.055555, -0.053458)
  (1.061728, -0.042767)
  (1.055555, -0.032075)
  (1.043209, -0.032075)
  (1.037037, -0.021383)
  (1.043209, -0.010692)
  (1.037037, 0.000000)
  (1.024691, 0.000000)
  (1.018518, 0.010692)
  (1.006172, 0.010692)
  (0.999999, 0.000000)
  (1.006172, -0.010692)
  (0.999999, -0.021383)
  (0.987654, -0.021383)
  (0.981481, -0.032075)
  (0.969135, -0.032075)
  (0.962962, -0.021383)
  (0.950617, -0.021383)
  (0.944444, -0.032075)
  (0.932098, -0.032075)
  (0.925925, -0.021383)
  (0.932098, -0.010692)
  (0.925925, 0.000000)
  (0.913580, 0.000000)
  (0.907407, 0.010692)
  (0.895061, 0.010692)
  (0.888888, 0.000000)
  (0.876543, 0.000000)
  (0.870370, 0.010692)
  (0.876543, 0.021383)
  (0.870370, 0.032075)
  (0.876543, 0.042767)
  (0.888888, 0.042767)
  (0.895061, 0.053458)
  (0.888888, 0.064150)
  (0.876543, 0.064150)
  (0.870370, 0.074842)
  (0.876543, 0.085533)
  (0.870370, 0.096225)
  (0.858024, 0.096225)
  (0.851851, 0.106917)
  (0.839506, 0.106917)
  (0.833333, 0.096225)
  (0.820987, 0.096225)
  (0.814814, 0.106917)
  (0.820987, 0.117608)
  (0.814814, 0.128300)
  (0.820987, 0.138992)
  (0.833333, 0.138992)
  (0.839506, 0.149683)
  (0.833333, 0.160375)
  (0.839506, 0.171067)
  (0.851851, 0.171067)
  (0.858024, 0.160375)
  (0.870370, 0.160375)
  (0.876543, 0.171067)
  (0.888888, 0.171067)
  (0.895061, 0.181758)
  (0.888888, 0.192450)
  (0.876543, 0.192450)
  (0.870370, 0.203142)
  (0.876543, 0.213833)
  (0.870370, 0.224525)
  (0.876543, 0.235217)
  (0.888888, 0.235217)
  (0.895061, 0.245908)
  (0.888888, 0.256600)
  (0.876543, 0.256600)
  (0.870370, 0.267292)
  (0.876543, 0.277984)
  (0.870370, 0.288675)
  (0.858024, 0.288675)
  (0.851851, 0.299367)
  (0.839506, 0.299367)
  (0.833333, 0.288675)
  (0.820987, 0.288675)
  (0.814814, 0.299367)
  (0.820987, 0.310059)
  (0.814814, 0.320750)
  (0.820987, 0.331442)
  (0.833333, 0.331442)
  (0.839506, 0.342134)
  (0.833333, 0.352825)
  (0.839506, 0.363517)
  (0.851851, 0.363517)
  (0.858024, 0.352825)
  (0.870370, 0.352825)
  (0.876543, 0.363517)
  (0.888888, 0.363517)
  (0.895061, 0.374209)
  (0.888888, 0.384900)
  (0.895061, 0.395592)
  (0.907407, 0.395592)
  (0.913580, 0.384900)
  (0.925925, 0.384900)
  (0.932098, 0.374209)
  (0.925925, 0.363517)
  (0.932098, 0.352825)
  (0.944444, 0.352825)
  (0.950617, 0.363517)
  (0.962962, 0.363517)
  (0.969135, 0.352825)
  (0.981481, 0.352825)
  (0.987654, 0.363517)
  (0.999999, 0.363517)
  (1.006172, 0.374209)
  (0.999999, 0.384900)
  (0.987654, 0.384900)
  (0.981481, 0.395592)
  (0.987654, 0.406284)
  (0.981481, 0.416975)
  (0.987654, 0.427667)
  (0.999999, 0.427667)
  (1.006172, 0.438359)
  (0.999999, 0.449050)
  (1.006172, 0.459742)
  (1.018518, 0.459742)
  (1.024691, 0.449050)
  (1.037037, 0.449050)
  (1.043209, 0.459742)
  (1.055555, 0.459742)
  (1.061728, 0.470434)
  (1.055555, 0.481125)
  (1.043209, 0.481125)
  (1.037037, 0.491817)
  (1.043209, 0.502509)
  (1.037037, 0.513200)
  (1.043209, 0.523892)
  (1.055555, 0.523892)
  (1.061728, 0.534584)
  (1.055555, 0.545275)
  (1.043209, 0.545275)
  (1.037037, 0.555967)
  (1.043209, 0.566659)
  (1.037037, 0.577350)
  (1.024691, 0.577350)
  (1.018518, 0.588042)
  (1.006172, 0.588042)
  (0.999999, 0.577350)
  (1.006172, 0.566658)
  (0.999999, 0.555967)
  (0.987654, 0.555967)
  (0.981481, 0.545275)
  (0.969135, 0.545275)
  (0.962962, 0.555967)
  (0.950617, 0.555967)
  (0.944444, 0.545275)
  (0.932098, 0.545275)
  (0.925925, 0.555967)
  (0.932098, 0.566658)
  (0.925925, 0.577350)
  (0.913580, 0.577350)
  (0.907407, 0.588042)
  (0.895061, 0.588042)
  (0.888888, 0.577350)
  (0.876543, 0.577350)
  (0.870370, 0.588042)
  (0.876543, 0.598733)
  (0.870370, 0.609425)
  (0.876543, 0.620117)
  (0.888888, 0.620117)
  (0.895061, 0.630808)
  (0.888888, 0.641500)
  (0.876543, 0.641500)
  (0.870370, 0.652192)
  (0.876543, 0.662883)
  (0.870370, 0.673575)
  (0.858024, 0.673575)
  (0.851851, 0.684267)
  (0.839506, 0.684267)
  (0.833333, 0.673575)
  (0.820987, 0.673575)
  (0.814814, 0.684267)
  (0.820987, 0.694958)
  (0.814814, 0.705650)
  (0.820987, 0.716341)
  (0.833333, 0.716341)
  (0.839506, 0.727033)
  (0.833333, 0.737725)
  (0.839506, 0.748416)
  (0.851851, 0.748416)
  (0.858024, 0.737725)
  (0.870370, 0.737725)
  (0.876543, 0.748416)
  (0.888888, 0.748416)
  (0.895061, 0.759108)
  (0.888888, 0.769800)
  (0.876543, 0.769800)
  (0.870370, 0.780491)
  (0.876543, 0.791183)
  (0.870370, 0.801875)
  (0.876543, 0.812566)
  (0.888888, 0.812566)
  (0.895061, 0.823258)
  (0.888888, 0.833950)
  (0.876543, 0.833950)
  (0.870370, 0.844641)
  (0.876543, 0.855333)
  (0.870370, 0.866024)
  (0.858024, 0.866024)
  (0.851851, 0.876716)
  (0.839506, 0.876716)
  (0.833333, 0.866024)
  (0.839506, 0.855333)
  (0.833333, 0.844641)
  (0.820987, 0.844641)
  (0.814814, 0.833949)
  (0.802469, 0.833949)
  (0.796296, 0.844641)
  (0.783950, 0.844641)
  (0.777777, 0.833949)
  (0.765432, 0.833949)
  (0.759259, 0.844641)
  (0.765432, 0.855333)
  (0.759259, 0.866024)
  (0.746913, 0.866024)
  (0.740740, 0.876716)
  (0.728395, 0.876716)
  (0.722222, 0.866024)
  (0.709876, 0.866024)
  (0.703703, 0.876716)
  (0.709876, 0.887408)
  (0.703703, 0.898099)
  (0.709876, 0.908791)
  (0.722222, 0.908791)
  (0.728395, 0.919483)
  (0.722222, 0.930174)
  (0.709876, 0.930174)
  (0.703703, 0.940866)
  (0.709876, 0.951558)
  (0.703703, 0.962249)
  (0.691358, 0.962249)
  (0.685185, 0.972941)
  (0.672839, 0.972941)
  (0.666666, 0.962249)
  (0.672839, 0.951557)
  (0.666666, 0.940866)
  (0.654321, 0.940866)
  (0.648148, 0.930174)
  (0.635802, 0.930174)
  (0.629629, 0.940866)
  (0.617284, 0.940866)
  (0.611111, 0.930174)
  (0.598765, 0.930174)
  (0.592592, 0.940866)
  (0.598765, 0.951557)
  (0.592592, 0.962249)
  (0.580247, 0.962249)
  (0.574074, 0.972941)
  (0.561728, 0.972941)
  (0.555555, 0.962249)
  (0.561728, 0.951557)
  (0.555555, 0.940866)
  (0.543210, 0.940866)
  (0.537037, 0.930174)
  (0.524691, 0.930174)
  (0.518518, 0.940866)
  (0.506173, 0.940866)
  (0.500000, 0.930174)
  (0.506173, 0.919482)
  (0.500000, 0.908791)
  (0.487654, 0.908791)
  (0.481481, 0.898099)
  (0.487654, 0.887407)
  (0.481481, 0.876716)
  (0.487654, 0.866024)
  (0.500000, 0.866024)
  (0.506173, 0.876716)
  (0.518518, 0.876716)
  (0.524691, 0.866024)
  (0.537037, 0.866024)
  (0.543210, 0.855332)
  (0.537037, 0.844641)
  (0.543210, 0.833949)
  (0.555555, 0.833949)
  (0.561728, 0.823257)
  (0.555555, 0.812566)
  (0.543210, 0.812566)
  (0.537037, 0.801874)
  (0.543210, 0.791182)
  (0.537037, 0.780491)
  (0.543210, 0.769799)
  (0.555555, 0.769799)
  (0.561728, 0.759107)
  (0.555555, 0.748416)
  (0.543210, 0.748416)
  (0.537037, 0.737724)
  (0.524691, 0.737724)
  (0.518518, 0.748416)
  (0.506173, 0.748416)
  (0.500000, 0.737724)
  (0.506173, 0.727032)
  (0.500000, 0.716341)
  (0.487654, 0.716341)
  (0.481481, 0.705649)
  (0.487654, 0.694957)
  (0.481481, 0.684266)
  (0.487654, 0.673574)
  (0.500000, 0.673574)
  (0.506173, 0.662882)
  (0.500000, 0.652191)
  (0.487654, 0.652191)
  (0.481481, 0.641499)
  (0.469136, 0.641499)
  (0.462963, 0.652191)
  (0.450617, 0.652191)
  (0.444444, 0.641499)
  (0.432099, 0.641499)
  (0.425926, 0.652191)
  (0.432099, 0.662882)
  (0.425926, 0.673574)
  (0.413580, 0.673574)
  (0.407407, 0.684266)
  (0.395062, 0.684265)
  (0.388889, 0.673574)
  (0.395062, 0.662882)
  (0.388889, 0.652190)
  (0.376543, 0.652191)
  (0.370370, 0.641499)
  (0.358025, 0.641499)
  (0.351852, 0.652191)
  (0.339506, 0.652191)
  (0.333333, 0.641499)
  (0.339506, 0.630807)
  (0.333333, 0.620115)
  (0.320988, 0.620116)
  (0.314815, 0.609424)
  (0.320988, 0.598732)
  (0.314815, 0.588041)
  (0.320988, 0.577349)
  (0.333333, 0.577349)
  (0.339506, 0.566657)
  (0.333333, 0.555965)
  (0.320988, 0.555965)
  (0.314815, 0.545274)
  (0.302469, 0.545274)
  (0.296296, 0.555966)
  (0.283951, 0.555965)
  (0.277778, 0.545274)
  (0.265432, 0.545274)
  (0.259259, 0.555966)
  (0.265432, 0.566657)
  (0.259259, 0.577349)
  (0.246913, 0.577349)
  (0.240741, 0.588040)
  (0.228395, 0.588040)
  (0.222222, 0.577349)
  (0.209876, 0.577349)
  (0.203704, 0.588040)
  (0.209876, 0.598732)
  (0.203704, 0.609424)
  (0.209876, 0.620115)
  (0.222222, 0.620115)
  (0.228395, 0.630807)
  (0.222222, 0.641499)
  (0.209876, 0.641499)
  (0.203704, 0.652190)
  (0.209876, 0.662882)
  (0.203704, 0.673574)
  (0.191358, 0.673574)
  (0.185185, 0.684265)
  (0.172839, 0.684265)
  (0.166667, 0.673574)
  (0.172839, 0.662882)
  (0.166667, 0.652190)
  (0.154321, 0.652190)
  (0.148148, 0.641499)
  (0.135802, 0.641499)
  (0.129630, 0.652190)
  (0.117284, 0.652190)
  (0.111111, 0.641499)
  (0.098765, 0.641499)
  (0.092593, 0.652190)
  (0.098765, 0.662882)
  (0.092593, 0.673574)
  (0.080247, 0.673573)
  (0.074074, 0.684265)
  (0.061728, 0.684265)
  (0.055556, 0.673573)
  (0.061728, 0.662882)
  (0.055556, 0.652190)
  (0.043210, 0.652190)
  (0.037037, 0.641498)
  (0.024691, 0.641499)
  (0.018518, 0.652190)
  (0.006173, 0.652190)
  (-0.000000, 0.641498)
  (0.006173, 0.630807)
  (-0.000000, 0.620115)
  (-0.012346, 0.620115)
  (-0.018519, 0.609423)
  (-0.012346, 0.598732)
  (-0.018519, 0.588040)
  (-0.012346, 0.577348)
  (-0.000000, 0.577348)
  (0.006173, 0.588040)
  (0.018518, 0.588040)
  (0.024691, 0.577348)
  (0.037037, 0.577348)
  (0.043210, 0.566657)
  (0.037037, 0.555965)
  (0.043210, 0.545273)
  (0.055555, 0.545273)
  (0.061728, 0.534582)
  (0.055555, 0.523890)
  (0.043210, 0.523890)
  (0.037037, 0.513198)
  (0.043210, 0.502507)
  (0.037037, 0.491815)
  (0.043210, 0.481123)
  (0.055555, 0.481123)
  (0.061728, 0.470432)
  (0.055555, 0.459740)
  (0.043210, 0.459740)
  (0.037037, 0.449048)
  (0.024691, 0.449048)
  (0.018518, 0.459740)
  (0.006173, 0.459740)
  (-0.000000, 0.449048)
  (0.006173, 0.438357)
  (-0.000000, 0.427665)
  (-0.012346, 0.427665)
  (-0.018519, 0.416973)
  (-0.012346, 0.406282)
  (-0.018519, 0.395590)
  (-0.012346, 0.384898)
  (-0.000000, 0.384898)
  (0.006173, 0.374207)
  (-0.000000, 0.363515)
  (-0.012346, 0.363515)
  (-0.018519, 0.352823)
  (-0.030864, 0.352823)
  (-0.037037, 0.363515)
  (-0.049383, 0.363515)
  (-0.055556, 0.352823)
  (-0.067901, 0.352823)
  (-0.074074, 0.363515)
  (-0.067901, 0.374207)
  (-0.074074, 0.384898)
  (-0.086420, 0.384898)
  (-0.092593, 0.395590)
  (-0.104938, 0.395590)
  (-0.111111, 0.384898)
  (-0.104938, 0.374207)
  (-0.111111, 0.363515)
  (-0.123457, 0.363515)
  (-0.129630, 0.352823)
  (-0.141975, 0.352823)
  (-0.148148, 0.363515)
  (-0.160494, 0.363515)
  (-0.166667, 0.352823)
  (-0.160494, 0.342132)
  (-0.166667, 0.331440)
  (-0.179012, 0.331440)
  (-0.185185, 0.320748)
  (-0.179012, 0.310057)
  (-0.185185, 0.299365)
  (-0.179012, 0.288673)
  (-0.166667, 0.288673)
  (-0.160494, 0.299365)
  (-0.148148, 0.299365)
  (-0.141975, 0.288673)
  (-0.129630, 0.288673)
  (-0.123457, 0.277982)
  (-0.129630, 0.267290)
  (-0.123457, 0.256598)
  (-0.111111, 0.256598)
  (-0.104938, 0.245907)
  (-0.111111, 0.235215)
  (-0.123457, 0.235215)
  (-0.129630, 0.224523)
  (-0.123457, 0.213832)
  (-0.129630, 0.203140)
  (-0.123457, 0.192448)
  (-0.111111, 0.192448)
  (-0.104938, 0.181756)
  (-0.111111, 0.171065)
  (-0.123457, 0.171065)
  (-0.129630, 0.160373)
  (-0.141975, 0.160373)
  (-0.148148, 0.171065)
  (-0.160494, 0.171065)
  (-0.166667, 0.160373)
  (-0.160494, 0.149681)
  (-0.166667, 0.138990)
  (-0.179013, 0.138990)
  (-0.185185, 0.128298)
  (-0.179013, 0.117606)
  (-0.185185, 0.106915)
  (-0.179013, 0.096223)
  (-0.166667, 0.096223)
  (-0.160494, 0.106915)
  (-0.148148, 0.106915)
  (-0.141975, 0.096223)
  (-0.129630, 0.096223)
  (-0.123457, 0.085531)
  (-0.129630, 0.074840)
  (-0.123457, 0.064148)
  (-0.111111, 0.064148)
  (-0.104938, 0.053456)
  (-0.111111, 0.042765)
  (-0.123457, 0.042765)
  (-0.129630, 0.032073)
  (-0.123457, 0.021381)
  (-0.129630, 0.010690)
  (-0.123457, -0.000002)
  (-0.111111, -0.000002)
  (-0.104938, 0.010690)
  (-0.092593, 0.010690)
  (-0.086420, -0.000002)
  (-0.074074, -0.000002)
  (-0.067901, -0.010694)
  (-0.074074, -0.021385)
  (-0.067901, -0.032077)
  (-0.055556, -0.032077)
  (-0.049383, -0.021385)
  (-0.037037, -0.021385)
  (-0.030864, -0.032077)
  (-0.018519, -0.032077)
  (-0.012346, -0.021385)
  (-0.000000, -0.021385)
  (0.006173, -0.010694)
  (-0.000000, -0.000002)
}
\end{pspicture}
%
\hskip0.1in\dots
%

\end{zztask}

%%%%%%%%%%%%%%%%%%%%%%%%%%%%%%%%%%%%%%%%%%%%%%%%%%%%%%%%%%%%%%%%%%%%%%%%%%%%%%


E. КЛЕТОЧНЫЕ АВТОМАТЫ И ИМИТАЦИОННЫЕ МОДЕЛИ

Пример задания

Общее условие

Клеточный автомат — это дискретная модель, представляющая собой бесконечную
регулярную (в данном случае квадратную) решетку из клеток, каждая из которых
может находиться в определенном состоянии (из заданного конечного набора).
Время в рассматриваемой системе тоже дискретно, состояние каждой клетки в
момент времени t есть функция от состояний ее соседей в момент времени t-1.
Вид функциональной зависимости длявсех клеток одинаков.

Для того, чтобы не иметь дела с бесконечным полем, часто рассматривают
конечное квадратное поле NxN с попарно склеенными сторонами (лево-право,
верх-низ). При этом образуется замкнутая тороидальная поверхность, по которой
можно бесконечно путешествовать в любом направлении.

Предлагается написать графическую программу, демонстрирующую развитие
(«жизнь») клеточного автомата на квадратной решетке с заданными правилами, в
зависимости от начальной конфигурации. Начальная конфигурация задается
пользователем — ему предлагается пустое поле (с клетками в нейтральном,
мертвом состоянии), на которое он может высаживать живые клетки (и удалять
непонравившиеся). Состояние клетки обычно обозначается ее цветом или простым
рисунком.

После того как начальная конфигурация задана, клавиша SPACE позволяет
переходить к следующему поколению (рассчитываемому автоматически по правилам).
В любой момент должно быть можно опять перейти в режим редактирования уже
текущей конфигурации с помощью клавиши ENTER, а потом продолжить наблюдения.
Выход из программы должен быть возможен в любое время по клавише ESC.

%%%%%%%%%%%%%%%%%%%%%%%%%%%%%%%%%%%%%%%%%%%%%%%%%%%%%%%%%%%%%%%%%%%%%%%%%%%%%%

Вариант E-1. «Жизнь» Конуэя

В рамках общего условия задачи написать программу моделирующую следующий
клеточный автомат. В некоторых клетках прямоугольной таблицы содержатся
бактерии (одна штука в одной клетке). Соседними являются клетки,
соприкасающиеся хотя бы одним углом (вершиной). Следующее поколение образуется
из предыдущего по правилам:

• бактерия выживает, если у нее ровно два или три соседа (из восьми соседних
  клеток); в противном случае она гибнет от одиночества или от перенаселения;
• бактерия рождается в пустой клетке, если на соседних клетках ровно три
  бактерии.

%%%%%%%%%%%%%%%%%%%%%%%%%%%%%%%%%%%%%%%%%%%%%%%%%%%%%%%%%%%%%%%%%%%%%%%%%%%%%%

Вариант E-2. Ресурсы

В рамках общего условия задачи написать программу моделирующую следующий
клеточный автомат. В некоторых клетках прямоугольной таблицы содержатся
бактерии (одна штука в одной клетке). В каждой клетке также присутствует
"ресурс" питания. Появившись бактерия за один ход съедает все ресурсы в восьми
соседних с ней клетках и по окончании хода умирает. В новом поколении бактерии
появятся в тех клетках, рядом с которыми были бактерии на предыдущем ходу и
рядом с которыми есть несъеденный ресурс. При обновлении поколений бактерий
происходит и обновление всех ресурсов.

%%%%%%%%%%%%%%%%%%%%%%%%%%%%%%%%%%%%%%%%%%%%%%%%%%%%%%%%%%%%%%%%%%%%%%%%%%%%%%

Вариант E-3. Стрелки

В рамках общего условия задачи написать программу моделирующую следующий
клеточный автомат. Каждая клетка прямоугольной таблицы содержит стрелку,
указывающую в одном из восьми направлений (влево и вверх, вверх, вправо и
вверх, влево, вправо, влево и вниз, вниз, вправо и вниз). В следующем
поколении направление стрелочки каждого поля меняется на направление той
соседней, на которую она указывала.

%%%%%%%%%%%%%%%%%%%%%%%%%%%%%%%%%%%%%%%%%%%%%%%%%%%%%%%%%%%%%%%%%%%%%%%%%%%%%%

Вариант E-4. «Лишай» Ван Тассела

В рамках общего условия задачи написать программу моделирующую следующий
клеточный автомат. Каждая клетка прямоугольной таблицы представляет собой
клетку кожи, которая может быть здоровой, зараженной стригущим лишаем или
невосприимчивой к инфекции. В каждый такт времени зараженная клетка может с
вероятностью 0.5 заразить каждую из соседних здоровых клеток. Через шесть
единиц времени зараженная клетка становится невосприимчивой к инфекции.
Возникший иммунитет действует в течение последующих четырех единиц времени, а
затем клетка выздоравливает.

%%%%%%%%%%%%%%%%%%%%%%%%%%%%%%%%%%%%%%%%%%%%%%%%%%%%%%%%%%%%%%%%%%%%%%%%%%%%%%

Вариант E-5. «Муравей» Лэнгтона

В рамках общего условия задачи написать программу моделирующую следующий
клеточный автомат. Клетки на плоскости раскрашены в черный и белый цвета. В
одну клетку помещается муравей, который может передвигаться в одном из четырех
направлений. Муравей двигается по следующим правилам:

• в черной клетке: повернуться направо, изменить цвет клетки и передвинуться
  вперед;
• в белой клетке: повернуться налево, изменить цвет клетки и передвинуться
  вперед.

%%%%%%%%%%%%%%%%%%%%%%%%%%%%%%%%%%%%%%%%%%%%%%%%%%%%%%%%%%%%%%%%%%%%%%%%%%%%%%

Вариант E-6. Лист

В рамках общего условия задачи написать программу моделирующую следующий
клеточный автомат. Развитие клеточного автомата начинается с единственной
исходной клетки, прилегающей своей нижней гранью к особой точке прикрепления.
В дальнейшем рост в этом направлении не происходит. На следующем шаге исходная
клетка начинает «делиться». Деление происходит в трех свободных направлениях —
вправо, вверх и влево. Далее каждая из вновь рожденных клеток также делится. В
отличие от исходной, все прочие клетки могут размножаться в любом из четырех
направлений. Однако если на какую-либо из свободных ячеек претендуют сразу
несколько вновь образующихся клеток, то они взаимно уничтожают друг друга и
ячейка остается свободной. Затравочных точек может быть и несколько,
направление первоначального роста тоже можно варьировать.

%%%%%%%%%%%%%%%%%%%%%%%%%%%%%%%%%%%%%%%%%%%%%%%%%%%%%%%%%%%%%%%%%%%%%%%%%%%%%%

Вариант E-7. «Волчий остров» Ван Тассела

В рамках общего условия задачи написать программу моделирующую следующую
ситуацию. Тороидальный остров заселен дикими кроликами, волками и волчицами.
Имеется несколько представителей каждого вида. Кролики довольно глупы: в
каждый момент времени они с одинаковой вероятностью 1/9 передвигаются в один
из восьми соседних квадратов или просто сидят неподвижно. Каждый кролик с
вероятностью 0.2 превращается в двух кроликов. Волчицы передвигаются случайным
образом до тех пор, пока в одном из соседних восьми квадратов не окажется
кролик. Если волчица и кролик оказываются в одном квадрате, волчица съедает
кролика и получает одно очко, в противном случае она теряет 0.1 очка за каждую
единицу времени. Волки и волчицы с нулевым количеством очков умирают. В
начальный момент времени все волки и волчицы имеют 1 очко. Волк ведет себя
подобно волчице до тех пор, пока в соседних квадратах не исчезнут все кролики;
в этом случае, если волчица находится в одном из восьми ближайших квадратов,
волк гонится за ней. Если волк и волчица окажутся в одном квадрате, они
производят потомство случайного пола. Проследите, как сказываются на эволюции
популяции изменение различных параметров модели.

%%%%%%%%%%%%%%%%%%%%%%%%%%%%%%%%%%%%%%%%%%%%%%%%%%%%%%%%%%%%%%%%%%%%%%%%%%%%%%

Вариант E-8. «Аква-тор» Дьюдни

В рамках общего условия задачи написать программу моделирующую следующую
ситуацию. В тороидальном аквариуме в некоторых клетках содержатся особи одного
из двух видов: рыбы и акулы. Правила существования популяций (параметры
следует подобрать самостоятельно):

• на каждом ходу рыба перемещается случайным образом в одном из восьми
  направлений (если клетка свободна), но из-за течения слева направо
  вероятность движения в правую сторону в два раза выше, чем в левую;
• акула, если видит в соседних клетках рыбу, перемещается туда и съедает ее
  (любую, если их несколько); если рыб нет, то направление движения случайно,
  течение на акул не влияет;
• если акула долго не может съесть рыбу, она погибает; время смерти является
  параметром;
• каждая особь периодически оставляет потомство а той клетке, из которой она
  переместилась; период является параметром; рыбы плодятся чаще чем акулы;
  период, через который акулы приносят потомство связан с временем голодной
  смерти (меньше на единицу).


%%%%%%%%%%%%%%%%%%%%%%%%%%%%%%%%%%%%%%%%%%%%%%%%%%%%%%%%%%%%%%%%%%%%%%%%%%%%%%
\zztaskgroup{PLT}{График функции}
%%%%%%%%%%%%%%%%%%%%%%%%%%%%%%%%%%%%%%%%%%%%%%%%%%%%%%%%%%%%%%%%%%%%%%%%%%%%%%

Написать программу, строящую график $y=f(x)$ --- функции одной переменной в заданной
области $x\in(a,b), y\in(c,d)$. Алгоритм следует оформить в виде отдельной
функции \textbf{в отдельном модуле} со своим заголовочным файлом.
Функция должна принимать в качестве параметров указатель на $f(x)$, границы по осям $x, y$
и прямоугольную область на экране для рисования.

Кроме самого графика (кривой линии, а не набора отдельных точек) функция должна 
рисовать прямоугольник, ограничивающий заданную область, и координатные оси, 
если они попадают в эту область.
И прямоугольник и оси должны быть подписаны и на них должны быть указаны крайние 
значения, а также нанесены риски, соответствующие большому 
шагу $\Delta = 1$ и маленькому шагу $\delta = 0.1$. Если масштаб выбран таким образом,
что риски (большие или маленькие) сливаются, то их показывать не надо. Если масштаб 
позволяет (риски достаточно далеко друг от друга), необходимо под (рядом с) риской 
указывать соответствующее значение координаты.

Программа должна уметь строить графики как непрерывных $(y=\sin(x))$, 
так и \textbf{разрывных} функций $(y=\lfloor x \rfloor)$,
функций, \textbf{обращающихся в бесконечность} $(y=1/x)$,
или \textbf{не полностью определенных} на заданном промежутке $(y=\sqrt x)$.
В качестве тестовых функций также можно взять:
%
\[
y = \sin(x)+2\cdot|\sin(x)|, \qquad
y = \sqrt{x^2-x-1}, \qquad
y = \frac{5(x-2)}{x^2}, \qquad
y = \frac{1+\cos x}{3-\sin x}.
\]

Внимание: алгоритм должен быть написан в общем виде, для построения различных 
функций. Для обработки исключительных ситуаций обратить внимание на: 
\texttt{matherr()}, \texttt{signal()}, \texttt{SIGFPE}.

%%%%%%%%%%%%%%%%%%%%%%%%%%%%%%%%%%%%%%%%%%%%%%%%%%%%%%%%%%%%%%%%%%%%%%%%%%%%%%
\zztaskgroup{GAM}{Мини-игры}
%%%%%%%%%%%%%%%%%%%%%%%%%%%%%%%%%%%%%%%%%%%%%%%%%%%%%%%%%%%%%%%%%%%%%%%%%%%%%%

В данной работе требуется написать программу, реализующую одну из известных 
графических компьютерных игр. Программа должна иметь простейший пользовательский 
интерфейс для взаимодействия и реализовывать обработку клавиш. Выход из игры 
должен осуществляться по клавише ESC, по F1 должна показываться помощь в виде 
списка клавиш игры. В некоторых играх уместна реализация игрового взаимодействия 
с помощью мыши. Для работы с графикой рекомендуется использовать средства WinAPI,
Open GL\zztodo{ссылки?} или графическую библиотеку labengine\zztodo{да?}. Использование
более продвинутых средств и библиотек приветствуется. Ниже перечислены 
приблизительные правила игр.

%%%%%%%%%%%%%%%%%%%%%%%%%%%%%%%%%%%%%%%%%%%%%%%%%%%%%%%%%%%%%%%%%%%%%%%%%%%%%%
\zzsectionCOMMENTS
%%%%%%%%%%%%%%%%%%%%%%%%%%%%%%%%%%%%%%%%%%%%%%%%%%%%%%%%%%%%%%%%%%%%%%%%%%%%%%

%%%%%%%%%%%%%%%%%%%%%%%%%%%%%%%%%%%%%%%%%%%%%%%%%%%%%%%%%%%%%%%%%%%%%%%%%%%%%%
\zzsectionPLAN
%%%%%%%%%%%%%%%%%%%%%%%%%%%%%%%%%%%%%%%%%%%%%%%%%%%%%%%%%%%%%%%%%%%%%%%%%%%%%%

%%%%%%%%%%%%%%%%%%%%%%%%%%%%%%%%%%%%%%%%%%%%%%%%%%%%%%%%%%%%%%%%%%%%%%%%%%%%%%
\zzsectionVARIATIONS
%%%%%%%%%%%%%%%%%%%%%%%%%%%%%%%%%%%%%%%%%%%%%%%%%%%%%%%%%%%%%%%%%%%%%%%%%%%%%%

\begin{zztask}[Tetris]
Фигуры тетрамино (всевозможные стыковки четырех квадратиков сторона к стороне)
в случайном порядке по очереди появляются наверху стакана и начинают медленно
падать вниз (дискретно). Долетая до дна или другого препятствия, фигура
останавливается, окаменевает и становится частью стакана. Пока фигура падает,
ей можно управлять: вращать или двигать вправо-влево в стакане, если позволяют
препятствия. Если после падения фигуры образуется непрерывный ряд
«окаменелостей» от левой до правой стенки стакана, этот ряд исчезает, все ряды
которые выше, сдвигаются вниз. Игра заканчивается, если очередная фигура не
может появиться наверху из-за препятствий. Очки начисляются пропорционально
количеству уничтоженных линий.
\end{zztask}

%%%%%%%%%%%%%%%%%%%%%%%%%%%%%%%%%%%%%%%%%%%%%%%%%%%%%%%%%%%%%%%%%%%%%%%%%%%%%%

\begin{zztask}[Columns]
Вертикальные столбики из трех квадратиков раскрашенных в случайные цвета (из
ограниченного набора, напр. три-четыре цвета) появляются наверху стакана и
начинают медленно падать вниз (дискретно). Долетая до дна или другого
препятствия, столбик останавливается. Пока столбик падает, им можно управлять:
двигать вправо-влево в стакане, если позволяют препятствия, и циклически
менять цвета. Если после падения столбика образуются непрерывные области из
трех или более квадратиков одного цвета, они исчезают. Все квадратики выше
исчезнувших падают по-отдельности на освободившиеся места. Повторно
производится проверка на области. Игра заканчивается, если очередная фигура не
может появиться наверху из-за препятствий. Очки начисляются пропорционально
количеству уничтоженных квадратиков и площади областей.
\end{zztask}

%%%%%%%%%%%%%%%%%%%%%%%%%%%%%%%%%%%%%%%%%%%%%%%%%%%%%%%%%%%%%%%%%%%%%%%%%%%%%%

\begin{zztask}[Gems]
Прямоугольное клеточное поле заполнено камушками разных цветов (набор из
четырех-пяти цветов). Игроку предоставляется возможность выбрать клетку, после
чего непрерывная область камушков одного цвета с выбранным (и прилегающая к
нему) исчезает (если ее площадь равна трем или больше). Все камушки выше
исчезнувших падают по-отдельности на освободившиеся места. Если образуется
пустой вертикальный ряд, то все ряды правее его сдвигаются влево. Игра (или
уровень игры) заканчивается если больше не удалить ни одну область (нет
смежных камней или они вообще закончились). Очки начисляются пропорционально
площади уничтоженных областей. За полную очистку дают бонусные очки.
\end{zztask}

%%%%%%%%%%%%%%%%%%%%%%%%%%%%%%%%%%%%%%%%%%%%%%%%%%%%%%%%%%%%%%%%%%%%%%%%%%%%%%

\begin{zztask}[Lines]
На квадратном клеточном поле в случайных местах находятся три разноцветных
шарика. Игрок делает ход, переставляя один шарик на другое место. Переместить
шарик можно только если существует путь из начальной в конечную точку (т.е.
шарик можно только «перекатывать»). Цель — выстроить 5 или более шариков
одного цвета в ряд, после чего они исчезают, а игроку дается дополнительный
ход. Если же ход игрока не привел к исчезновению пятерки, то на поле
выбрасываются еще три шарика (если автоматически получается ряд, то шарики
убираются с поля, но очки игроку не даются). Игра заканчивается, когда игрок
не может сделать ход (поле заполнено). Очки начисляются пропорционально
количеству убранных с поля шариков и длине убираемого ряда. За несколько
убранных рядов подряд начисляются бонус-очки.
\end{zztask}

%%%%%%%%%%%%%%%%%%%%%%%%%%%%%%%%%%%%%%%%%%%%%%%%%%%%%%%%%%%%%%%%%%%%%%%%%%%%%%

\begin{zztask}[Arcanoid]
Наверху игрового поля выстроена стенка из нескольких рядов кирпичей. Внизу
поля расположена ракетка, управляемая игроком. Ракетка может двигаться только
влево и вправо, отбивая мячик, летающий по полю и отскакивающий от границ поля
и от кирпичей. При ударе о кирпичи мячик их уничтожает. Если ракетка не
отбивает мячик, он улетает за нижнюю границу поля и теряется. В распоряжении
игрока три мячика. Игра заканчивается при потере всех мячиков. При очистке
всего поля происходит переход на следующий уровень. Можно ввести кирпичи
разной стойкости, которые разрушаются не от одного, а от двух, трех ударов.
Также можно в некоторых кирпичах прятать бонусы, которые после разрушения
выпадают и летят вниз. Если ракеткой поймать такой бонус, то можно заработать
очки, дополнительную жизнь, увеличить или уменьшить размер ракетки, увеличить
скорость мячика, сделать его всепробивающим (уничтожающим простые кирпичи без
отражения от них), размножить мячик...
\end{zztask}

%%%%%%%%%%%%%%%%%%%%%%%%%%%%%%%%%%%%%%%%%%%%%%%%%%%%%%%%%%%%%%%%%%%%%%%%%%%%%%

\begin{zztask}[Snake]
По клеточному игровому полю бегает маленький питон, которой управляет игрок с
помощью стрелок. На поле периодически появляются лягушки, которых должен есть
питон. Через некоторое время несъеденная лягушка исчезает. При съедании
лягушки питон увеличивается на одну клеточку. На поле также могут появляться
камни, съедание которых укорачивает питона. Также питон укорачивается при
длительном голодании. При врезании в стену (в границу поля) или кусании самого
себя питон умирает. Также на поле могут быть дополнительные стены, в
зависимости от уровня сложности. Через некоторое время (через N появившихся
лягушек) происходит переход на следующий уровень. Очки начисляются
пропорционально длине питона на момент конца уровня. Можно вводить
дополнительные денежные бонусы на поле.
\end{zztask}

%%%%%%%%%%%%%%%%%%%%%%%%%%%%%%%%%%%%%%%%%%%%%%%%%%%%%%%%%%%%%%%%%%%%%%%%%%%%%%


\endinput

%%%%%%%%%%%%%%%%%%%%%%%%%%%%%%%%%%%%%%%%%%%%%%%%%%%%%%%%%%%%%%%%%%%%%%%%%%%%%%
\chapter{На доработку}
%%%%%%%%%%%%%%%%%%%%%%%%%%%%%%%%%%%%%%%%%%%%%%%%%%%%%%%%%%%%%%%%%%%%%%%%%%%%%%

С этими задачами уже не столь очевидно, как надо поступить. Хотелось бы их
рано или поздно довести до ума и включить в общий котёл --- они пригодятся.
К сожалению, доработка требуется достаточно серъёзная.

%%%%%%%%%%%%%%%%%%%%%%%%%%%%%%%%%%%%%%%%%%%%%%%%%%%%%%%%%%%%%%%%%%%%%%%%%%%%%%
\zztaskgroup{PLY}{Полиномы}
%%%%%%%%%%%%%%%%%%%%%%%%%%%%%%%%%%%%%%%%%%%%%%%%%%%%%%%%%%%%%%%%%%%%%%%%%%%%%%

Как я понимаю, эта завуалированная задачка на массивы.

В следующих задачах требуется написать программу, 

Примеры диалога программы и пользователя:

\begin{zzoutput}
  Задание \thezztaskgroup-1: Значение полинома
  Введите N: \zzuser{5}
  Случайный полином: 3x^5 + 2x^4 + 9x^2 + 7x + 5
  Введите x: \zzuser{0.0}
  Значение полинома при x = 0: 5.00000
\end{zzoutput}


\begin{zztask}[Значение полинома]
В рамках общего условия задачи составить случайный полином и вывести на экран
значение полинома в указанной пользователем точке $x$.
\end{zztask}

\begin{itemize}
\item изменить его коэффициенты на дополнения до 9 ($x^2+5x+3$ $\rightarrow$ $8x^2+4x+6$)
\item перевернуть его коэффициенты задом наперед($x^2+5x+3$ $\rightarrow$ $3x^2+5x+1$)
\item \dots
\end{itemize}

����� ������� ������ F: �����.

� ��������� ������� ������������ �������� � ���������� �������, � �����
������� �� ������� ������������. ���������� � ������ ������������� � �������,
�� ������ �� ���� ��������. ���� � ������ ��������� ��������, ������
����������� ���������� �����. ��������� ������� ��������� ���� ���� ������� �
������, ��������� ���� ������ � ������������ ������� �������� � ������
����������� ����, � �����, ��������� �� ������ ���� � ���������� ��������,
�������� �� ��� ���������� ����������.

��� ����� ����������� � ������������. �� ������� ���������� �� ��, ��� ���
������� ����� ����� � ���������� �������, ������ ������� ������� ������������
� �������� ������� ��������������. ������� ����� ��� �������� �����������
��������������.

%%%%%%%%%%%%%%%%%%%%%%%%%%%%%%%%%%%%%%%%%%%%%%%%%%%%%%%%%%%%%%%%%%%%%%%%%%%%%%

������ F-1.

� ������ ������ ������� ������ ��������� ����, ���������� � ����� �������
���������� � ������ ���������: ����� ������, ���� � ����� ������, �����
������� � ���� � ����� �������. ��������� ������������ ��������� �������
������������:

�	������� ��� ����� �� ��������� �����,
�	������� ��� ����� �� ���������� ������,
������� ��� ����� � ��������� ��������� ���������,
������� ��� ����� �������������� ������ ���������� �������.

%%%%%%%%%%%%%%%%%%%%%%%%%%%%%%%%%%%%%%%%%%%%%%%%%%%%%%%%%%%%%%%%%%%%%%%%%%%%%%

������ F-2.

� ������ ������ ������� ������ ��������� ����, ���������� � ����� �������
���������� �� ������������ ���������: �������, ���, ��������, ����� ������,
�������, ������� ������ �� ����� �������� (���� ��� ���������). ���������
������������ ��������� ������� ������������:

�	������� ���� ���������, ���������� ������ �� ���������� ��������,
�	������� �������� � ��������� ������� ������ �� ���������� ��������,
������� ���� ��������� �� ������� ������ ���� (����) ����������.
��� � �������� �������� �� ����� ������ ����������.

%%%%%%%%%%%%%%%%%%%%%%%%%%%%%%%%%%%%%%%%%%%%%%%%%%%%%%%%%%%%%%%%%%%%%%%%%%%%%%

������ F-3.

� ������ ������ ������� ������ ��������� ����, ���������� � ����� �������
���������� � ������������ �� ������� �������: �������������, ������, ���,
������, ���������, ���� ������ �������. ��������� ������������ ���������
������� ������������:

�	������� ��� ����������� ����� ����� ���������� �������������,
�	������� ��� ����������� � ��������� ������� ���������,
������� ��� �����������, ����������� � ���� ������ ���������� �������,
������� ����� ������ ������ �� ������� �������������.

%%%%%%%%%%%%%%%%%%%%%%%%%%%%%%%%%%%%%%%%%%%%%%%%%%%%%%%%%%%%%%%%%%%%%%%%%%%%%%

������ F-4.

� ������ ������ ������� ������ ��������� ����, ���������� � ����� �������
���������� � ���������� ����� ���������� ����� �����: �������, ���, ��������,
����, �����, ����������� �������, ����� ����������. ��������� ������������
��������� ������� ������������:

�	������� ���� ������������ �� ����������� � ��������� ������ �������,
�	������� ���� ����������� � ��������� ����� �� ��� �����.

%%%%%%%%%%%%%%%%%%%%%%%%%%%%%%%%%%%%%%%%%%%%%%%%%%%%%%%%%%%%%%%%%%%%%%%%%%%%%%

Имеется английский словарь. Пользователь вводит набор цифр, и программа должна
предложить ему набор слов той же длины, буквы в словах соответствуют цифрам
как на кнопках телефона. Например, числовой последовательности 2233
соответствует слово cafe и некоторые другие.

%%%%%%%%%%%%%%%%%%%%%%%%%%%%%%%%%%%%%%%%%%%%%%%%%%%%%%%%%%%%%%%%%%%%%%%%%%%%%%

Имеется английский словарь. Пользователь вводит набор букв, программа должна
выдать набор слов, которые можно составить из этих букв (возможно не используя
всех), отсортированный по суммарному количеству очков. Буквам назначаются
баллы от 1 до 5 в соответствие с частотой встречаемости (определить по словарю
и оценить самостоятельно). Например, набору букв “o a d r a b” соответствует
набор слов (неотсортированный) road, abroad, board, aboard, rad,… Решение
задачи составить из двух частей — построение по словарю промежуточной
структуры данных (придуманной самостоятельно для решения этой задачи), и
собственно осуществление запросов (возможно множественных). Во второй части
полный перебор всех слов словаря запрещен как неэффективный. Запрещена
рекурсия.

%%%%%%%%%%%%%%%%%%%%%%%%%%%%%%%%%%%%%%%%%%%%%%%%%%%%%%%%%%%%%%%%%%%%%%%%%%%%%%

Имеется английский словарь. Пользователь «задает» (можно рандомно) матрицу из
букв (порядка 20 на 20), программа находит в этой матрице слова (по прямой во
всех 8 направлениях) и выводит, сортируя по суммарному количеству очков за
буквы умноженному на длину слова. Буквам назначаются баллы от 1 до 5 в
соответствие с частотой встречаемости (определить по словарю и оценить
самостоятельно). Решение задачи составить из двух частей — построение по
словарю промежуточной структуры данных (придуманной самостоятельно для решения
этой задачи), и собственно осуществление запросов (возможно множественных). Во
второй части полный перебор всех слов словаря запрещен как неэффективный, но
перебор всех клеток неизбежен. Запрещена рекурсия.

%%%%%%%%%%%%%%%%%%%%%%%%%%%%%%%%%%%%%%%%%%%%%%%%%%%%%%%%%%%%%%%%%%%%%%%%%%%%%%

Имеется английский словарь. Пользователь задает матрицу из букв (порядка 6 на
6), программа находит в этой матрице слова (червяком, от буквы к букве
движение возможно во всех 8 направлениях, если там еще не были) и выводит,
сортируя по суммарному количеству очков за буквы, умноженному на длину слова.
Буквам назначаются баллы от 1 до 5 в соответствие с частотой встречаемости
(определить по словарю и оценить самостоятельно). Решение задачи составить из
двух частей — построение по словарю промежуточной структуры данных
(придуманной самостоятельно для решения этой задачи), и собственно
осуществление запросов (возможно множественных). Во второй части полный
перебор всех слов словаря запрещен как неэффективный, но перебор всех клеток
неизбежен. Запрещена рекурсия.

%%%%%%%%%%%%%%%%%%%%%%%%%%%%%%%%%%%%%%%%%%%%%%%%%%%%%%%%%%%%%%%%%%%%%%%%%%%%%%

Имеется английский словарь. Определить частоту встречаемости букв в словаре.

%%%%%%%%%%%%%%%%%%%%%%%%%%%%%%%%%%%%%%%%%%%%%%%%%%%%%%%%%%%%%%%%%%%%%%%%%%%%%%

Имеется английский словарь. Для пар букв определить частоту встречаемости их
друг за другом в словах. Какие буквосочетания самые «вероятные»?



%%%%%%%%%%%%%%%%%%%%%%%%%%%%%%%%%%%%%%%%%%%%%%%%%%%%%%%%%%%%%%%%%%%%%%%%%%%%%%
\zztaskgroup{CMD}{Командная строка}
%%%%%%%%%%%%%%%%%%%%%%%%%%%%%%%%%%%%%%%%%%%%%%%%%%%%%%%%%%%%%%%%%%%%%%%%%%%%%%

Написать консольную программу, принимающую на вход параметры командной строки
и выдающую результаты на экран. Заголовочный файл conio.h и функции из него
(для расширенной работы с экраном и клавиатурой в текстовом режиме, типа
gotoxy или getch) запрещены. Если программа вызвана с меньшим или большим
числом параметров чем нужно, выдать help. Если параметр один и он равен «/?»
или «-?», тоже вывести помощь.


%%%%%%%%%%%%%%%%%%%%%%%%%%%%%%%%%%%%%%%%%%%%%%%%%%%%%%%%%%%%%%%%%%%%%%%%%%%%%%

VIII-1 Поиск по маске

Параметры программы — стартовый каталог и набор масок файлов для поиска. Маска
кроме обычных букв может содержать спецсимволы * и ?. Вопросик обозначает, что
на этом месте может быть один любой символ, а звездочка обозначает любую
последовательность символов произвольной длины (в т.ч. и пустую). Искать надо
только в указанном каталоге (без подкаталогов). Использовать
findfirst()/findnext(). Имена найденных файлов вывести по одному в строчке.


%%%%%%%%%%%%%%%%%%%%%%%%%%%%%%%%%%%%%%%%%%%%%%%%%%%%%%%%%%%%%%%%%%%%%%%%%%%%%%

VIII-2 Поиск по маске в подкаталогах

Параметры программы — стартовый каталог и набор масок файлов для поиска. Маска
кроме обычных букв может содержать спецсимволы * и ?. Вопросик обозначает, что
на этом месте может быть один любой символ, а звездочка обозначает любую
последовательность символов произвольной длины (в т.ч. и пустую). Искать надо
в указанном каталоге и всех подкаталогах. Использовать findfirst()/findnext().
Имена найденных файлов вывести по одному в строчке, с полным путем и буквой
диска.


%%%%%%%%%%%%%%%%%%%%%%%%%%%%%%%%%%%%%%%%%%%%%%%%%%%%%%%%%%%%%%%%%%%%%%%%%%%%%%

VIII-3 Поиск строки в файлах

Параметры программы — стартовый каталог и подстрока для поиска. Искать
подстроку надо во всех файлах с расширением txt в указанном каталоге и всех
подкаталогах. Использовать findfirst()/findnext(). Вывести имена файлов с
относительным путем (от указанного каталога) и после двоеточия номера строк
(через запятую), в которых встретилась подстрока.


%%%%%%%%%%%%%%%%%%%%%%%%%%%%%%%%%%%%%%%%%%%%%%%%%%%%%%%%%%%%%%%%%%%%%%%%%%%%%%

VIII-4 Подсчет слов (wc)

Параметры программы — набор опций-ключей (начинаются с минуса) и одно или
несколько имен файлов. Вместо имени файла можно указывать одинокий знак минус,
или вообще опустить этот параметр, тогда читать надо не из файла, а со
стандартного ввода (stdin). Требуется вывести на одной строчке имя файла и
после двоеточия три числа, разделенных знаками табуляции: количество переводов
строк, количество слов, количество байт в файле. Если указано больше одного
файла, то вывести еще дополнительную строчку с суммарными значениями. Ключи: C
– выводить количество байт, L – количество строк, W – количество слов.


%%%%%%%%%%%%%%%%%%%%%%%%%%%%%%%%%%%%%%%%%%%%%%%%%%%%%%%%%%%%%%%%%%%%%%%%%%%%%%

VIII-5 Статистика кода

Параметры программы — набор стартовых каталогов. Требуется просмотреть все эти
каталоги с подкаталогами, найти файлы *.c, *.cpp *.h и вывести статистику про
каждый файл (по строчке на файл): размер файла в байтах, количество строк в
файле, объем коментариев в тексте в байтах и процентах от размера, средняя
длина строки в файле. Вывести общую статистику: суммарный объем кода в байтах,
суммарный объем коментариев в байтах и процентах, суммарное количество строк.
Вывести максимальное и среднее количество строк в файле по группам (по строчке
для .c, .h, .cpp).


%%%%%%%%%%%%%%%%%%%%%%%%%%%%%%%%%%%%%%%%%%%%%%%%%%%%%%%%%%%%%%%%%%%%%%%%%%%%%%

VIII-6 Dir

Написать программу-аналог команды dir, которая бы выводила информацию о файлах
в таком же виде. Поддержать два ключа: /B (без заголовков, только имена
файлов) и /W (вывод в несколько столбцов).


%%%%%%%%%%%%%%%%%%%%%%%%%%%%%%%%%%%%%%%%%%%%%%%%%%%%%%%%%%%%%%%%%%%%%%%%%%%%%%

VIII-7 Сравнение файлов

Написать программу сравнения двух файлов, имена которых передаются как
параметры. Необходимо проверить и выдать сообщение – файлы одинаковы или
различаются.


%%%%%%%%%%%%%%%%%%%%%%%%%%%%%%%%%%%%%%%%%%%%%%%%%%%%%%%%%%%%%%%%%%%%%%%%%%%%%%

VIII-8 Преобразование регистра

Написать программу преобразования текстового файла в заглавные или строчные
буквы. Программа принимает на входе имя файла и тип преобразования: /U -
преобразовать все буквы в заглавные, /L - преобразовать все буквы в строчные.
Новый файл под тем же самым именем должен быть записать вместо исходного.


%%%%%%%%%%%%%%%%%%%%%%%%%%%%%%%%%%%%%%%%%%%%%%%%%%%%%%%%%%%%%%%%%%%%%%%%%%%%%%

VIII-9 Шифрование

Написать программу шифрования/расшифровки текстового файла. Шифровка
осуществляется путем сдвигания букв на одну (замена ‘a’ на ‘b’, ‘b’ на ‘c’ и
т.д., т.е. увеличение ASCII кода символа на 1). Для шифрования файла
используется ключ /C, для дешифрования файла используется /D. Новый файл под
тем же самым именем должен быть записать вместо исходного.

При написании программы учтите, что файл должен остаться текстовым, то есть
символу 255 будет соответствовать символ 32, а служебные символы с кодами от 0
до 31 должны остаться без изменений.


%%%%%%%%%%%%%%%%%%%%%%%%%%%%%%%%%%%%%%%%%%%%%%%%%%%%%%%%%%%%%%%%%%%%%%%%%%%%%%

%%%%%%%%%%%%%%%%%%%%%%%%%%%%%%%%%%%%%%%%%%%%%%%%%%%%%%%%%%%%%%%%%%%%%%%%%%%%%%
\zztaskgroup{BIN}{Бинарные файлы}
%%%%%%%%%%%%%%%%%%%%%%%%%%%%%%%%%%%%%%%%%%%%%%%%%%%%%%%%%%%%%%%%%%%%%%%%%%%%%%

Эти задачки на работу с файлами как с бинарными данными: fopen() в режимах
“rb” / ”wb”, чтение и запись с помощью fread() / fwrite(). Дополнительные
функции fseek(), ftell().

%%%%%%%%%%%%%%%%%%%%%%%%%%%%%%%%%%%%%%%%%%%%%%%%%%%%%%%%%%%%%%%%%%%%%%%%%%%%%%

IX-1 Степень компрессии BMP/TIFF/…

Зная внутреннюю структуру файла, хранящего сжатое изображение в формате
BMP/TIFF/… (без знания алгоритма сжатия), определить степень компрессии
изображения как отношение полного размера файла к размеру памяти, необходимой
для хранения несжатого изображения (ширина * высота * <<байт на точку>> +
заголовок).

%%%%%%%%%%%%%%%%%%%%%%%%%%%%%%%%%%%%%%%%%%%%%%%%%%%%%%%%%%%%%%%%%%%%%%%%%%%%%%

IX-2 Гистограмма из BMP/TIFF/…

Зная внутреннюю структуру файла, хранящего изображение в формате BMP/TIFF/…
(только непакованное изображение), проанализировать содержимое и собрать
статистику: сколько пикселов изображения имеют яркость N (N=0..255) и
построить гистограмму (столбчатую диаграмму зависимости количества пикселов от
яркости). Яркость вычислять по формуле (R+G+B)/3, где R,G,B – красная, зеленая
и синяя компонента цвета пиксела.

%%%%%%%%%%%%%%%%%%%%%%%%%%%%%%%%%%%%%%%%%%%%%%%%%%%%%%%%%%%%%%%%%%%%%%%%%%%%%%

IX-3 Deinterlace BMP/TIFF/…

Зная внутреннюю структуру файла, хранящего изображение в формате BMP/TIFF/…
(только непакованное изображение), изменить изображение следующим образом:
собрать все четные строки изображения (0, 2, 4...) в верхней половине
изображения, а нечетные — в нижней. Новый файл под тем же самым именем должен
быть записать вместо исходного.

%%%%%%%%%%%%%%%%%%%%%%%%%%%%%%%%%%%%%%%%%%%%%%%%%%%%%%%%%%%%%%%%%%%%%%%%%%%%%%

IX-4 Interlace BMP/TIFF/…

Зная внутреннюю структуру файла, хранящего изображение в формате BMP/TIFF/…
(только непакованное изображение), изменить изображение следующим образом:
перемешать верхнюю и нижнюю половины изображения, перемежая строчки через
одну. Четные строки конечного изображения (0, 2, 4...) брать в верхней
половине изображения, а нечетные — в нижней. Новый файл под тем же самым
именем должен быть записать вместо исходного.

%%%%%%%%%%%%%%%%%%%%%%%%%%%%%%%%%%%%%%%%%%%%%%%%%%%%%%%%%%%%%%%%%%%%%%%%%%%%%%

IX-5 Rotate BMP/TIFF/…

Зная внутреннюю структуру файла, хранящего изображение в формате BMP/TIFF/…
(только непакованное изображение), повернуть изображение на 90 градусов против
часовой стрелки. Новый файл под тем же самым именем должен быть записать
вместо исходного.

%%%%%%%%%%%%%%%%%%%%%%%%%%%%%%%%%%%%%%%%%%%%%%%%%%%%%%%%%%%%%%%%%%%%%%%%%%%%%%

IX-6 Список файлов в ZIP/RAR/…

Зная внутренню структуру архива ZIP/RAR/… (без необходимости компрессии /
декомпрессии), вывести список файлов, находящихся в нем. Если задан ключ /m,
изменить в архиве регистр букв в именах файлов (заглавные на строчные и
наоборот).

%%%%%%%%%%%%%%%%%%%%%%%%%%%%%%%%%%%%%%%%%%%%%%%%%%%%%%%%%%%%%%%%%%%%%%%%%%%%%%

IX-7 Сохранение файлов в ZIP/RAR/…

Зная внутренню структуру архива ZIP/RAR/… (без необходимости компрессии /
декомпрессии), создать архив содержащий файлы, переданные как параметры.
Использовать степень компрессии 0 (store), т.е. не сжимать данные.

%%%%%%%%%%%%%%%%%%%%%%%%%%%%%%%%%%%%%%%%%%%%%%%%%%%%%%%%%%%%%%%%%%%%%%%%%%%%%%


%%%%%%%%%%%%%%%%%%%%%%%%%%%%%%%%%%%%%%%%%%%%%%%%%%%%%%%%%%%%%%%%%%%%%%%%%%%%%%
\chapter{Подумать} % TEMPORARY! TODO: chapter per file
%%%%%%%%%%%%%%%%%%%%%%%%%%%%%%%%%%%%%%%%%%%%%%%%%%%%%%%%%%%%%%%%%%%%%%%%%%%%%%

Эти задачи скорее всего уже морально устарели. Можно над ними помедитировать
и какие-то идеи перенести в существующие задачи.

%%%%%%%%%%%%%%%%%%%%%%%%%%%%%%%%%%%%%%%%%%%%%%%%%%%%%%%%%%%%%%%%%%%%%%%%%%%%%%
\zztaskgroup{ARY}{Одномерные массивы}
%%%%%%%%%%%%%%%%%%%%%%%%%%%%%%%%%%%%%%%%%%%%%%%%%%%%%%%%%%%%%%%%%%%%%%%%%%%%%%

Элементы языка: простые массивы, генератор псевдослучайных чисел.

Здесь приводятся упражнения на работу с одномерными массивами и индексацию в
массив. Показаны стандартные приемы нахождения минимального и максимального
элементов массива. Также вводится понятие генератора псевдослучайных чисел,
закрепляются навыки его использования.\zztodo{Не очень понятно, что даст эта лаба, как-то тут кучно все. Все классе тут не успеть нифига,а домашка вроде как для имбецилов получается. }


%%%%%%%%%%%%%%%%%%%%%%%%%%%%%%%%%%%%%%%%%%%%%%%%%%%%%%%%%%%%%%%%%%%%%%%%%%%%%%

Задача C-1

В начало массива из 20 элементов записать N случайных натуральных чисел из
диапазона от 1 до N, воспользовавшись функцией rand(). Число N вводится с
клавиатуры. Сосчитать сумму элементов. В другой массив записать суммы пар
элементов (N сумм): первого и последнего, второго и предпоследнего, третьего и
пред-предпоследнего... Вывести на экран результирующий массив и сумму его
элементов, а также минимальный и максимальный эл-т. Повторять выполнение
программы до тех пор, пока в качестве N не введут 0.\zztodo{Зачем дублировать парные суммы?}

Иванов И.И. (1057/1): Массивы, нат.случайные числа от 1 до N.
Введите размер (или 0 для завершения): 5
Сгенерированный массив A = { 4 3 1 4 5 }, сумма = 17
Результат B = { 9 7 2 7 9 }, сумма = 34
Минимум = 2, максимум = 9
Введите размер (или 0 для завершения): 0

%%%%%%%%%%%%%%%%%%%%%%%%%%%%%%%%%%%%%%%%%%%%%%%%%%%%%%%%%%%%%%%%%%%%%%%%%%%%%%

Задача C-2

В начало массива из 20 элементов записать N случайных натуральных чисел из
диапазона от 1 до N, воспользовавшись функцией rand(). Число N вводится с
клавиатуры. Сосчитать сумму элементов. В другой массив записать суммы пар
элементов (N сумм): первого и второго, второго и третьего, ..., последнего и
первого. Вывести на экран результирующий массив и сумму его элементов, а также
минимальный и максимальный эл-т. Повторять выполнение программы до тех пор,
пока в качестве N не введут 0.

Иванов И.И. (1057/1): Массивы, нат.случайные числа от 1 до N.
Введите размер (или 0 для завершения): 5
Сгенерированный массив A = { 4 3 1 4 5 }, сумма = 17
Результат B = { 7 4 5 9 9 }, сумма = 34
Минимум = 4, максимум = 9
Введите размер (или 0 для завершения): 0

%%%%%%%%%%%%%%%%%%%%%%%%%%%%%%%%%%%%%%%%%%%%%%%%%%%%%%%%%%%%%%%%%%%%%%%%%%%%%%

Задача C-3

В начало массива из 20 элементов записать N случайных целых чисел из диапазона
от A до B, воспользовавшись функцией rand(). Числа N, A, B вводятся с
клавиатуры. Сосчитать сумму элементов. В другой массив записать суммы пар
элементов (N сумм): первого и последнего, второго и предпоследнего, третьего и
пред-предпоследнего... Вывести на экран результирующий массив и сумму его
элементов, а также минимальный и максимальный эл-т. Повторять выполнение
программы до тех пор, пока в качестве N не введут 0.

Иванов И.И. (1057/1): Массивы, целые случайные числа от A до B.
Введите размер (или 0 для завершения): 5
Введите диапазон A B через пробел: 3 7
Сгенерированный массив A = { 4 3 7 6 5 }, сумма = 25
Результат B = { 9 9 14 9 9 }, сумма = 50
Минимум = 9, максимум = 14
Введите размер (или 0 для завершения): 0

%%%%%%%%%%%%%%%%%%%%%%%%%%%%%%%%%%%%%%%%%%%%%%%%%%%%%%%%%%%%%%%%%%%%%%%%%%%%%%

Задача C-4

В начало массива из 20 элементов записать N случайных целых чисел из диапазона
от A до B, воспользовавшись функцией rand(). Числа N, A, B вводятся с
клавиатуры. Сосчитать сумму элементов. В другой массив записать суммы пар
элементов (N сумм): первого и второго, второго и третьего, ..., последнего и
первого. Вывести на экран результирующий массив и сумму его элементов, а также
минимальный и максимальный эл-т. Повторять выполнение программы до тех пор,
пока в качестве N не введут 0.

Иванов И.И. (1057/1): Массивы, целые случайные числа от A до B.
Введите размер (или 0 для завершения): 5
Введите диапазон A B через пробел: 3 7
Сгенерированный массив A = { 4 3 7 6 5 }, сумма = 25
Результат B = { 7 10 13 11 9 }, сумма = 50
Минимум = 7, максимум = 13
Введите размер (или 0 для завершения): 0

%%%%%%%%%%%%%%%%%%%%%%%%%%%%%%%%%%%%%%%%%%%%%%%%%%%%%%%%%%%%%%%%%%%%%%%%%%%%%%

Задача C-5

В начало массива из 20 элементов записать N случайных вещественных чисел из
диапазона от 0 до 1, воспользовавшись функцией rand(). Число N вводится с
клавиатуры. Сосчитать сумму элементов. В другой массив записать суммы пар
элементов (N сумм): первого и последнего, второго и предпоследнего, третьего и
пред-предпоследнего... Вывести на экран результирующий массив и сумму его
элементов, а также минимальный и максимальный эл-т. Повторять выполнение
программы до тех пор, пока в качестве N не введут 0.

Иванов И.И. (1057/1): Массивы, вещ. случайные числа от 0 до 1.
Введите размер (или 0 для завершения): 5
Сгенерированный массив A = { 0.43 0.31 0.74 0.16 0.50 }, сумма = 2.14
Результат B = { 0.93 0.47 1.48 0.47 0.93 }, сумма = 4.28
Минимум = 0.47, максимум = 1.48
Введите размер (или 0 для завершения): 0

%%%%%%%%%%%%%%%%%%%%%%%%%%%%%%%%%%%%%%%%%%%%%%%%%%%%%%%%%%%%%%%%%%%%%%%%%%%%%%

Задача C-6

В начало массива из 20 элементов записать N случайных вещественных чисел из
диапазона от 0 до 1, воспользовавшись функцией rand(). Число N вводится с
клавиатуры. Сосчитать сумму элементов. В другой массив записать суммы пар
элементов (N сумм): первого и второго, второго и третьего, ..., последнего и
первого. Вывести на экран результирующий массив и сумму его элементов, а также
минимальный и максимальный эл-т. Повторять выполнение программы до тех пор,
пока в качестве N не введут 0.

Иванов И.И. (1057/1): Массивы, вещ. случайные числа от 0 до 1.
Введите размер (или 0 для завершения): 5
Сгенерированный массив A = { 0.43 0.31 0.74 0.16 0.50 }, сумма = 2.14
Результат B = { 0.74 1.05 0.90 0.66 0.93 }, сумма = 4.28
Минимум = 0.66, максимум = 1.05
Введите размер (или 0 для завершения): 0

%%%%%%%%%%%%%%%%%%%%%%%%%%%%%%%%%%%%%%%%%%%%%%%%%%%%%%%%%%%%%%%%%%%%%%%%%%%%%%

Задача C-7

В начало массива из 20 элементов записать N случайных вещественных чисел из
диапазона от 0 до N с точностью 0.1, воспользовавшись функцией rand(). Число N
вводится с клавиатуры. Сосчитать сумму элементов. В другой массив записать
суммы пар элементов (N сумм): первого и последнего, второго и предпоследнего,
третьего и пред-предпоследнего... Вывести на экран результирующий массив и
сумму его элементов, а также минимальный и максимальный эл-т. Повторять
выполнение программы до тех пор, пока в качестве N не введут 0.

Иванов И.И. (1057/1): Массивы, вещ. случайные числа от 0 до N.
Введите размер (или 0 для завершения): 5
Сгенерированный массив A = { 3.40 0.30 4.70 1.10 4.50 }, сумма = 14
Результат B = { 7.90 1.40 9.40 1.40 7.90 }, сумма = 28
Минимум = 1.40, максимум = 9.40
Введите размер (или 0 для завершения): 0

%%%%%%%%%%%%%%%%%%%%%%%%%%%%%%%%%%%%%%%%%%%%%%%%%%%%%%%%%%%%%%%%%%%%%%%%%%%%%%

Задача C-8

В начало массива из 20 элементов записать N случайных вещественных чисел из
диапазона от 0 до N с точностью 0.1, воспользовавшись функцией rand(). Число N
вводится с клавиатуры. Сосчитать сумму элементов. В другой массив записать
суммы пар элементов (N сумм): первого и второго, второго и третьего, ...,
последнего и первого. Вывести на экран результирующий массив и сумму его
элементов, а также минимальный и максимальный эл-т. Повторять выполнение
программы до тех пор, пока в качестве N не введут 0.

Иванов И.И. (1057/1): Массивы, вещ. случайные числа от 0 до N.
Введите размер (или 0 для завершения): 5
Сгенерированный массив A = { 3.40 0.30 4.70 1.10 4.50 }, сумма = 14
Результат B = { 3.70 5.00 5.80 5.60 7.90 }, сумма = 28
Минимум = 3.70, максимум = 7.90
Введите размер (или 0 для завершения): 0

%%%%%%%%%%%%%%%%%%%%%%%%%%%%%%%%%%%%%%%%%%%%%%%%%%%%%%%%%%%%%%%%%%%%%%%%%%%%%%

Задача C-9

В начало массива из 20 элементов записать N случайных натуральных чисел из
диапазона от 1 до N, воспользовавшись функцией rand(). Число N вводится с
клавиатуры. Сосчитать сумму элементов. В другой массив записать суммы пар
элементов (N сумм): первого и последнего, второго и предпоследнего, третьего и
пред-предпоследнего... Вывести на экран результирующий массив и сумму его
элементов, а также индексы минимального и максимального эл-та (если таких
несколько, то первый минимальный и последний максимальный). Повторять
выполнение программы до тех пор, пока в качестве N не введут 0.

Иванов И.И. (1057/1): Массивы, нат.случайные числа от 1 до N.
Введите размер (или 0 для завершения): 5
Сгенерированный массив A = { 4 3 1 4 5 }, сумма = 17
Результат B = { 9 7 2 7 9 }, сумма = 34
Минимум = B[2], максимум = B[4]
Введите размер (или 0 для завершения): 0

%%%%%%%%%%%%%%%%%%%%%%%%%%%%%%%%%%%%%%%%%%%%%%%%%%%%%%%%%%%%%%%%%%%%%%%%%%%%%%

Задача C-10

В начало массива из 20 элементов записать N случайных натуральных чисел из
диапазона от 1 до N, воспользовавшись функцией rand(). Число N вводится с
клавиатуры. Сосчитать сумму элементов. В другой массив записать суммы пар
элементов (N сумм): первого и второго, второго и третьего, ..., последнего и
первого. Вывести на экран результирующий массив и сумму его элементов, а также
индексы минимального и максимального эл-та (если таких несколько, то первый
минимальный и последний максимальный). Повторять выполнение программы до тех
пор, пока в качестве N не введут 0.

Иванов И.И. (1057/1): Массивы, нат.случайные числа от 1 до N.
Введите размер (или 0 для завершения): 5
Сгенерированный массив A = { 4 3 1 4 5 }, сумма = 17
Результат B = { 7 4 5 9 9 }, сумма = 34
Минимум = B[1], максимум = B[4]
Введите размер (или 0 для завершения): 0

%%%%%%%%%%%%%%%%%%%%%%%%%%%%%%%%%%%%%%%%%%%%%%%%%%%%%%%%%%%%%%%%%%%%%%%%%%%%%%

Задача C-11

В начало массива из 20 элементов записать N случайных целых чисел из диапазона
от A до B, воспользовавшись функцией rand(). Числа N, A, B вводятся с
клавиатуры. Сосчитать сумму элементов. В другой массив записать суммы пар
элементов (N сумм): первого и последнего, второго и предпоследнего, третьего и
пред-предпоследнего... Вывести на экран результирующий массив и сумму его
элементов, а также индексы минимального и максимального эл-та (если таких
несколько, то первый минимальный и последний максимальный). Повторять
выполнение программы до тех пор, пока в качестве N не введут 0.

Иванов И.И. (1057/1): Массивы, целые случайные числа от A до B.
Введите размер (или 0 для завершения): 5
Введите диапазон A B через пробел: 3 7
Сгенерированный массив A = { 4 3 7 6 5 }, сумма = 25
Результат B = { 9 9 14 9 9 }, сумма = 50
Минимум = B[0], максимум = B[2]
Введите размер (или 0 для завершения): 0

%%%%%%%%%%%%%%%%%%%%%%%%%%%%%%%%%%%%%%%%%%%%%%%%%%%%%%%%%%%%%%%%%%%%%%%%%%%%%%

Задача C-12

В начало массива из 20 элементов записать N случайных целых чисел из диапазона
от A до B, воспользовавшись функцией rand(). Числа N, A, B вводятся с
клавиатуры. Сосчитать сумму элементов. В другой массив записать суммы пар
элементов (N сумм): первого и второго, второго и третьего, ..., последнего и
первого. Вывести на экран результирующий массив и сумму его элементов, а также
индексы минимального и максимального эл-та (если таких несколько, то первый
минимальный и последний максимальный). Повторять выполнение программы до тех
пор, пока в качестве N не введут 0.

Иванов И.И. (1057/1): Массивы, целые случайные числа от A до B.
Введите размер (или 0 для завершения): 5
Введите диапазон A B через пробел: 3 7
Сгенерированный массив A = { 4 3 7 6 5 }, сумма = 25
Результат B = { 7 10 13 11 9 }, сумма = 50
Минимум = B[0], максимум = B[2]
Введите размер (или 0 для завершения): 0

%%%%%%%%%%%%%%%%%%%%%%%%%%%%%%%%%%%%%%%%%%%%%%%%%%%%%%%%%%%%%%%%%%%%%%%%%%%%%%

Задача C-13

В начало массива из 20 элементов записать N случайных вещественных чисел из
диапазона от 0 до 1, воспользовавшись функцией rand(). Число N вводится с
клавиатуры. Сосчитать сумму элементов. В другой массив записать суммы пар
элементов (N сумм): первого и последнего, второго и предпоследнего, третьего и
пред-предпоследнего... Вывести на экран результирующий массив и сумму его
элементов, а также индексы минимального и максимального эл-та (если таких
несколько, то первый минимальный и последний максимальный). Повторять
выполнение программы до тех пор, пока в качестве N не введут 0.

Иванов И.И. (1057/1): Массивы, вещ. случайные числа от 0 до 1.
Введите размер (или 0 для завершения): 5
Сгенерированный массив A = { 0.43 0.31 0.74 0.16 0.50 }, сумма = 2.14
Результат B = { 0.93 0.47 1.48 0.47 0.93 }, сумма = 4.28
Минимум = B[1], максимум = B[2]
Введите размер (или 0 для завершения): 0

%%%%%%%%%%%%%%%%%%%%%%%%%%%%%%%%%%%%%%%%%%%%%%%%%%%%%%%%%%%%%%%%%%%%%%%%%%%%%%

Задача C-14

В начало массива из 20 элементов записать N случайных вещественных чисел из
диапазона от 0 до 1, воспользовавшись функцией rand(). Число N вводится с
клавиатуры. Сосчитать сумму элементов. В другой массив записать суммы пар
элементов (N сумм): первого и второго, второго и третьего, ..., последнего и
первого. Вывести на экран результирующий массив и сумму его элементов, а также
индексы минимального и максимального эл-та (если таких несколько, то первый
минимальный и последний максимальный). Повторять выполнение программы до тех
пор, пока в качестве N не введут 0.

Иванов И.И. (1057/1): Массивы, вещ. случайные числа от 0 до 1.
Введите размер (или 0 для завершения): 5
Сгенерированный массив A = { 0.43 0.31 0.74 0.16 0.50 }, сумма = 2.14
Результат B = { 0.74 1.05 0.90 0.66 0.93 }, сумма = 4.28
Минимум = B[3], максимум = B[1]
Введите размер (или 0 для завершения): 0

%%%%%%%%%%%%%%%%%%%%%%%%%%%%%%%%%%%%%%%%%%%%%%%%%%%%%%%%%%%%%%%%%%%%%%%%%%%%%%

Задача C-15

В начало массива из 20 элементов записать N случайных вещественных чисел из
диапазона от 0 до N с точностью 0.1, воспользовавшись функцией rand(). Число N
вводится с клавиатуры. Сосчитать сумму элементов. В другой массив записать
суммы пар элементов (N сумм): первого и последнего, второго и предпоследнего,
третьего и пред-предпоследнего... Вывести на экран результирующий массив и
сумму его элементов, а также индексы минимального и максимального эл-та (если
таких несколько, то первый минимальный и последний максимальный). Повторять
выполнение программы до тех пор, пока в качестве N не введут 0.

Иванов И.И. (1057/1): Массивы, вещ. случайные числа от 0 до N.
Введите размер (или 0 для завершения): 5
Сгенерированный массив A = { 3.40 0.30 4.70 1.10 4.50 }, сумма = 14
Результат B = { 7.90 1.40 9.40 1.40 7.90 }, сумма = 28
Минимум = B[1], максимум = B[2]
Введите размер (или 0 для завершения): 0

%%%%%%%%%%%%%%%%%%%%%%%%%%%%%%%%%%%%%%%%%%%%%%%%%%%%%%%%%%%%%%%%%%%%%%%%%%%%%%

Задача C-16

В начало массива из 20 элементов записать N случайных вещественных чисел из
диапазона от 0 до N с точностью 0.1, воспользовавшись функцией rand(). Число N
вводится с клавиатуры. Сосчитать сумму элементов. В другой массив записать
суммы пар элементов (N сумм): первого и второго, второго и третьего, ...,
последнего и первого. Вывести на экран результирующий массив и сумму его
элементов, а также индексы минимального и максимального эл-та (если таких
несколько, то первый минимальный и последний максимальный). Повторять
выполнение программы до тех пор, пока в качестве N не введут 0.

Иванов И.И. (1057/1): Массивы, вещ. случайные числа от 0 до N.
Введите размер (или 0 для завершения): 5
Сгенерированный массив A = { 3.40 0.30 4.70 1.10 4.50 }, сумма = 14
Результат B = { 3.70 5.00 5.80 5.60 7.90 }, сумма = 28
Минимум = B[0], максимум = B[4]
Введите размер (или 0 для завершения): 0

%%%%%%%%%%%%%%%%%%%%%%%%%%%%%%%%%%%%%%%%%%%%%%%%%%%%%%%%%%%%%%%%%%%%%%%%%%%%%%

Задача C-17

В начало массива из 20 элементов записать N случайных натуральных чисел из
диапазона от 1 до N, воспользовавшись функцией rand(). Число N вводится с
клавиатуры. Найти индексы минимального и максимального эл-та (если таких
несколько, то первый минимальный и последний максимальный). В другой массив
записать элементы, лежащие между найденными индексами включительно. Вывести на
экран результирующий массив и сумму его элементов. Повторять выполнение
программы до тех пор, пока в качестве N не введут 0.

Иванов И.И. (1057/1): Массивы, нат.случайные числа от 1 до N.
Введите размер (или 0 для завершения): 5
Сгенерированный массив A = { 4 3 1 4 5 }
Минимум = A[2], максимум = A[4]
Результат B = { 1 4 5 }, сумма = 10
Введите размер (или 0 для завершения): 0

%%%%%%%%%%%%%%%%%%%%%%%%%%%%%%%%%%%%%%%%%%%%%%%%%%%%%%%%%%%%%%%%%%%%%%%%%%%%%%

Задача C-18

В начало массива из 20 элементов записать N случайных целых чисел из диапазона
от A до B, воспользовавшись функцией rand(). Числа N, A, B вводятся с
клавиатуры. Найти индексы минимального и максимального эл-та (если таких
несколько, то первый минимальный и последний максимальный). В другой массив
записать элементы, лежащие между найденными индексами включительно. Вывести на
экран результирующий массив и сумму его элементов. Повторять выполнение
программы до тех пор, пока в качестве N не введут 0.

Иванов И.И. (1057/1): Массивы, целые случайные числа от A до B.
Введите размер (или 0 для завершения): 5
Введите диапазон A B через пробел: 3 7
Сгенерированный массив A = { 4 3 7 4 5 }
Минимум = A[1], максимум = A[2]
Результат B = { 3 7 }, сумма = 10
Введите размер (или 0 для завершения): 0

%%%%%%%%%%%%%%%%%%%%%%%%%%%%%%%%%%%%%%%%%%%%%%%%%%%%%%%%%%%%%%%%%%%%%%%%%%%%%%

Задача C-19

В начало массива из 20 элементов записать N случайных вещественных чисел из
диапазона от 0 до 1, воспользовавшись функцией rand(). Число N вводится с
клавиатуры. Найти индексы минимального и максимального эл-та (если таких
несколько, то первый минимальный и последний максимальный). В другой массив
записать элементы, лежащие между найденными индексами включительно. Вывести на
экран результирующий массив и сумму его элементов. Повторять выполнение
программы до тех пор, пока в качестве N не введут 0.

Иванов И.И. (1057/1): Массивы, вещ. случайные числа от 0 до 1.
Введите размер (или 0 для завершения): 5
Сгенерированный массив A = { 0.42 0.11 0.29 0.94 0.75 }
Минимум = A[1], максимум = A[3]
Результат B = { 0.11 0.29 0.94 }, сумма = 1.34
Введите размер (или 0 для завершения): 0

%%%%%%%%%%%%%%%%%%%%%%%%%%%%%%%%%%%%%%%%%%%%%%%%%%%%%%%%%%%%%%%%%%%%%%%%%%%%%%

Задача C-20

В начало массива из 20 элементов записать N случайных вещественных чисел из
диапазона от 0 до N с точностью 0.1, воспользовавшись функцией rand(). Число N
вводится с клавиатуры. Найти индексы минимального и максимального эл-та (если
таких несколько, то первый минимальный и последний максимальный). В другой
массив записать элементы, лежащие между найденными индексами включительно.
Вывести на экран результирующий массив и сумму его элементов. Повторять
выполнение программы до тех пор, пока в качестве N не введут 0.

Иванов И.И. (1057/1): Массивы, вещ. случайные числа от 0 до N.
Введите размер (или 0 для завершения): 5
Сгенерированный массив A = { 0.40 0.90 0.20 0.10 0.70 }
Минимум = A[3], максимум = A[1]
Результат B = { 0.90 0.20 0.10 }, сумма = 1.20
Введите размер (или 0 для завершения): 0

%%%%%%%%%%%%%%%%%%%%%%%%%%%%%%%%%%%%%%%%%%%%%%%%%%%%%%%%%%%%%%%%%%%%%%%%%%%%%%

Задача C-21

В начало массива из 20 элементов записать N случайных вещественных чисел из
диапазона от 0 до N с точностью 0.1, воспользовавшись функцией rand(). Число N
вводится с клавиатуры. Найти индексы минимального и максимального эл-та (если
таких несколько, то первый минимальный и последний максимальный). В другой
массив записать элементы, лежащие между найденными индексами включительно,
начиная с минимального индекса (может оказаться в обратном порядке). Вывести
на экран результирующий массив и сумму его элементов. Повторять выполнение
программы до тех пор, пока в качестве N не введут 0.

Иванов И.И. (1057/1): Массивы, вещ. случайные числа от 0 до N.
Введите размер (или 0 для завершения): 5
Сгенерированный массив A = { 0.40 0.90 0.20 0.10 0.70 }
Минимум = A[3], максимум = A[1]
Результат B = { 0.10 0.20 0.90 }, сумма = 1.20
Введите размер (или 0 для завершения): 0

%%%%%%%%%%%%%%%%%%%%%%%%%%%%%%%%%%%%%%%%%%%%%%%%%%%%%%%%%%%%%%%%%%%%%%%%%%%%%%


Пример задания

Вариант A-1. Линейное уравнение

Найти и вывести множество вещественных значений x, удовлетворяющих равенству
ax+b=0. Параметры a, b — любые целые числа (в том числе и нули), вводятся
пользователем с клавиатуры. Необходимо корректно разобрать все случаи.

Решение

Известно, что уравнение имеет одно решение -b/a, если a отлично от нуля.
Заметим, что при a=0 уравнение вырождается в равенство b=0, которое может быть
как истинным, так и ложным вне зависимости от x. Эти рассуждения дают нам
последовательность условий для проверки.

Поскольку в языке Си результат операции деления двух целых чисел тоже является
целым, необходимо воспользоваться операцией преобразования аргументов к
вещественному типу.

#include <stdio.h>

int main(void)
{
  int a, b;

  printf("Введите A и B через пробел: ");
  scanf("%i%i", &a, &b);

  if (a != 0)
    printf("Решение x = %f\n", -(double)b / (double)a);
  else if (b != 0)
    printf("Нет решения.\n");
  else
    printf("Верно при любом x.\n");

  return 0;
}

%%%%%%%%%%%%%%%%%%%%%%%%%%%%%%%%%%%%%%%%%%%%%%%%%%%%%%%%%%%%%%%%%%%%%%%%%%%%%%

Вариант A-2. Квадратное уравнение

Найти и вывести множество вещественных (комплексных) значений x,
удовлетворяющих равенству ax2+bx+c=0. Параметры a, b, c — любые целые числа (в
том числе и нули), вводятся пользователем с клавиатуры. Необходимо корректно
разобрать все случаи.

%%%%%%%%%%%%%%%%%%%%%%%%%%%%%%%%%%%%%%%%%%%%%%%%%%%%%%%%%%%%%%%%%%%%%%%%%%%%%%

Вариант A-3. Кубическое уравнение

Найти и вывести множество вещественных (комплексных) значений x,
удовлетворяющих равенству ax3+bx2+cx+d=0. Параметры a, b, c, d — любые целые
числа (в том числе и нули), вводятся пользователем с клавиатуры. Необходимо
корректно разобрать все случаи.

%%%%%%%%%%%%%%%%%%%%%%%%%%%%%%%%%%%%%%%%%%%%%%%%%%%%%%%%%%%%%%%%%%%%%%%%%%%%%%

Вариант A-4. Система уравнений

Найти и вывести все пары x, y удовлетворяющие системе двух равенств:
a1x+b1y=c1, a2x+b2y=c2. Параметры ak, bk, ck — любые целые числа (в том числе
и нули), вводятся пользователем с клавиатуры. Необходимо корректно разобрать
все случаи.

%%%%%%%%%%%%%%%%%%%%%%%%%%%%%%%%%%%%%%%%%%%%%%%%%%%%%%%%%%%%%%%%%%%%%%%%%%%%%%
\zztaskgroup{STR}{Строки (старые)}
%%%%%%%%%%%%%%%%%%%%%%%%%%%%%%%%%%%%%%%%%%%%%%%%%%%%%%%%%%%%%%%%%%%%%%%%%%%%%%

Задача D-1

В буферы фиксированного размера прочитать с клавиатуры длинную и короткую
строку. Не используя страндартные функции (из string.h), найти все места, где
в длинной строке встречается короткая, вывести номера позиций (считая с нуля).
При поиске не менять исходных буферов. Повторять выполнение программы до тех
пор, пока в качестве длинной строки не введут пустую.

Иванов И.И. (1057/1): Строки. Поиск подстроки.
Введите строку-1 (<40 символов): abcd efab cdabcdef
Введите строку-2 (<40 символов): abcd
Позиции: 0 12
Введите строку-1 (<40 символов):

%%%%%%%%%%%%%%%%%%%%%%%%%%%%%%%%%%%%%%%%%%%%%%%%%%%%%%%%%%%%%%%%%%%%%%%%%%%%%%

Задача D-2

В буферы фиксированного размера прочитать с клавиатуры длинную и короткую
строку. Не используя страндартные функции (из string.h), найти все места, где
в длинной строке встречается короткая, записанная задом наперед, вывести
номера позиций (считая с нуля). При поиске не менять исходных буферов.
Повторять выполнение программы до тех пор, пока в качестве длинной строки не
введут пустую.

Иванов И.И. (1057/1): Строки. Поиск обратной подстроки.
Введите строку-1 (<40 символов): abcd efab cdabcdef
Введите строку-2 (<40 символов): dcba
Позиции: 0 12
Введите строку-1 (<40 символов):

%%%%%%%%%%%%%%%%%%%%%%%%%%%%%%%%%%%%%%%%%%%%%%%%%%%%%%%%%%%%%%%%%%%%%%%%%%%%%%

Задача D-3

В буфер фиксированного размера прочитать с клавиатуры строку. Не используя
страндартные функции (из string.h), развернуть строку задом наперед. Повторять
выполнение программы до тех пор, пока в качестве строки не введут пустую.

Иванов И.И. (1057/1): Строки. Разворот строки.
Введите строку (<40 символов): abcd efab
Строка наоборот: bafe dcba
Введите строку (<40 символов):

%%%%%%%%%%%%%%%%%%%%%%%%%%%%%%%%%%%%%%%%%%%%%%%%%%%%%%%%%%%%%%%%%%%%%%%%%%%%%%

Задача D-4

В буфер фиксированного размера прочитать с клавиатуры строку. Не используя
страндартные функции (из string.h и др.), подсчитать количество заглавных и
строчных букв. Другие символы игнорировать. Повторять выполнение программы до
тех пор, пока в качестве строки не введут пустую.

Иванов И.И. (1057/1): Строки. Подсчет букв.
Введите строку (<40 символов): abCD eFab
В строке 3 заглавные и 5 строчных букв 
Введите строку (<40 символов):

%%%%%%%%%%%%%%%%%%%%%%%%%%%%%%%%%%%%%%%%%%%%%%%%%%%%%%%%%%%%%%%%%%%%%%%%%%%%%%

Задача D-5

В буфер фиксированного размера прочитать с клавиатуры строку. Не используя
страндартные функции (из string.h и др.), заменить в строке заглавные буквы на
пробелы. Повторять выполнение программы до тех пор, пока в качестве строки не
введут пустую.

Иванов И.И. (1057/1): Строки. Забивка заглавных букв.
Введите строку (<40 символов): abCD eFab
Результат: ab   e ab
Введите строку (<40 символов):

%%%%%%%%%%%%%%%%%%%%%%%%%%%%%%%%%%%%%%%%%%%%%%%%%%%%%%%%%%%%%%%%%%%%%%%%%%%%%%

Задача D-6

В буфер фиксированного размера прочитать с клавиатуры строку. Не используя
страндартные функции (из string.h и др.), во всех последовательностях
повторяющихся букв заменить все буквы кроме первой на пробелы. Остальные
символы не трогать. Повторять выполнение программы до тех пор, пока в качестве
строки не введут пустую.

Иванов И.И. (1057/1): Строки. Забивка заглавных букв.
Введите строку (<40 символов): abbbaacdbbaaaaffd
Результат: ab  a cdb a   f d
Введите строку (<40 символов):

%%%%%%%%%%%%%%%%%%%%%%%%%%%%%%%%%%%%%%%%%%%%%%%%%%%%%%%%%%%%%%%%%%%%%%%%%%%%%%

Задача D-7

В буфер фиксированного размера прочитать с клавиатуры строку. Не используя
страндартные функции (из string.h), развернуть слова в строке задом наперед
(слова разделены пробелами). Повторять выполнение программы до тех пор, пока в
качестве строки не введут пустую.

Иванов И.И. (1057/1): Строки. Разворот слов.
Введите строку (<40 символов): abcd efab
Слова наоборот: dcba bafe
Введите строку (<40 символов):

%%%%%%%%%%%%%%%%%%%%%%%%%%%%%%%%%%%%%%%%%%%%%%%%%%%%%%%%%%%%%%%%%%%%%%%%%%%%%%

Задача D-8

В буфер фиксированного размера прочитать с клавиатуры строку. Не используя
страндартные функции (из string.h), найти подстроку максимальной длины, такую
что: строка начинается с нее, а заканчивается ее зеркальным отображением (см.
пример). Повторять выполнение программы до тех пор, пока в качестве строки не
введут пустую.

Иванов И.И. (1057/1): Строки. Поиск зеркального начала.
Введите строку (<40 символов): abcxyzcba
Зеркальное начало: abc
Введите строку (<40 символов):

%%%%%%%%%%%%%%%%%%%%%%%%%%%%%%%%%%%%%%%%%%%%%%%%%%%%%%%%%%%%%%%%%%%%%%%%%%%%%%

Задача D-9

В буфер фиксированного размера прочитать с клавиатуры строку. Не используя
страндартные функции (из string.h и др.), удалить из строки все пробелы,
прижав друг к другу слова. Повторять выполнение программы до тех пор, пока в
качестве строки не введут пустую.

Иванов И.И. (1057/1): Строки. Удаление пробелов.
Введите строку (<40 символов): I am a program
Результат: Iamaprogram
Введите строку (<40 символов):

%%%%%%%%%%%%%%%%%%%%%%%%%%%%%%%%%%%%%%%%%%%%%%%%%%%%%%%%%%%%%%%%%%%%%%%%%%%%%%

Задача D-10

В буфер фиксированного размера прочитать с клавиатуры строку, в которой
записан полное имя файла. Не используя страндартные функции (из string.h),
выделить в три буфера три части этого имени: путь к файлу, само имя файла,
расширение файла. Вывести их на экран. Повторять выполнение программы до тех
пор, пока в качестве строки не введут пустую.

Иванов И.И. (1057/1): Строки. Длинное имя файла.
Введите полное имя файла (<40 символов): \verb|c:\windows\system32\xcopy.exe|
Путь: \verb|с:\windows\system32|
Имя: xcopy
Расширение: exe
Введите полное имя файла (<40 символов):

%%%%%%%%%%%%%%%%%%%%%%%%%%%%%%%%%%%%%%%%%%%%%%%%%%%%%%%%%%%%%%%%%%%%%%%%%%%%%%

Задача D-11

В буфер фиксированного размера прочитать с клавиатуры строку. Не используя
страндартные функции (из string.h и др.), перевести все английские буквы в
заглавные. Если в строке встречаются другие, «неправильные» символы, также
сообщить об этом. Повторять выполнение программы до тех пор, пока в качестве
строки не введут пустую.

Иванов И.И. (1057/1): Строки. Заглавные буквы.
Введите строку (<40 символов): abcbax
Результат: ABCBAX
Введите строку (<40 символов):

%%%%%%%%%%%%%%%%%%%%%%%%%%%%%%%%%%%%%%%%%%%%%%%%%%%%%%%%%%%%%%%%%%%%%%%%%%%%%%

Задача D-12

В буфер фиксированного размера прочитать с клавиатуры строку. Не используя
страндартные функции (из string.h и др.), перевести все английские буквы в
строчные. Если в строке встречаются другие, «неправильные» символы, также
сообщить об этом. Повторять выполнение программы до тех пор, пока в качестве
строки не введут пустую.

Иванов И.И. (1057/1): Строки. Строчные буквы.
Введите строку (<40 символов): ABCBAX
Результат: abcbax
Введите строку (<40 символов):

%%%%%%%%%%%%%%%%%%%%%%%%%%%%%%%%%%%%%%%%%%%%%%%%%%%%%%%%%%%%%%%%%%%%%%%%%%%%%%

Задача D-13

В буфер фиксированного размера прочитать с клавиатуры строку. Не используя
страндартные функции (из string.h и др.), перевести все заглавные английские
буквы в строчные и наоборот. Если в строке встречаются другие, «неправильные»
символы, также сообщить об этом. Повторять выполнение программы до тех пор,
пока в качестве строки не введут пустую.

Иванов И.И. (1057/1): Строки. Инвертирование регистра букв.
Введите строку (<40 символов): ABcBax
Результат: abCbAX
Введите строку (<40 символов):

%%%%%%%%%%%%%%%%%%%%%%%%%%%%%%%%%%%%%%%%%%%%%%%%%%%%%%%%%%%%%%%%%%%%%%%%%%%%%%

Задача D-14

В буфер фиксированного размера прочитать с клавиатуры строку. Не используя
страндартные функции (из string.h и др.), проверить правильность написания
предложений в тексте: первая буква предложения должна быть заглавной,
предложения завершаются точкой, между точкой и следующим предложением должен
быть минимум один пробел. Сообщить все ошибки (и индексы символов с ошибкой).
Повторять выполнение программы до тех пор, пока в качестве строки не введут
пустую.

Иванов И.И. (1057/1): Строки. Проверка предложений.
Введите строку (<40 символов): AbcBax. F dcc. Sa nfmmc dfldk.
Предложение правильное.
Введите строку (<40 символов): Abc. f dcc.Sa nf.
Ошибка в символе 5: должна быть заглавная.
Ошибка в символе 11: должен быть пробел.
Введите строку (<40 символов):

%%%%%%%%%%%%%%%%%%%%%%%%%%%%%%%%%%%%%%%%%%%%%%%%%%%%%%%%%%%%%%%%%%%%%%%%%%%%%%

Задача D-15

В буфер фиксированного размера прочитать с клавиатуры строку. Не используя
страндартные функции (из string.h), найти все подстроки длиной больше одного
символа, такие, что зеркальные их отражения тоже присутствуют в ней. Повторять
выполнение программы до тех пор, пока в качестве строки не введут пустую.

Иванов И.И. (1057/1): Строки. Поиск зеркальных подстрок.
Введите строку (<40 символов): abcbaxxyzcb
Подстрока 1: ab
Подстрока 2: abс
Подстрока 3: bс
Подстрока 4: xx
...
Введите строку (<40 символов):

%%%%%%%%%%%%%%%%%%%%%%%%%%%%%%%%%%%%%%%%%%%%%%%%%%%%%%%%%%%%%%%%%%%%%%%%%%%%%%

Задача D-16

В буфер фиксированного размера на 60 символов прочитать с клавиатуры строку.
Не используя страндартные функции (из string.h и др.) и дополнительный буфер,
выровнять строку по центру этого буфера (вставить в буфер необходимое
количество пробелов перед и после текста, так чтобы суммарная длина строки
была равна 60 — проверить с помощью strlen()). Вывести строку на экран в
квадратных скобках, чтобы была видна ширина текста. Повторять выполнение
программы до тех пор, пока в качестве строки не введут пустую. Пример для 20
символов:

Иванов И.И. (1057/1): Строки. Выравнивание по центру.
Введите строку (<60 символов): hello, world!
Результат: [    hello, world!   ]
Введите строку (<60 символов):

%%%%%%%%%%%%%%%%%%%%%%%%%%%%%%%%%%%%%%%%%%%%%%%%%%%%%%%%%%%%%%%%%%%%%%%%%%%%%%

Задача D-17

В буфер фиксированного размера прочитать с клавиатуры строку. Не используя
страндартные функции (из string.h и др.), выбросить из нее все символы,
начиная с последовательности «/*» и заканчивая «*/». Если указанной
последовательности нет, не делать ничего. Если встречается только открывающая
или закрывающая последовательность, выдать сообщение об ошибке. Повторять
выполнение программы до тех пор, пока в качестве строки не введут пустую.

Иванов И.И. (1057/1): Строки. Удаление коментариев.
Введите строку (<40 символов): hello,/*bc12df fh*/world
Результат: hello,world
Введите строку (<40 символов):

%%%%%%%%%%%%%%%%%%%%%%%%%%%%%%%%%%%%%%%%%%%%%%%%%%%%%%%%%%%%%%%%%%%%%%%%%%%%%%

Задача D-18

В буфер фиксированного размера прочитать с клавиатуры строку. Не используя
страндартные функции (из string.h и др.), найти в ней слово минимальной длины.
Слова — это последовательности букв. Остальные символы — разделители.
Повторять выполнение программы до тех пор, пока в качестве строки не введут
пустую.

Иванов И.И. (1057/1): Строки. Минимальное слово.
Введите строку (<40 символов): read a string
Результат: a
Введите строку (<40 символов):

%%%%%%%%%%%%%%%%%%%%%%%%%%%%%%%%%%%%%%%%%%%%%%%%%%%%%%%%%%%%%%%%%%%%%%%%%%%%%%

Задача D-19

В буфер фиксированного размера прочитать с клавиатуры строку. Не используя
страндартные функции (из string.h и др.), найти в ней слово максимальной
длины. Слова — это последовательности букв. Остальные символы — разделители.
Повторять выполнение программы до тех пор, пока в качестве строки не введут
пустую.

Иванов И.И. (1057/1): Строки. Максимальное слово.
Введите строку (<40 символов): read a string
Результат: string
Введите строку (<40 символов):

%%%%%%%%%%%%%%%%%%%%%%%%%%%%%%%%%%%%%%%%%%%%%%%%%%%%%%%%%%%%%%%%%%%%%%%%%%%%%%

Задача D-20

В буфер фиксированного размера на 60 символов прочитать с клавиатуры строку.
Не используя страндартные функции (из string.h и др.) и дополнительный буфер,
равномерно вставить между словами дополнительные пробелы так, чтобы, чтобы
строка заполнила буфер (проверить с помощью strlen()). Вывести строку на экран
в квадратных скобках, чтобы была видна ширина текста. Повторять выполнение
программы до тех пор, пока в качестве строки не введут пустую. Пример для 20
символов:

Иванов И.И. (1057/1): Строки. Выравнивание по обоим краям.
Введите строку (<60 символов): I am a program
Результат: [I   am   a   program]
Введите строку (<60 символов):

%%%%%%%%%%%%%%%%%%%%%%%%%%%%%%%%%%%%%%%%%%%%%%%%%%%%%%%%%%%%%%%%%%%%%%%%%%%%%%

Задача D-21

В буфер фиксированного размера прочитать с клавиатуры строку. Не используя
страндартные функции (из string.h и др.), найти максимальное по модулю целое
число, встречающееся в этой строке. Повторять выполнение программы до тех пор,
пока в качестве строки не введут пустую.

Иванов И.И. (1057/1): Строки. Поиск чисел.
Введите строку (<40 символов): abc12df fh 84ii-10z
Результат: 84
Введите строку (<40 символов):

%%%%%%%%%%%%%%%%%%%%%%%%%%%%%%%%%%%%%%%%%%%%%%%%%%%%%%%%%%%%%%%%%%%%%%%%%%%%%%

