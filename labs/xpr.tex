%%%%%%%%%%%%%%%%%%%%%%%%%%%%%%%%%%%%%%%%%%%%%%%%%%%%%%%%%%%%%%%%%%%%%%%%%%%%%%
\zztaskgroup{XPR}{Вычисление выражений и взаимодействие с пользователем}
%%%%%%%%%%%%%%%%%%%%%%%%%%%%%%%%%%%%%%%%%%%%%%%%%%%%%%%%%%%%%%%%%%%%%%%%%%%%%%

В следующих задачах требуется написать программу, автоматизирующую вычисления
по заданной формуле. Формула может включать в себя несколько заранее известных
констант и заранее неизвестных переменных (параметров), которые должен ввести
в программу пользователь после соответствующего приглашения. Запуская
программу несколько раз, вводя разные значения параметров, пользователь сможет
получать соответствующее этим параметрам решение задачи.

Приложение должно быть дружелюбным к пользователю (user friendly), то есть
вести с ним разумный диалог. Пользователь в каждый момент времени должен
знать, чего ожидает от него программа. Программа должна производить проверку
корректности исходных данных, вводимых пользователем.
Обычно вводимые величины имеют минимум одно ограничение --- масса, длина,
скорость должны быть неотрицательны.
В случае недопустимых данных программа должна выдавать осмысленное сообщение
и повторно запрашивать соответствующее значение. Запросы должны повторяться
до тех пор пока пользователь не введет корректное значение. Так должно происходить
для каждого вводимого параметра.

Примеры диалога программы и пользователя:% \zztodo{АТ. Задача ясно сформулирована, простая, является хорошим "разгоночным" упражнением. Может стоит сделать наружный бесконечный цикл? Здесь формулировка гораздо подробнее, чем на классной(все константы есть и тп), но стоит ли ее делать домашкой?}:

\begin{zzoutput}
  Задание \thezztaskgroup-1: Диаметр шара заданной массы и плотности
  Введите массу (кг): \zzuser{-1.5}
  Ошибка ввода, масса должна быть больше нуля!
  Введите массу (кг): \zzuser{0}
  Ошибка ввода, масса должна быть больше нуля!
  Введите массу (кг): \zzuser{1.5}
  Введите плотность (кг/м3): \zzuser{0}
  Ошибка ввода, плотность должна быть больше нуля!
  Введите плотность (кг/м3): \zzuser{19320}
  Диаметр получившегося шара (м): d = 0.052929
\end{zzoutput}
 
Рекомендуется проверять программу на реальных значениях параметров, при
которых ответ на задачу заранее известен или может быть легко проверен на
адекватность. В тестировании могут пригодиться некоторые табличные данные.
Для сведения:
%
\begin{itemize}
%
\item плотность~$\rho$ золота, железа, льда и пробки ---
19320, 7870, 916 и 240~кг/м\textsuperscript{3} соответственно;
%
\item ускорение свободного падения $g$ примерно равно 
9.8~м/с\textsuperscript{2};
%
\item человек привык измерять углы в градусах, а компьютер ---
в радианах, и это уже учтено в предлагаемых ниже формулах;
полный круг $360^{\circ}$ равен $2\pi$ радиан ($\pi = 3.14159265358979323846\dots$);
%
\item человек привык измерять проценты в диапазоне от 0 до 100\%,
а компьютер -- в долях, от 0 до 1, что тоже уже учитывают формулы.
\end{itemize}


%%%%%%%%%%%%%%%%%%%%%%%%%%%%%%%%%%%%%%%%%%%%%%%%%%%%%%%%%%%%%%%%%%%%%%%%%%%%%%
\bigskip
%%%%%%%%%%%%%%%%%%%%%%%%%%%%%%%%%%%%%%%%%%%%%%%%%%%%%%%%%%%%%%%%%%%%%%%%%%%%%%


\begin{zztask}
Какого диаметра получится шар массы $m$, изготовленный из материала с
плотностью~$\rho$?
%
\[
d = 2\cdot \sqrt[3]{\frac{3m}{4\pi\rho}}
\]
\end{zztask}

%%%%%%%%%%%%%%%%%%%%%%%%%%%%%%%%%%%%%%%%%%%%%%%%%%%%%%%%%%%%%%%%%%%%%%%%%%%%%%

\begin{zztask}
Какой массы получится шар диаметра $d$, изготовленный из материала с
плотностью~$\rho$?
%
\[
m = \frac{1}{6}\pi\rho d^3
\]
\end{zztask}

%%%%%%%%%%%%%%%%%%%%%%%%%%%%%%%%%%%%%%%%%%%%%%%%%%%%%%%%%%%%%%%%%%%%%%%%%%%%%%

\begin{zztask}
Какого диаметра получится стержень массы $m$ и длины $l$, изготовленный из
материала с плотностью~$\rho$?
%
\[
d = 2\cdot \sqrt{\frac{m}{\pi\rho l}}
\]
\end{zztask}

%%%%%%%%%%%%%%%%%%%%%%%%%%%%%%%%%%%%%%%%%%%%%%%%%%%%%%%%%%%%%%%%%%%%%%%%%%%%%%

\begin{zztask}
Какой длины получится стержень массы $m$ диаметром $d$, изготовленный из
материала с плотностью~$\rho$?
%
\[
l = \frac{4m}{\pi\rho d^2}
\]
\end{zztask}

%%%%%%%%%%%%%%%%%%%%%%%%%%%%%%%%%%%%%%%%%%%%%%%%%%%%%%%%%%%%%%%%%%%%%%%%%%%%%%

\begin{zztask}
Какой массы получится стержень диаметром $d$ длиной $l$, изготовленный из
материала с плотностью~$\rho$?
%
\[
m = \frac{1}{4}\pi\rho d^2 l
\]
\end{zztask}

%%%%%%%%%%%%%%%%%%%%%%%%%%%%%%%%%%%%%%%%%%%%%%%%%%%%%%%%%%%%%%%%%%%%%%%%%%%%%%

\begin{zztask}
Какой массы получится пирамида с длиной ребра $a$, изготовленная из
материала с плотностью~$\rho$?
%
\[
m = \frac{\sqrt2}{12}\rho a^3
\]
\end{zztask}

%%%%%%%%%%%%%%%%%%%%%%%%%%%%%%%%%%%%%%%%%%%%%%%%%%%%%%%%%%%%%%%%%%%%%%%%%%%%%%

\begin{zztask}
Какой высоты получится пирамида массы $m$, изготовленная из
материала с плотностью~$\rho$?
%
\[
h = \frac{\sqrt6}{3} \sqrt[3]{\frac{6\sqrt2 m}{\rho}}
\]
\end{zztask}

%%%%%%%%%%%%%%%%%%%%%%%%%%%%%%%%%%%%%%%%%%%%%%%%%%%%%%%%%%%%%%%%%%%%%%%%%%%%%%

\begin{zztask}
Как далеко улетит тело, если его бросить под углом $\alpha$ к горизонту
со скоростью $v$?
%
\[
l = \frac{v^2}{g}\sin\frac{\alpha\pi}{90}
\]
\end{zztask}

%%%%%%%%%%%%%%%%%%%%%%%%%%%%%%%%%%%%%%%%%%%%%%%%%%%%%%%%%%%%%%%%%%%%%%%%%%%%%%

\begin{zztask}
Под каким углом к горизонту нужно бросить тело, чтобы оно улетело на
расстояние $l$ со скоростью $v$?
%
\[
\alpha = \frac{90}{\pi}\arcsin\frac{lg}{v^2}
\]
\end{zztask}

%%%%%%%%%%%%%%%%%%%%%%%%%%%%%%%%%%%%%%%%%%%%%%%%%%%%%%%%%%%%%%%%%%%%%%%%%%%%%%

\begin{zztask}
С какой скоростью нужно бросить тело под углом $\alpha$ к горизонту,
чтобы оно улетело на расстояние $l$?
%
\[
v = \sqrt{\frac{lg}{\sin(\alpha\pi/90)}}
\]
\end{zztask}

%%%%%%%%%%%%%%%%%%%%%%%%%%%%%%%%%%%%%%%%%%%%%%%%%%%%%%%%%%%%%%%%%%%%%%%%%%%%%%

\begin{zztask}
Какой доход за год принесет сумма $m$ на счёте в банке при годовой
процентной ставке $p$, если вклад подразумевает ежемесячную
капитализацию процентов?
%
\[
i = m \left(1 + \frac{p}{1200}\right)^{12} - m
\]
\end{zztask}

%%%%%%%%%%%%%%%%%%%%%%%%%%%%%%%%%%%%%%%%%%%%%%%%%%%%%%%%%%%%%%%%%%%%%%%%%%%%%%

\begin{zztask}
Сколько денег надо положить на счет в банк, чтобы при ежемесячной
капитализации процентов получить годовой доход $i$ при процентной
ставке $p$?
%
\[
m = \frac{i}{\left(1 + p/1200\right)^{12} - 1}
\]
\end{zztask}

%%%%%%%%%%%%%%%%%%%%%%%%%%%%%%%%%%%%%%%%%%%%%%%%%%%%%%%%%%%%%%%%%%%%%%%%%%%%%%

\begin{zztask}
Какой должна быть ежемесячная выплата по кредиту, если сумма $m$ берётся 
под процент $p$ на $k$ месяцев?
%
\[
a = \frac{mp}{1200}\left(1 + \frac{1}{(1 + p/1200)^k - 1}\right)
\]
\end{zztask}

%%%%%%%%%%%%%%%%%%%%%%%%%%%%%%%%%%%%%%%%%%%%%%%%%%%%%%%%%%%%%%%%%%%%%%%%%%%%%%

\begin{zztask}
Какой будет суммарная переплата по кредиту, если сумма $m$ берётся
под процент $p$ на $k$ месяцев?
%
\[
i = \frac{mkp}{1200}\left(1 + \frac{1}{(1 + p/1200)^k - 1}\right) - m
\]
\end{zztask}

%%%%%%%%%%%%%%%%%%%%%%%%%%%%%%%%%%%%%%%%%%%%%%%%%%%%%%%%%%%%%%%%%%%%%%%%%%%%%%

\begin{zztask}
На какую максимальную сумму кредита под процент $p$ на $k$ месяцев
можно расчитывать, если есть возможность отдавать ежемесячно сумму $a$?
%
\[
m = \frac{a}{p/1200}\left(1 - \frac{1}{(1 + p/1200)^k}\right)
\]
\end{zztask}

%%%%%%%%%%%%%%%%%%%%%%%%%%%%%%%%%%%%%%%%%%%%%%%%%%%%%%%%%%%%%%%%%%%%%%%%%%%%%%

