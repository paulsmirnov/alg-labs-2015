%%%%%%%%%%%%%%%%%%%%%%%%%%%%%%%%%%%%%%%%%%%%%%%%%%%%%%%%%%%%%%%%%%%%%%%%%%%%%%
\zztaskgroup{CUR}{Самоподобные (фрактальные) кривые}
%%%%%%%%%%%%%%%%%%%%%%%%%%%%%%%%%%%%%%%%%%%%%%%%%%%%%%%%%%%%%%%%%%%%%%%%%%%%%%

%Графические программы пишутся в среде Borland C с использованием BGI-графики.
%Режим обычный (автоопределяемый), 640х480 (VGAHI), 16 цветов. При желании можно 
%использовать режим 640х350 (VGAMED), позволяющий иметь 2 видеостраницы.
\begin{itemize}
    \item Здесь хорошо бы добавить общие правила того, как это нужно реализовывать (я, кстати, не знаю как)
    \item Возможно, в приложениях можно добавить шаблон программки, чтобы это было легко реализовывать
    \item + наверняка ты что-то вещаешь на доске, хорошо бы это тоже сюда поместить.
    \item Используется ли здесь библиотека? если да, то описать процесс подключения. 
\end{itemize}
Написать программу, строящую N-ное приближение (итерацию) одной из 
нижеперечисленных рекурентно заданных фрактальных кривых. После запуска 
программа позволяет управляемым с клавиатуры курсором (который рисуется 
самостоятельно) выбрать начальную и конечную точки кривой и отображает нулевую 
итерацию.\zztodo{почему именно с клавы и будем ли тут кнопки регламентировать?} После этого стрелки вправо/влево на клавиатуре позволяют 
увеличивать/уменьшать номер итерации, и кривая перерисовывается.\zztodo{мне кажется, что тут не хватает форматирования}

Рекурентно заданые кривые должны быть реализованы в виде рекурсивной (вызывающей 
саму себя) функции.\zztodo{тут я тоже считаю, что это домашка онли, разобраться с этим в классе лично я был бы не готов.}

%%%%%%%%%%%%%%%%%%%%%%%%%%%%%%%%%%%%%%%%%%%%%%%%%%%%%%%%%%%%%%%%%%%%%%%%%%%%%%

\begin{zztask}[Koch Curve]
В рамках общего условия задачи построить кривую Коха. База для построения кривой
(нулевое приближение) представляет собой отрезок. Переход от приближения
$n$ к приближению $(n+1)$ осуществляется заменой каждого отрезка фигуры на четыре
(см. рис. при $n=1$). Все отрезки (включая пустой промежуток) имеют одинаковую 
длину. Угол между отрезками составляет $60^\circ$.\zztodo{а у меня не отображаются рисунки}
\par\noindent
%
\begin{zzfrac}{$n = 0$}
\begin{scope}[shift={(0,0.5)}]
\draw[very thick]
  (0,0) node (A) {} --
  (1,0) node (B) {};
\zzfracdots{A,B}
\end{scope}
\end{zzfrac}
%
\zzfracskip
%
\begin{zzfrac}{$n = 1$}
\begin{scope}[shift={(0,0.5)}]
\draw[very thick]
  (0.000000, 0.000000) --
  (0.333333, 0.000000) node (A1) {} --
  (0.500000, 0.288675) node (C) {} --
  (0.666667, 0.000000) node (B1) {} --
  (1.000000, 0.000000);
\zzfracdots{A,A1,C,B1,B}
\end{scope}
\end{zzfrac}
%
\zzfracskip
%
\begin{zzfrac}{$n = 2$}
\begin{scope}[shift={(0,0.5)}]
\draw[thin]
  (0.000000, 0.000000) --
  (0.111111, 0.000000) --
  (0.166667, 0.096225) --
  (0.222222, 0.000000) --
  (0.333333, 0.000000) --
  (0.388889, 0.096225) --
  (0.333333, 0.192450) --
  (0.444444, 0.192450) --
  (0.500000, 0.288675) --
  (0.555556, 0.192450) --
  (0.666667, 0.192450) --
  (0.611111, 0.096225) --
  (0.666667, 0.000000) --
  (0.777778, 0.000000) --
  (0.833333, 0.096225) --
  (0.888889, 0.000000) --
  (1.000000, 0.000000);
\end{scope}
\end{zzfrac}
%
\zzfracskip
%
\begin{zzfrac}{$n = 3$}
\begin{scope}[shift={(0,0.5)}]
\draw[thin]
  (0.000000, 0.000000) --
  (0.037037, 0.000000) --
  (0.055556, 0.032075) --
  (0.074074, 0.000000) --
  (0.111111, 0.000000) --
  (0.129630, 0.032075) --
  (0.111111, 0.064150) --
  (0.148148, 0.064150) --
  (0.166667, 0.096225) --
  (0.185185, 0.064150) --
  (0.222222, 0.064150) --
  (0.203704, 0.032075) --
  (0.222222, 0.000000) --
  (0.259259, 0.000000) --
  (0.277778, 0.032075) --
  (0.296296, 0.000000) --
  (0.333333, 0.000000) --
  (0.351852, 0.032075) --
  (0.333333, 0.064150) --
  (0.370370, 0.064150) --
  (0.388889, 0.096225) --
  (0.370370, 0.128300) --
  (0.333333, 0.128300) --
  (0.351852, 0.160375) --
  (0.333333, 0.192450) --
  (0.370370, 0.192450) --
  (0.388889, 0.224525) --
  (0.407407, 0.192450) --
  (0.444444, 0.192450) --
  (0.462963, 0.224525) --
  (0.444444, 0.256600) --
  (0.481481, 0.256600) --
  (0.500000, 0.288675) --
  (0.518518, 0.256600) --
  (0.555555, 0.256600) --
  (0.537037, 0.224525) --
  (0.555555, 0.192450) --
  (0.592592, 0.192450) --
  (0.611111, 0.224525) --
  (0.629629, 0.192450) --
  (0.666666, 0.192450) --
  (0.648148, 0.160375) --
  (0.666666, 0.128300) --
  (0.629629, 0.128300) --
  (0.611111, 0.096225) --
  (0.629629, 0.064150) --
  (0.666666, 0.064150) --
  (0.648148, 0.032075) --
  (0.666666, 0.000000) --
  (0.703703, 0.000000) --
  (0.722222, 0.032075) --
  (0.740740, 0.000000) --
  (0.777777, 0.000000) --
  (0.796296, 0.032075) --
  (0.777777, 0.064150) --
  (0.814815, 0.064150) --
  (0.833333, 0.096225) --
  (0.851852, 0.064150) --
  (0.888889, 0.064150) --
  (0.870370, 0.032075) --
  (0.888889, 0.000000) --
  (0.925926, 0.000000) --
  (0.944444, 0.032075) --
  (0.962963, 0.000000) --
  (1.000000, 0.000000);
\end{scope}
\end{zzfrac}
%
\zzfracskip \dots
%

\end{zztask}

%%%%%%%%%%%%%%%%%%%%%%%%%%%%%%%%%%%%%%%%%%%%%%%%%%%%%%%%%%%%%%%%%%%%%%%%%%%%%%

\begin{zztask}[Koch Snow{f}lake]
В рамках общего условия задачи построить снежинку Коха, составленную из
трех одинаковых кривых Коха. База для построения кривой (нулевое приближение) 
представляет собой равносторонний треугольник. Переход от приближения
$n$ к приближению $(n+1)$ осуществляется заменой каждого отрезка фигуры на четыре
(см. рис. при $n=1$, выделено жирным). Все отрезки (включая пустой промежуток)
имеют одинаковую длину. Угол между отрезками \mbox{составляет $60^\circ$}.
\par\endinput
\begin{pspicture}(-0.5,-0.5)(1.5,1.5)
\psgrid
\rput(0.5,-0.25){$n=0$}
\rput(0,0.21){
\psline
  (0.500000, 0.866025)
  (1.000000, 0.000000)
  (-0.000000, 0.000000)
\psline[linewidth=0.06cm, arrows=o-o]
  (0, 0)
  (0.500000, 0.866025)
}
\end{pspicture}
%
\hskip0.1in
%
\begin{pspicture}(-0.5,-0.5)(1.5,1.5)
\psgrid
\rput(0.5,-0.25){$n=1$}
\rput(0,0.21){
\psline
  (0.500000, 0.866025)
  (0.666667, 0.577350)
  (1.000000, 0.577350)
  (0.833333, 0.288675)
  (1.000000, 0.000000)
  (0.666667, 0.000000)
  (0.500000, -0.288675)
  (0.333333, 0.000000)
  (0.000000, 0.000000)
\psline[linewidth=0.06cm, arrows=o-o, showpoints=true]
  (0, 0)
  (0.166667, 0.288675)
  (-0.000000, 0.577350)
  (0.333333, 0.577350)
  (0.500000, 0.866025)
}
\end{pspicture}
%
\hskip0.1in
%
\begin{pspicture}(-0.5,-0.5)(1.5,1.5)
\psgrid
\rput(0.5,-0.25){$n=2$}
\rput(0,0.21){
\psline
  (0, 0)
  (0.055556, 0.096225)
  (-0.000000, 0.192450)
  (0.111111, 0.192450)
  (0.166667, 0.288675)
  (0.111111, 0.384900)
  (-0.000000, 0.384900)
  (0.055556, 0.481125)
  (-0.000000, 0.577350)
  (0.111111, 0.577350)
  (0.166667, 0.673575)
  (0.222222, 0.577350)
  (0.333333, 0.577350)
  (0.388889, 0.673575)
  (0.333333, 0.769800)
  (0.444444, 0.769800)
  (0.500000, 0.866025)
  (0.555556, 0.769800)
  (0.666667, 0.769800)
  (0.611111, 0.673575)
  (0.666667, 0.577350)
  (0.777778, 0.577350)
  (0.833333, 0.673575)
  (0.888889, 0.577350)
  (1.000000, 0.577350)
  (0.944444, 0.481125)
  (1.000000, 0.384900)
  (0.888889, 0.384900)
  (0.833333, 0.288675)
  (0.888889, 0.192450)
  (1.000000, 0.192450)
  (0.944444, 0.096225)
  (1.000000, 0.000000)
  (0.888889, 0.000000)
  (0.833333, -0.096225)
  (0.777778, 0.000000)
  (0.666667, 0.000000)
  (0.611111, -0.096225)
  (0.666667, -0.192450)
  (0.555556, -0.192450)
  (0.500000, -0.288675)
  (0.444444, -0.192450)
  (0.333333, -0.192450)
  (0.388889, -0.096225)
  (0.333333, 0.000000)
  (0.222222, 0.000000)
  (0.166667, -0.096225)
  (0.111111, 0.000000)
  (-0.000000, 0.000000)
}
\end{pspicture}
%
\hskip0.1in
%
\begin{pspicture}(-0.5,-0.5)(1.5,1.5)
\psgrid
\rput(0.5,-0.25){$n=3$}
\rput(0,0.21){
\psline
  (0, 0)
  (0.018519, 0.032075)
  (-0.000000, 0.064150)
  (0.037037, 0.064150)
  (0.055556, 0.096225)
  (0.037037, 0.128300)
  (-0.000000, 0.128300)
  (0.018519, 0.160375)
  (-0.000000, 0.192450)
  (0.037037, 0.192450)
  (0.055556, 0.224525)
  (0.074074, 0.192450)
  (0.111111, 0.192450)
  (0.129630, 0.224525)
  (0.111111, 0.256600)
  (0.148148, 0.256600)
  (0.166667, 0.288675)
  (0.148148, 0.320750)
  (0.111111, 0.320750)
  (0.129630, 0.352825)
  (0.111111, 0.384900)
  (0.074074, 0.384900)
  (0.055556, 0.352825)
  (0.037037, 0.384900)
  (-0.000000, 0.384900)
  (0.018519, 0.416975)
  (-0.000000, 0.449050)
  (0.037037, 0.449050)
  (0.055556, 0.481125)
  (0.037037, 0.513200)
  (-0.000000, 0.513200)
  (0.018519, 0.545275)
  (-0.000000, 0.577350)
  (0.037037, 0.577350)
  (0.055556, 0.609425)
  (0.074074, 0.577350)
  (0.111111, 0.577350)
  (0.129630, 0.609425)
  (0.111111, 0.641500)
  (0.148148, 0.641500)
  (0.166667, 0.673575)
  (0.185185, 0.641500)
  (0.222222, 0.641500)
  (0.203704, 0.609425)
  (0.222222, 0.577350)
  (0.259259, 0.577350)
  (0.277778, 0.609425)
  (0.296296, 0.577350)
  (0.333333, 0.577350)
  (0.351852, 0.609425)
  (0.333333, 0.641500)
  (0.370370, 0.641500)
  (0.388889, 0.673575)
  (0.370370, 0.705650)
  (0.333333, 0.705650)
  (0.351852, 0.737725)
  (0.333333, 0.769800)
  (0.370370, 0.769800)
  (0.388889, 0.801875)
  (0.407407, 0.769800)
  (0.444444, 0.769800)
  (0.462963, 0.801875)
  (0.444444, 0.833950)
  (0.481481, 0.833950)
  (0.500000, 0.866025)
  (0.518518, 0.833950)
  (0.555555, 0.833950)
  (0.537037, 0.801875)
  (0.555555, 0.769800)
  (0.592592, 0.769800)
  (0.611111, 0.801875)
  (0.629629, 0.769800)
  (0.666666, 0.769800)
  (0.648148, 0.737725)
  (0.666666, 0.705650)
  (0.629629, 0.705650)
  (0.611111, 0.673575)
  (0.629629, 0.641500)
  (0.666666, 0.641500)
  (0.648148, 0.609425)
  (0.666666, 0.577350)
  (0.703703, 0.577350)
  (0.722222, 0.609425)
  (0.740740, 0.577350)
  (0.777777, 0.577350)
  (0.796296, 0.609425)
  (0.777777, 0.641500)
  (0.814815, 0.641500)
  (0.833333, 0.673575)
  (0.851852, 0.641500)
  (0.888889, 0.641500)
  (0.870370, 0.609425)
  (0.888889, 0.577350)
  (0.925926, 0.577350)
  (0.944444, 0.609425)
  (0.962963, 0.577350)
  (1.000000, 0.577350)
  (0.981481, 0.545275)
  (1.000000, 0.513200)
  (0.962963, 0.513200)
  (0.944444, 0.481125)
  (0.962963, 0.449050)
  (1.000000, 0.449050)
  (0.981481, 0.416975)
  (1.000000, 0.384900)
  (0.962963, 0.384900)
  (0.944444, 0.352825)
  (0.925926, 0.384900)
  (0.888889, 0.384900)
  (0.870370, 0.352825)
  (0.888889, 0.320750)
  (0.851852, 0.320750)
  (0.833333, 0.288675)
  (0.851852, 0.256600)
  (0.888889, 0.256600)
  (0.870370, 0.224525)
  (0.888889, 0.192450)
  (0.925926, 0.192450)
  (0.944444, 0.224525)
  (0.962963, 0.192450)
  (1.000000, 0.192450)
  (0.981481, 0.160375)
  (1.000000, 0.128300)
  (0.962963, 0.128300)
  (0.944444, 0.096225)
  (0.962963, 0.064150)
  (1.000000, 0.064150)
  (0.981481, 0.032075)
  (1.000000, -0.000000)
  (0.962963, 0.000000)
  (0.944444, -0.032075)
  (0.925926, 0.000000)
  (0.888889, 0.000000)
  (0.870370, -0.032075)
  (0.888889, -0.064150)
  (0.851852, -0.064150)
  (0.833333, -0.096225)
  (0.814815, -0.064150)
  (0.777777, -0.064150)
  (0.796296, -0.032075)
  (0.777777, 0.000000)
  (0.740740, 0.000000)
  (0.722222, -0.032075)
  (0.703703, 0.000000)
  (0.666666, 0.000000)
  (0.648148, -0.032075)
  (0.666666, -0.064150)
  (0.629629, -0.064150)
  (0.611111, -0.096225)
  (0.629629, -0.128300)
  (0.666666, -0.128300)
  (0.648148, -0.160375)
  (0.666666, -0.192450)
  (0.629629, -0.192450)
  (0.611111, -0.224525)
  (0.592592, -0.192450)
  (0.555555, -0.192450)
  (0.537037, -0.224525)
  (0.555555, -0.256600)
  (0.518518, -0.256600)
  (0.500000, -0.288675)
  (0.481481, -0.256600)
  (0.444444, -0.256600)
  (0.462963, -0.224525)
  (0.444444, -0.192450)
  (0.407407, -0.192450)
  (0.388889, -0.224525)
  (0.370370, -0.192450)
  (0.333333, -0.192450)
  (0.351852, -0.160375)
  (0.333333, -0.128300)
  (0.370370, -0.128300)
  (0.388889, -0.096225)
  (0.370370, -0.064150)
  (0.333333, -0.064150)
  (0.351852, -0.032075)
  (0.333333, 0.000000)
  (0.296296, 0.000000)
  (0.277778, -0.032075)
  (0.259259, 0.000000)
  (0.222222, 0.000000)
  (0.203704, -0.032075)
  (0.222222, -0.064150)
  (0.185185, -0.064150)
  (0.166666, -0.096225)
  (0.148148, -0.064150)
  (0.111111, -0.064150)
  (0.129629, -0.032075)
  (0.111111, 0.000000)
  (0.074074, 0.000000)
  (0.055555, -0.032075)
  (0.037037, 0.000000)
  (-0.000000, 0.000000)
}
\end{pspicture}
%
\hskip0.1in\dots
%

\end{zztask}

%%%%%%%%%%%%%%%%%%%%%%%%%%%%%%%%%%%%%%%%%%%%%%%%%%%%%%%%%%%%%%%%%%%%%%%%%%%%%%

\begin{zztask}[Harter-Heighway Dragon]
В рамках общего условия задачи построить кривую дракона Хейуэя. База для 
построения кривой (нулевое приближение) представляет собой отрезок. Переход 
от приближения $n$ к приближению $(n+1)$ осуществляется заменой каждого 
отрезка фигуры на два (см. рис. при $n=1$). Все отрезки 
имеют одинаковую длину, угол между отрезками составляет $90^\circ$.
Выгиб происходит чередуясь, то в одну, то в другую сторону (см. рис. при $n=2$).
\par\endinput
\begin{pspicture}(-0.5,-0.5)(1.5,1.5)
\psgrid
\rput(0.5,-0.25){$n=0$}
\rput(0,0.5){
\psline[linewidth=0.06cm, arrows=o-o]
  (0, 0)
  (1.000000, 0.000000)
}
\end{pspicture}
%
\hskip0.1in
%
\begin{pspicture}(-0.5,-0.5)(1.5,1.5)
\psgrid
\rput(0.5,-0.25){$n=1$}
\rput(0,0.5){
\psline[linewidth=0.01cm, linestyle=dashed]
  (0, 0)
  (1.000000, 0.000000)
\psline[linewidth=0.06cm, arrows=o-o, showpoints=true]
  (0, 0)
  (0.500000, 0.500000)
  (1.000000, -0.000000)
}
\end{pspicture}
%
\hskip0.1in
%
\begin{pspicture}(-0.5,-0.5)(1.5,1.5)
\psgrid
\rput(0.5,-0.25){$n=2$}
\rput(0,0.5){
\psline[linewidth=0.01cm, linestyle=dashed]
  (0, 0)
  (0.500000, 0.500000)
  (1.000000, -0.000000)
\psline[linewidth=0.06cm, arrows=o-o, showpoints=true]
  (0, 0)
  (-0.000000, 0.500000)
  (0.500000, 0.500000)
\psline[linewidth=0.06cm, arrows=o-o, showpoints=true]
  (0.500000, 0.500000)
  (0.500000, 0.000000)
  (1.000000, 0.000000)
}
\end{pspicture}
%
\hskip0.1in
%
\begin{pspicture}(-0.5,-0.5)(1.5,1.5)
\psgrid
\rput(0.5,-0.25){$n=3$}
\rput(0,0.5){
\psline
  (0, 0)
  (-0.250000, 0.250000)
  (0.000000, 0.500000)
  (0.250000, 0.250000)
  (0.500000, 0.500000)
  (0.750000, 0.250000)
  (0.500000, -0.000000)
  (0.750000, -0.250000)
  (1.000000, -0.000000)
}
\end{pspicture}
%
\hskip0.1in \dots \hskip0.1in
%
\begin{pspicture}(-0.5,-0.5)(1.5,1.5)
\psgrid
\rput(0.5,-0.25){$n=9$}
\rput(0,0.5){
\psline
  (0, 0)
  (0.031250, 0.031250)
  (0.062500, -0.000000)
  (0.031250, -0.031250)
  (0.062500, -0.062500)
  (0.031250, -0.093750)
  (0.000000, -0.062500)
  (-0.031250, -0.093750)
  (-0.000000, -0.125000)
  (-0.031250, -0.156250)
  (-0.062500, -0.125000)
  (-0.031250, -0.093750)
  (-0.062500, -0.062500)
  (-0.093750, -0.093750)
  (-0.125000, -0.062500)
  (-0.156250, -0.093750)
  (-0.125000, -0.125000)
  (-0.156250, -0.156250)
  (-0.187500, -0.125000)
  (-0.156250, -0.093750)
  (-0.187500, -0.062500)
  (-0.156250, -0.031250)
  (-0.125000, -0.062500)
  (-0.093750, -0.031250)
  (-0.125000, -0.000000)
  (-0.156250, -0.031250)
  (-0.187500, 0.000000)
  (-0.156250, 0.031250)
  (-0.187500, 0.062500)
  (-0.218750, 0.031250)
  (-0.250000, 0.062500)
  (-0.281250, 0.031250)
  (-0.250000, 0.000000)
  (-0.281250, -0.031250)
  (-0.312500, 0.000000)
  (-0.281250, 0.031250)
  (-0.312500, 0.062500)
  (-0.281250, 0.093750)
  (-0.250000, 0.062500)
  (-0.218750, 0.093750)
  (-0.250000, 0.125000)
  (-0.218750, 0.156250)
  (-0.187500, 0.125000)
  (-0.218750, 0.093750)
  (-0.187500, 0.062500)
  (-0.156250, 0.093750)
  (-0.125000, 0.062500)
  (-0.093750, 0.093750)
  (-0.125000, 0.125000)
  (-0.156250, 0.093750)
  (-0.187500, 0.125000)
  (-0.156250, 0.156250)
  (-0.187500, 0.187500)
  (-0.156250, 0.218750)
  (-0.125000, 0.187500)
  (-0.093750, 0.218750)
  (-0.125000, 0.250000)
  (-0.156250, 0.218750)
  (-0.187500, 0.250000)
  (-0.156250, 0.281250)
  (-0.187500, 0.312500)
  (-0.218750, 0.281250)
  (-0.250000, 0.312500)
  (-0.281250, 0.281250)
  (-0.250000, 0.250000)
  (-0.281250, 0.218750)
  (-0.312500, 0.250000)
  (-0.281250, 0.281250)
  (-0.312500, 0.312500)
  (-0.281250, 0.343750)
  (-0.250000, 0.312500)
  (-0.218750, 0.343750)
  (-0.250000, 0.375000)
  (-0.218750, 0.406250)
  (-0.187500, 0.375000)
  (-0.218750, 0.343750)
  (-0.187500, 0.312500)
  (-0.156250, 0.343750)
  (-0.125000, 0.312500)
  (-0.093750, 0.343750)
  (-0.125000, 0.375000)
  (-0.093750, 0.406250)
  (-0.062500, 0.375000)
  (-0.093750, 0.343750)
  (-0.062500, 0.312500)
  (-0.093750, 0.281250)
  (-0.125000, 0.312500)
  (-0.156250, 0.281250)
  (-0.125000, 0.250000)
  (-0.093750, 0.281250)
  (-0.062500, 0.250000)
  (-0.093750, 0.218750)
  (-0.062500, 0.187500)
  (-0.031250, 0.218750)
  (-0.000000, 0.187500)
  (0.031250, 0.218750)
  (0.000000, 0.250000)
  (-0.031250, 0.218750)
  (-0.062500, 0.250000)
  (-0.031250, 0.281250)
  (-0.062500, 0.312500)
  (-0.031250, 0.343750)
  (0.000000, 0.312500)
  (0.031250, 0.343750)
  (0.000000, 0.375000)
  (0.031250, 0.406250)
  (0.062500, 0.375000)
  (0.031250, 0.343750)
  (0.062500, 0.312500)
  (0.093750, 0.343750)
  (0.125000, 0.312500)
  (0.156250, 0.343750)
  (0.125000, 0.375000)
  (0.093750, 0.343750)
  (0.062500, 0.375000)
  (0.093750, 0.406250)
  (0.062500, 0.437500)
  (0.093750, 0.468750)
  (0.125000, 0.437500)
  (0.156250, 0.468750)
  (0.125000, 0.500000)
  (0.093750, 0.468750)
  (0.062500, 0.500000)
  (0.093750, 0.531250)
  (0.062500, 0.562500)
  (0.031250, 0.531250)
  (0.000000, 0.562500)
  (-0.031250, 0.531250)
  (0.000000, 0.500000)
  (-0.031250, 0.468750)
  (-0.062500, 0.500000)
  (-0.031250, 0.531250)
  (-0.062500, 0.562500)
  (-0.031250, 0.593750)
  (0.000000, 0.562500)
  (0.031250, 0.593750)
  (0.000000, 0.625000)
  (0.031250, 0.656250)
  (0.062500, 0.625000)
  (0.031250, 0.593750)
  (0.062500, 0.562500)
  (0.093750, 0.593750)
  (0.125000, 0.562500)
  (0.156250, 0.593750)
  (0.125000, 0.625000)
  (0.156250, 0.656250)
  (0.187500, 0.625000)
  (0.156250, 0.593750)
  (0.187500, 0.562500)
  (0.156250, 0.531250)
  (0.125000, 0.562500)
  (0.093750, 0.531250)
  (0.125000, 0.500000)
  (0.156250, 0.531250)
  (0.187500, 0.500000)
  (0.156250, 0.468750)
  (0.187500, 0.437500)
  (0.218750, 0.468750)
  (0.250000, 0.437500)
  (0.281250, 0.468750)
  (0.250000, 0.500000)
  (0.281250, 0.531250)
  (0.312500, 0.500000)
  (0.281250, 0.468750)
  (0.312500, 0.437500)
  (0.281250, 0.406250)
  (0.250000, 0.437500)
  (0.218750, 0.406250)
  (0.250000, 0.375000)
  (0.218750, 0.343750)
  (0.187500, 0.375000)
  (0.218750, 0.406250)
  (0.187500, 0.437500)
  (0.156250, 0.406250)
  (0.125000, 0.437500)
  (0.093750, 0.406250)
  (0.125000, 0.375000)
  (0.156250, 0.406250)
  (0.187500, 0.375000)
  (0.156250, 0.343750)
  (0.187500, 0.312500)
  (0.156250, 0.281250)
  (0.125000, 0.312500)
  (0.093750, 0.281250)
  (0.125000, 0.250000)
  (0.156250, 0.281250)
  (0.187500, 0.250000)
  (0.156250, 0.218750)
  (0.187500, 0.187500)
  (0.218750, 0.218750)
  (0.250000, 0.187500)
  (0.281250, 0.218750)
  (0.250000, 0.250000)
  (0.218750, 0.218750)
  (0.187500, 0.250000)
  (0.218750, 0.281250)
  (0.187500, 0.312500)
  (0.218750, 0.343750)
  (0.250000, 0.312500)
  (0.281250, 0.343750)
  (0.250000, 0.375000)
  (0.281250, 0.406250)
  (0.312500, 0.375000)
  (0.281250, 0.343750)
  (0.312500, 0.312500)
  (0.343750, 0.343750)
  (0.375000, 0.312500)
  (0.406250, 0.343750)
  (0.375000, 0.375000)
  (0.406250, 0.406250)
  (0.437500, 0.375000)
  (0.406250, 0.343750)
  (0.437500, 0.312500)
  (0.406250, 0.281250)
  (0.375000, 0.312500)
  (0.343750, 0.281250)
  (0.375000, 0.250000)
  (0.406250, 0.281250)
  (0.437500, 0.250000)
  (0.406250, 0.218750)
  (0.437500, 0.187500)
  (0.468750, 0.218750)
  (0.500000, 0.187500)
  (0.531250, 0.218750)
  (0.500000, 0.250000)
  (0.468750, 0.218750)
  (0.437500, 0.250000)
  (0.468750, 0.281250)
  (0.437500, 0.312500)
  (0.468750, 0.343750)
  (0.500000, 0.312500)
  (0.531250, 0.343750)
  (0.500000, 0.375000)
  (0.531250, 0.406250)
  (0.562500, 0.375000)
  (0.531250, 0.343750)
  (0.562500, 0.312500)
  (0.593750, 0.343750)
  (0.625000, 0.312500)
  (0.656250, 0.343750)
  (0.625000, 0.375000)
  (0.593750, 0.343750)
  (0.562500, 0.375000)
  (0.593750, 0.406250)
  (0.562500, 0.437500)
  (0.593750, 0.468750)
  (0.625000, 0.437500)
  (0.656250, 0.468750)
  (0.625000, 0.500000)
  (0.593750, 0.468750)
  (0.562500, 0.500000)
  (0.593750, 0.531250)
  (0.562500, 0.562500)
  (0.531250, 0.531250)
  (0.500000, 0.562500)
  (0.468750, 0.531250)
  (0.500000, 0.500000)
  (0.468750, 0.468750)
  (0.437500, 0.500000)
  (0.468750, 0.531250)
  (0.437500, 0.562500)
  (0.468750, 0.593750)
  (0.500000, 0.562500)
  (0.531250, 0.593750)
  (0.500000, 0.625000)
  (0.531250, 0.656250)
  (0.562500, 0.625000)
  (0.531250, 0.593750)
  (0.562500, 0.562500)
  (0.593750, 0.593750)
  (0.625000, 0.562500)
  (0.656250, 0.593750)
  (0.625000, 0.625000)
  (0.656250, 0.656250)
  (0.687500, 0.625000)
  (0.656250, 0.593750)
  (0.687500, 0.562500)
  (0.656250, 0.531250)
  (0.625000, 0.562500)
  (0.593750, 0.531250)
  (0.625000, 0.500000)
  (0.656250, 0.531250)
  (0.687500, 0.500000)
  (0.656250, 0.468750)
  (0.687500, 0.437500)
  (0.718750, 0.468750)
  (0.750000, 0.437500)
  (0.781250, 0.468750)
  (0.750000, 0.500000)
  (0.781250, 0.531250)
  (0.812500, 0.500000)
  (0.781250, 0.468750)
  (0.812500, 0.437500)
  (0.781250, 0.406250)
  (0.750000, 0.437500)
  (0.718750, 0.406250)
  (0.750000, 0.375000)
  (0.718750, 0.343750)
  (0.687500, 0.375000)
  (0.718750, 0.406250)
  (0.687500, 0.437500)
  (0.656250, 0.406250)
  (0.625000, 0.437500)
  (0.593750, 0.406250)
  (0.625000, 0.375000)
  (0.656250, 0.406250)
  (0.687500, 0.375000)
  (0.656250, 0.343750)
  (0.687500, 0.312500)
  (0.656250, 0.281250)
  (0.625000, 0.312500)
  (0.593750, 0.281250)
  (0.625000, 0.250000)
  (0.656250, 0.281250)
  (0.687500, 0.250000)
  (0.656250, 0.218750)
  (0.687500, 0.187500)
  (0.718750, 0.218750)
  (0.750000, 0.187500)
  (0.781250, 0.218750)
  (0.750000, 0.250000)
  (0.781250, 0.281250)
  (0.812500, 0.250000)
  (0.781250, 0.218750)
  (0.812500, 0.187500)
  (0.781250, 0.156251)
  (0.750000, 0.187501)
  (0.718750, 0.156251)
  (0.750000, 0.125001)
  (0.718750, 0.093751)
  (0.687500, 0.125001)
  (0.718750, 0.156251)
  (0.687500, 0.187501)
  (0.656250, 0.156251)
  (0.625000, 0.187501)
  (0.593750, 0.156251)
  (0.625000, 0.125001)
  (0.593750, 0.093751)
  (0.562500, 0.125001)
  (0.593750, 0.156251)
  (0.562500, 0.187501)
  (0.593750, 0.218751)
  (0.625000, 0.187501)
  (0.656250, 0.218751)
  (0.625000, 0.250001)
  (0.593750, 0.218751)
  (0.562500, 0.250001)
  (0.593750, 0.281251)
  (0.562500, 0.312501)
  (0.531250, 0.281251)
  (0.500000, 0.312501)
  (0.468750, 0.281251)
  (0.500000, 0.250001)
  (0.531250, 0.281251)
  (0.562500, 0.250001)
  (0.531250, 0.218751)
  (0.562500, 0.187501)
  (0.531250, 0.156251)
  (0.500000, 0.187501)
  (0.468750, 0.156251)
  (0.500000, 0.125001)
  (0.468750, 0.093751)
  (0.437500, 0.125001)
  (0.468750, 0.156251)
  (0.437500, 0.187501)
  (0.406250, 0.156251)
  (0.375000, 0.187501)
  (0.343750, 0.156251)
  (0.375000, 0.125001)
  (0.406250, 0.156251)
  (0.437500, 0.125001)
  (0.406250, 0.093751)
  (0.437500, 0.062501)
  (0.406250, 0.031251)
  (0.375000, 0.062501)
  (0.343750, 0.031251)
  (0.375000, 0.000001)
  (0.406250, 0.031251)
  (0.437500, 0.000001)
  (0.406250, -0.031249)
  (0.437500, -0.062499)
  (0.468750, -0.031249)
  (0.500000, -0.062499)
  (0.531250, -0.031249)
  (0.500000, 0.000001)
  (0.468750, -0.031249)
  (0.437500, 0.000001)
  (0.468750, 0.031251)
  (0.437500, 0.062501)
  (0.468750, 0.093751)
  (0.500000, 0.062501)
  (0.531250, 0.093751)
  (0.500000, 0.125001)
  (0.531250, 0.156251)
  (0.562500, 0.125001)
  (0.531250, 0.093751)
  (0.562500, 0.062501)
  (0.593750, 0.093751)
  (0.625000, 0.062501)
  (0.656250, 0.093751)
  (0.625000, 0.125001)
  (0.656250, 0.156251)
  (0.687500, 0.125001)
  (0.656250, 0.093751)
  (0.687500, 0.062501)
  (0.656250, 0.031251)
  (0.625000, 0.062501)
  (0.593750, 0.031251)
  (0.625000, 0.000001)
  (0.656250, 0.031251)
  (0.687500, 0.000001)
  (0.656250, -0.031249)
  (0.687500, -0.062499)
  (0.718750, -0.031249)
  (0.750000, -0.062499)
  (0.781250, -0.031249)
  (0.750000, 0.000001)
  (0.781250, 0.031251)
  (0.812500, 0.000001)
  (0.781250, -0.031249)
  (0.812500, -0.062499)
  (0.781250, -0.093749)
  (0.750000, -0.062499)
  (0.718750, -0.093749)
  (0.750000, -0.124999)
  (0.718750, -0.156249)
  (0.687500, -0.124999)
  (0.718750, -0.093749)
  (0.687500, -0.062499)
  (0.656250, -0.093749)
  (0.625000, -0.062499)
  (0.593750, -0.093749)
  (0.625000, -0.124999)
  (0.656250, -0.093749)
  (0.687500, -0.124999)
  (0.656250, -0.156249)
  (0.687500, -0.187499)
  (0.656250, -0.218749)
  (0.625000, -0.187499)
  (0.593750, -0.218749)
  (0.625000, -0.249999)
  (0.656250, -0.218749)
  (0.687500, -0.249999)
  (0.656250, -0.281249)
  (0.687500, -0.312499)
  (0.718750, -0.281249)
  (0.750000, -0.312499)
  (0.781250, -0.281249)
  (0.750000, -0.249999)
  (0.718750, -0.281249)
  (0.687500, -0.249999)
  (0.718750, -0.218749)
  (0.687500, -0.187499)
  (0.718750, -0.156249)
  (0.750000, -0.187499)
  (0.781250, -0.156249)
  (0.750000, -0.124999)
  (0.781250, -0.093749)
  (0.812500, -0.124999)
  (0.781250, -0.156249)
  (0.812500, -0.187499)
  (0.843750, -0.156249)
  (0.875000, -0.187499)
  (0.906250, -0.156249)
  (0.875000, -0.124999)
  (0.906250, -0.093749)
  (0.937500, -0.124999)
  (0.906250, -0.156249)
  (0.937500, -0.187499)
  (0.906250, -0.218749)
  (0.875000, -0.187499)
  (0.843750, -0.218749)
  (0.875000, -0.249999)
  (0.906250, -0.218749)
  (0.937500, -0.249999)
  (0.906250, -0.281249)
  (0.937500, -0.312499)
  (0.968750, -0.281249)
  (1.000000, -0.312499)
  (1.031250, -0.281249)
  (1.000000, -0.249999)
  (0.968750, -0.281249)
  (0.937500, -0.249999)
  (0.968750, -0.218749)
  (0.937500, -0.187499)
  (0.968750, -0.156249)
  (1.000000, -0.187499)
  (1.031250, -0.156249)
  (1.000000, -0.124999)
  (1.031250, -0.093749)
  (1.062500, -0.124999)
  (1.031250, -0.156249)
  (1.062500, -0.187499)
  (1.093750, -0.156249)
  (1.125000, -0.187499)
  (1.156250, -0.156249)
  (1.125000, -0.124999)
  (1.093750, -0.156249)
  (1.062500, -0.124999)
  (1.093750, -0.093749)
  (1.062500, -0.062499)
  (1.093750, -0.031249)
  (1.125000, -0.062499)
  (1.156250, -0.031249)
  (1.125000, 0.000001)
  (1.093750, -0.031249)
  (1.062500, 0.000001)
  (1.093750, 0.031251)
  (1.062500, 0.062501)
  (1.031250, 0.031251)
  (1.000000, 0.062501)
  (0.968750, 0.031251)
  (1.000000, 0.000001)
}
\end{pspicture}
%
\hskip0.1in \dots
%

\end{zztask}

%\clearpage

%%%%%%%%%%%%%%%%%%%%%%%%%%%%%%%%%%%%%%%%%%%%%%%%%%%%%%%%%%%%%%%%%%%%%%%%%%%%%%

\begin{zztask}[Knuth's Terdragon]
В рамках общего условия задачи построить кривую дракона Кнута. База для 
построения кривой (нулевое приближение) представляет собой отрезок. Переход 
от приближения $n$ к приближению $(n+1)$ осуществляется заменой каждого 
отрезка фигуры на три (см. рис. при $n=1$). Все отрезки 
имеют одинаковую длину, угол между отрезками составляет $60^\circ$.
\par\endinput
\begin{pspicture}(-0.5,-0.5)(1.5,1.5)
\psgrid
\rput(0.5,-0.25){$n=0$}
\rput(0,0.5){
\psline[linewidth=0.06cm, arrows=o-o]
  (0, 0)
  (1.000000, 0.000000)
}
\end{pspicture}
%
\hskip0.1in
%
\begin{pspicture}(-0.5,-0.5)(1.5,1.5)
\psgrid
\rput(0.5,-0.25){$n=1$}
\rput(0,0.5){
\psline[linewidth=0.01cm, linestyle=dashed]
  (0, 0)
  (1.000000, 0.000000)
\psline[linewidth=0.06cm, arrows=o-o, showpoints=true]
  (0, 0)
  (0.500000, 0.288675)
  (0.500000, -0.288675)
  (1.000000, 0.000000)
}
\end{pspicture}
%
\hskip0.1in
%
\begin{pspicture}(-0.5,-0.5)(1.5,1.5)
\psgrid
\rput(0.5,-0.25){$n=2$}
\rput(0,0.5){
\psline[linewidth=0.01cm, linestyle=dashed]
  (0, 0)
  (0.500000, 0.288675)
  (0.500000, -0.288675)
  (1.000000, 0.000000)
\psline
  (0, 0)
  (0.166667, 0.288675)
  (0.333333, 0.000000)
  (0.500000, 0.288675)
  (0.666667, 0.000000)
  (0.333333, -0.000000)
  (0.500000, -0.288675)
  (0.666667, -0.000000)
  (0.833333, -0.288675)
  (1.000000, -0.000000)
\psline[linewidth=0.06cm, arrows=o-o, showpoints=true]
  (0, 0)
  (0.166667, 0.288675)
  (0.333333, 0.000000)
  (0.500000, 0.288675)
}
\end{pspicture}
%
\hskip0.1in
%
\begin{pspicture}(-0.5,-0.5)(1.5,1.5)
\psgrid
\rput(0.5,-0.25){$n=3$}
\rput(0,0.5){
\psline
  (0, 0)
  (-0.000000, 0.192450)
  (0.166667, 0.096225)
  (0.166667, 0.288675)
  (0.333333, 0.192450)
  (0.166667, 0.096225)
  (0.333333, -0.000000)
  (0.333333, 0.192450)
  (0.500000, 0.096225)
  (0.500000, 0.288675)
  (0.666667, 0.192450)
  (0.500000, 0.096225)
  (0.666667, -0.000000)
  (0.500000, -0.096225)
  (0.500000, 0.096225)
  (0.333333, -0.000000)
  (0.500000, -0.096225)
  (0.333333, -0.192450)
  (0.500000, -0.288675)
  (0.500000, -0.096225)
  (0.666666, -0.192450)
  (0.666666, -0.000000)
  (0.833333, -0.096225)
  (0.666666, -0.192450)
  (0.833333, -0.288675)
  (0.833333, -0.096225)
  (1.000000, -0.192450)
  (1.000000, -0.000000)
}
\end{pspicture}
%
\hskip0.1in\dots\hskip0.1in
%
\begin{pspicture}(-0.5,-0.5)(1.5,1.5)
\psgrid
\rput(0.5,-0.25){$n=5$}
\rput(0,0.5){
\psline
  (0, 0)
  (-0.055556, 0.032075)
  (0.000000, 0.064150)
  (-0.055556, 0.096225)
  (0.000000, 0.128300)
  (-0.000000, 0.064150)
  (0.055556, 0.096225)
  (-0.000000, 0.128300)
  (0.055556, 0.160375)
  (-0.000000, 0.192450)
  (0.055556, 0.224525)
  (0.055556, 0.160375)
  (0.111111, 0.192450)
  (0.111111, 0.128300)
  (0.055556, 0.160375)
  (0.055556, 0.096225)
  (0.111111, 0.128300)
  (0.111111, 0.064150)
  (0.166667, 0.096225)
  (0.111111, 0.128300)
  (0.166667, 0.160375)
  (0.111111, 0.192450)
  (0.166667, 0.224525)
  (0.166667, 0.160375)
  (0.222222, 0.192450)
  (0.166667, 0.224525)
  (0.222222, 0.256600)
  (0.166667, 0.288675)
  (0.222222, 0.320750)
  (0.222222, 0.256600)
  (0.277778, 0.288675)
  (0.277778, 0.224525)
  (0.222222, 0.256600)
  (0.222222, 0.192450)
  (0.277778, 0.224525)
  (0.277778, 0.160375)
  (0.333333, 0.192450)
  (0.333333, 0.128300)
  (0.277778, 0.160375)
  (0.277778, 0.096225)
  (0.222222, 0.128300)
  (0.277778, 0.160375)
  (0.222222, 0.192450)
  (0.222222, 0.128300)
  (0.166667, 0.160375)
  (0.166667, 0.096225)
  (0.222222, 0.128300)
  (0.222222, 0.064150)
  (0.277778, 0.096225)
  (0.277778, 0.032075)
  (0.222222, 0.064150)
  (0.222222, 0.000000)
  (0.277778, 0.032075)
  (0.277778, -0.032075)
  (0.333333, 0.000000)
  (0.277778, 0.032075)
  (0.333333, 0.064150)
  (0.277778, 0.096225)
  (0.333333, 0.128300)
  (0.333333, 0.064150)
  (0.388889, 0.096225)
  (0.333333, 0.128300)
  (0.388889, 0.160375)
  (0.333333, 0.192450)
  (0.388889, 0.224525)
  (0.388889, 0.160375)
  (0.444444, 0.192450)
  (0.444444, 0.128300)
  (0.388889, 0.160375)
  (0.388889, 0.096225)
  (0.444444, 0.128300)
  (0.444444, 0.064150)
  (0.500000, 0.096225)
  (0.444444, 0.128300)
  (0.500000, 0.160375)
  (0.444444, 0.192450)
  (0.500000, 0.224525)
  (0.500000, 0.160375)
  (0.555555, 0.192450)
  (0.500000, 0.224525)
  (0.555555, 0.256600)
  (0.500000, 0.288675)
  (0.555555, 0.320750)
  (0.555555, 0.256600)
  (0.611111, 0.288675)
  (0.611111, 0.224525)
  (0.555555, 0.256600)
  (0.555555, 0.192450)
  (0.611111, 0.224525)
  (0.611111, 0.160375)
  (0.666667, 0.192450)
  (0.666667, 0.128300)
  (0.611111, 0.160375)
  (0.611111, 0.096225)
  (0.555555, 0.128300)
  (0.611111, 0.160375)
  (0.555555, 0.192450)
  (0.555555, 0.128300)
  (0.500000, 0.160375)
  (0.500000, 0.096225)
  (0.555555, 0.128300)
  (0.555555, 0.064150)
  (0.611111, 0.096225)
  (0.611111, 0.032075)
  (0.555555, 0.064150)
  (0.555555, 0.000000)
  (0.611111, 0.032075)
  (0.611111, -0.032075)
  (0.666667, 0.000000)
  (0.666667, -0.064150)
  (0.611111, -0.032075)
  (0.611111, -0.096225)
  (0.555555, -0.064150)
  (0.611111, -0.032075)
  (0.555555, 0.000000)
  (0.555555, -0.064150)
  (0.500000, -0.032075)
  (0.500000, -0.096225)
  (0.444444, -0.064150)
  (0.500000, -0.032075)
  (0.444444, 0.000000)
  (0.500000, 0.032075)
  (0.500000, -0.032075)
  (0.555555, 0.000000)
  (0.500000, 0.032075)
  (0.555555, 0.064150)
  (0.500000, 0.096225)
  (0.500000, 0.032075)
  (0.444444, 0.064150)
  (0.444444, 0.000000)
  (0.388889, 0.032075)
  (0.444444, 0.064150)
  (0.388889, 0.096225)
  (0.388889, 0.032075)
  (0.333333, 0.064150)
  (0.333333, 0.000000)
  (0.388889, 0.032075)
  (0.388889, -0.032075)
  (0.444444, 0.000000)
  (0.444444, -0.064150)
  (0.388889, -0.032075)
  (0.388889, -0.096225)
  (0.444444, -0.064150)
  (0.444444, -0.128300)
  (0.500000, -0.096225)
  (0.500000, -0.160375)
  (0.444444, -0.128300)
  (0.444444, -0.192450)
  (0.388889, -0.160375)
  (0.444444, -0.128300)
  (0.388889, -0.096225)
  (0.388889, -0.160375)
  (0.333333, -0.128300)
  (0.333333, -0.192450)
  (0.388889, -0.160375)
  (0.388889, -0.224525)
  (0.444444, -0.192450)
  (0.444444, -0.256600)
  (0.388889, -0.224525)
  (0.388889, -0.288675)
  (0.444444, -0.256600)
  (0.444444, -0.320750)
  (0.500000, -0.288675)
  (0.444444, -0.256600)
  (0.500000, -0.224525)
  (0.444444, -0.192450)
  (0.500000, -0.160375)
  (0.500000, -0.224525)
  (0.555555, -0.192450)
  (0.500000, -0.160375)
  (0.555555, -0.128300)
  (0.500000, -0.096225)
  (0.555555, -0.064150)
  (0.555555, -0.128300)
  (0.611111, -0.096225)
  (0.611111, -0.160375)
  (0.555555, -0.128300)
  (0.555555, -0.192450)
  (0.611111, -0.160375)
  (0.611111, -0.224525)
  (0.666666, -0.192450)
  (0.611111, -0.160375)
  (0.666666, -0.128300)
  (0.611111, -0.096225)
  (0.666666, -0.064150)
  (0.666666, -0.128300)
  (0.722222, -0.096225)
  (0.666666, -0.064150)
  (0.722222, -0.032075)
  (0.666666, 0.000000)
  (0.722222, 0.032075)
  (0.722222, -0.032075)
  (0.777778, 0.000000)
  (0.777778, -0.064150)
  (0.722222, -0.032075)
  (0.722222, -0.096225)
  (0.777778, -0.064150)
  (0.777778, -0.128300)
  (0.833333, -0.096225)
  (0.833333, -0.160375)
  (0.777778, -0.128300)
  (0.777778, -0.192450)
  (0.722222, -0.160375)
  (0.777778, -0.128300)
  (0.722222, -0.096225)
  (0.722222, -0.160375)
  (0.666666, -0.128300)
  (0.666666, -0.192450)
  (0.722222, -0.160375)
  (0.722222, -0.224525)
  (0.777778, -0.192450)
  (0.777778, -0.256600)
  (0.722222, -0.224525)
  (0.722222, -0.288675)
  (0.777778, -0.256600)
  (0.777778, -0.320750)
  (0.833333, -0.288675)
  (0.777778, -0.256600)
  (0.833333, -0.224525)
  (0.777778, -0.192450)
  (0.833333, -0.160375)
  (0.833333, -0.224525)
  (0.888889, -0.192450)
  (0.833333, -0.160375)
  (0.888889, -0.128300)
  (0.833333, -0.096225)
  (0.888889, -0.064150)
  (0.888889, -0.128300)
  (0.944444, -0.096225)
  (0.944444, -0.160375)
  (0.888889, -0.128300)
  (0.888889, -0.192450)
  (0.944444, -0.160375)
  (0.944444, -0.224525)
  (1.000000, -0.192450)
  (0.944444, -0.160375)
  (1.000000, -0.128300)
  (0.944444, -0.096225)
  (1.000000, -0.064150)
  (1.000000, -0.128300)
  (1.055555, -0.096225)
  (1.000000, -0.064150)
  (1.055555, -0.032075)
  (1.000000, 0.000000)
}
\end{pspicture}
%
\hskip0.1in\dots
%
\end{zztask}

%%%%%%%%%%%%%%%%%%%%%%%%%%%%%%%%%%%%%%%%%%%%%%%%%%%%%%%%%%%%%%%%%%%%%%%%%%%%%%

\begin{zztask}[L\'evy C Сurve]
В рамках общего условия задачи построить кривую Леви. База для 
построения кривой (нулевое приближение) представляет собой отрезок. Переход 
от приближения $n$ к приближению $(n+1)$ осуществляется заменой каждого 
отрезка фигуры на два (см. рис. при $n=1$). Все отрезки 
имеют одинаковую длину, угол между отрезками составляет $90^\circ$.
\par\endinput
\begin{pspicture}(-0.5,-0.5)(1.5,1.5)
\psgrid
\rput(0.5,-0.25){$n=0$}
\rput(0,0.5){
\psline[linewidth=0.06cm, arrows=o-o]
  (0, 0)
  (1.000000, 0.000000)
}
\end{pspicture}
%
\hskip0.1in
%
\begin{pspicture}(-0.5,-0.5)(1.5,1.5)
\psgrid
\rput(0.5,-0.25){$n=1$}
\rput(0,0.5){
\psline[linewidth=0.01cm, linestyle=dashed]
  (0, 0)
  (1.000000, 0.000000)
\psline[linewidth=0.06cm, arrows=o-o, showpoints=true]
  (0, 0)
  (0.500000, 0.500000)
  (1.000000, -0.000000)
}
\end{pspicture}
%
\hskip0.1in
%
\begin{pspicture}(-0.5,-0.5)(1.5,1.5)
\psgrid
\rput(0.5,-0.25){$n=2$}
\rput(0,0.5){
\psline[linewidth=0.01cm, linestyle=dashed]
  (0, 0)
  (0.500000, 0.500000)
  (1.000000, -0.000000)
\psline[linewidth=0.06cm, arrows=o-o, showpoints=true]
  (0, 0)
  (-0.000000, 0.500000)
  (0.500000, 0.500000)
\psline[linewidth=0.06cm, arrows=o-o, showpoints=true]
  (0.500000, 0.500000)
  (1.000000, 0.500000)
  (1.000000, 0.000000)
}
\end{pspicture}
%
\hskip0.1in
%
\begin{pspicture}(-0.5,-0.5)(1.5,1.5)
\psgrid
\rput(0.5,-0.25){$n=3$}
\rput(0,0.5){
\psline
  (0, 0)
  (-0.250000, 0.250000)
  (0.000000, 0.500000)
  (0.250000, 0.750000)
  (0.500000, 0.500000)
  (0.750000, 0.750000)
  (1.000000, 0.500000)
  (1.250000, 0.250000)
  (1.000000, -0.000000)
}
\end{pspicture}
%
\hskip0.1in \dots \hskip0.1in
%
\begin{pspicture}(-0.5,-0.5)(1.5,1.5)
\psgrid
\rput(0.5,-0.25){$n=8$}
\rput(0,0.5){
\psline
  (0, 0)
  (0.062500, -0.000000)
  (0.062500, -0.062500)
  (0.062500, -0.125000)
  (0.000000, -0.125000)
  (0.000000, -0.187500)
  (-0.062500, -0.187500)
  (-0.125000, -0.187500)
  (-0.125000, -0.125000)
  (-0.125000, -0.187500)
  (-0.187500, -0.187500)
  (-0.250000, -0.187500)
  (-0.250000, -0.125000)
  (-0.312500, -0.125000)
  (-0.312500, -0.062500)
  (-0.312500, 0.000000)
  (-0.250000, 0.000000)
  (-0.250000, -0.062500)
  (-0.312500, -0.062500)
  (-0.375000, -0.062500)
  (-0.375000, 0.000000)
  (-0.437500, 0.000000)
  (-0.437500, 0.062500)
  (-0.437500, 0.125000)
  (-0.375000, 0.125000)
  (-0.437500, 0.125000)
  (-0.437500, 0.187500)
  (-0.437500, 0.250000)
  (-0.375000, 0.250000)
  (-0.375000, 0.312500)
  (-0.312500, 0.312500)
  (-0.250000, 0.312500)
  (-0.250000, 0.250000)
  (-0.250000, 0.187500)
  (-0.312500, 0.187500)
  (-0.375000, 0.187500)
  (-0.375000, 0.250000)
  (-0.437500, 0.250000)
  (-0.437500, 0.312500)
  (-0.437500, 0.375000)
  (-0.375000, 0.375000)
  (-0.437500, 0.375000)
  (-0.437500, 0.437500)
  (-0.437500, 0.500000)
  (-0.375000, 0.500000)
  (-0.375000, 0.562500)
  (-0.312500, 0.562500)
  (-0.250000, 0.562500)
  (-0.250000, 0.500000)
  (-0.312500, 0.500000)
  (-0.312500, 0.562500)
  (-0.312499, 0.625000)
  (-0.250000, 0.625000)
  (-0.249999, 0.687500)
  (-0.187500, 0.687500)
  (-0.125000, 0.687500)
  (-0.125000, 0.625000)
  (-0.125000, 0.687500)
  (-0.062500, 0.687500)
  (0.000000, 0.687500)
  (0.000000, 0.625000)
  (0.062500, 0.625000)
  (0.062500, 0.562500)
  (0.062500, 0.500000)
  (0.000000, 0.500000)
  (0.000000, 0.437500)
  (-0.062500, 0.437500)
  (-0.125000, 0.437500)
  (-0.125000, 0.500000)
  (-0.187500, 0.500000)
  (-0.187500, 0.562500)
  (-0.187500, 0.625000)
  (-0.125000, 0.625000)
  (-0.187500, 0.625000)
  (-0.187500, 0.687500)
  (-0.187499, 0.750000)
  (-0.125000, 0.750000)
  (-0.124999, 0.812500)
  (-0.062499, 0.812500)
  (0.000001, 0.812500)
  (0.000000, 0.750000)
  (-0.062500, 0.750000)
  (-0.062499, 0.812500)
  (-0.062499, 0.875000)
  (0.000001, 0.875000)
  (0.000001, 0.937500)
  (0.062501, 0.937500)
  (0.125001, 0.937500)
  (0.125000, 0.875000)
  (0.125001, 0.937500)
  (0.187500, 0.937500)
  (0.250000, 0.937500)
  (0.250000, 0.875000)
  (0.312500, 0.875000)
  (0.312500, 0.812500)
  (0.312500, 0.750000)
  (0.250000, 0.750000)
  (0.187500, 0.750000)
  (0.187500, 0.812500)
  (0.187500, 0.875000)
  (0.250000, 0.875000)
  (0.250000, 0.937500)
  (0.312500, 0.937500)
  (0.375000, 0.937500)
  (0.375000, 0.875000)
  (0.375000, 0.937500)
  (0.437500, 0.937500)
  (0.500000, 0.937500)
  (0.500000, 0.875000)
  (0.562500, 0.875000)
  (0.562500, 0.812500)
  (0.562500, 0.750000)
  (0.500000, 0.750000)
  (0.500000, 0.812500)
  (0.562500, 0.812500)
  (0.625000, 0.812500)
  (0.625000, 0.750000)
  (0.687500, 0.750000)
  (0.687500, 0.687500)
  (0.687500, 0.625000)
  (0.625000, 0.625000)
  (0.687500, 0.625000)
  (0.687500, 0.562500)
  (0.687500, 0.500000)
  (0.625000, 0.500000)
  (0.625000, 0.437500)
  (0.562500, 0.437500)
  (0.500000, 0.437500)
  (0.500000, 0.500000)
  (0.500000, 0.437500)
  (0.437500, 0.437500)
  (0.375000, 0.437500)
  (0.375001, 0.500000)
  (0.312501, 0.500000)
  (0.312501, 0.562500)
  (0.312501, 0.625000)
  (0.375001, 0.625000)
  (0.312501, 0.625000)
  (0.312501, 0.687500)
  (0.312501, 0.750000)
  (0.375001, 0.750000)
  (0.375001, 0.812500)
  (0.437501, 0.812500)
  (0.500001, 0.812500)
  (0.500001, 0.750000)
  (0.437501, 0.750000)
  (0.437501, 0.812500)
  (0.437501, 0.875000)
  (0.500001, 0.875000)
  (0.500001, 0.937500)
  (0.562501, 0.937500)
  (0.625001, 0.937500)
  (0.625001, 0.875000)
  (0.625001, 0.937500)
  (0.687501, 0.937500)
  (0.750001, 0.937500)
  (0.750001, 0.875000)
  (0.812501, 0.875000)
  (0.812501, 0.812500)
  (0.812501, 0.750000)
  (0.750001, 0.750000)
  (0.687501, 0.750000)
  (0.687501, 0.812500)
  (0.687501, 0.875000)
  (0.750001, 0.875000)
  (0.750001, 0.937500)
  (0.812501, 0.937500)
  (0.875001, 0.937500)
  (0.875001, 0.875000)
  (0.875001, 0.937500)
  (0.937501, 0.937500)
  (1.000001, 0.937500)
  (1.000001, 0.875000)
  (1.062501, 0.875000)
  (1.062501, 0.812500)
  (1.062501, 0.750000)
  (1.000001, 0.750000)
  (1.000001, 0.812500)
  (1.062501, 0.812500)
  (1.125001, 0.812500)
  (1.125001, 0.750000)
  (1.187501, 0.750000)
  (1.187501, 0.687500)
  (1.187501, 0.625000)
  (1.125001, 0.625000)
  (1.187501, 0.625000)
  (1.187501, 0.562500)
  (1.187501, 0.500000)
  (1.125001, 0.500000)
  (1.125001, 0.437500)
  (1.062501, 0.437501)
  (1.000001, 0.437501)
  (1.000001, 0.500001)
  (0.937501, 0.500001)
  (0.937501, 0.562501)
  (0.937501, 0.625001)
  (1.000001, 0.625001)
  (1.000001, 0.687501)
  (1.062501, 0.687501)
  (1.125001, 0.687501)
  (1.125001, 0.625001)
  (1.125001, 0.687501)
  (1.187501, 0.687501)
  (1.250001, 0.687501)
  (1.250001, 0.625001)
  (1.312501, 0.625001)
  (1.312501, 0.562501)
  (1.312501, 0.500001)
  (1.250001, 0.500001)
  (1.250001, 0.562501)
  (1.312501, 0.562501)
  (1.375001, 0.562501)
  (1.375001, 0.500001)
  (1.437501, 0.500001)
  (1.437501, 0.437501)
  (1.437501, 0.375001)
  (1.375001, 0.375001)
  (1.437501, 0.375001)
  (1.437501, 0.312501)
  (1.437501, 0.250001)
  (1.375001, 0.250001)
  (1.375001, 0.187501)
  (1.312501, 0.187501)
  (1.250001, 0.187501)
  (1.250001, 0.250001)
  (1.250001, 0.312501)
  (1.312501, 0.312501)
  (1.375001, 0.312501)
  (1.375001, 0.250001)
  (1.437501, 0.250001)
  (1.437501, 0.187501)
  (1.437501, 0.125001)
  (1.375001, 0.125001)
  (1.437501, 0.125001)
  (1.437501, 0.062501)
  (1.437501, 0.000001)
  (1.375001, 0.000001)
  (1.375001, -0.062499)
  (1.312501, -0.062499)
  (1.250001, -0.062499)
  (1.250001, 0.000001)
  (1.312501, 0.000001)
  (1.312501, -0.062499)
  (1.312501, -0.124999)
  (1.250001, -0.124999)
  (1.250001, -0.187499)
  (1.187501, -0.187499)
  (1.125001, -0.187499)
  (1.125001, -0.124999)
  (1.125001, -0.187499)
  (1.062501, -0.187499)
  (1.000001, -0.187499)
  (1.000001, -0.124999)
  (0.937501, -0.124999)
  (0.937501, -0.062499)
  (0.937501, 0.000001)
  (1.000001, 0.000001)
}
\end{pspicture}
%
\hskip0.1in \dots
%

\end{zztask}

%%%%%%%%%%%%%%%%%%%%%%%%%%%%%%%%%%%%%%%%%%%%%%%%%%%%%%%%%%%%%%%%%%%%%%%%%%%%%%

\begin{zztask}[Sierpi\'nski Arrowhead Curve]
В рамках общего условия задачи построить стреловидную кривую Серпинского.
База для построения кривой (нулевое приближение) представляет собой отрезок.
Переход от приближения $n$ к приближению $(n+1)$ осуществляется заменой каждого 
отрезка фигуры на три (см. рис. при $n=1$). 
Выгиб происходит чередуясь, то в одну, то в другую сторону (см. рис. при $n=2$).
Все отрезки 
имеют одинаковую длину, угол между отрезками составляет $120^\circ$. Вся фигура
как бы ``вписывается'' в равносторонний треугольник (заполняя треугольник 
Серпинского).
\par\endinput
\begin{pspicture}(-0.5,-0.5)(1.5,1.5)
\psgrid
\rput(0.5,-0.25){$n=0$}
\rput(0,0){
\psline[linewidth=0.06cm, arrows=o-o]
  (0, 0)
  (1.000000, 0.000000)
}
\end{pspicture}
%
\hskip0.1in
%
\begin{pspicture}(-0.5,-0.5)(1.5,1.5)
\psgrid
\rput(0.5,-0.25){$n=1$}
\rput(0,0){
\psline[linewidth=0.01cm, linestyle=dashed]
  (0, 0)
  (1.000000, 0.000000)
\psline[linewidth=0.06cm, arrows=o-o, showpoints=true]
  (0, 0)
  (0.250000, 0.433013)
  (0.750000, 0.433013)
  (1.000000, 0.000000)
}
\end{pspicture}
%
\hskip0.1in
%
\begin{pspicture}(-0.5,-0.5)(1.5,1.5)
\psgrid
\rput(0.5,-0.25){$n=2$}
\rput(0,0){
\psline[linewidth=0.01cm, linestyle=dashed]
  (0, 0)
  (0.250000, 0.433013)
  (0.750000, 0.433013)
  (1.000000, 0.000000)
\psline[linewidth=0.06cm, arrows=o-, showpoints=true]
  (0, 0)
  (0.250000, 0.000000)
  (0.375000, 0.216506)
  (0.250000, 0.433013)
\psline[linewidth=0.06cm, arrows=o-, showpoints=true]
  (0.250000, 0.433013)
  (0.375000, 0.649519)
  (0.625000, 0.649519)
  (0.750000, 0.433013)
\psline[linewidth=0.06cm, arrows=o-o, showpoints=true]
  (0.750000, 0.433013)
  (0.625000, 0.216506)
  (0.750000, -0.000000)
  (1.000000, -0.000000)
}
\end{pspicture}
%
\hskip0.1in
%
\begin{pspicture}(-0.5,-0.5)(1.5,1.5)
\psgrid
\rput(0.5,-0.25){$n=3$}
\rput(0,0){
\psline
  (0, 0)
  (0.062500, 0.108253)
  (0.187500, 0.108253)
  (0.250000, 0.000000)
  (0.375000, 0.000000)
  (0.437500, 0.108253)
  (0.375000, 0.216506)
  (0.250000, 0.216506)
  (0.187500, 0.324760)
  (0.250000, 0.433013)
  (0.375000, 0.433013)
  (0.437500, 0.541266)
  (0.375000, 0.649519)
  (0.437500, 0.757772)
  (0.562500, 0.757772)
  (0.625000, 0.649519)
  (0.562500, 0.541266)
  (0.625000, 0.433013)
  (0.750000, 0.433013)
  (0.812500, 0.324759)
  (0.750000, 0.216506)
  (0.625000, 0.216506)
  (0.562500, 0.108253)
  (0.625000, -0.000000)
  (0.750000, -0.000000)
  (0.812500, 0.108253)
  (0.937500, 0.108253)
  (1.000000, -0.000000)
}
\end{pspicture}
%
\hskip0.1in\dots\hskip0.1in
%
\begin{pspicture}(-0.5,-0.5)(1.5,1.5)
\psgrid
\rput(0.5,-0.25){$n=6$}
\rput(0,0){
\psline
  (0, 0)
  (0.015625, 0.000000)
  (0.023438, 0.013532)
  (0.015625, 0.027063)
  (0.023437, 0.040595)
  (0.039063, 0.040595)
  (0.046875, 0.027063)
  (0.039063, 0.013532)
  (0.046875, -0.000000)
  (0.062500, -0.000000)
  (0.070313, 0.013532)
  (0.085938, 0.013532)
  (0.093750, -0.000000)
  (0.109375, -0.000000)
  (0.117188, 0.013532)
  (0.109375, 0.027063)
  (0.093750, 0.027063)
  (0.085938, 0.040595)
  (0.093750, 0.054127)
  (0.085938, 0.067658)
  (0.070313, 0.067658)
  (0.062500, 0.054127)
  (0.046875, 0.054127)
  (0.039063, 0.067658)
  (0.046875, 0.081190)
  (0.062500, 0.081190)
  (0.070313, 0.094722)
  (0.062500, 0.108253)
  (0.070313, 0.121785)
  (0.085938, 0.121785)
  (0.093750, 0.108253)
  (0.109375, 0.108253)
  (0.117188, 0.121785)
  (0.109375, 0.135316)
  (0.093750, 0.135316)
  (0.085938, 0.148848)
  (0.093750, 0.162380)
  (0.109375, 0.162380)
  (0.117188, 0.175911)
  (0.109375, 0.189443)
  (0.117188, 0.202975)
  (0.132813, 0.202975)
  (0.140625, 0.189443)
  (0.132813, 0.175911)
  (0.140625, 0.162380)
  (0.156250, 0.162380)
  (0.164063, 0.148848)
  (0.156250, 0.135316)
  (0.140625, 0.135316)
  (0.132813, 0.121785)
  (0.140625, 0.108253)
  (0.156250, 0.108253)
  (0.164063, 0.121785)
  (0.179688, 0.121785)
  (0.187500, 0.108253)
  (0.179688, 0.094722)
  (0.187500, 0.081190)
  (0.203125, 0.081190)
  (0.210938, 0.067658)
  (0.203125, 0.054127)
  (0.187500, 0.054127)
  (0.179688, 0.067658)
  (0.164063, 0.067658)
  (0.156250, 0.054127)
  (0.164063, 0.040595)
  (0.156250, 0.027063)
  (0.140625, 0.027063)
  (0.132813, 0.013532)
  (0.140625, -0.000000)
  (0.156250, -0.000000)
  (0.164063, 0.013532)
  (0.179688, 0.013532)
  (0.187500, -0.000000)
  (0.203125, -0.000000)
  (0.210938, 0.013532)
  (0.203125, 0.027063)
  (0.210938, 0.040595)
  (0.226563, 0.040595)
  (0.234375, 0.027063)
  (0.226563, 0.013532)
  (0.234375, -0.000000)
  (0.250000, -0.000000)
  (0.257813, 0.013532)
  (0.273438, 0.013532)
  (0.281250, -0.000000)
  (0.296875, -0.000000)
  (0.304688, 0.013532)
  (0.296875, 0.027063)
  (0.281250, 0.027063)
  (0.273438, 0.040595)
  (0.281250, 0.054127)
  (0.296875, 0.054127)
  (0.304688, 0.067658)
  (0.296875, 0.081190)
  (0.304688, 0.094722)
  (0.320313, 0.094722)
  (0.328125, 0.081190)
  (0.320313, 0.067658)
  (0.328125, 0.054127)
  (0.343750, 0.054127)
  (0.351563, 0.040595)
  (0.343750, 0.027063)
  (0.328125, 0.027063)
  (0.320313, 0.013532)
  (0.328125, -0.000000)
  (0.343750, -0.000000)
  (0.351563, 0.013532)
  (0.367188, 0.013532)
  (0.375000, -0.000000)
  (0.390625, -0.000000)
  (0.398438, 0.013532)
  (0.390625, 0.027063)
  (0.398438, 0.040595)
  (0.414063, 0.040595)
  (0.421875, 0.027063)
  (0.414063, 0.013532)
  (0.421875, -0.000000)
  (0.437500, -0.000000)
  (0.445313, 0.013532)
  (0.460938, 0.013532)
  (0.468750, -0.000000)
  (0.484375, -0.000000)
  (0.492188, 0.013532)
  (0.484375, 0.027063)
  (0.468750, 0.027063)
  (0.460938, 0.040595)
  (0.468750, 0.054127)
  (0.460938, 0.067658)
  (0.445313, 0.067658)
  (0.437500, 0.054127)
  (0.421875, 0.054127)
  (0.414063, 0.067658)
  (0.421875, 0.081190)
  (0.437500, 0.081190)
  (0.445313, 0.094722)
  (0.437500, 0.108253)
  (0.421875, 0.108253)
  (0.414063, 0.121785)
  (0.421875, 0.135316)
  (0.414063, 0.148848)
  (0.398438, 0.148848)
  (0.390625, 0.135316)
  (0.398438, 0.121785)
  (0.390625, 0.108253)
  (0.375000, 0.108253)
  (0.367188, 0.121785)
  (0.351563, 0.121785)
  (0.343750, 0.108253)
  (0.328125, 0.108253)
  (0.320313, 0.121785)
  (0.328125, 0.135316)
  (0.343750, 0.135316)
  (0.351563, 0.148848)
  (0.343750, 0.162380)
  (0.351563, 0.175911)
  (0.367188, 0.175911)
  (0.375000, 0.162380)
  (0.390625, 0.162380)
  (0.398438, 0.175911)
  (0.390625, 0.189443)
  (0.375000, 0.189443)
  (0.367188, 0.202975)
  (0.375000, 0.216506)
  (0.367188, 0.230038)
  (0.351563, 0.230038)
  (0.343750, 0.216506)
  (0.328125, 0.216506)
  (0.320313, 0.230038)
  (0.328125, 0.243570)
  (0.343750, 0.243570)
  (0.351563, 0.257101)
  (0.343750, 0.270633)
  (0.328125, 0.270633)
  (0.320313, 0.284165)
  (0.328125, 0.297696)
  (0.320313, 0.311228)
  (0.304688, 0.311228)
  (0.296875, 0.297696)
  (0.304688, 0.284165)
  (0.296875, 0.270633)
  (0.281250, 0.270633)
  (0.273438, 0.257101)
  (0.281250, 0.243570)
  (0.296875, 0.243570)
  (0.304688, 0.230038)
  (0.296875, 0.216506)
  (0.281250, 0.216506)
  (0.273438, 0.230038)
  (0.257813, 0.230038)
  (0.250000, 0.216506)
  (0.234375, 0.216506)
  (0.226563, 0.230038)
  (0.234375, 0.243570)
  (0.226563, 0.257101)
  (0.210938, 0.257101)
  (0.203125, 0.243570)
  (0.210938, 0.230038)
  (0.203125, 0.216506)
  (0.187500, 0.216506)
  (0.179688, 0.230038)
  (0.164063, 0.230038)
  (0.156250, 0.216506)
  (0.140625, 0.216506)
  (0.132813, 0.230038)
  (0.140625, 0.243569)
  (0.156250, 0.243569)
  (0.164063, 0.257101)
  (0.156250, 0.270633)
  (0.164063, 0.284164)
  (0.179688, 0.284164)
  (0.187500, 0.270633)
  (0.203125, 0.270633)
  (0.210938, 0.284164)
  (0.203125, 0.297696)
  (0.187500, 0.297696)
  (0.179688, 0.311228)
  (0.187500, 0.324759)
  (0.203125, 0.324759)
  (0.210938, 0.338291)
  (0.203125, 0.351823)
  (0.210938, 0.365354)
  (0.226563, 0.365354)
  (0.234375, 0.351823)
  (0.226563, 0.338291)
  (0.234375, 0.324759)
  (0.250000, 0.324759)
  (0.257813, 0.338291)
  (0.273438, 0.338291)
  (0.281250, 0.324759)
  (0.296875, 0.324759)
  (0.304688, 0.338291)
  (0.296875, 0.351823)
  (0.281250, 0.351823)
  (0.273438, 0.365354)
  (0.281250, 0.378886)
  (0.273438, 0.392418)
  (0.257813, 0.392418)
  (0.250000, 0.378886)
  (0.234375, 0.378886)
  (0.226563, 0.392418)
  (0.234375, 0.405949)
  (0.250000, 0.405949)
  (0.257813, 0.419481)
  (0.250000, 0.433013)
  (0.257813, 0.446544)
  (0.273438, 0.446544)
  (0.281250, 0.433013)
  (0.296875, 0.433013)
  (0.304688, 0.446544)
  (0.296875, 0.460076)
  (0.281250, 0.460076)
  (0.273438, 0.473608)
  (0.281250, 0.487139)
  (0.296875, 0.487139)
  (0.304688, 0.500671)
  (0.296875, 0.514202)
  (0.304688, 0.527734)
  (0.320313, 0.527734)
  (0.328125, 0.514202)
  (0.320313, 0.500671)
  (0.328125, 0.487139)
  (0.343750, 0.487139)
  (0.351563, 0.473608)
  (0.343750, 0.460076)
  (0.328125, 0.460076)
  (0.320313, 0.446544)
  (0.328125, 0.433013)
  (0.343750, 0.433013)
  (0.351563, 0.446544)
  (0.367188, 0.446544)
  (0.375000, 0.433013)
  (0.390625, 0.433013)
  (0.398438, 0.446544)
  (0.390625, 0.460076)
  (0.398438, 0.473608)
  (0.414063, 0.473608)
  (0.421875, 0.460076)
  (0.414063, 0.446544)
  (0.421875, 0.433013)
  (0.437500, 0.433013)
  (0.445313, 0.446544)
  (0.460938, 0.446544)
  (0.468750, 0.433013)
  (0.484375, 0.433013)
  (0.492188, 0.446544)
  (0.484375, 0.460076)
  (0.468750, 0.460076)
  (0.460938, 0.473608)
  (0.468750, 0.487139)
  (0.460938, 0.500671)
  (0.445313, 0.500671)
  (0.437500, 0.487139)
  (0.421875, 0.487139)
  (0.414063, 0.500671)
  (0.421875, 0.514202)
  (0.437500, 0.514202)
  (0.445313, 0.527734)
  (0.437500, 0.541266)
  (0.421875, 0.541266)
  (0.414063, 0.554797)
  (0.421875, 0.568329)
  (0.414063, 0.581861)
  (0.398438, 0.581861)
  (0.390625, 0.568329)
  (0.398438, 0.554797)
  (0.390625, 0.541266)
  (0.375000, 0.541266)
  (0.367188, 0.554797)
  (0.351563, 0.554797)
  (0.343750, 0.541266)
  (0.328125, 0.541266)
  (0.320313, 0.554797)
  (0.328125, 0.568329)
  (0.343750, 0.568329)
  (0.351563, 0.581861)
  (0.343750, 0.595392)
  (0.351563, 0.608924)
  (0.367188, 0.608924)
  (0.375000, 0.595392)
  (0.390625, 0.595392)
  (0.398438, 0.608924)
  (0.390625, 0.622455)
  (0.375000, 0.622455)
  (0.367188, 0.635987)
  (0.375000, 0.649519)
  (0.390625, 0.649519)
  (0.398438, 0.663050)
  (0.390625, 0.676582)
  (0.398438, 0.690114)
  (0.414063, 0.690114)
  (0.421875, 0.676582)
  (0.414063, 0.663050)
  (0.421875, 0.649519)
  (0.437500, 0.649519)
  (0.445313, 0.663050)
  (0.460938, 0.663050)
  (0.468750, 0.649519)
  (0.484375, 0.649519)
  (0.492188, 0.663050)
  (0.484375, 0.676582)
  (0.468750, 0.676582)
  (0.460938, 0.690114)
  (0.468750, 0.703645)
  (0.460938, 0.717177)
  (0.445313, 0.717177)
  (0.437500, 0.703645)
  (0.421875, 0.703645)
  (0.414063, 0.717177)
  (0.421875, 0.730709)
  (0.437500, 0.730708)
  (0.445313, 0.744240)
  (0.437500, 0.757772)
  (0.445313, 0.771303)
  (0.460938, 0.771303)
  (0.468750, 0.757772)
  (0.484375, 0.757772)
  (0.492188, 0.771303)
  (0.484375, 0.784835)
  (0.468750, 0.784835)
  (0.460938, 0.798367)
  (0.468750, 0.811898)
  (0.484375, 0.811898)
  (0.492188, 0.825430)
  (0.484375, 0.838962)
  (0.492188, 0.852493)
  (0.507813, 0.852493)
  (0.515625, 0.838961)
  (0.507813, 0.825430)
  (0.515625, 0.811898)
  (0.531250, 0.811898)
  (0.539063, 0.798367)
  (0.531250, 0.784835)
  (0.515625, 0.784835)
  (0.507813, 0.771303)
  (0.515625, 0.757772)
  (0.531250, 0.757772)
  (0.539063, 0.771303)
  (0.554688, 0.771303)
  (0.562500, 0.757772)
  (0.554688, 0.744240)
  (0.562500, 0.730708)
  (0.578125, 0.730708)
  (0.585938, 0.717177)
  (0.578125, 0.703645)
  (0.562500, 0.703645)
  (0.554688, 0.717177)
  (0.539063, 0.717177)
  (0.531250, 0.703645)
  (0.539063, 0.690114)
  (0.531250, 0.676582)
  (0.515625, 0.676582)
  (0.507813, 0.663050)
  (0.515625, 0.649519)
  (0.531250, 0.649519)
  (0.539063, 0.663050)
  (0.554688, 0.663050)
  (0.562500, 0.649519)
  (0.578125, 0.649519)
  (0.585938, 0.663050)
  (0.578125, 0.676582)
  (0.585938, 0.690114)
  (0.601563, 0.690114)
  (0.609375, 0.676582)
  (0.601563, 0.663050)
  (0.609375, 0.649519)
  (0.625000, 0.649519)
  (0.632813, 0.635987)
  (0.625000, 0.622455)
  (0.609375, 0.622455)
  (0.601563, 0.608924)
  (0.609375, 0.595392)
  (0.625000, 0.595392)
  (0.632813, 0.608924)
  (0.648438, 0.608924)
  (0.656250, 0.595392)
  (0.648438, 0.581861)
  (0.656250, 0.568329)
  (0.671875, 0.568329)
  (0.679688, 0.554797)
  (0.671875, 0.541266)
  (0.656250, 0.541266)
  (0.648438, 0.554797)
  (0.632813, 0.554797)
  (0.625000, 0.541266)
  (0.609375, 0.541266)
  (0.601563, 0.554797)
  (0.609375, 0.568329)
  (0.601563, 0.581861)
  (0.585938, 0.581861)
  (0.578125, 0.568329)
  (0.585938, 0.554797)
  (0.578125, 0.541266)
  (0.562500, 0.541266)
  (0.554688, 0.527734)
  (0.562500, 0.514202)
  (0.578125, 0.514202)
  (0.585938, 0.500671)
  (0.578125, 0.487139)
  (0.562500, 0.487139)
  (0.554688, 0.500671)
  (0.539063, 0.500671)
  (0.531250, 0.487139)
  (0.539063, 0.473608)
  (0.531250, 0.460076)
  (0.515625, 0.460076)
  (0.507813, 0.446544)
  (0.515625, 0.433013)
  (0.531250, 0.433013)
  (0.539063, 0.446544)
  (0.554688, 0.446544)
  (0.562500, 0.433013)
  (0.578125, 0.433013)
  (0.585938, 0.446544)
  (0.578125, 0.460076)
  (0.585938, 0.473608)
  (0.601563, 0.473608)
  (0.609375, 0.460076)
  (0.601563, 0.446544)
  (0.609375, 0.433013)
  (0.625000, 0.433013)
  (0.632813, 0.446544)
  (0.648438, 0.446544)
  (0.656250, 0.433013)
  (0.671875, 0.433013)
  (0.679688, 0.446544)
  (0.671875, 0.460076)
  (0.656250, 0.460076)
  (0.648438, 0.473608)
  (0.656250, 0.487139)
  (0.671875, 0.487139)
  (0.679688, 0.500671)
  (0.671875, 0.514202)
  (0.679688, 0.527734)
  (0.695313, 0.527734)
  (0.703125, 0.514202)
  (0.695313, 0.500671)
  (0.703125, 0.487139)
  (0.718750, 0.487139)
  (0.726563, 0.473608)
  (0.718750, 0.460076)
  (0.703125, 0.460076)
  (0.695313, 0.446544)
  (0.703125, 0.433013)
  (0.718750, 0.433013)
  (0.726563, 0.446544)
  (0.742188, 0.446544)
  (0.750000, 0.433013)
  (0.742188, 0.419481)
  (0.750000, 0.405949)
  (0.765625, 0.405949)
  (0.773438, 0.392418)
  (0.765625, 0.378886)
  (0.750000, 0.378886)
  (0.742188, 0.392418)
  (0.726563, 0.392418)
  (0.718750, 0.378886)
  (0.726563, 0.365354)
  (0.718750, 0.351823)
  (0.703125, 0.351823)
  (0.695313, 0.338291)
  (0.703125, 0.324759)
  (0.718750, 0.324759)
  (0.726563, 0.338291)
  (0.742188, 0.338291)
  (0.750000, 0.324759)
  (0.765625, 0.324759)
  (0.773438, 0.338291)
  (0.765625, 0.351823)
  (0.773438, 0.365354)
  (0.789063, 0.365354)
  (0.796875, 0.351823)
  (0.789063, 0.338291)
  (0.796875, 0.324759)
  (0.812500, 0.324759)
  (0.820313, 0.311228)
  (0.812500, 0.297696)
  (0.796875, 0.297696)
  (0.789063, 0.284164)
  (0.796875, 0.270633)
  (0.812500, 0.270633)
  (0.820313, 0.284164)
  (0.835938, 0.284164)
  (0.843750, 0.270633)
  (0.835938, 0.257101)
  (0.843750, 0.243569)
  (0.859375, 0.243569)
  (0.867188, 0.230038)
  (0.859375, 0.216506)
  (0.843750, 0.216506)
  (0.835938, 0.230038)
  (0.820313, 0.230038)
  (0.812500, 0.216506)
  (0.796875, 0.216506)
  (0.789063, 0.230038)
  (0.796875, 0.243569)
  (0.789063, 0.257101)
  (0.773438, 0.257101)
  (0.765625, 0.243569)
  (0.773438, 0.230038)
  (0.765625, 0.216506)
  (0.750000, 0.216506)
  (0.742188, 0.230038)
  (0.726563, 0.230038)
  (0.718750, 0.216506)
  (0.703125, 0.216506)
  (0.695313, 0.230038)
  (0.703125, 0.243570)
  (0.718750, 0.243570)
  (0.726563, 0.257101)
  (0.718750, 0.270633)
  (0.703125, 0.270633)
  (0.695313, 0.284164)
  (0.703125, 0.297696)
  (0.695313, 0.311228)
  (0.679688, 0.311228)
  (0.671875, 0.297696)
  (0.679688, 0.284164)
  (0.671875, 0.270633)
  (0.656250, 0.270633)
  (0.648438, 0.257101)
  (0.656250, 0.243570)
  (0.671875, 0.243570)
  (0.679688, 0.230038)
  (0.671875, 0.216506)
  (0.656250, 0.216506)
  (0.648438, 0.230038)
  (0.632813, 0.230038)
  (0.625000, 0.216506)
  (0.632813, 0.202975)
  (0.625000, 0.189443)
  (0.609375, 0.189443)
  (0.601563, 0.175911)
  (0.609375, 0.162380)
  (0.625000, 0.162380)
  (0.632813, 0.175911)
  (0.648438, 0.175911)
  (0.656250, 0.162380)
  (0.648438, 0.148848)
  (0.656250, 0.135316)
  (0.671875, 0.135316)
  (0.679688, 0.121785)
  (0.671875, 0.108253)
  (0.656250, 0.108253)
  (0.648438, 0.121785)
  (0.632813, 0.121785)
  (0.625000, 0.108253)
  (0.609375, 0.108253)
  (0.601563, 0.121785)
  (0.609375, 0.135316)
  (0.601563, 0.148848)
  (0.585938, 0.148848)
  (0.578125, 0.135316)
  (0.585938, 0.121785)
  (0.578125, 0.108253)
  (0.562500, 0.108253)
  (0.554688, 0.094722)
  (0.562500, 0.081190)
  (0.578125, 0.081190)
  (0.585938, 0.067658)
  (0.578125, 0.054127)
  (0.562500, 0.054127)
  (0.554688, 0.067658)
  (0.539063, 0.067658)
  (0.531250, 0.054127)
  (0.539063, 0.040595)
  (0.531250, 0.027063)
  (0.515625, 0.027063)
  (0.507813, 0.013532)
  (0.515625, -0.000000)
  (0.531250, -0.000000)
  (0.539063, 0.013532)
  (0.554688, 0.013532)
  (0.562500, -0.000000)
  (0.578125, -0.000000)
  (0.585938, 0.013532)
  (0.578125, 0.027063)
  (0.585938, 0.040595)
  (0.601563, 0.040595)
  (0.609375, 0.027063)
  (0.601563, 0.013532)
  (0.609375, -0.000000)
  (0.625000, -0.000000)
  (0.632813, 0.013532)
  (0.648438, 0.013532)
  (0.656250, -0.000000)
  (0.671875, -0.000000)
  (0.679688, 0.013532)
  (0.671875, 0.027063)
  (0.656250, 0.027063)
  (0.648438, 0.040595)
  (0.656250, 0.054127)
  (0.671875, 0.054127)
  (0.679688, 0.067658)
  (0.671875, 0.081190)
  (0.679688, 0.094722)
  (0.695313, 0.094722)
  (0.703125, 0.081190)
  (0.695313, 0.067658)
  (0.703125, 0.054127)
  (0.718750, 0.054127)
  (0.726563, 0.040595)
  (0.718750, 0.027063)
  (0.703125, 0.027063)
  (0.695313, 0.013532)
  (0.703125, -0.000000)
  (0.718750, -0.000000)
  (0.726563, 0.013532)
  (0.742188, 0.013532)
  (0.750000, -0.000000)
  (0.765625, -0.000000)
  (0.773438, 0.013532)
  (0.765625, 0.027063)
  (0.773438, 0.040595)
  (0.789063, 0.040595)
  (0.796875, 0.027063)
  (0.789063, 0.013532)
  (0.796875, -0.000000)
  (0.812500, -0.000000)
  (0.820313, 0.013532)
  (0.835938, 0.013532)
  (0.843750, -0.000000)
  (0.859375, -0.000000)
  (0.867188, 0.013532)
  (0.859375, 0.027063)
  (0.843750, 0.027063)
  (0.835938, 0.040595)
  (0.843750, 0.054127)
  (0.835938, 0.067658)
  (0.820313, 0.067658)
  (0.812500, 0.054127)
  (0.796875, 0.054127)
  (0.789063, 0.067658)
  (0.796875, 0.081190)
  (0.812500, 0.081190)
  (0.820313, 0.094722)
  (0.812500, 0.108253)
  (0.820313, 0.121785)
  (0.835938, 0.121785)
  (0.843750, 0.108253)
  (0.859375, 0.108253)
  (0.867188, 0.121785)
  (0.859375, 0.135316)
  (0.843750, 0.135316)
  (0.835938, 0.148848)
  (0.843750, 0.162380)
  (0.859375, 0.162380)
  (0.867188, 0.175911)
  (0.859375, 0.189443)
  (0.867188, 0.202975)
  (0.882813, 0.202975)
  (0.890625, 0.189443)
  (0.882813, 0.175911)
  (0.890625, 0.162380)
  (0.906250, 0.162380)
  (0.914063, 0.148848)
  (0.906250, 0.135316)
  (0.890625, 0.135316)
  (0.882813, 0.121785)
  (0.890625, 0.108253)
  (0.906250, 0.108253)
  (0.914063, 0.121785)
  (0.929688, 0.121785)
  (0.937500, 0.108253)
  (0.929688, 0.094722)
  (0.937500, 0.081190)
  (0.953125, 0.081190)
  (0.960938, 0.067658)
  (0.953125, 0.054127)
  (0.937500, 0.054127)
  (0.929688, 0.067658)
  (0.914063, 0.067658)
  (0.906250, 0.054127)
  (0.914063, 0.040595)
  (0.906250, 0.027063)
  (0.890625, 0.027063)
  (0.882813, 0.013532)
  (0.890625, -0.000000)
  (0.906250, -0.000000)
  (0.914063, 0.013532)
  (0.929688, 0.013532)
  (0.937500, -0.000000)
  (0.953125, -0.000000)
  (0.960938, 0.013532)
  (0.953125, 0.027063)
  (0.960938, 0.040595)
  (0.976563, 0.040595)
  (0.984375, 0.027063)
  (0.976563, 0.013532)
  (0.984375, -0.000000)
  (1.000000, -0.000000)
}
\end{pspicture}
%
\hskip0.1in\dots
%

\end{zztask}

%%%%%%%%%%%%%%%%%%%%%%%%%%%%%%%%%%%%%%%%%%%%%%%%%%%%%%%%%%%%%%%%%%%%%%%%%%%%%%

\begin{zztask}[Minkowski Sausage]
В рамках общего условия задачи построить колбасу Минковского.
База для построения кривой (нулевое приближение) представляет собой отрезок.
Переход от приближения $n$ к приближению $(n+1)$ осуществляется заменой каждого 
отрезка фигуры на восемь (см. рис. при $n=1$). 
Все отрезки имеют одинаковую длину, угол между отрезками составляет $90^\circ$.
\par\endinput
\begin{pspicture}(-0.5,-0.5)(1.5,1.5)
\psgrid
\rput(0.5,-0.25){$n=0$}
\rput(0,0.5){
\psline[linewidth=0.06cm, arrows=o-o]
  (0, 0)
  (1.000000, 0.000000)
}
\end{pspicture}
%
\hskip0.1in
%
\begin{pspicture}(-0.5,-0.5)(1.5,1.5)
\psgrid
\rput(0.5,-0.25){$n=1$}
\rput(0,0.5){
\psline[linewidth=0.06cm, arrows=o-o, showpoints=true]
  (0, 0)
  (0.250000, 0.000000)
  (0.250000, -0.250000)
  (0.500000, -0.250000)
  (0.500000, -0.000000)
  (0.500000, 0.250000)
  (0.750000, 0.250000)
  (0.750000, 0.000000)
  (1.000000, 0.000000)
}
\end{pspicture}
%
\hskip0.1in
%
\begin{pspicture}(-0.5,-0.5)(1.5,1.5)
\psgrid
\rput(0.5,-0.25){$n=2$}
\rput(0,0.5){
\psline
  (0, 0)
  (0.062500, 0.000000)
  (0.062500, -0.062500)
  (0.125000, -0.062500)
  (0.125000, -0.000000)
  (0.125000, 0.062500)
  (0.187500, 0.062500)
  (0.187500, 0.000000)
  (0.250000, 0.000000)
  (0.250000, -0.062500)
  (0.187500, -0.062500)
  (0.187500, -0.125000)
  (0.250000, -0.125000)
  (0.312500, -0.125000)
  (0.312500, -0.187500)
  (0.250000, -0.187500)
  (0.250000, -0.250000)
  (0.312500, -0.250000)
  (0.312500, -0.312500)
  (0.375000, -0.312500)
  (0.375000, -0.250000)
  (0.375000, -0.187500)
  (0.437500, -0.187500)
  (0.437500, -0.250000)
  (0.500000, -0.250000)
  (0.500000, -0.187500)
  (0.562500, -0.187500)
  (0.562500, -0.125000)
  (0.500000, -0.125000)
  (0.437500, -0.125000)
  (0.437500, -0.062500)
  (0.500000, -0.062500)
  (0.500000, 0.000000)
  (0.500000, 0.062500)
  (0.562500, 0.062500)
  (0.562500, 0.125000)
  (0.500000, 0.125000)
  (0.437500, 0.125000)
  (0.437500, 0.187500)
  (0.500000, 0.187500)
  (0.500000, 0.250000)
  (0.562500, 0.250000)
  (0.562500, 0.187500)
  (0.625000, 0.187500)
  (0.625000, 0.250000)
  (0.625000, 0.312500)
  (0.687500, 0.312500)
  (0.687500, 0.250000)
  (0.750000, 0.250000)
  (0.750000, 0.187500)
  (0.687500, 0.187500)
  (0.687500, 0.125000)
  (0.750000, 0.125000)
  (0.812500, 0.125000)
  (0.812500, 0.062500)
  (0.750000, 0.062500)
  (0.750000, 0.000000)
  (0.812500, 0.000000)
  (0.812500, -0.062500)
  (0.875000, -0.062500)
  (0.875000, 0.000000)
  (0.875000, 0.062500)
  (0.937500, 0.062500)
  (0.937500, 0.000000)
  (1.000000, 0.000000)
}
\end{pspicture}
%
\hskip0.1in
%
\begin{pspicture}(-0.5,-0.5)(1.5,1.5)
\psgrid
\rput(0.5,-0.25){$n=3$}
\rput(0,0.5){
\psline
  (0, 0)
  (0.015625, 0.000000)
  (0.015625, -0.015625)
  (0.031250, -0.015625)
  (0.031250, -0.000000)
  (0.031250, 0.015625)
  (0.046875, 0.015625)
  (0.046875, 0.000000)
  (0.062500, 0.000000)
  (0.062500, -0.015625)
  (0.046875, -0.015625)
  (0.046875, -0.031250)
  (0.062500, -0.031250)
  (0.078125, -0.031250)
  (0.078125, -0.046875)
  (0.062500, -0.046875)
  (0.062500, -0.062500)
  (0.078125, -0.062500)
  (0.078125, -0.078125)
  (0.093750, -0.078125)
  (0.093750, -0.062500)
  (0.093750, -0.046875)
  (0.109375, -0.046875)
  (0.109375, -0.062500)
  (0.125000, -0.062500)
  (0.125000, -0.046875)
  (0.140625, -0.046875)
  (0.140625, -0.031250)
  (0.125000, -0.031250)
  (0.109375, -0.031250)
  (0.109375, -0.015625)
  (0.125000, -0.015625)
  (0.125000, 0.000000)
  (0.125000, 0.015625)
  (0.140625, 0.015625)
  (0.140625, 0.031250)
  (0.125000, 0.031250)
  (0.109375, 0.031250)
  (0.109375, 0.046875)
  (0.125000, 0.046875)
  (0.125000, 0.062500)
  (0.140625, 0.062500)
  (0.140625, 0.046875)
  (0.156250, 0.046875)
  (0.156250, 0.062500)
  (0.156250, 0.078125)
  (0.171875, 0.078125)
  (0.171875, 0.062500)
  (0.187500, 0.062500)
  (0.187500, 0.046875)
  (0.171875, 0.046875)
  (0.171875, 0.031250)
  (0.187500, 0.031250)
  (0.203125, 0.031250)
  (0.203125, 0.015625)
  (0.187500, 0.015625)
  (0.187500, 0.000000)
  (0.203125, 0.000000)
  (0.203125, -0.015625)
  (0.218750, -0.015625)
  (0.218750, 0.000000)
  (0.218750, 0.015625)
  (0.234375, 0.015625)
  (0.234375, 0.000000)
  (0.250000, 0.000000)
  (0.250000, -0.015625)
  (0.234375, -0.015625)
  (0.234375, -0.031250)
  (0.250000, -0.031250)
  (0.265625, -0.031250)
  (0.265625, -0.046875)
  (0.250000, -0.046875)
  (0.250000, -0.062500)
  (0.234375, -0.062500)
  (0.234375, -0.046875)
  (0.218750, -0.046875)
  (0.218750, -0.062500)
  (0.218750, -0.078125)
  (0.203125, -0.078125)
  (0.203125, -0.062500)
  (0.187500, -0.062500)
  (0.187500, -0.078125)
  (0.171875, -0.078125)
  (0.171875, -0.093750)
  (0.187500, -0.093750)
  (0.203125, -0.093750)
  (0.203125, -0.109375)
  (0.187500, -0.109375)
  (0.187500, -0.125000)
  (0.203125, -0.125000)
  (0.203125, -0.140625)
  (0.218750, -0.140625)
  (0.218750, -0.125000)
  (0.218750, -0.109375)
  (0.234375, -0.109375)
  (0.234375, -0.125000)
  (0.250000, -0.125000)
  (0.265625, -0.125000)
  (0.265625, -0.140625)
  (0.281250, -0.140625)
  (0.281250, -0.125000)
  (0.281250, -0.109375)
  (0.296875, -0.109375)
  (0.296875, -0.125000)
  (0.312500, -0.125000)
  (0.312500, -0.140625)
  (0.296875, -0.140625)
  (0.296875, -0.156250)
  (0.312500, -0.156250)
  (0.328125, -0.156250)
  (0.328125, -0.171875)
  (0.312500, -0.171875)
  (0.312500, -0.187500)
  (0.296875, -0.187500)
  (0.296875, -0.171875)
  (0.281250, -0.171875)
  (0.281250, -0.187500)
  (0.281250, -0.203125)
  (0.265625, -0.203125)
  (0.265625, -0.187500)
  (0.250000, -0.187500)
  (0.250000, -0.203125)
  (0.234375, -0.203125)
  (0.234375, -0.218750)
  (0.250000, -0.218750)
  (0.265625, -0.218750)
  (0.265625, -0.234375)
  (0.250000, -0.234375)
  (0.250000, -0.250000)
  (0.265625, -0.250000)
  (0.265625, -0.265625)
  (0.281250, -0.265625)
  (0.281250, -0.250000)
  (0.281250, -0.234375)
  (0.296875, -0.234375)
  (0.296875, -0.250000)
  (0.312500, -0.250000)
  (0.312500, -0.265625)
  (0.296875, -0.265625)
  (0.296875, -0.281250)
  (0.312500, -0.281250)
  (0.328125, -0.281250)
  (0.328125, -0.296875)
  (0.312500, -0.296875)
  (0.312500, -0.312500)
  (0.328125, -0.312500)
  (0.328125, -0.328125)
  (0.343750, -0.328125)
  (0.343750, -0.312500)
  (0.343750, -0.296875)
  (0.359375, -0.296875)
  (0.359375, -0.312500)
  (0.375000, -0.312500)
  (0.375000, -0.296875)
  (0.390625, -0.296875)
  (0.390625, -0.281250)
  (0.375000, -0.281250)
  (0.359375, -0.281250)
  (0.359375, -0.265625)
  (0.375000, -0.265625)
  (0.375000, -0.250000)
  (0.375000, -0.234375)
  (0.390625, -0.234375)
  (0.390625, -0.218750)
  (0.375000, -0.218750)
  (0.359375, -0.218750)
  (0.359375, -0.203125)
  (0.375000, -0.203125)
  (0.375000, -0.187500)
  (0.390625, -0.187500)
  (0.390625, -0.203125)
  (0.406250, -0.203125)
  (0.406250, -0.187500)
  (0.406250, -0.171875)
  (0.421875, -0.171875)
  (0.421875, -0.187500)
  (0.437500, -0.187500)
  (0.437500, -0.203125)
  (0.421875, -0.203125)
  (0.421875, -0.218750)
  (0.437500, -0.218750)
  (0.453125, -0.218750)
  (0.453125, -0.234375)
  (0.437500, -0.234375)
  (0.437500, -0.250000)
  (0.453125, -0.250000)
  (0.453125, -0.265625)
  (0.468750, -0.265625)
  (0.468750, -0.250000)
  (0.468750, -0.234375)
  (0.484375, -0.234375)
  (0.484375, -0.250000)
  (0.500000, -0.250000)
  (0.500000, -0.234375)
  (0.515625, -0.234375)
  (0.515625, -0.218750)
  (0.500000, -0.218750)
  (0.484375, -0.218750)
  (0.484375, -0.203125)
  (0.500000, -0.203125)
  (0.500000, -0.187500)
  (0.515625, -0.187500)
  (0.515625, -0.203125)
  (0.531250, -0.203125)
  (0.531250, -0.187500)
  (0.531250, -0.171875)
  (0.546875, -0.171875)
  (0.546875, -0.187500)
  (0.562500, -0.187500)
  (0.562500, -0.171875)
  (0.578125, -0.171875)
  (0.578125, -0.156250)
  (0.562500, -0.156250)
  (0.546875, -0.156250)
  (0.546875, -0.140625)
  (0.562500, -0.140625)
  (0.562500, -0.125000)
  (0.546875, -0.125000)
  (0.546875, -0.109375)
  (0.531250, -0.109375)
  (0.531250, -0.125000)
  (0.531250, -0.140625)
  (0.515625, -0.140625)
  (0.515625, -0.125000)
  (0.500000, -0.125000)
  (0.484375, -0.125000)
  (0.484375, -0.109375)
  (0.468750, -0.109375)
  (0.468750, -0.125000)
  (0.468750, -0.140625)
  (0.453125, -0.140625)
  (0.453125, -0.125000)
  (0.437500, -0.125000)
  (0.437500, -0.109375)
  (0.453125, -0.109375)
  (0.453125, -0.093750)
  (0.437500, -0.093750)
  (0.421875, -0.093750)
  (0.421875, -0.078125)
  (0.437500, -0.078125)
  (0.437500, -0.062500)
  (0.453125, -0.062500)
  (0.453125, -0.078125)
  (0.468750, -0.078125)
  (0.468750, -0.062500)
  (0.468750, -0.046875)
  (0.484375, -0.046875)
  (0.484375, -0.062500)
  (0.500000, -0.062500)
  (0.500000, -0.046875)
  (0.515625, -0.046875)
  (0.515625, -0.031250)
  (0.500000, -0.031250)
  (0.484375, -0.031250)
  (0.484375, -0.015625)
  (0.500000, -0.015625)
  (0.500000, -0.000000)
  (0.500000, 0.015625)
  (0.515625, 0.015625)
  (0.515625, 0.031250)
  (0.500000, 0.031250)
  (0.484375, 0.031250)
  (0.484375, 0.046875)
  (0.500000, 0.046875)
  (0.500000, 0.062500)
  (0.515625, 0.062500)
  (0.515625, 0.046875)
  (0.531250, 0.046875)
  (0.531250, 0.062500)
  (0.531250, 0.078125)
  (0.546875, 0.078125)
  (0.546875, 0.062500)
  (0.562500, 0.062500)
  (0.562500, 0.078125)
  (0.578125, 0.078125)
  (0.578125, 0.093750)
  (0.562500, 0.093750)
  (0.546875, 0.093750)
  (0.546875, 0.109375)
  (0.562500, 0.109375)
  (0.562500, 0.125000)
  (0.546875, 0.125000)
  (0.546875, 0.140625)
  (0.531250, 0.140625)
  (0.531250, 0.125000)
  (0.531250, 0.109375)
  (0.515625, 0.109375)
  (0.515625, 0.125000)
  (0.500000, 0.125000)
  (0.484375, 0.125000)
  (0.484375, 0.140625)
  (0.468750, 0.140625)
  (0.468750, 0.125000)
  (0.468750, 0.109375)
  (0.453125, 0.109375)
  (0.453125, 0.125000)
  (0.437500, 0.125000)
  (0.437500, 0.140625)
  (0.453125, 0.140625)
  (0.453125, 0.156250)
  (0.437500, 0.156250)
  (0.421875, 0.156250)
  (0.421875, 0.171875)
  (0.437500, 0.171875)
  (0.437500, 0.187500)
  (0.453125, 0.187500)
  (0.453125, 0.171875)
  (0.468750, 0.171875)
  (0.468750, 0.187500)
  (0.468750, 0.203125)
  (0.484375, 0.203125)
  (0.484375, 0.187500)
  (0.500000, 0.187500)
  (0.500000, 0.203125)
  (0.515625, 0.203125)
  (0.515625, 0.218750)
  (0.500000, 0.218750)
  (0.484375, 0.218750)
  (0.484375, 0.234375)
  (0.500000, 0.234375)
  (0.500000, 0.250000)
  (0.515625, 0.250000)
  (0.515625, 0.234375)
  (0.531250, 0.234375)
  (0.531250, 0.250000)
  (0.531250, 0.265625)
  (0.546875, 0.265625)
  (0.546875, 0.250000)
  (0.562500, 0.250000)
  (0.562500, 0.234375)
  (0.546875, 0.234375)
  (0.546875, 0.218750)
  (0.562500, 0.218750)
  (0.578125, 0.218750)
  (0.578125, 0.203125)
  (0.562500, 0.203125)
  (0.562500, 0.187500)
  (0.578125, 0.187500)
  (0.578125, 0.171875)
  (0.593750, 0.171875)
  (0.593750, 0.187500)
  (0.593750, 0.203125)
  (0.609375, 0.203125)
  (0.609375, 0.187500)
  (0.625000, 0.187500)
  (0.625000, 0.203125)
  (0.640625, 0.203125)
  (0.640625, 0.218750)
  (0.625000, 0.218750)
  (0.609375, 0.218750)
  (0.609375, 0.234375)
  (0.625000, 0.234375)
  (0.625000, 0.250000)
  (0.625000, 0.265625)
  (0.640625, 0.265625)
  (0.640625, 0.281250)
  (0.625000, 0.281250)
  (0.609375, 0.281250)
  (0.609375, 0.296875)
  (0.625000, 0.296875)
  (0.625000, 0.312500)
  (0.640625, 0.312500)
  (0.640625, 0.296875)
  (0.656250, 0.296875)
  (0.656250, 0.312500)
  (0.656250, 0.328125)
  (0.671875, 0.328125)
  (0.671875, 0.312500)
  (0.687500, 0.312500)
  (0.687500, 0.296875)
  (0.671875, 0.296875)
  (0.671875, 0.281250)
  (0.687500, 0.281250)
  (0.703125, 0.281250)
  (0.703125, 0.265625)
  (0.687500, 0.265625)
  (0.687500, 0.250000)
  (0.703125, 0.250000)
  (0.703125, 0.234375)
  (0.718750, 0.234375)
  (0.718750, 0.250000)
  (0.718750, 0.265625)
  (0.734375, 0.265625)
  (0.734375, 0.250000)
  (0.750000, 0.250000)
  (0.750000, 0.234375)
  (0.734375, 0.234375)
  (0.734375, 0.218750)
  (0.750000, 0.218750)
  (0.765625, 0.218750)
  (0.765625, 0.203125)
  (0.750000, 0.203125)
  (0.750000, 0.187500)
  (0.734375, 0.187500)
  (0.734375, 0.203125)
  (0.718750, 0.203125)
  (0.718750, 0.187500)
  (0.718750, 0.171875)
  (0.703125, 0.171875)
  (0.703125, 0.187500)
  (0.687500, 0.187500)
  (0.687500, 0.171875)
  (0.671875, 0.171875)
  (0.671875, 0.156250)
  (0.687500, 0.156250)
  (0.703125, 0.156250)
  (0.703125, 0.140625)
  (0.687500, 0.140625)
  (0.687500, 0.125000)
  (0.703125, 0.125000)
  (0.703125, 0.109375)
  (0.718750, 0.109375)
  (0.718750, 0.125000)
  (0.718750, 0.140625)
  (0.734375, 0.140625)
  (0.734375, 0.125000)
  (0.750000, 0.125000)
  (0.765625, 0.125000)
  (0.765625, 0.109375)
  (0.781250, 0.109375)
  (0.781250, 0.125000)
  (0.781250, 0.140625)
  (0.796875, 0.140625)
  (0.796875, 0.125000)
  (0.812500, 0.125000)
  (0.812500, 0.109375)
  (0.796875, 0.109375)
  (0.796875, 0.093750)
  (0.812500, 0.093750)
  (0.828125, 0.093750)
  (0.828125, 0.078125)
  (0.812500, 0.078125)
  (0.812500, 0.062500)
  (0.796875, 0.062500)
  (0.796875, 0.078125)
  (0.781250, 0.078125)
  (0.781250, 0.062500)
  (0.781250, 0.046875)
  (0.765625, 0.046875)
  (0.765625, 0.062500)
  (0.750000, 0.062500)
  (0.750000, 0.046875)
  (0.734375, 0.046875)
  (0.734375, 0.031250)
  (0.750000, 0.031250)
  (0.765625, 0.031250)
  (0.765625, 0.015625)
  (0.750000, 0.015625)
  (0.750000, -0.000000)
  (0.765625, -0.000000)
  (0.765625, -0.015625)
  (0.781250, -0.015625)
  (0.781250, -0.000000)
  (0.781250, 0.015625)
  (0.796875, 0.015625)
  (0.796875, -0.000000)
  (0.812500, -0.000000)
  (0.812500, -0.015625)
  (0.796875, -0.015625)
  (0.796875, -0.031250)
  (0.812500, -0.031250)
  (0.828125, -0.031250)
  (0.828125, -0.046875)
  (0.812500, -0.046875)
  (0.812500, -0.062500)
  (0.828125, -0.062500)
  (0.828125, -0.078125)
  (0.843750, -0.078125)
  (0.843750, -0.062500)
  (0.843750, -0.046875)
  (0.859375, -0.046875)
  (0.859375, -0.062500)
  (0.875000, -0.062500)
  (0.875000, -0.046875)
  (0.890625, -0.046875)
  (0.890625, -0.031250)
  (0.875000, -0.031250)
  (0.859375, -0.031250)
  (0.859375, -0.015625)
  (0.875000, -0.015625)
  (0.875000, -0.000000)
  (0.875000, 0.015625)
  (0.890625, 0.015625)
  (0.890625, 0.031250)
  (0.875000, 0.031250)
  (0.859375, 0.031250)
  (0.859375, 0.046875)
  (0.875000, 0.046875)
  (0.875000, 0.062500)
  (0.890625, 0.062500)
  (0.890625, 0.046875)
  (0.906250, 0.046875)
  (0.906250, 0.062500)
  (0.906250, 0.078125)
  (0.921875, 0.078125)
  (0.921875, 0.062500)
  (0.937500, 0.062500)
  (0.937500, 0.046875)
  (0.921875, 0.046875)
  (0.921875, 0.031250)
  (0.937500, 0.031250)
  (0.953125, 0.031250)
  (0.953125, 0.015625)
  (0.937500, 0.015625)
  (0.937500, -0.000000)
  (0.953125, -0.000000)
  (0.953125, -0.015625)
  (0.968750, -0.015625)
  (0.968750, -0.000000)
  (0.968750, 0.015625)
  (0.984375, 0.015625)
  (0.984375, -0.000000)
  (1.000000, -0.000000)
}
\end{pspicture}
%
\hskip0.1in\dots
%

\end{zztask}

%%%%%%%%%%%%%%%%%%%%%%%%%%%%%%%%%%%%%%%%%%%%%%%%%%%%%%%%%%%%%%%%%%%%%%%%%%%%%%

\begin{zztask}[Quadratic Koch Island]
В рамках общего условия задачи построить остров Коха. База для построения кривой 
(нулевое приближение) представляет собой квадрат.
Переход от приближения $n$ к приближению $(n+1)$ осуществляется заменой каждого 
отрезка фигуры на три (см. рис. при $n=1$). 
Все отрезки имеют одинаковую длину, угол между отрезками составляет $90^\circ$.
\par\endinput
\begin{pspicture}(-0.5,-0.5)(1.5,1.5)
\psgrid
\rput(0.0,-0.25){$n=0$}
\rput(0,0){
\psline
  (-0.000000, 1.000000)
  (1.000000, 1.000000)
  (1.000000, 0.000000)
  (-0.000000, 0.000000)
\psline[linewidth=0.06cm, arrows=o-o]
  (0, 0)
  (-0.000000, 1.000000)
}
\end{pspicture}
%
\hskip0.1in
%
\begin{pspicture}(-0.5,-0.5)(1.5,1.5)
\psgrid
\rput(0.0,-0.25){$n=1$}
\rput(0,0){
\psline
  (0.000000, 1.000000)
  (0.400000, 1.200000)
  (0.600000, 0.800000)
  (1.000000, 1.000000)
  (1.200000, 0.600000)
  (0.800000, 0.400000)
  (1.000000, -0.000000)
  (0.600000, -0.200000)
  (0.400000, 0.200000)
  (0.000000, -0.000000)
\psline[linewidth=0.06cm, arrows=o-o, showpoints=true]
  (0, 0)
  (-0.200000, 0.400000)
  (0.200000, 0.600000)
  (0.000000, 1.000000)
}
\end{pspicture}
%
\hskip0.1in
%
\begin{pspicture}(-0.5,-0.5)(1.5,1.5)
\psgrid
\rput(0.0,-0.25){$n=2$}
\rput(0,0){
\psline
  (0, 0)
  (-0.160000, 0.120000)
  (-0.040000, 0.280000)
  (-0.200000, 0.400000)
  (-0.080000, 0.560000)
  (0.080000, 0.440000)
  (0.200000, 0.600000)
  (0.040000, 0.720000)
  (0.160000, 0.880000)
  (-0.000000, 1.000000)
  (0.120000, 1.160000)
  (0.280000, 1.040000)
  (0.400000, 1.200000)
  (0.560000, 1.080000)
  (0.440000, 0.920000)
  (0.600000, 0.800000)
  (0.720000, 0.960000)
  (0.880000, 0.840000)
  (1.000000, 1.000000)
  (1.160000, 0.880000)
  (1.040000, 0.720000)
  (1.200000, 0.600000)
  (1.080000, 0.440000)
  (0.920000, 0.560000)
  (0.800000, 0.400000)
  (0.960000, 0.280000)
  (0.840000, 0.120000)
  (1.000000, 0.000000)
  (0.880000, -0.160000)
  (0.720000, -0.040000)
  (0.600000, -0.200000)
  (0.440000, -0.080000)
  (0.560000, 0.080000)
  (0.400000, 0.200000)
  (0.280000, 0.040000)
  (0.120000, 0.160000)
  (0.000000, 0.000000)
}
\end{pspicture}
%
\hskip0.1in
%
\begin{pspicture}(-0.5,-0.5)(1.5,1.5)
\psgrid
\rput(0.0,-0.25){$n=3$}
\rput(0,0){
\psline
  (0, 0)
  (-0.088000, 0.016000)
  (-0.072000, 0.104000)
  (-0.160000, 0.120000)
  (-0.144000, 0.208000)
  (-0.056000, 0.192000)
  (-0.040000, 0.280000)
  (-0.128000, 0.296000)
  (-0.112000, 0.384000)
  (-0.200000, 0.400000)
  (-0.184000, 0.488000)
  (-0.096000, 0.472000)
  (-0.080000, 0.560000)
  (0.008000, 0.544000)
  (-0.008000, 0.456000)
  (0.080000, 0.440000)
  (0.096000, 0.528000)
  (0.184000, 0.512000)
  (0.200000, 0.600000)
  (0.112000, 0.616000)
  (0.128000, 0.704000)
  (0.040000, 0.720000)
  (0.056000, 0.808000)
  (0.144000, 0.792000)
  (0.160000, 0.880000)
  (0.072000, 0.896000)
  (0.088000, 0.984000)
  (0.000000, 1.000000)
  (0.016000, 1.088000)
  (0.104000, 1.072000)
  (0.120000, 1.160000)
  (0.208000, 1.144000)
  (0.192000, 1.056000)
  (0.280000, 1.040000)
  (0.296000, 1.128000)
  (0.384000, 1.112000)
  (0.400000, 1.200000)
  (0.488000, 1.184000)
  (0.472000, 1.096000)
  (0.560000, 1.080000)
  (0.544000, 0.992000)
  (0.456000, 1.008000)
  (0.440000, 0.920000)
  (0.528000, 0.904000)
  (0.512000, 0.816000)
  (0.600000, 0.800000)
  (0.616000, 0.888000)
  (0.704000, 0.872000)
  (0.720000, 0.960000)
  (0.808000, 0.944000)
  (0.792000, 0.856000)
  (0.880000, 0.840000)
  (0.896000, 0.927999)
  (0.984000, 0.911999)
  (1.000000, 0.999999)
  (1.088000, 0.983999)
  (1.072000, 0.895999)
  (1.160000, 0.879999)
  (1.144000, 0.791999)
  (1.056000, 0.807999)
  (1.040000, 0.719999)
  (1.128000, 0.703999)
  (1.112000, 0.615999)
  (1.200000, 0.599999)
  (1.184000, 0.511999)
  (1.096000, 0.527999)
  (1.080000, 0.439999)
  (0.992000, 0.455999)
  (1.008000, 0.543999)
  (0.920000, 0.559999)
  (0.904000, 0.471999)
  (0.816000, 0.487999)
  (0.800000, 0.400000)
  (0.888000, 0.383999)
  (0.872000, 0.296000)
  (0.960000, 0.279999)
  (0.944000, 0.192000)
  (0.856000, 0.208000)
  (0.840000, 0.120000)
  (0.928000, 0.104000)
  (0.912000, 0.016000)
  (1.000000, -0.000001)
  (0.984000, -0.088000)
  (0.896000, -0.072000)
  (0.880000, -0.160000)
  (0.792000, -0.144000)
  (0.808000, -0.056000)
  (0.720000, -0.040000)
  (0.704000, -0.128000)
  (0.616000, -0.112000)
  (0.599999, -0.200000)
  (0.511999, -0.184000)
  (0.528000, -0.096000)
  (0.440000, -0.080000)
  (0.456000, 0.008000)
  (0.544000, -0.008000)
  (0.560000, 0.080000)
  (0.472000, 0.096000)
  (0.488000, 0.184000)
  (0.400000, 0.200000)
  (0.384000, 0.112000)
  (0.296000, 0.128000)
  (0.280000, 0.040000)
  (0.192000, 0.056000)
  (0.208000, 0.144000)
  (0.120000, 0.160000)
  (0.104000, 0.072000)
  (0.016000, 0.088000)
  (-0.000000, -0.000000)
}
\end{pspicture}
%
\hskip0.1in\dots
%

\end{zztask}

%%%%%%%%%%%%%%%%%%%%%%%%%%%%%%%%%%%%%%%%%%%%%%%%%%%%%%%%%%%%%%%%%%%%%%%%%%%%%%

\begin{zztask}[Minkowski Island]
В рамках общего условия задачи построить остров Минковского, составленный из
четырех одинаковых колбас Минковского. База для построения кривой 
(нулевое приближение) представляет собой квадрат.
Переход от приближения $n$ к приближению $(n+1)$ осуществляется заменой каждого 
отрезка фигуры на восемь (см. рис. при $n=1$). 
Все отрезки имеют одинаковую длину, угол между отрезками составляет $90^\circ$.
\par\endinput
\begin{pspicture}(-0.5,-0.5)(1.5,1.5)
\psgrid
\rput(0.0,-0.25){$n=0$}
\rput(0,0){
\psline
  (0.000000, 0.000000)
  (1.000000, 0.000000)
  (1.000000, 1.000000)
  (-0.000000, 1.000000)
\psline[linewidth=0.06cm, arrows=o-o]
  (-0.000000, 1.000000)
  (-0.000000, -0.000000)
}
\end{pspicture}
%
\hskip0.1in
%
\begin{pspicture}(-0.5,-0.5)(1.5,1.5)
\psgrid
\rput(0.0,-0.25){$n=1$}
\rput(0,0){
\psline
  (0, 0)
  (0.250000, 0.000000)
  (0.250000, 0.250000)
  (0.500000, 0.250000)
  (0.500000, 0.000000)
  (0.500000, -0.250000)
  (0.750000, -0.250000)
  (0.750000, -0.000000)
  (1.000000, -0.000000)
  (1.000000, 0.250000)
  (0.750000, 0.250000)
  (0.750000, 0.500000)
  (1.000000, 0.500000)
  (1.250000, 0.500000)
  (1.250000, 0.750000)
  (1.000000, 0.750000)
  (1.000000, 1.000000)
  (0.750000, 1.000000)
  (0.750000, 0.750000)
  (0.500000, 0.750000)
  (0.500000, 1.000000)
  (0.500000, 1.250000)
  (0.250000, 1.250000)
  (0.250000, 1.000000)
  (0.000000, 1.000000)
\psline[linewidth=0.06cm, arrows=o-o, showpoints=true]
  (0.000000, 1.000000)
  (0.000000, 0.750000)
  (0.250000, 0.750000)
  (0.250000, 0.500000)
  (0.000000, 0.500000)
  (-0.250000, 0.500000)
  (-0.250000, 0.250000)
  (0.000000, 0.250000)
  (0.000000, 0.000000)
}
\end{pspicture}
%
\hskip0.1in
%
\begin{pspicture}(-0.5,-0.5)(1.5,1.5)
\psgrid
\rput(0.0,-0.25){$n=2$}
\rput(0,0){
\psline
  (0, 0)
  (0.062500, 0.000000)
  (0.062500, 0.062500)
  (0.125000, 0.062500)
  (0.125000, 0.000000)
  (0.125000, -0.062500)
  (0.187500, -0.062500)
  (0.187500, -0.000000)
  (0.250000, -0.000000)
  (0.250000, 0.062500)
  (0.187500, 0.062500)
  (0.187500, 0.125000)
  (0.250000, 0.125000)
  (0.312500, 0.125000)
  (0.312500, 0.187500)
  (0.250000, 0.187500)
  (0.250000, 0.250000)
  (0.312500, 0.250000)
  (0.312500, 0.312500)
  (0.375000, 0.312500)
  (0.375000, 0.250000)
  (0.375000, 0.187500)
  (0.437500, 0.187500)
  (0.437500, 0.250000)
  (0.500000, 0.250000)
  (0.500000, 0.187500)
  (0.562500, 0.187500)
  (0.562500, 0.125000)
  (0.500000, 0.125000)
  (0.437500, 0.125000)
  (0.437500, 0.062500)
  (0.500000, 0.062500)
  (0.500000, -0.000000)
  (0.500000, -0.062500)
  (0.562500, -0.062500)
  (0.562500, -0.125000)
  (0.500000, -0.125000)
  (0.437500, -0.125000)
  (0.437500, -0.187500)
  (0.500000, -0.187500)
  (0.500000, -0.250000)
  (0.562500, -0.250000)
  (0.562500, -0.187500)
  (0.625000, -0.187500)
  (0.625000, -0.250000)
  (0.625000, -0.312500)
  (0.687500, -0.312500)
  (0.687500, -0.250000)
  (0.750000, -0.250000)
  (0.750000, -0.187500)
  (0.687500, -0.187500)
  (0.687500, -0.125000)
  (0.750000, -0.125000)
  (0.812500, -0.125000)
  (0.812500, -0.062500)
  (0.750000, -0.062500)
  (0.750000, -0.000000)
  (0.812500, -0.000000)
  (0.812500, 0.062500)
  (0.875000, 0.062500)
  (0.875000, -0.000000)
  (0.875000, -0.062500)
  (0.937500, -0.062500)
  (0.937500, -0.000000)
  (1.000000, -0.000000)
  (1.000000, 0.062500)
  (0.937500, 0.062500)
  (0.937500, 0.125000)
  (1.000000, 0.125000)
  (1.062500, 0.125000)
  (1.062500, 0.187500)
  (1.000000, 0.187500)
  (1.000000, 0.250000)
  (0.937500, 0.250000)
  (0.937500, 0.187500)
  (0.875000, 0.187500)
  (0.875000, 0.250000)
  (0.875000, 0.312500)
  (0.812500, 0.312500)
  (0.812500, 0.250000)
  (0.750000, 0.250000)
  (0.750000, 0.312500)
  (0.687500, 0.312500)
  (0.687500, 0.375000)
  (0.750000, 0.375000)
  (0.812500, 0.375000)
  (0.812500, 0.437500)
  (0.750000, 0.437500)
  (0.750000, 0.500000)
  (0.812500, 0.500000)
  (0.812500, 0.562500)
  (0.875000, 0.562500)
  (0.875000, 0.500000)
  (0.875000, 0.437500)
  (0.937500, 0.437500)
  (0.937500, 0.500000)
  (1.000000, 0.500000)
  (1.062500, 0.500000)
  (1.062500, 0.562500)
  (1.125000, 0.562500)
  (1.125000, 0.500000)
  (1.125000, 0.437500)
  (1.187500, 0.437500)
  (1.187500, 0.500000)
  (1.250000, 0.500000)
  (1.250000, 0.562500)
  (1.187500, 0.562500)
  (1.187500, 0.625000)
  (1.250000, 0.625000)
  (1.312500, 0.625000)
  (1.312500, 0.687500)
  (1.250000, 0.687500)
  (1.250000, 0.750000)
  (1.187500, 0.750000)
  (1.187500, 0.687500)
  (1.125000, 0.687500)
  (1.125000, 0.750000)
  (1.125000, 0.812500)
  (1.062500, 0.812500)
  (1.062500, 0.750000)
  (1.000000, 0.750000)
  (1.000000, 0.812500)
  (0.937500, 0.812500)
  (0.937500, 0.875000)
  (1.000000, 0.875000)
  (1.062500, 0.875000)
  (1.062500, 0.937500)
  (1.000000, 0.937500)
  (1.000000, 1.000000)
  (0.937500, 1.000000)
  (0.937500, 0.937500)
  (0.875000, 0.937500)
  (0.875000, 1.000000)
  (0.875000, 1.062500)
  (0.812500, 1.062500)
  (0.812500, 1.000000)
  (0.750000, 1.000000)
  (0.750000, 0.937500)
  (0.812500, 0.937500)
  (0.812500, 0.875000)
  (0.750000, 0.875000)
  (0.687500, 0.875000)
  (0.687500, 0.812500)
  (0.750000, 0.812500)
  (0.750000, 0.750000)
  (0.687500, 0.750000)
  (0.687500, 0.687500)
  (0.625000, 0.687500)
  (0.625000, 0.750000)
  (0.625000, 0.812500)
  (0.562500, 0.812500)
  (0.562500, 0.750000)
  (0.500000, 0.750000)
  (0.500000, 0.812500)
  (0.437500, 0.812500)
  (0.437500, 0.875000)
  (0.500000, 0.875000)
  (0.562500, 0.875000)
  (0.562500, 0.937500)
  (0.500000, 0.937500)
  (0.500000, 1.000000)
  (0.500000, 1.062500)
  (0.437500, 1.062500)
  (0.437500, 1.125000)
  (0.500000, 1.125000)
  (0.562500, 1.125000)
  (0.562500, 1.187500)
  (0.500000, 1.187500)
  (0.500000, 1.250000)
  (0.437500, 1.250000)
  (0.437500, 1.187500)
  (0.375000, 1.187500)
  (0.375000, 1.250000)
  (0.375000, 1.312500)
  (0.312500, 1.312500)
  (0.312500, 1.250000)
  (0.250000, 1.250000)
  (0.250000, 1.187500)
  (0.312500, 1.187500)
  (0.312500, 1.125000)
  (0.250000, 1.125000)
  (0.187500, 1.125000)
  (0.187500, 1.062500)
  (0.250000, 1.062500)
  (0.250000, 1.000000)
  (0.187500, 1.000000)
  (0.187500, 0.937500)
  (0.125000, 0.937500)
  (0.125000, 1.000000)
  (0.125000, 1.062500)
  (0.062500, 1.062500)
  (0.062500, 1.000000)
  (0.000000, 1.000000)
  (0.000000, 0.937500)
  (0.062500, 0.937500)
  (0.062500, 0.875000)
  (0.000000, 0.875000)
  (-0.062500, 0.875000)
  (-0.062500, 0.812500)
  (0.000000, 0.812500)
  (0.000000, 0.750000)
  (0.062500, 0.750000)
  (0.062500, 0.812500)
  (0.125000, 0.812500)
  (0.125000, 0.750000)
  (0.125000, 0.687500)
  (0.187500, 0.687500)
  (0.187500, 0.750000)
  (0.250000, 0.750000)
  (0.250000, 0.687500)
  (0.312500, 0.687500)
  (0.312500, 0.625000)
  (0.250000, 0.625000)
  (0.187500, 0.625000)
  (0.187500, 0.562500)
  (0.250000, 0.562500)
  (0.250000, 0.500000)
  (0.187500, 0.500000)
  (0.187500, 0.437500)
  (0.125000, 0.437500)
  (0.125000, 0.500000)
  (0.125000, 0.562500)
  (0.062500, 0.562500)
  (0.062500, 0.500000)
  (0.000000, 0.500000)
  (-0.062500, 0.500000)
  (-0.062500, 0.437500)
  (-0.125000, 0.437500)
  (-0.125000, 0.500000)
  (-0.125000, 0.562500)
  (-0.187500, 0.562500)
  (-0.187500, 0.500000)
  (-0.250000, 0.500000)
  (-0.250000, 0.437500)
  (-0.187500, 0.437500)
  (-0.187500, 0.375000)
  (-0.250000, 0.375000)
  (-0.312500, 0.375000)
  (-0.312500, 0.312500)
  (-0.250000, 0.312500)
  (-0.250000, 0.250000)
  (-0.187500, 0.250000)
  (-0.187500, 0.312500)
  (-0.125000, 0.312500)
  (-0.125000, 0.250000)
  (-0.125000, 0.187500)
  (-0.062500, 0.187500)
  (-0.062500, 0.250000)
  (-0.000000, 0.250000)
  (-0.000000, 0.187500)
  (0.062500, 0.187500)
  (0.062500, 0.125000)
  (-0.000000, 0.125000)
  (-0.062500, 0.125000)
  (-0.062500, 0.062500)
  (-0.000000, 0.062500)
  (-0.000000, 0.000000)
}
\end{pspicture}
%
\hskip0.1in\dots
%

\end{zztask}

%%%%%%%%%%%%%%%%%%%%%%%%%%%%%%%%%%%%%%%%%%%%%%%%%%%%%%%%%%%%%%%%%%%%%%%%%%%%%%

\begin{zztask}[T-Square Curve]
В рамках общего условия задачи построить кривую T-Square.
База для построения кривой (нулевое приближение) представляет собой квадрат.
Переход от приближения $n$ к приближению $(n+1)$ осуществляется заменой каждого 
угла квадрата на квадрат без одной четверти (см. рис. при $n=1$). 
Угол (заменяемая часть) захватывает $1/4$ от каждой стороны.
\par\endinput
\begin{pspicture}(-0.5,-0.5)(1.5,1.5)
\psgrid
\rput(0.5,0.5){$n=0$}
\rput(0,0.5){
\psline
  (0, 0)
  (-0.000000, 0.500000)
  (0.500000, 0.500000)
  (1.000000, 0.500000)
  (1.000000, 0.000000)
  (1.000000, -0.500000)
  (0.500000, -0.500000)
  (0.000000, -0.500000)
  (0.000000, 0.000000)
\psline[linewidth=0.06cm, arrows=o-o]
  (0,0.25)
  (0,0.5)
  (0.25,0.5)
}
\end{pspicture}
%
\hskip0.1in
%
\begin{pspicture}(-0.5,-0.5)(1.5,1.5)
\psgrid
\rput(0.5,0.5){$n=1$}
\rput(0,0.5){
\psline
  (0, 0)
  (-0.000000, 0.250000)
  (-0.250000, 0.250000)
  (-0.250000, 0.500000)
  (-0.250000, 0.750000)
  (0.000000, 0.750000)
  (0.250000, 0.750000)
  (0.250000, 0.500000)
  (0.500000, 0.500000)
  (0.750000, 0.500000)
  (0.750000, 0.750000)
  (1.000000, 0.750000)
  (1.250000, 0.750000)
  (1.250000, 0.500000)
  (1.250000, 0.250000)
  (1.000000, 0.250000)
  (1.000000, -0.000000)
  (1.000000, -0.250000)
  (1.250000, -0.250000)
  (1.250000, -0.500000)
  (1.250000, -0.750000)
  (1.000000, -0.750000)
  (0.750000, -0.750000)
  (0.750000, -0.500000)
  (0.500000, -0.500000)
  (0.250000, -0.500000)
  (0.250000, -0.750000)
  (0.000000, -0.750000)
  (-0.250000, -0.750000)
  (-0.250000, -0.500000)
  (-0.250000, -0.250000)
  (0.000000, -0.250000)
  (0.000000, -0.000000)
\psline[linewidth=0.06cm, arrows=o-o]
  (0,0.25)
  (-0.25,0.25)
  (-0.25,0.75)
  (0.25,0.75)
  (0.25,0.5)
}
\end{pspicture}
%
\hskip0.1in
%
\begin{pspicture}(-0.5,-0.5)(1.5,1.5)
\psgrid
\rput(0.5,0.5){$n=2$}
\rput(0,0.5){
\psline
  (0, 0)
  (-0.000000, 0.125000)
  (-0.000000, 0.250000)
  (-0.125000, 0.250000)
  (-0.125000, 0.125000)
  (-0.250000, 0.125000)
  (-0.375000, 0.125000)
  (-0.375000, 0.250000)
  (-0.375000, 0.375000)
  (-0.250000, 0.375000)
  (-0.250000, 0.500000)
  (-0.250000, 0.625000)
  (-0.375000, 0.625000)
  (-0.375000, 0.750000)
  (-0.375000, 0.875000)
  (-0.250000, 0.875000)
  (-0.125000, 0.875000)
  (-0.125000, 0.750000)
  (0.000000, 0.750000)
  (0.125000, 0.750000)
  (0.125000, 0.875000)
  (0.250000, 0.875000)
  (0.375000, 0.875000)
  (0.375000, 0.750000)
  (0.375000, 0.625000)
  (0.250000, 0.625000)
  (0.250000, 0.500000)
  (0.375000, 0.500000)
  (0.500000, 0.500000)
  (0.625000, 0.500000)
  (0.750000, 0.500000)
  (0.750000, 0.625000)
  (0.625000, 0.625000)
  (0.625000, 0.750000)
  (0.625000, 0.875000)
  (0.750000, 0.875000)
  (0.875000, 0.875000)
  (0.875000, 0.750000)
  (1.000000, 0.750000)
  (1.125000, 0.750000)
  (1.125000, 0.875000)
  (1.250000, 0.875000)
  (1.375000, 0.875000)
  (1.375000, 0.750000)
  (1.375000, 0.625000)
  (1.250000, 0.625000)
  (1.250000, 0.500000)
  (1.250000, 0.375000)
  (1.375000, 0.375000)
  (1.375000, 0.250000)
  (1.375000, 0.125000)
  (1.250000, 0.125000)
  (1.125000, 0.125000)
  (1.125000, 0.250000)
  (1.000000, 0.250000)
  (1.000000, 0.125000)
  (1.000000, -0.000000)
  (1.000000, -0.125000)
  (1.000000, -0.250000)
  (1.125000, -0.250000)
  (1.125000, -0.125000)
  (1.250000, -0.125000)
  (1.375000, -0.125000)
  (1.375000, -0.250000)
  (1.375000, -0.375000)
  (1.250000, -0.375000)
  (1.250000, -0.500000)
  (1.250000, -0.625000)
  (1.375000, -0.625000)
  (1.375000, -0.750000)
  (1.375000, -0.875000)
  (1.250000, -0.875000)
  (1.125000, -0.875000)
  (1.125000, -0.750000)
  (1.000000, -0.750000)
  (0.875000, -0.750000)
  (0.875000, -0.875000)
  (0.750000, -0.875000)
  (0.625000, -0.875000)
  (0.625000, -0.750000)
  (0.625000, -0.625000)
  (0.750000, -0.625000)
  (0.750000, -0.500000)
  (0.625000, -0.500000)
  (0.500000, -0.500000)
  (0.375000, -0.500000)
  (0.250000, -0.500000)
  (0.250000, -0.625000)
  (0.375000, -0.625000)
  (0.375000, -0.750000)
  (0.375000, -0.875000)
  (0.250000, -0.875000)
  (0.125000, -0.875000)
  (0.125000, -0.750000)
  (0.000000, -0.750000)
  (-0.125000, -0.750000)
  (-0.125000, -0.875000)
  (-0.250000, -0.874999)
  (-0.375000, -0.874999)
  (-0.375000, -0.749999)
  (-0.375000, -0.624999)
  (-0.250000, -0.624999)
  (-0.250000, -0.499999)
  (-0.249999, -0.374999)
  (-0.374999, -0.374999)
  (-0.374999, -0.249999)
  (-0.374999, -0.124999)
  (-0.249999, -0.125000)
  (-0.124999, -0.125000)
  (-0.124999, -0.250000)
  (0.000001, -0.250000)
  (0.000001, -0.125000)
  (0.000001, 0.000000)
}
\end{pspicture}
%
\hskip0.1in
%
\begin{pspicture}(-0.5,-0.5)(1.5,1.5)
\psgrid
\rput(0.5,0.5){$n=3$}
\rput(0,0.5){
\psline
  (0, 0)
  (-0.000000, 0.062500)
  (-0.000000, 0.125000)
  (-0.000000, 0.187500)
  (-0.000000, 0.250000)
  (-0.062500, 0.250000)
  (-0.125000, 0.250000)
  (-0.125000, 0.187500)
  (-0.062500, 0.187500)
  (-0.062500, 0.125000)
  (-0.062500, 0.062500)
  (-0.125000, 0.062500)
  (-0.187500, 0.062500)
  (-0.187500, 0.125000)
  (-0.250000, 0.125000)
  (-0.312500, 0.125000)
  (-0.312500, 0.062500)
  (-0.375000, 0.062500)
  (-0.437500, 0.062500)
  (-0.437500, 0.125000)
  (-0.437500, 0.187500)
  (-0.375000, 0.187500)
  (-0.375000, 0.250000)
  (-0.375000, 0.312500)
  (-0.437500, 0.312500)
  (-0.437500, 0.375000)
  (-0.437500, 0.437500)
  (-0.375000, 0.437500)
  (-0.312500, 0.437500)
  (-0.312500, 0.375000)
  (-0.250000, 0.375000)
  (-0.250000, 0.437500)
  (-0.250000, 0.500000)
  (-0.250000, 0.562500)
  (-0.250000, 0.625000)
  (-0.312500, 0.625000)
  (-0.312500, 0.562500)
  (-0.375000, 0.562500)
  (-0.437500, 0.562500)
  (-0.437500, 0.625000)
  (-0.437500, 0.687500)
  (-0.375000, 0.687500)
  (-0.375000, 0.750000)
  (-0.375000, 0.812500)
  (-0.437500, 0.812500)
  (-0.437500, 0.875000)
  (-0.437500, 0.937500)
  (-0.375000, 0.937500)
  (-0.312500, 0.937500)
  (-0.312500, 0.875000)
  (-0.250000, 0.875000)
  (-0.187500, 0.875000)
  (-0.187500, 0.937500)
  (-0.125000, 0.937500)
  (-0.062500, 0.937500)
  (-0.062500, 0.875000)
  (-0.062500, 0.812500)
  (-0.125000, 0.812500)
  (-0.125000, 0.750000)
  (-0.062500, 0.750000)
  (-0.000000, 0.750000)
  (0.062500, 0.750000)
  (0.125000, 0.750000)
  (0.125000, 0.812500)
  (0.062500, 0.812500)
  (0.062500, 0.875000)
  (0.062500, 0.937500)
  (0.125000, 0.937500)
  (0.187500, 0.937500)
  (0.187500, 0.875000)
  (0.250000, 0.875000)
  (0.312500, 0.875000)
  (0.312500, 0.937500)
  (0.375000, 0.937500)
  (0.437500, 0.937500)
  (0.437500, 0.875000)
  (0.437500, 0.812500)
  (0.375000, 0.812500)
  (0.375000, 0.750000)
  (0.375000, 0.687500)
  (0.437500, 0.687500)
  (0.437500, 0.625000)
  (0.437500, 0.562500)
  (0.375000, 0.562500)
  (0.312500, 0.562500)
  (0.312500, 0.625000)
  (0.250000, 0.625000)
  (0.250000, 0.562500)
  (0.250000, 0.500000)
  (0.312500, 0.500000)
  (0.375000, 0.500000)
  (0.437500, 0.500000)
  (0.500000, 0.500000)
  (0.562500, 0.500000)
  (0.625000, 0.500000)
  (0.687500, 0.500000)
  (0.750000, 0.500000)
  (0.750000, 0.562500)
  (0.750000, 0.625000)
  (0.687500, 0.625000)
  (0.687500, 0.562500)
  (0.625000, 0.562500)
  (0.562500, 0.562500)
  (0.562500, 0.625000)
  (0.562500, 0.687500)
  (0.625000, 0.687500)
  (0.625000, 0.750000)
  (0.625000, 0.812500)
  (0.562500, 0.812500)
  (0.562500, 0.875000)
  (0.562500, 0.937500)
  (0.625000, 0.937500)
  (0.687500, 0.937500)
  (0.687500, 0.875000)
  (0.750000, 0.875000)
  (0.812500, 0.875000)
  (0.812500, 0.937500)
  (0.875000, 0.937500)
  (0.937500, 0.937500)
  (0.937500, 0.875000)
  (0.937500, 0.812500)
  (0.875000, 0.812500)
  (0.875000, 0.750000)
  (0.937500, 0.750000)
  (1.000000, 0.750000)
  (1.062500, 0.750000)
  (1.125000, 0.750000)
  (1.125000, 0.812500)
  (1.062500, 0.812500)
  (1.062500, 0.875000)
  (1.062500, 0.937500)
  (1.125000, 0.937500)
  (1.187500, 0.937500)
  (1.187500, 0.875000)
  (1.250000, 0.875000)
  (1.312500, 0.875000)
  (1.312500, 0.937500)
  (1.375000, 0.937500)
  (1.437500, 0.937500)
  (1.437500, 0.875000)
  (1.437500, 0.812500)
  (1.375000, 0.812500)
  (1.375000, 0.750000)
  (1.375000, 0.687500)
  (1.437500, 0.687500)
  (1.437500, 0.625000)
  (1.437500, 0.562500)
  (1.375000, 0.562500)
  (1.312500, 0.562500)
  (1.312500, 0.625000)
  (1.250000, 0.625000)
  (1.250000, 0.562500)
  (1.250000, 0.500000)
  (1.250000, 0.437500)
  (1.250000, 0.375000)
  (1.312500, 0.375000)
  (1.312500, 0.437500)
  (1.375000, 0.437500)
  (1.437500, 0.437500)
  (1.437500, 0.375000)
  (1.437500, 0.312500)
  (1.375000, 0.312500)
  (1.375000, 0.250000)
  (1.375000, 0.187500)
  (1.437500, 0.187500)
  (1.437500, 0.125000)
  (1.437500, 0.062500)
  (1.375000, 0.062500)
  (1.312500, 0.062500)
  (1.312500, 0.125000)
  (1.250000, 0.125000)
  (1.187500, 0.125000)
  (1.187500, 0.062500)
  (1.125000, 0.062500)
  (1.062500, 0.062500)
  (1.062500, 0.125000)
  (1.062500, 0.187500)
  (1.125000, 0.187500)
  (1.125000, 0.250000)
  (1.062500, 0.250000)
  (1.000000, 0.250000)
  (1.000000, 0.187500)
  (1.000000, 0.125000)
  (1.000000, 0.062500)
  (1.000000, 0.000000)
  (1.000000, -0.062500)
  (1.000000, -0.125000)
  (1.000000, -0.187500)
  (1.000000, -0.250000)
  (1.062500, -0.250000)
  (1.125000, -0.250000)
  (1.125000, -0.187500)
  (1.062500, -0.187500)
  (1.062500, -0.125000)
  (1.062500, -0.062500)
  (1.125000, -0.062500)
  (1.187500, -0.062500)
  (1.187500, -0.125000)
  (1.250000, -0.125000)
  (1.312500, -0.125000)
  (1.312500, -0.062500)
  (1.375000, -0.062500)
  (1.437500, -0.062500)
  (1.437500, -0.125000)
  (1.437500, -0.187500)
  (1.375000, -0.187500)
  (1.375000, -0.250000)
  (1.375000, -0.312500)
  (1.437500, -0.312500)
  (1.437500, -0.375000)
  (1.437500, -0.437500)
  (1.375000, -0.437500)
  (1.312500, -0.437500)
  (1.312500, -0.375000)
  (1.250000, -0.375000)
  (1.250000, -0.437500)
  (1.250000, -0.500000)
  (1.250000, -0.562500)
  (1.250000, -0.625000)
  (1.312500, -0.625000)
  (1.312500, -0.562500)
  (1.375000, -0.562500)
  (1.437500, -0.562500)
  (1.437500, -0.625000)
  (1.437500, -0.687500)
  (1.375000, -0.687500)
  (1.375000, -0.750000)
  (1.375000, -0.812500)
  (1.437500, -0.812500)
  (1.437500, -0.875000)
  (1.437500, -0.937500)
  (1.375000, -0.937500)
  (1.312500, -0.937500)
  (1.312500, -0.875000)
  (1.250000, -0.875000)
  (1.187500, -0.875000)
  (1.187500, -0.937500)
  (1.125000, -0.937500)
  (1.062500, -0.937500)
  (1.062500, -0.875000)
  (1.062500, -0.812500)
  (1.125000, -0.812500)
  (1.125000, -0.750000)
  (1.062500, -0.750000)
  (1.000000, -0.750000)
  (0.937500, -0.750000)
  (0.875000, -0.750000)
  (0.875000, -0.812500)
  (0.937500, -0.812500)
  (0.937500, -0.875000)
  (0.937500, -0.937500)
  (0.875000, -0.937500)
  (0.812500, -0.937500)
  (0.812500, -0.875000)
  (0.750000, -0.875000)
  (0.687500, -0.875000)
  (0.687500, -0.937500)
  (0.625000, -0.937500)
  (0.562500, -0.937500)
  (0.562500, -0.875000)
  (0.562500, -0.812500)
  (0.625000, -0.812500)
  (0.625000, -0.750000)
  (0.625000, -0.687500)
  (0.562500, -0.687500)
  (0.562500, -0.625000)
  (0.562500, -0.562500)
  (0.625000, -0.562500)
  (0.687500, -0.562500)
  (0.687500, -0.625000)
  (0.750000, -0.625000)
  (0.750000, -0.562500)
  (0.750000, -0.500000)
  (0.687500, -0.500000)
  (0.625000, -0.500000)
  (0.562500, -0.500000)
  (0.500000, -0.500000)
  (0.437500, -0.500000)
  (0.375000, -0.500000)
  (0.312500, -0.500000)
  (0.250000, -0.500000)
  (0.250000, -0.562500)
  (0.250000, -0.625000)
  (0.312500, -0.625000)
  (0.312500, -0.562500)
  (0.375000, -0.562500)
  (0.437500, -0.562500)
  (0.437500, -0.625000)
  (0.437500, -0.687500)
  (0.375000, -0.687500)
  (0.375000, -0.750000)
  (0.375000, -0.812500)
  (0.437500, -0.812500)
  (0.437500, -0.875000)
  (0.437500, -0.937500)
  (0.375000, -0.937500)
  (0.312500, -0.937500)
  (0.312500, -0.875000)
  (0.250000, -0.875000)
  (0.187500, -0.875000)
  (0.187500, -0.937500)
  (0.125000, -0.937500)
  (0.062500, -0.937500)
  (0.062500, -0.875000)
  (0.062500, -0.812500)
  (0.125000, -0.812500)
  (0.125000, -0.750000)
  (0.062500, -0.750000)
  (0.000000, -0.750000)
  (-0.062500, -0.750000)
  (-0.125000, -0.750000)
  (-0.125000, -0.812500)
  (-0.062500, -0.812500)
  (-0.062500, -0.875000)
  (-0.062500, -0.937500)
  (-0.125000, -0.937500)
  (-0.187500, -0.937500)
  (-0.187500, -0.875000)
  (-0.250000, -0.875000)
  (-0.312500, -0.875000)
  (-0.312500, -0.937500)
  (-0.375000, -0.937500)
  (-0.437500, -0.937500)
  (-0.437500, -0.875000)
  (-0.437500, -0.812500)
  (-0.375000, -0.812500)
  (-0.375000, -0.750000)
  (-0.375000, -0.687500)
  (-0.437500, -0.687500)
  (-0.437500, -0.625000)
  (-0.437500, -0.562500)
  (-0.375000, -0.562500)
  (-0.312500, -0.562500)
  (-0.312500, -0.625000)
  (-0.250000, -0.625000)
  (-0.250000, -0.562500)
  (-0.250000, -0.500000)
  (-0.250000, -0.437500)
  (-0.250000, -0.375000)
  (-0.312500, -0.375000)
  (-0.312500, -0.437500)
  (-0.375000, -0.437500)
  (-0.437500, -0.437500)
  (-0.437500, -0.375000)
  (-0.437500, -0.312500)
  (-0.375000, -0.312500)
  (-0.375000, -0.250000)
  (-0.375000, -0.187500)
  (-0.437500, -0.187500)
  (-0.437500, -0.125000)
  (-0.437500, -0.062500)
  (-0.375000, -0.062500)
  (-0.312500, -0.062500)
  (-0.312500, -0.125000)
  (-0.250000, -0.125000)
  (-0.187500, -0.125000)
  (-0.187500, -0.062500)
  (-0.125000, -0.062500)
  (-0.062500, -0.062501)
  (-0.062500, -0.125001)
  (-0.062500, -0.187501)
  (-0.125000, -0.187500)
  (-0.125000, -0.250000)
  (-0.062500, -0.250000)
  (0.000000, -0.250001)
  (0.000000, -0.187501)
  (0.000000, -0.125001)
  (0.000000, -0.062501)
  (0.000000, -0.000001)
}
\end{pspicture}
%
\hskip0.1in\dots
%

\end{zztask}

%%%%%%%%%%%%%%%%%%%%%%%%%%%%%%%%%%%%%%%%%%%%%%%%%%%%%%%%%%%%%%%%%%%%%%%%%%%%%%

\begin{zztask}[Cross-Stitch Curve]
В рамках общего условия задачи построить ``вышивку крестиком''.
База для построения кривой (нулевое приближение) представляет собой квадрат.
Переход от приближения $n$ к приближению $(n+1)$ осуществляется заменой каждого 
отрезка фигуры на пять (см. рис. при $n=1$). 
Все отрезки имеют одинаковую длину, угол между отрезками составляет $90^\circ$.
\par\endinput
\begin{pspicture}(-0.5,-0.5)(1.5,1.5)
\psgrid
\rput(0.5,0.5){$n=0$}
\rput(0,0){
\psline
  (0, 0)
  (-0.000000, 1.000000)
  (1.000000, 1.000000)
  (1.000000, 0.000000)
  (0, 0)
\psline[linewidth=0.06cm, arrows=o-o]
  (0, 0)
  (-0.000000, 1.000000)
}
\end{pspicture}
%
\hskip0.1in
%
\begin{pspicture}(-0.5,-0.5)(1.5,1.5)
\psgrid
\rput(0.5,0.5){$n=1$}
\rput(0,0){
\psline
  (0, 0)
  (-0.000000, 0.333333)
  (-0.333333, 0.333333)
  (-0.333333, 0.666667)
  (0.000000, 0.666667)
  (-0.000000, 1.000000)
  (0.333333, 1.000000)
  (0.333333, 1.333333)
  (0.666667, 1.333333)
  (0.666667, 1.000000)
  (1.000000, 1.000000)
  (1.000000, 0.666667)
  (1.333333, 0.666667)
  (1.333333, 0.333333)
  (1.000000, 0.333333)
  (1.000000, -0.000000)
  (0.666667, -0.000000)
  (0.666667, -0.333333)
  (0.333333, -0.333333)
  (0.333333, 0.000000)
  (0.000000, -0.000000)
\psline[linewidth=0.06cm, arrows=o-o, showpoints=true]
  (0, 0)
  (-0.000000, 0.333333)
  (-0.333333, 0.333333)
  (-0.333333, 0.666667)
  (0.000000, 0.666667)
  (-0.000000, 1.000000)
}
\end{pspicture}
%
\hskip0.1in
%
\begin{pspicture}(-0.5,-0.5)(1.5,1.5)
\psgrid
\rput(0.5,0.5){$n=2$}
\rput(0,0){
\psline
  (0, 0)
  (-0.000000, 0.111111)
  (-0.111111, 0.111111)
  (-0.111111, 0.222222)
  (-0.000000, 0.222222)
  (-0.000000, 0.333333)
  (-0.111111, 0.333333)
  (-0.111111, 0.222222)
  (-0.222222, 0.222222)
  (-0.222222, 0.333333)
  (-0.333333, 0.333333)
  (-0.333333, 0.444444)
  (-0.444444, 0.444444)
  (-0.444444, 0.555556)
  (-0.333333, 0.555556)
  (-0.333333, 0.666667)
  (-0.222222, 0.666667)
  (-0.222222, 0.777778)
  (-0.111111, 0.777778)
  (-0.111111, 0.666667)
  (0.000000, 0.666667)
  (0.000000, 0.777778)
  (-0.111111, 0.777778)
  (-0.111111, 0.888889)
  (0.000000, 0.888889)
  (0.000000, 1.000000)
  (0.111111, 1.000000)
  (0.111111, 1.111111)
  (0.222222, 1.111111)
  (0.222222, 1.000000)
  (0.333333, 1.000000)
  (0.333334, 1.111111)
  (0.222222, 1.111111)
  (0.222222, 1.222222)
  (0.333334, 1.222222)
  (0.333334, 1.333334)
  (0.444445, 1.333334)
  (0.444445, 1.444445)
  (0.555556, 1.444445)
  (0.555556, 1.333334)
  (0.666667, 1.333334)
  (0.666667, 1.222222)
  (0.777778, 1.222222)
  (0.777778, 1.111111)
  (0.666667, 1.111111)
  (0.666667, 1.000000)
  (0.777778, 1.000000)
  (0.777778, 1.111111)
  (0.888889, 1.111111)
  (0.888889, 1.000000)
  (1.000000, 1.000000)
  (1.000000, 0.888889)
  (1.111111, 0.888889)
  (1.111111, 0.777778)
  (1.000000, 0.777778)
  (1.000000, 0.666667)
  (1.111111, 0.666667)
  (1.111111, 0.777778)
  (1.222222, 0.777778)
  (1.222222, 0.666667)
  (1.333333, 0.666667)
  (1.333333, 0.555556)
  (1.444445, 0.555556)
  (1.444445, 0.444445)
  (1.333333, 0.444445)
  (1.333333, 0.333334)
  (1.222222, 0.333334)
  (1.222222, 0.222222)
  (1.111111, 0.222223)
  (1.111111, 0.333334)
  (1.000000, 0.333334)
  (1.000000, 0.222223)
  (1.111111, 0.222223)
  (1.111111, 0.111111)
  (1.000000, 0.111111)
  (1.000000, 0.000000)
  (0.888889, 0.000000)
  (0.888889, -0.111111)
  (0.777778, -0.111111)
  (0.777778, 0.000000)
  (0.666667, 0.000000)
  (0.666667, -0.111111)
  (0.777778, -0.111111)
  (0.777778, -0.222222)
  (0.666666, -0.222222)
  (0.666666, -0.333333)
  (0.555555, -0.333333)
  (0.555555, -0.444444)
  (0.444444, -0.444444)
  (0.444444, -0.333333)
  (0.333333, -0.333333)
  (0.333333, -0.222222)
  (0.222222, -0.222222)
  (0.222222, -0.111111)
  (0.333333, -0.111111)
  (0.333333, 0.000000)
  (0.222222, 0.000001)
  (0.222222, -0.111111)
  (0.111111, -0.111111)
  (0.111111, 0.000001)
  (-0.000000, 0.000001)
}
\end{pspicture}
%
\hskip0.1in
%
\begin{pspicture}(-0.5,-0.5)(1.5,1.5)
\psgrid
\rput(0.5,0.5){$n=3$}
\rput(0,0){
\psline
  (0, 0)
  (-0.000000, 0.037037)
  (-0.037037, 0.037037)
  (-0.037037, 0.074074)
  (0.000000, 0.074074)
  (-0.000000, 0.111111)
  (-0.037037, 0.111111)
  (-0.037037, 0.074074)
  (-0.074074, 0.074074)
  (-0.074074, 0.111111)
  (-0.111111, 0.111111)
  (-0.111111, 0.148148)
  (-0.148148, 0.148148)
  (-0.148148, 0.185185)
  (-0.111111, 0.185185)
  (-0.111111, 0.222222)
  (-0.074074, 0.222222)
  (-0.074074, 0.259259)
  (-0.037037, 0.259259)
  (-0.037037, 0.222222)
  (0.000000, 0.222222)
  (0.000000, 0.259259)
  (-0.037037, 0.259259)
  (-0.037037, 0.296296)
  (0.000000, 0.296296)
  (0.000000, 0.333333)
  (-0.037037, 0.333333)
  (-0.037037, 0.296296)
  (-0.074074, 0.296296)
  (-0.074074, 0.333333)
  (-0.111111, 0.333333)
  (-0.111111, 0.296296)
  (-0.074074, 0.296296)
  (-0.074074, 0.259259)
  (-0.111111, 0.259259)
  (-0.111111, 0.222222)
  (-0.148148, 0.222222)
  (-0.148148, 0.185185)
  (-0.185185, 0.185185)
  (-0.185185, 0.222222)
  (-0.222222, 0.222222)
  (-0.222222, 0.259259)
  (-0.259259, 0.259259)
  (-0.259259, 0.296296)
  (-0.222222, 0.296296)
  (-0.222222, 0.333333)
  (-0.259259, 0.333333)
  (-0.259259, 0.296296)
  (-0.296296, 0.296296)
  (-0.296296, 0.333333)
  (-0.333333, 0.333333)
  (-0.333333, 0.370370)
  (-0.370370, 0.370370)
  (-0.370370, 0.407407)
  (-0.333333, 0.407407)
  (-0.333333, 0.444444)
  (-0.370370, 0.444444)
  (-0.370370, 0.407407)
  (-0.407407, 0.407407)
  (-0.407407, 0.444444)
  (-0.444444, 0.444444)
  (-0.444444, 0.481481)
  (-0.481482, 0.481481)
  (-0.481482, 0.518519)
  (-0.444444, 0.518518)
  (-0.444444, 0.555556)
  (-0.407407, 0.555556)
  (-0.407407, 0.592593)
  (-0.370370, 0.592593)
  (-0.370370, 0.555555)
  (-0.333333, 0.555556)
  (-0.333333, 0.592593)
  (-0.370370, 0.592593)
  (-0.370370, 0.629630)
  (-0.333333, 0.629630)
  (-0.333333, 0.666667)
  (-0.296296, 0.666667)
  (-0.296296, 0.703704)
  (-0.259259, 0.703704)
  (-0.259259, 0.666667)
  (-0.222222, 0.666667)
  (-0.222222, 0.703704)
  (-0.259259, 0.703704)
  (-0.259259, 0.740741)
  (-0.222222, 0.740741)
  (-0.222222, 0.777778)
  (-0.185185, 0.777778)
  (-0.185185, 0.814815)
  (-0.148148, 0.814815)
  (-0.148148, 0.777778)
  (-0.111111, 0.777778)
  (-0.111111, 0.740741)
  (-0.074074, 0.740741)
  (-0.074074, 0.703704)
  (-0.111111, 0.703704)
  (-0.111111, 0.666667)
  (-0.074074, 0.666667)
  (-0.074074, 0.703704)
  (-0.037037, 0.703704)
  (-0.037037, 0.666667)
  (-0.000000, 0.666667)
  (0.000000, 0.703704)
  (-0.037037, 0.703704)
  (-0.037037, 0.740741)
  (0.000000, 0.740741)
  (0.000000, 0.777778)
  (-0.037037, 0.777778)
  (-0.037037, 0.740741)
  (-0.074074, 0.740741)
  (-0.074074, 0.777778)
  (-0.111111, 0.777778)
  (-0.111111, 0.814815)
  (-0.148148, 0.814815)
  (-0.148148, 0.851852)
  (-0.111111, 0.851852)
  (-0.111111, 0.888889)
  (-0.074074, 0.888889)
  (-0.074074, 0.925926)
  (-0.037037, 0.925926)
  (-0.037037, 0.888889)
  (0.000000, 0.888889)
  (0.000000, 0.925926)
  (-0.037037, 0.925926)
  (-0.037037, 0.962963)
  (0.000000, 0.962963)
  (0.000000, 1.000000)
  (0.037037, 1.000000)
  (0.037037, 1.037037)
  (0.074074, 1.037037)
  (0.074074, 1.000000)
  (0.111111, 1.000000)
  (0.111111, 1.037037)
  (0.074074, 1.037037)
  (0.074074, 1.074074)
  (0.111111, 1.074074)
  (0.111111, 1.111111)
  (0.148148, 1.111111)
  (0.148148, 1.148148)
  (0.185185, 1.148148)
  (0.185185, 1.111111)
  (0.222222, 1.111111)
  (0.222222, 1.074074)
  (0.259259, 1.074074)
  (0.259259, 1.037037)
  (0.222222, 1.037037)
  (0.222222, 1.000000)
  (0.259259, 1.000000)
  (0.259259, 1.037037)
  (0.296296, 1.037037)
  (0.296296, 1.000000)
  (0.333333, 1.000000)
  (0.333333, 1.037037)
  (0.296296, 1.037037)
  (0.296296, 1.074074)
  (0.333333, 1.074074)
  (0.333333, 1.111111)
  (0.296296, 1.111111)
  (0.296296, 1.074074)
  (0.259259, 1.074074)
  (0.259259, 1.111111)
  (0.222222, 1.111111)
  (0.222222, 1.148148)
  (0.185185, 1.148148)
  (0.185185, 1.185185)
  (0.222222, 1.185185)
  (0.222222, 1.222222)
  (0.259259, 1.222222)
  (0.259259, 1.259259)
  (0.296296, 1.259259)
  (0.296296, 1.222222)
  (0.333333, 1.222222)
  (0.333333, 1.259259)
  (0.296296, 1.259259)
  (0.296296, 1.296296)
  (0.333333, 1.296296)
  (0.333333, 1.333333)
  (0.370370, 1.333333)
  (0.370370, 1.370370)
  (0.407407, 1.370370)
  (0.407407, 1.333333)
  (0.444445, 1.333333)
  (0.444445, 1.370370)
  (0.407407, 1.370370)
  (0.407407, 1.407407)
  (0.444445, 1.407407)
  (0.444445, 1.444444)
  (0.481482, 1.444444)
  (0.481482, 1.481481)
  (0.518519, 1.481481)
  (0.518519, 1.444444)
  (0.555556, 1.444444)
  (0.555556, 1.407407)
  (0.592593, 1.407407)
  (0.592593, 1.370370)
  (0.555556, 1.370370)
  (0.555556, 1.333333)
  (0.592593, 1.333333)
  (0.592593, 1.370370)
  (0.629630, 1.370370)
  (0.629630, 1.333333)
  (0.666667, 1.333333)
  (0.666667, 1.296296)
  (0.703704, 1.296296)
  (0.703704, 1.259259)
  (0.666667, 1.259259)
  (0.666667, 1.222222)
  (0.703704, 1.222222)
  (0.703704, 1.259259)
  (0.740741, 1.259259)
  (0.740741, 1.222222)
  (0.777778, 1.222222)
  (0.777778, 1.185185)
  (0.814815, 1.185185)
  (0.814815, 1.148148)
  (0.777778, 1.148148)
  (0.777778, 1.111111)
  (0.740741, 1.111111)
  (0.740741, 1.074074)
  (0.703704, 1.074074)
  (0.703704, 1.111111)
  (0.666667, 1.111111)
  (0.666667, 1.074074)
  (0.703704, 1.074074)
  (0.703704, 1.037037)
  (0.666667, 1.037037)
  (0.666667, 1.000000)
  (0.703704, 1.000000)
  (0.703704, 1.037037)
  (0.740741, 1.037037)
  (0.740741, 1.000000)
  (0.777778, 1.000000)
  (0.777778, 1.037037)
  (0.740741, 1.037037)
  (0.740741, 1.074074)
  (0.777778, 1.074074)
  (0.777778, 1.111111)
  (0.814815, 1.111111)
  (0.814815, 1.148148)
  (0.851852, 1.148148)
  (0.851852, 1.111111)
  (0.888889, 1.111111)
  (0.888889, 1.074074)
  (0.925926, 1.074074)
  (0.925926, 1.037037)
  (0.888889, 1.037037)
  (0.888889, 1.000000)
  (0.925926, 1.000000)
  (0.925926, 1.037037)
  (0.962963, 1.037037)
  (0.962963, 1.000000)
  (1.000000, 1.000000)
  (1.000000, 0.962963)
  (1.037037, 0.962963)
  (1.037037, 0.925926)
  (1.000000, 0.925926)
  (1.000000, 0.888889)
  (1.037037, 0.888889)
  (1.037037, 0.925926)
  (1.074074, 0.925926)
  (1.074074, 0.888889)
  (1.111111, 0.888889)
  (1.111111, 0.851852)
  (1.148148, 0.851852)
  (1.148148, 0.814815)
  (1.111111, 0.814815)
  (1.111111, 0.777778)
  (1.074074, 0.777778)
  (1.074074, 0.740741)
  (1.037037, 0.740741)
  (1.037037, 0.777778)
  (1.000000, 0.777778)
  (1.000000, 0.740741)
  (1.037037, 0.740741)
  (1.037037, 0.703704)
  (1.000000, 0.703704)
  (1.000000, 0.666667)
  (1.037037, 0.666667)
  (1.037037, 0.703704)
  (1.074074, 0.703704)
  (1.074074, 0.666667)
  (1.111111, 0.666667)
  (1.111111, 0.703704)
  (1.074074, 0.703704)
  (1.074074, 0.740741)
  (1.111111, 0.740741)
  (1.111111, 0.777778)
  (1.148148, 0.777778)
  (1.148148, 0.814815)
  (1.185185, 0.814815)
  (1.185185, 0.777778)
  (1.222222, 0.777778)
  (1.222222, 0.740741)
  (1.259259, 0.740741)
  (1.259259, 0.703704)
  (1.222222, 0.703704)
  (1.222222, 0.666667)
  (1.259259, 0.666667)
  (1.259259, 0.703704)
  (1.296296, 0.703704)
  (1.296296, 0.666667)
  (1.333333, 0.666667)
  (1.333333, 0.629630)
  (1.370370, 0.629630)
  (1.370370, 0.592593)
  (1.333333, 0.592593)
  (1.333333, 0.555556)
  (1.370370, 0.555556)
  (1.370370, 0.592593)
  (1.407407, 0.592593)
  (1.407407, 0.555556)
  (1.444444, 0.555556)
  (1.444444, 0.518519)
  (1.481481, 0.518519)
  (1.481481, 0.481482)
  (1.444444, 0.481482)
  (1.444444, 0.444444)
  (1.407407, 0.444444)
  (1.407407, 0.407407)
  (1.370370, 0.407407)
  (1.370370, 0.444444)
  (1.333333, 0.444444)
  (1.333333, 0.407407)
  (1.370370, 0.407407)
  (1.370370, 0.370370)
  (1.333333, 0.370370)
  (1.333333, 0.333333)
  (1.296296, 0.333333)
  (1.296296, 0.296296)
  (1.259259, 0.296296)
  (1.259259, 0.333333)
  (1.222222, 0.333333)
  (1.222222, 0.296296)
  (1.259259, 0.296296)
  (1.259259, 0.259259)
  (1.222222, 0.259259)
  (1.222222, 0.222222)
  (1.185185, 0.222222)
  (1.185185, 0.185185)
  (1.148148, 0.185185)
  (1.148148, 0.222222)
  (1.111111, 0.222222)
  (1.111111, 0.259259)
  (1.074074, 0.259259)
  (1.074074, 0.296296)
  (1.111111, 0.296296)
  (1.111111, 0.333333)
  (1.074074, 0.333333)
  (1.074074, 0.296296)
  (1.037037, 0.296296)
  (1.037037, 0.333333)
  (1.000000, 0.333333)
  (1.000000, 0.296296)
  (1.037037, 0.296296)
  (1.037037, 0.259259)
  (1.000000, 0.259259)
  (1.000000, 0.222222)
  (1.037037, 0.222222)
  (1.037037, 0.259259)
  (1.074074, 0.259259)
  (1.074074, 0.222222)
  (1.111111, 0.222222)
  (1.111111, 0.185185)
  (1.148148, 0.185185)
  (1.148148, 0.148148)
  (1.111111, 0.148148)
  (1.111111, 0.111111)
  (1.074074, 0.111111)
  (1.074074, 0.074074)
  (1.037037, 0.074074)
  (1.037037, 0.111111)
  (1.000000, 0.111111)
  (1.000000, 0.074074)
  (1.037037, 0.074074)
  (1.037037, 0.037037)
  (1.000000, 0.037037)
  (1.000000, -0.000000)
  (0.962963, -0.000000)
  (0.962963, -0.037037)
  (0.925926, -0.037037)
  (0.925926, -0.000000)
  (0.888889, 0.000000)
  (0.888889, -0.037037)
  (0.925926, -0.037037)
  (0.925926, -0.074074)
  (0.888889, -0.074074)
  (0.888889, -0.111111)
  (0.851852, -0.111111)
  (0.851852, -0.148148)
  (0.814815, -0.148148)
  (0.814815, -0.111111)
  (0.777778, -0.111111)
  (0.777778, -0.074074)
  (0.740741, -0.074074)
  (0.740741, -0.037037)
  (0.777778, -0.037037)
  (0.777778, 0.000000)
  (0.740741, 0.000000)
  (0.740741, -0.037037)
  (0.703704, -0.037037)
  (0.703704, 0.000000)
  (0.666667, 0.000000)
  (0.666667, -0.037037)
  (0.703704, -0.037037)
  (0.703704, -0.074074)
  (0.666667, -0.074074)
  (0.666667, -0.111111)
  (0.703704, -0.111111)
  (0.703704, -0.074074)
  (0.740741, -0.074074)
  (0.740741, -0.111111)
  (0.777778, -0.111111)
  (0.777778, -0.148148)
  (0.814815, -0.148148)
  (0.814815, -0.185185)
  (0.777778, -0.185185)
  (0.777778, -0.222222)
  (0.740741, -0.222222)
  (0.740741, -0.259259)
  (0.703704, -0.259259)
  (0.703704, -0.222222)
  (0.666667, -0.222222)
  (0.666667, -0.259259)
  (0.703704, -0.259259)
  (0.703704, -0.296296)
  (0.666667, -0.296296)
  (0.666667, -0.333333)
  (0.629630, -0.333333)
  (0.629630, -0.370370)
  (0.592593, -0.370370)
  (0.592593, -0.333333)
  (0.555556, -0.333333)
  (0.555556, -0.370370)
  (0.592593, -0.370370)
  (0.592593, -0.407407)
  (0.555556, -0.407407)
  (0.555556, -0.444444)
  (0.518519, -0.444444)
  (0.518519, -0.481481)
  (0.481482, -0.481481)
  (0.481482, -0.444444)
  (0.444445, -0.444444)
  (0.444445, -0.407407)
  (0.407408, -0.407407)
  (0.407408, -0.370370)
  (0.444445, -0.370370)
  (0.444445, -0.333333)
  (0.407408, -0.333333)
  (0.407408, -0.370370)
  (0.370371, -0.370370)
  (0.370371, -0.333333)
  (0.333334, -0.333333)
  (0.333334, -0.296296)
  (0.296297, -0.296296)
  (0.296297, -0.259259)
  (0.333334, -0.259259)
  (0.333334, -0.222222)
  (0.296297, -0.222222)
  (0.296297, -0.259259)
  (0.259260, -0.259259)
  (0.259260, -0.222222)
  (0.222223, -0.222222)
  (0.222223, -0.185185)
  (0.185186, -0.185185)
  (0.185186, -0.148148)
  (0.222223, -0.148148)
  (0.222223, -0.111111)
  (0.259260, -0.111111)
  (0.259260, -0.074074)
  (0.296297, -0.074074)
  (0.296297, -0.111111)
  (0.333334, -0.111111)
  (0.333334, -0.074074)
  (0.296297, -0.074074)
  (0.296297, -0.037037)
  (0.333334, -0.037037)
  (0.333334, 0.000000)
  (0.296297, 0.000000)
  (0.296297, -0.037037)
  (0.259260, -0.037037)
  (0.259260, 0.000000)
  (0.222223, 0.000000)
  (0.222223, -0.037037)
  (0.259260, -0.037037)
  (0.259260, -0.074074)
  (0.222223, -0.074074)
  (0.222223, -0.111111)
  (0.185186, -0.111111)
  (0.185186, -0.148148)
  (0.148149, -0.148148)
  (0.148149, -0.111111)
  (0.111112, -0.111111)
  (0.111112, -0.074074)
  (0.074075, -0.074074)
  (0.074075, -0.037037)
  (0.111112, -0.037037)
  (0.111112, 0.000000)
  (0.074075, 0.000000)
  (0.074075, -0.037037)
  (0.037038, -0.037037)
  (0.037038, 0.000000)
  (0.000001, 0.000000)
}
\end{pspicture}
%
\hskip0.1in\dots
%

\end{zztask}

%%%%%%%%%%%%%%%%%%%%%%%%%%%%%%%%%%%%%%%%%%%%%%%%%%%%%%%%%%%%%%%%%%%%%%%%%%%%%%

\begin{zztask}[Anti-Cross-Stitch Curve]
В рамках общего условия задачи построить анти-``вышивку крестиком''.
База для построения кривой (нулевое приближение) представляет собой квадрат.
Переход от приближения $n$ к приближению $(n+1)$ осуществляется заменой каждого 
отрезка фигуры на пять (см. рис. при $n=1$). 
Все отрезки имеют одинаковую длину, угол между отрезками составляет $90^\circ$.
\par\noindent
%
\begin{zzfrac}{$n = 0$}
\draw[thin]
  (0.000000, 0.000000) --
  (0.000000, 1.000000) --
  (1.000000, 1.000000) --
  (1.000000, 0.000000) --
  (0.000000, 0.000000);
\draw[very thick]
  (0.000000, 0.000000) node (A) {} --
  (0.000000, 1.000000) node (B) {};
\zzfracdots{A,B}
\end{zzfrac}
%
\zzfracskip
%
\begin{zzfrac}{$n = 1$}
\draw[thin]
  (0.000000, 0.000000) --
  (0.000000, 0.333333) --
  (0.333333, 0.333333) --
  (0.333333, 0.666667) --
  (0.000000, 0.666667) --
  (0.000000, 1.000000) --
  (0.333333, 1.000000) --
  (0.333333, 0.666667) --
  (0.666667, 0.666667) --
  (0.666667, 1.000000) --
  (1.000000, 1.000000) --
  (1.000000, 0.666667) --
  (0.666667, 0.666667) --
  (0.666667, 0.333333) --
  (1.000000, 0.333333) --
  (1.000000, 0.000000) --
  (0.666667, 0.000000) --
  (0.666667, 0.333333) --
  (0.333333, 0.333333) --
  (0.333333, 0.000000) --
  (0.000000, 0.000000);
\draw[very thick]
  (0.000000, 0.000000) --
  (0.000000, 0.333333) node (A1) {} --
  (0.333333, 0.333333) node (A2) {} --
  (0.333333, 0.666667) node (B2) {} --
  (0.000000, 0.666667) node (B1) {} --
  (0.000000, 1.000000);
\zzfracdots{A,A1,A2,B2,B1,B}
\end{zzfrac}
%
\zzfracskip
%
\begin{zzfrac}{$n = 2$}
\draw[thin]
  (0.000000, 0.000000) --
  (0.000000, 0.111111) --
  (0.111111, 0.111111) --
  (0.111111, 0.222222) --
  (0.000000, 0.222222) --
  (0.000000, 0.333333) --
  (0.111111, 0.333333) --
  (0.111111, 0.222222) --
  (0.222222, 0.222222) --
  (0.222222, 0.333333) --
  (0.333333, 0.333333) --
  (0.333333, 0.444444) --
  (0.444444, 0.444444) --
  (0.444444, 0.555556) --
  (0.333333, 0.555556) --
  (0.333333, 0.666667) --
  (0.222222, 0.666667) --
  (0.222222, 0.777778) --
  (0.111111, 0.777778) --
  (0.111111, 0.666667) --
  (0.000000, 0.666667) --
  (0.000000, 0.777778) --
  (0.111111, 0.777778) --
  (0.111111, 0.888889) --
  (0.000000, 0.888889) --
  (0.000000, 1.000000) --
  (0.111111, 1.000000) --
  (0.111111, 0.888889) --
  (0.222222, 0.888889) --
  (0.222222, 1.000000) --
  (0.333333, 1.000000) --
  (0.333333, 0.888889) --
  (0.222222, 0.888889) --
  (0.222222, 0.777778) --
  (0.333333, 0.777778) --
  (0.333333, 0.666667) --
  (0.444444, 0.666667) --
  (0.444444, 0.555556) --
  (0.555555, 0.555556) --
  (0.555555, 0.666667) --
  (0.666667, 0.666667) --
  (0.666667, 0.777778) --
  (0.777778, 0.777778) --
  (0.777778, 0.888889) --
  (0.666667, 0.888889) --
  (0.666667, 1.000000) --
  (0.777778, 1.000000) --
  (0.777778, 0.888889) --
  (0.888889, 0.888889) --
  (0.888889, 1.000000) --
  (1.000000, 1.000000) --
  (1.000000, 0.888889) --
  (0.888889, 0.888889) --
  (0.888889, 0.777778) --
  (1.000000, 0.777778) --
  (1.000000, 0.666667) --
  (0.888889, 0.666667) --
  (0.888889, 0.777778) --
  (0.777777, 0.777778) --
  (0.777777, 0.666667) --
  (0.666666, 0.666667) --
  (0.666666, 0.555556) --
  (0.555555, 0.555556) --
  (0.555555, 0.444445) --
  (0.666666, 0.444445) --
  (0.666666, 0.333334) --
  (0.777777, 0.333334) --
  (0.777777, 0.222223) --
  (0.888888, 0.222223) --
  (0.888888, 0.333334) --
  (0.999999, 0.333334) --
  (0.999999, 0.222223) --
  (0.888888, 0.222223) --
  (0.888888, 0.111111) --
  (0.999999, 0.111111) --
  (0.999999, 0.000000) --
  (0.888888, 0.000000) --
  (0.888888, 0.111111) --
  (0.777777, 0.111112) --
  (0.777777, 0.000000) --
  (0.666666, 0.000000) --
  (0.666666, 0.111112) --
  (0.777777, 0.111111) --
  (0.777777, 0.222223) --
  (0.666666, 0.222223) --
  (0.666666, 0.333334) --
  (0.555555, 0.333334) --
  (0.555555, 0.444445) --
  (0.444444, 0.444445) --
  (0.444444, 0.333334) --
  (0.333333, 0.333334) --
  (0.333333, 0.222223) --
  (0.222222, 0.222223) --
  (0.222222, 0.111112) --
  (0.333333, 0.111112) --
  (0.333333, 0.000000) --
  (0.222222, 0.000000) --
  (0.222222, 0.111112) --
  (0.111111, 0.111112) --
  (0.111111, 0.000001) --
  (0.000000, 0.000000);
\end{zzfrac}
%
\zzfracskip
%
\begin{zzfrac}{$n = 3$}
\draw[thin]
  (0.000000, 0.000000) --
  (0.000000, 0.037037) --
  (0.037037, 0.037037) --
  (0.037037, 0.074074) --
  (0.000000, 0.074074) --
  (0.000000, 0.111111) --
  (0.037037, 0.111111) --
  (0.037037, 0.074074) --
  (0.074074, 0.074074) --
  (0.074074, 0.111111) --
  (0.111111, 0.111111) --
  (0.111111, 0.148148) --
  (0.148148, 0.148148) --
  (0.148148, 0.185185) --
  (0.111111, 0.185185) --
  (0.111111, 0.222222) --
  (0.074074, 0.222222) --
  (0.074074, 0.259259) --
  (0.037037, 0.259259) --
  (0.037037, 0.222222) --
  (0.000000, 0.222222) --
  (0.000000, 0.259259) --
  (0.037037, 0.259259) --
  (0.037037, 0.296296) --
  (0.000000, 0.296296) --
  (0.000000, 0.333333) --
  (0.037037, 0.333333) --
  (0.037037, 0.296296) --
  (0.074074, 0.296296) --
  (0.074074, 0.333333) --
  (0.111111, 0.333333) --
  (0.111111, 0.296296) --
  (0.074074, 0.296296) --
  (0.074074, 0.259259) --
  (0.111111, 0.259259) --
  (0.111111, 0.222222) --
  (0.148148, 0.222222) --
  (0.148148, 0.185185) --
  (0.185185, 0.185185) --
  (0.185185, 0.222222) --
  (0.222222, 0.222222) --
  (0.222222, 0.259259) --
  (0.259259, 0.259259) --
  (0.259259, 0.296296) --
  (0.222222, 0.296296) --
  (0.222222, 0.333333) --
  (0.259259, 0.333333) --
  (0.259259, 0.296296) --
  (0.296296, 0.296296) --
  (0.296296, 0.333333) --
  (0.333333, 0.333333) --
  (0.333333, 0.370370) --
  (0.370370, 0.370370) --
  (0.370370, 0.407407) --
  (0.333333, 0.407407) --
  (0.333333, 0.444444) --
  (0.370370, 0.444444) --
  (0.370370, 0.407407) --
  (0.407407, 0.407407) --
  (0.407407, 0.444444) --
  (0.444445, 0.444444) --
  (0.444445, 0.481481) --
  (0.481482, 0.481481) --
  (0.481482, 0.518519) --
  (0.444445, 0.518519) --
  (0.444445, 0.555556) --
  (0.407407, 0.555556) --
  (0.407407, 0.592593) --
  (0.370370, 0.592593) --
  (0.370370, 0.555555) --
  (0.333333, 0.555556) --
  (0.333333, 0.592593) --
  (0.370370, 0.592593) --
  (0.370370, 0.629630) --
  (0.333333, 0.629630) --
  (0.333333, 0.666667) --
  (0.296296, 0.666667) --
  (0.296296, 0.703704) --
  (0.259259, 0.703704) --
  (0.259259, 0.666667) --
  (0.222222, 0.666667) --
  (0.222222, 0.703704) --
  (0.259259, 0.703704) --
  (0.259259, 0.740741) --
  (0.222222, 0.740741) --
  (0.222222, 0.777778) --
  (0.185185, 0.777778) --
  (0.185185, 0.814815) --
  (0.148148, 0.814815) --
  (0.148148, 0.777778) --
  (0.111111, 0.777778) --
  (0.111111, 0.740741) --
  (0.074074, 0.740741) --
  (0.074074, 0.703704) --
  (0.111111, 0.703704) --
  (0.111111, 0.666667) --
  (0.074074, 0.666667) --
  (0.074074, 0.703704) --
  (0.037037, 0.703704) --
  (0.037037, 0.666667) --
  (0.000000, 0.666667) --
  (0.000000, 0.703704) --
  (0.037037, 0.703704) --
  (0.037037, 0.740741) --
  (0.000000, 0.740741) --
  (0.000000, 0.777778) --
  (0.037037, 0.777778) --
  (0.037037, 0.740741) --
  (0.074074, 0.740741) --
  (0.074074, 0.777778) --
  (0.111111, 0.777778) --
  (0.111111, 0.814815) --
  (0.148148, 0.814815) --
  (0.148148, 0.851852) --
  (0.111111, 0.851852) --
  (0.111111, 0.888889) --
  (0.074074, 0.888889) --
  (0.074074, 0.925926) --
  (0.037037, 0.925926) --
  (0.037037, 0.888889) --
  (0.000000, 0.888889) --
  (0.000000, 0.925926) --
  (0.037037, 0.925926) --
  (0.037037, 0.962963) --
  (0.000000, 0.962963) --
  (0.000000, 1.000000) --
  (0.037037, 1.000000) --
  (0.037037, 0.962963) --
  (0.074074, 0.962963) --
  (0.074074, 1.000000) --
  (0.111111, 1.000000) --
  (0.111111, 0.962963) --
  (0.074074, 0.962963) --
  (0.074074, 0.925926) --
  (0.111111, 0.925926) --
  (0.111111, 0.888889) --
  (0.148148, 0.888889) --
  (0.148148, 0.851852) --
  (0.185185, 0.851852) --
  (0.185185, 0.888889) --
  (0.222222, 0.888889) --
  (0.222222, 0.925926) --
  (0.259259, 0.925926) --
  (0.259259, 0.962963) --
  (0.222222, 0.962963) --
  (0.222222, 1.000000) --
  (0.259259, 1.000000) --
  (0.259259, 0.962963) --
  (0.296296, 0.962963) --
  (0.296296, 1.000000) --
  (0.333333, 1.000000) --
  (0.333333, 0.962963) --
  (0.296296, 0.962963) --
  (0.296296, 0.925926) --
  (0.333333, 0.925926) --
  (0.333333, 0.888889) --
  (0.296296, 0.888889) --
  (0.296296, 0.925926) --
  (0.259259, 0.925926) --
  (0.259259, 0.888889) --
  (0.222222, 0.888889) --
  (0.222222, 0.851852) --
  (0.185185, 0.851852) --
  (0.185185, 0.814815) --
  (0.222222, 0.814815) --
  (0.222222, 0.777778) --
  (0.259259, 0.777778) --
  (0.259259, 0.740741) --
  (0.296296, 0.740741) --
  (0.296296, 0.777778) --
  (0.333333, 0.777778) --
  (0.333333, 0.740741) --
  (0.296296, 0.740741) --
  (0.296296, 0.703704) --
  (0.333333, 0.703704) --
  (0.333333, 0.666667) --
  (0.370370, 0.666667) --
  (0.370370, 0.629630) --
  (0.407407, 0.629630) --
  (0.407407, 0.666667) --
  (0.444445, 0.666667) --
  (0.444445, 0.629630) --
  (0.407407, 0.629630) --
  (0.407407, 0.592593) --
  (0.444445, 0.592593) --
  (0.444445, 0.555555) --
  (0.481482, 0.555556) --
  (0.481482, 0.518518) --
  (0.518519, 0.518518) --
  (0.518519, 0.555556) --
  (0.555556, 0.555556) --
  (0.555556, 0.592593) --
  (0.592593, 0.592593) --
  (0.592593, 0.629630) --
  (0.555556, 0.629630) --
  (0.555556, 0.666667) --
  (0.592593, 0.666667) --
  (0.592593, 0.629630) --
  (0.629630, 0.629630) --
  (0.629630, 0.666667) --
  (0.666667, 0.666667) --
  (0.666667, 0.703704) --
  (0.703704, 0.703704) --
  (0.703704, 0.740741) --
  (0.666667, 0.740741) --
  (0.666667, 0.777778) --
  (0.703704, 0.777778) --
  (0.703704, 0.740741) --
  (0.740741, 0.740741) --
  (0.740741, 0.777778) --
  (0.777778, 0.777778) --
  (0.777778, 0.814815) --
  (0.814815, 0.814815) --
  (0.814815, 0.851852) --
  (0.777778, 0.851852) --
  (0.777778, 0.888889) --
  (0.740741, 0.888889) --
  (0.740741, 0.925926) --
  (0.703704, 0.925926) --
  (0.703704, 0.888889) --
  (0.666667, 0.888889) --
  (0.666667, 0.925926) --
  (0.703704, 0.925926) --
  (0.703704, 0.962963) --
  (0.666667, 0.962963) --
  (0.666667, 1.000000) --
  (0.703704, 1.000000) --
  (0.703704, 0.962963) --
  (0.740741, 0.962963) --
  (0.740741, 1.000000) --
  (0.777778, 1.000000) --
  (0.777778, 0.962963) --
  (0.740741, 0.962963) --
  (0.740741, 0.925926) --
  (0.777778, 0.925926) --
  (0.777778, 0.888889) --
  (0.814815, 0.888889) --
  (0.814815, 0.851852) --
  (0.851852, 0.851852) --
  (0.851852, 0.888889) --
  (0.888889, 0.888889) --
  (0.888889, 0.925926) --
  (0.925926, 0.925926) --
  (0.925926, 0.962963) --
  (0.888889, 0.962963) --
  (0.888889, 1.000000) --
  (0.925926, 1.000000) --
  (0.925926, 0.962963) --
  (0.962963, 0.962963) --
  (0.962963, 1.000000) --
  (1.000000, 1.000000) --
  (1.000000, 0.962963) --
  (0.962963, 0.962963) --
  (0.962963, 0.925926) --
  (1.000000, 0.925926) --
  (1.000000, 0.888889) --
  (0.962963, 0.888889) --
  (0.962963, 0.925926) --
  (0.925926, 0.925926) --
  (0.925926, 0.888889) --
  (0.888889, 0.888889) --
  (0.888889, 0.851852) --
  (0.851852, 0.851852) --
  (0.851852, 0.814815) --
  (0.888889, 0.814815) --
  (0.888889, 0.777778) --
  (0.925926, 0.777778) --
  (0.925926, 0.740741) --
  (0.962963, 0.740741) --
  (0.962963, 0.777778) --
  (1.000000, 0.777778) --
  (1.000000, 0.740741) --
  (0.962963, 0.740741) --
  (0.962963, 0.703704) --
  (1.000000, 0.703704) --
  (1.000000, 0.666667) --
  (0.962963, 0.666667) --
  (0.962963, 0.703704) --
  (0.925926, 0.703704) --
  (0.925926, 0.666667) --
  (0.888889, 0.666667) --
  (0.888889, 0.703704) --
  (0.925926, 0.703704) --
  (0.925926, 0.740741) --
  (0.888889, 0.740741) --
  (0.888889, 0.777778) --
  (0.851852, 0.777778) --
  (0.851852, 0.814815) --
  (0.814815, 0.814815) --
  (0.814815, 0.777778) --
  (0.777778, 0.777778) --
  (0.777778, 0.740741) --
  (0.740741, 0.740741) --
  (0.740741, 0.703704) --
  (0.777778, 0.703704) --
  (0.777778, 0.666667) --
  (0.740741, 0.666667) --
  (0.740741, 0.703704) --
  (0.703704, 0.703704) --
  (0.703704, 0.666667) --
  (0.666667, 0.666667) --
  (0.666667, 0.629630) --
  (0.629630, 0.629630) --
  (0.629630, 0.592593) --
  (0.666667, 0.592593) --
  (0.666667, 0.555555) --
  (0.629630, 0.555556) --
  (0.629630, 0.592593) --
  (0.592593, 0.592593) --
  (0.592593, 0.555555) --
  (0.555556, 0.555556) --
  (0.555556, 0.518518) --
  (0.518519, 0.518519) --
  (0.518519, 0.481481) --
  (0.555556, 0.481481) --
  (0.555556, 0.444444) --
  (0.592593, 0.444444) --
  (0.592593, 0.407407) --
  (0.629630, 0.407407) --
  (0.629630, 0.444444) --
  (0.666667, 0.444444) --
  (0.666667, 0.407407) --
  (0.629630, 0.407407) --
  (0.629630, 0.370370) --
  (0.666667, 0.370370) --
  (0.666667, 0.333333) --
  (0.703704, 0.333333) --
  (0.703704, 0.296296) --
  (0.740741, 0.296296) --
  (0.740741, 0.333333) --
  (0.777778, 0.333333) --
  (0.777778, 0.296296) --
  (0.740741, 0.296296) --
  (0.740741, 0.259259) --
  (0.777778, 0.259259) --
  (0.777778, 0.222222) --
  (0.814815, 0.222222) --
  (0.814815, 0.185185) --
  (0.851852, 0.185185) --
  (0.851852, 0.222222) --
  (0.888889, 0.222222) --
  (0.888889, 0.259259) --
  (0.925926, 0.259259) --
  (0.925926, 0.296296) --
  (0.888889, 0.296296) --
  (0.888889, 0.333333) --
  (0.925926, 0.333333) --
  (0.925926, 0.296296) --
  (0.962963, 0.296296) --
  (0.962963, 0.333333) --
  (1.000000, 0.333333) --
  (1.000000, 0.296296) --
  (0.962963, 0.296296) --
  (0.962963, 0.259259) --
  (1.000000, 0.259259) --
  (1.000000, 0.222222) --
  (0.962963, 0.222222) --
  (0.962963, 0.259259) --
  (0.925926, 0.259259) --
  (0.925926, 0.222222) --
  (0.888889, 0.222222) --
  (0.888889, 0.185185) --
  (0.851852, 0.185185) --
  (0.851852, 0.148148) --
  (0.888889, 0.148148) --
  (0.888889, 0.111111) --
  (0.925926, 0.111111) --
  (0.925926, 0.074074) --
  (0.962963, 0.074074) --
  (0.962963, 0.111111) --
  (1.000000, 0.111111) --
  (1.000000, 0.074074) --
  (0.962963, 0.074074) --
  (0.962963, 0.037037) --
  (1.000000, 0.037037) --
  (1.000000, 0.000000) --
  (0.962963, 0.000000) --
  (0.962963, 0.037037) --
  (0.925926, 0.037037) --
  (0.925926, 0.000000) --
  (0.888889, 0.000000) --
  (0.888889, 0.037037) --
  (0.925926, 0.037037) --
  (0.925926, 0.074074) --
  (0.888889, 0.074074) --
  (0.888889, 0.111111) --
  (0.851852, 0.111111) --
  (0.851852, 0.148148) --
  (0.814815, 0.148148) --
  (0.814815, 0.111111) --
  (0.777778, 0.111111) --
  (0.777778, 0.074074) --
  (0.740741, 0.074074) --
  (0.740741, 0.037037) --
  (0.777778, 0.037037) --
  (0.777778, 0.000000) --
  (0.740741, 0.000000) --
  (0.740741, 0.037037) --
  (0.703704, 0.037037) --
  (0.703704, 0.000000) --
  (0.666667, 0.000000) --
  (0.666667, 0.037037) --
  (0.703704, 0.037037) --
  (0.703704, 0.074074) --
  (0.666667, 0.074074) --
  (0.666667, 0.111111) --
  (0.703704, 0.111111) --
  (0.703704, 0.074074) --
  (0.740741, 0.074074) --
  (0.740741, 0.111111) --
  (0.777778, 0.111111) --
  (0.777778, 0.148148) --
  (0.814815, 0.148148) --
  (0.814815, 0.185185) --
  (0.777778, 0.185185) --
  (0.777778, 0.222222) --
  (0.740741, 0.222222) --
  (0.740741, 0.259259) --
  (0.703704, 0.259259) --
  (0.703704, 0.222222) --
  (0.666667, 0.222222) --
  (0.666667, 0.259259) --
  (0.703704, 0.259259) --
  (0.703704, 0.296296) --
  (0.666667, 0.296296) --
  (0.666667, 0.333333) --
  (0.629630, 0.333333) --
  (0.629630, 0.370370) --
  (0.592593, 0.370370) --
  (0.592593, 0.333333) --
  (0.555556, 0.333333) --
  (0.555556, 0.370370) --
  (0.592593, 0.370370) --
  (0.592593, 0.407407) --
  (0.555556, 0.407407) --
  (0.555556, 0.444444) --
  (0.518519, 0.444444) --
  (0.518519, 0.481481) --
  (0.481481, 0.481481) --
  (0.481481, 0.444444) --
  (0.444444, 0.444444) --
  (0.444444, 0.407407) --
  (0.407407, 0.407407) --
  (0.407407, 0.370370) --
  (0.444444, 0.370370) --
  (0.444444, 0.333333) --
  (0.407407, 0.333333) --
  (0.407407, 0.370370) --
  (0.370370, 0.370370) --
  (0.370370, 0.333333) --
  (0.333333, 0.333333) --
  (0.333333, 0.296296) --
  (0.296296, 0.296296) --
  (0.296296, 0.259259) --
  (0.333333, 0.259259) --
  (0.333333, 0.222222) --
  (0.296296, 0.222222) --
  (0.296296, 0.259259) --
  (0.259259, 0.259259) --
  (0.259259, 0.222222) --
  (0.222222, 0.222222) --
  (0.222222, 0.185185) --
  (0.185185, 0.185185) --
  (0.185185, 0.148148) --
  (0.222222, 0.148148) --
  (0.222222, 0.111111) --
  (0.259259, 0.111111) --
  (0.259259, 0.074074) --
  (0.296296, 0.074074) --
  (0.296296, 0.111111) --
  (0.333333, 0.111111) --
  (0.333333, 0.074074) --
  (0.296296, 0.074074) --
  (0.296296, 0.037037) --
  (0.333333, 0.037037) --
  (0.333333, 0.000000) --
  (0.296296, 0.000000) --
  (0.296296, 0.037037) --
  (0.259259, 0.037037) --
  (0.259259, 0.000000) --
  (0.222222, 0.000000) --
  (0.222222, 0.037037) --
  (0.259259, 0.037037) --
  (0.259259, 0.074074) --
  (0.222222, 0.074074) --
  (0.222222, 0.111111) --
  (0.185185, 0.111111) --
  (0.185185, 0.148148) --
  (0.148148, 0.148148) --
  (0.148148, 0.111111) --
  (0.111111, 0.111111) --
  (0.111111, 0.074074) --
  (0.074074, 0.074074) --
  (0.074074, 0.037037) --
  (0.111111, 0.037037) --
  (0.111111, 0.000000) --
  (0.074074, 0.000000) --
  (0.074074, 0.037037) --
  (0.037037, 0.037037) --
  (0.037037, 0.000000) --
  (0.000000, 0.000000);
\end{zzfrac}
%
\zzfracskip\dots
%

\end{zztask}

%%%%%%%%%%%%%%%%%%%%%%%%%%%%%%%%%%%%%%%%%%%%%%%%%%%%%%%%%%%%%%%%%%%%%%%%%%%%%%

\begin{zztask}[Fudge{f}lake]
В рамках общего условия задачи построить кривую снежинку. База для построения
кривой (нулевое приближение) представляет собой равносторонний треугольник.
Переход от приближения
$n$ к приближению $(n+1)$ осуществляется заменой каждого отрезка фигуры на два
(см. рис. при $n=1$, выделено жирным). Все отрезки
имеют одинаковую длину, угол между отрезками \mbox{составляет $120^\circ$}.
В дальнейшем, выгиб происходит чередуясь, то в одну, то в другую сторону 
(см. рис. при $n=2$).
\par\endinput
\begin{pspicture}(-0.5,-0.5)(1.5,1.5)
\psgrid
\rput(0.5,-0.25){$n=0$}
\rput(0,0.21){
\psline
  (0, 0)
  (1.000000, 0.000000)
  (0.500000, 0.866025)
  (-0.000000, 0.000000)
\psline[linewidth=0.06cm, arrows=o-o]
  (0.500000, 0.866025)
  (-0.000000, 0.000000)
}
\end{pspicture}
%
\hskip0.1in
%
\begin{pspicture}(-0.5,-0.5)(1.5,1.5)
\psgrid
\rput(0.5,-0.25){$n=1$}
\rput(0,0.21){
\psline[linewidth=0.01cm, linestyle=dashed]
  (0, 0)
  (1.000000, 0.000000)
  (0.500000, 0.866025)
  (-0.000000, 0.000000)
\psline
  (0, 0)
  (0.500000, -0.288675)
  (1.000000, 0.000000)
  (1.000000, 0.577350)
  (0.500000, 0.866025)
  (-0.000000, 0.577350)
  (-0.000000, 0.000000)
\psline[linewidth=0.06cm, arrows=o-o, showpoints=true]
  (0.500000, 0.866025)
  (-0.000000, 0.577350)
  (-0.000000, 0.000000)
}
\end{pspicture}
%
\hskip0.1in
%
\begin{pspicture}(-0.5,-0.5)(1.5,1.5)
\psgrid
\rput(0.5,-0.25){$n=2$}
\rput(0,0.21){
\psline[linewidth=0.01cm, linestyle=dashed]
  (0, 0)
  (0.500000, -0.288675)
  (1.000000, 0.000000)
  (1.000000, 0.577350)
  (0.500000, 0.866025)
  (-0.000000, 0.577350)
  (-0.000000, 0.000000)
\psline
  (0, 0)
  (0.333333, 0.000000)
  (0.500000, -0.288675)
  (0.833333, -0.288675)
  (1.000000, -0.000000)
  (0.833333, 0.288675)
  (1.000000, 0.577350)
  (0.833333, 0.866025)
  (0.500000, 0.866025)
  (0.333333, 0.577350)
  (-0.000000, 0.577350)
  (-0.166667, 0.288675)
  (-0.000001, 0.000000)
\psline[linewidth=0.06cm, arrows=o-o, showpoints=true]
  (0.500000, 0.866025)
  (0.333333, 0.577350)
  (-0.000000, 0.577350)
\psline[linewidth=0.06cm, arrows=o-o, showpoints=true]
  (-0.000000, 0.577350)
  (-0.166667, 0.288675)
  (-0.000001, 0.000000)
}
\end{pspicture}
%
\hskip0.1in
%
\begin{pspicture}(-0.5,-0.5)(1.5,1.5)
\psgrid
\rput(0.5,-0.25){$n=3$}
\rput(0,0.21){
\psline
  (0, 0)
  (0.166667, 0.096225)
  (0.333333, -0.000000)
  (0.333333, -0.192450)
  (0.500000, -0.288675)
  (0.666667, -0.192450)
  (0.833333, -0.288675)
  (1.000000, -0.192450)
  (1.000000, -0.000000)
  (0.833333, 0.096225)
  (0.833333, 0.288675)
  (1.000000, 0.384900)
  (1.000000, 0.577350)
  (0.833333, 0.673575)
  (0.833333, 0.866025)
  (0.666667, 0.962250)
  (0.500000, 0.866025)
  (0.500000, 0.673575)
  (0.333333, 0.577350)
  (0.166667, 0.673575)
  (-0.000000, 0.577350)
  (-0.000000, 0.384900)
  (-0.166667, 0.288675)
  (-0.166667, 0.096225)
  (-0.000000, -0.000000)
}
\end{pspicture}
%
\hskip0.1in\dots\hskip0.1in
%
\begin{pspicture}(-0.5,-0.5)(1.5,1.5)
\psgrid
\rput(0.5,-0.25){$n=8$}
\rput(0,0.21){
\psline
  (0, 0)
  (-0.012346, 0.000000)
  (-0.018519, 0.010692)
  (-0.012346, 0.021383)
  (-0.018519, 0.032075)
  (-0.012346, 0.042767)
  (-0.000000, 0.042767)
  (0.006173, 0.053458)
  (-0.000000, 0.064150)
  (0.006173, 0.074842)
  (0.018519, 0.074842)
  (0.024691, 0.064150)
  (0.037037, 0.064150)
  (0.043210, 0.074842)
  (0.055556, 0.074842)
  (0.061728, 0.085533)
  (0.055556, 0.096225)
  (0.061728, 0.106917)
  (0.074074, 0.106917)
  (0.080247, 0.096225)
  (0.092593, 0.096225)
  (0.098765, 0.085533)
  (0.092593, 0.074842)
  (0.098765, 0.064150)
  (0.111111, 0.064150)
  (0.117284, 0.074842)
  (0.129630, 0.074842)
  (0.135802, 0.064150)
  (0.148148, 0.064150)
  (0.154321, 0.074842)
  (0.166667, 0.074842)
  (0.172839, 0.085533)
  (0.166667, 0.096225)
  (0.172839, 0.106917)
  (0.185185, 0.106917)
  (0.191358, 0.096225)
  (0.203704, 0.096225)
  (0.209876, 0.085533)
  (0.203704, 0.074842)
  (0.209876, 0.064150)
  (0.222222, 0.064150)
  (0.228395, 0.053458)
  (0.222222, 0.042767)
  (0.209876, 0.042767)
  (0.203704, 0.032075)
  (0.209876, 0.021383)
  (0.203704, 0.010692)
  (0.209876, -0.000000)
  (0.222222, -0.000000)
  (0.228395, 0.010692)
  (0.240741, 0.010692)
  (0.246913, -0.000000)
  (0.259259, -0.000000)
  (0.265432, -0.010692)
  (0.259259, -0.021383)
  (0.265432, -0.032075)
  (0.277778, -0.032075)
  (0.283951, -0.021383)
  (0.296296, -0.021383)
  (0.302469, -0.032075)
  (0.314815, -0.032075)
  (0.320988, -0.021383)
  (0.333333, -0.021383)
  (0.339506, -0.010692)
  (0.333333, 0.000000)
  (0.339506, 0.010692)
  (0.351852, 0.010692)
  (0.358025, 0.000000)
  (0.370370, 0.000000)
  (0.376543, -0.010692)
  (0.370370, -0.021383)
  (0.376543, -0.032075)
  (0.388889, -0.032075)
  (0.395062, -0.042767)
  (0.388889, -0.053458)
  (0.376543, -0.053458)
  (0.370370, -0.064150)
  (0.376543, -0.074842)
  (0.370370, -0.085533)
  (0.376543, -0.096225)
  (0.388889, -0.096225)
  (0.395062, -0.106917)
  (0.388889, -0.117608)
  (0.376543, -0.117608)
  (0.370370, -0.128300)
  (0.358025, -0.128300)
  (0.351852, -0.117608)
  (0.339506, -0.117608)
  (0.333333, -0.128300)
  (0.339506, -0.138992)
  (0.333333, -0.149683)
  (0.320988, -0.149683)
  (0.314815, -0.160375)
  (0.320988, -0.171067)
  (0.314815, -0.181758)
  (0.320988, -0.192450)
  (0.333333, -0.192450)
  (0.339506, -0.181758)
  (0.351852, -0.181758)
  (0.358025, -0.192450)
  (0.370370, -0.192450)
  (0.376543, -0.203142)
  (0.370370, -0.213833)
  (0.376543, -0.224525)
  (0.388889, -0.224525)
  (0.395062, -0.235217)
  (0.388889, -0.245908)
  (0.376543, -0.245908)
  (0.370370, -0.256600)
  (0.376543, -0.267292)
  (0.370370, -0.277983)
  (0.376543, -0.288675)
  (0.388889, -0.288675)
  (0.395062, -0.277983)
  (0.407407, -0.277983)
  (0.413580, -0.288675)
  (0.425926, -0.288675)
  (0.432099, -0.299367)
  (0.425926, -0.310058)
  (0.432099, -0.320750)
  (0.444444, -0.320750)
  (0.450617, -0.310058)
  (0.462963, -0.310058)
  (0.469136, -0.320750)
  (0.481481, -0.320750)
  (0.487654, -0.310058)
  (0.500000, -0.310058)
  (0.506173, -0.299367)
  (0.500000, -0.288675)
  (0.487654, -0.288675)
  (0.481481, -0.277983)
  (0.487654, -0.267292)
  (0.481481, -0.256600)
  (0.487654, -0.245908)
  (0.500000, -0.245908)
  (0.506173, -0.235217)
  (0.500000, -0.224525)
  (0.506173, -0.213833)
  (0.518518, -0.213833)
  (0.524691, -0.224525)
  (0.537037, -0.224525)
  (0.543210, -0.213833)
  (0.555555, -0.213833)
  (0.561728, -0.203142)
  (0.555555, -0.192450)
  (0.561728, -0.181758)
  (0.574074, -0.181758)
  (0.580247, -0.192450)
  (0.592592, -0.192450)
  (0.598765, -0.203142)
  (0.592592, -0.213833)
  (0.598765, -0.224525)
  (0.611111, -0.224525)
  (0.617284, -0.213833)
  (0.629629, -0.213833)
  (0.635802, -0.224525)
  (0.648148, -0.224525)
  (0.654321, -0.213833)
  (0.666666, -0.213833)
  (0.672839, -0.203142)
  (0.666666, -0.192450)
  (0.672839, -0.181758)
  (0.685185, -0.181758)
  (0.691358, -0.192450)
  (0.703703, -0.192450)
  (0.709876, -0.203142)
  (0.703703, -0.213833)
  (0.709876, -0.224525)
  (0.722222, -0.224525)
  (0.728395, -0.235217)
  (0.722222, -0.245908)
  (0.709876, -0.245908)
  (0.703703, -0.256600)
  (0.709876, -0.267292)
  (0.703703, -0.277983)
  (0.709876, -0.288675)
  (0.722222, -0.288675)
  (0.728395, -0.277983)
  (0.740740, -0.277983)
  (0.746913, -0.288675)
  (0.759259, -0.288675)
  (0.765432, -0.299367)
  (0.759259, -0.310058)
  (0.765432, -0.320750)
  (0.777777, -0.320750)
  (0.783950, -0.310058)
  (0.796296, -0.310058)
  (0.802469, -0.320750)
  (0.814814, -0.320750)
  (0.820987, -0.310058)
  (0.833333, -0.310058)
  (0.839506, -0.299367)
  (0.833333, -0.288675)
  (0.820987, -0.288675)
  (0.814814, -0.277983)
  (0.820987, -0.267292)
  (0.814814, -0.256600)
  (0.820987, -0.245908)
  (0.833333, -0.245908)
  (0.839506, -0.235217)
  (0.833333, -0.224525)
  (0.839506, -0.213833)
  (0.851851, -0.213833)
  (0.858024, -0.224525)
  (0.870370, -0.224525)
  (0.876543, -0.213833)
  (0.888888, -0.213833)
  (0.895061, -0.203142)
  (0.888888, -0.192450)
  (0.895061, -0.181758)
  (0.907407, -0.181758)
  (0.913580, -0.192450)
  (0.925925, -0.192450)
  (0.932098, -0.203142)
  (0.925925, -0.213833)
  (0.932098, -0.224525)
  (0.944444, -0.224525)
  (0.950617, -0.213833)
  (0.962962, -0.213833)
  (0.969135, -0.224525)
  (0.981481, -0.224525)
  (0.987654, -0.213833)
  (0.999999, -0.213833)
  (1.006172, -0.203142)
  (0.999999, -0.192450)
  (0.987654, -0.192450)
  (0.981481, -0.181758)
  (0.987654, -0.171067)
  (0.981481, -0.160375)
  (0.987654, -0.149683)
  (0.999999, -0.149683)
  (1.006172, -0.138992)
  (0.999999, -0.128300)
  (1.006172, -0.117608)
  (1.018518, -0.117608)
  (1.024691, -0.128300)
  (1.037037, -0.128300)
  (1.043209, -0.117608)
  (1.055555, -0.117608)
  (1.061728, -0.106917)
  (1.055555, -0.096225)
  (1.043209, -0.096225)
  (1.037037, -0.085533)
  (1.043209, -0.074842)
  (1.037037, -0.064150)
  (1.043209, -0.053458)
  (1.055555, -0.053458)
  (1.061728, -0.042767)
  (1.055555, -0.032075)
  (1.043209, -0.032075)
  (1.037037, -0.021383)
  (1.043209, -0.010692)
  (1.037037, 0.000000)
  (1.024691, 0.000000)
  (1.018518, 0.010692)
  (1.006172, 0.010692)
  (0.999999, 0.000000)
  (1.006172, -0.010692)
  (0.999999, -0.021383)
  (0.987654, -0.021383)
  (0.981481, -0.032075)
  (0.969135, -0.032075)
  (0.962962, -0.021383)
  (0.950617, -0.021383)
  (0.944444, -0.032075)
  (0.932098, -0.032075)
  (0.925925, -0.021383)
  (0.932098, -0.010692)
  (0.925925, 0.000000)
  (0.913580, 0.000000)
  (0.907407, 0.010692)
  (0.895061, 0.010692)
  (0.888888, 0.000000)
  (0.876543, 0.000000)
  (0.870370, 0.010692)
  (0.876543, 0.021383)
  (0.870370, 0.032075)
  (0.876543, 0.042767)
  (0.888888, 0.042767)
  (0.895061, 0.053458)
  (0.888888, 0.064150)
  (0.876543, 0.064150)
  (0.870370, 0.074842)
  (0.876543, 0.085533)
  (0.870370, 0.096225)
  (0.858024, 0.096225)
  (0.851851, 0.106917)
  (0.839506, 0.106917)
  (0.833333, 0.096225)
  (0.820987, 0.096225)
  (0.814814, 0.106917)
  (0.820987, 0.117608)
  (0.814814, 0.128300)
  (0.820987, 0.138992)
  (0.833333, 0.138992)
  (0.839506, 0.149683)
  (0.833333, 0.160375)
  (0.839506, 0.171067)
  (0.851851, 0.171067)
  (0.858024, 0.160375)
  (0.870370, 0.160375)
  (0.876543, 0.171067)
  (0.888888, 0.171067)
  (0.895061, 0.181758)
  (0.888888, 0.192450)
  (0.876543, 0.192450)
  (0.870370, 0.203142)
  (0.876543, 0.213833)
  (0.870370, 0.224525)
  (0.876543, 0.235217)
  (0.888888, 0.235217)
  (0.895061, 0.245908)
  (0.888888, 0.256600)
  (0.876543, 0.256600)
  (0.870370, 0.267292)
  (0.876543, 0.277984)
  (0.870370, 0.288675)
  (0.858024, 0.288675)
  (0.851851, 0.299367)
  (0.839506, 0.299367)
  (0.833333, 0.288675)
  (0.820987, 0.288675)
  (0.814814, 0.299367)
  (0.820987, 0.310059)
  (0.814814, 0.320750)
  (0.820987, 0.331442)
  (0.833333, 0.331442)
  (0.839506, 0.342134)
  (0.833333, 0.352825)
  (0.839506, 0.363517)
  (0.851851, 0.363517)
  (0.858024, 0.352825)
  (0.870370, 0.352825)
  (0.876543, 0.363517)
  (0.888888, 0.363517)
  (0.895061, 0.374209)
  (0.888888, 0.384900)
  (0.895061, 0.395592)
  (0.907407, 0.395592)
  (0.913580, 0.384900)
  (0.925925, 0.384900)
  (0.932098, 0.374209)
  (0.925925, 0.363517)
  (0.932098, 0.352825)
  (0.944444, 0.352825)
  (0.950617, 0.363517)
  (0.962962, 0.363517)
  (0.969135, 0.352825)
  (0.981481, 0.352825)
  (0.987654, 0.363517)
  (0.999999, 0.363517)
  (1.006172, 0.374209)
  (0.999999, 0.384900)
  (0.987654, 0.384900)
  (0.981481, 0.395592)
  (0.987654, 0.406284)
  (0.981481, 0.416975)
  (0.987654, 0.427667)
  (0.999999, 0.427667)
  (1.006172, 0.438359)
  (0.999999, 0.449050)
  (1.006172, 0.459742)
  (1.018518, 0.459742)
  (1.024691, 0.449050)
  (1.037037, 0.449050)
  (1.043209, 0.459742)
  (1.055555, 0.459742)
  (1.061728, 0.470434)
  (1.055555, 0.481125)
  (1.043209, 0.481125)
  (1.037037, 0.491817)
  (1.043209, 0.502509)
  (1.037037, 0.513200)
  (1.043209, 0.523892)
  (1.055555, 0.523892)
  (1.061728, 0.534584)
  (1.055555, 0.545275)
  (1.043209, 0.545275)
  (1.037037, 0.555967)
  (1.043209, 0.566659)
  (1.037037, 0.577350)
  (1.024691, 0.577350)
  (1.018518, 0.588042)
  (1.006172, 0.588042)
  (0.999999, 0.577350)
  (1.006172, 0.566658)
  (0.999999, 0.555967)
  (0.987654, 0.555967)
  (0.981481, 0.545275)
  (0.969135, 0.545275)
  (0.962962, 0.555967)
  (0.950617, 0.555967)
  (0.944444, 0.545275)
  (0.932098, 0.545275)
  (0.925925, 0.555967)
  (0.932098, 0.566658)
  (0.925925, 0.577350)
  (0.913580, 0.577350)
  (0.907407, 0.588042)
  (0.895061, 0.588042)
  (0.888888, 0.577350)
  (0.876543, 0.577350)
  (0.870370, 0.588042)
  (0.876543, 0.598733)
  (0.870370, 0.609425)
  (0.876543, 0.620117)
  (0.888888, 0.620117)
  (0.895061, 0.630808)
  (0.888888, 0.641500)
  (0.876543, 0.641500)
  (0.870370, 0.652192)
  (0.876543, 0.662883)
  (0.870370, 0.673575)
  (0.858024, 0.673575)
  (0.851851, 0.684267)
  (0.839506, 0.684267)
  (0.833333, 0.673575)
  (0.820987, 0.673575)
  (0.814814, 0.684267)
  (0.820987, 0.694958)
  (0.814814, 0.705650)
  (0.820987, 0.716341)
  (0.833333, 0.716341)
  (0.839506, 0.727033)
  (0.833333, 0.737725)
  (0.839506, 0.748416)
  (0.851851, 0.748416)
  (0.858024, 0.737725)
  (0.870370, 0.737725)
  (0.876543, 0.748416)
  (0.888888, 0.748416)
  (0.895061, 0.759108)
  (0.888888, 0.769800)
  (0.876543, 0.769800)
  (0.870370, 0.780491)
  (0.876543, 0.791183)
  (0.870370, 0.801875)
  (0.876543, 0.812566)
  (0.888888, 0.812566)
  (0.895061, 0.823258)
  (0.888888, 0.833950)
  (0.876543, 0.833950)
  (0.870370, 0.844641)
  (0.876543, 0.855333)
  (0.870370, 0.866024)
  (0.858024, 0.866024)
  (0.851851, 0.876716)
  (0.839506, 0.876716)
  (0.833333, 0.866024)
  (0.839506, 0.855333)
  (0.833333, 0.844641)
  (0.820987, 0.844641)
  (0.814814, 0.833949)
  (0.802469, 0.833949)
  (0.796296, 0.844641)
  (0.783950, 0.844641)
  (0.777777, 0.833949)
  (0.765432, 0.833949)
  (0.759259, 0.844641)
  (0.765432, 0.855333)
  (0.759259, 0.866024)
  (0.746913, 0.866024)
  (0.740740, 0.876716)
  (0.728395, 0.876716)
  (0.722222, 0.866024)
  (0.709876, 0.866024)
  (0.703703, 0.876716)
  (0.709876, 0.887408)
  (0.703703, 0.898099)
  (0.709876, 0.908791)
  (0.722222, 0.908791)
  (0.728395, 0.919483)
  (0.722222, 0.930174)
  (0.709876, 0.930174)
  (0.703703, 0.940866)
  (0.709876, 0.951558)
  (0.703703, 0.962249)
  (0.691358, 0.962249)
  (0.685185, 0.972941)
  (0.672839, 0.972941)
  (0.666666, 0.962249)
  (0.672839, 0.951557)
  (0.666666, 0.940866)
  (0.654321, 0.940866)
  (0.648148, 0.930174)
  (0.635802, 0.930174)
  (0.629629, 0.940866)
  (0.617284, 0.940866)
  (0.611111, 0.930174)
  (0.598765, 0.930174)
  (0.592592, 0.940866)
  (0.598765, 0.951557)
  (0.592592, 0.962249)
  (0.580247, 0.962249)
  (0.574074, 0.972941)
  (0.561728, 0.972941)
  (0.555555, 0.962249)
  (0.561728, 0.951557)
  (0.555555, 0.940866)
  (0.543210, 0.940866)
  (0.537037, 0.930174)
  (0.524691, 0.930174)
  (0.518518, 0.940866)
  (0.506173, 0.940866)
  (0.500000, 0.930174)
  (0.506173, 0.919482)
  (0.500000, 0.908791)
  (0.487654, 0.908791)
  (0.481481, 0.898099)
  (0.487654, 0.887407)
  (0.481481, 0.876716)
  (0.487654, 0.866024)
  (0.500000, 0.866024)
  (0.506173, 0.876716)
  (0.518518, 0.876716)
  (0.524691, 0.866024)
  (0.537037, 0.866024)
  (0.543210, 0.855332)
  (0.537037, 0.844641)
  (0.543210, 0.833949)
  (0.555555, 0.833949)
  (0.561728, 0.823257)
  (0.555555, 0.812566)
  (0.543210, 0.812566)
  (0.537037, 0.801874)
  (0.543210, 0.791182)
  (0.537037, 0.780491)
  (0.543210, 0.769799)
  (0.555555, 0.769799)
  (0.561728, 0.759107)
  (0.555555, 0.748416)
  (0.543210, 0.748416)
  (0.537037, 0.737724)
  (0.524691, 0.737724)
  (0.518518, 0.748416)
  (0.506173, 0.748416)
  (0.500000, 0.737724)
  (0.506173, 0.727032)
  (0.500000, 0.716341)
  (0.487654, 0.716341)
  (0.481481, 0.705649)
  (0.487654, 0.694957)
  (0.481481, 0.684266)
  (0.487654, 0.673574)
  (0.500000, 0.673574)
  (0.506173, 0.662882)
  (0.500000, 0.652191)
  (0.487654, 0.652191)
  (0.481481, 0.641499)
  (0.469136, 0.641499)
  (0.462963, 0.652191)
  (0.450617, 0.652191)
  (0.444444, 0.641499)
  (0.432099, 0.641499)
  (0.425926, 0.652191)
  (0.432099, 0.662882)
  (0.425926, 0.673574)
  (0.413580, 0.673574)
  (0.407407, 0.684266)
  (0.395062, 0.684265)
  (0.388889, 0.673574)
  (0.395062, 0.662882)
  (0.388889, 0.652190)
  (0.376543, 0.652191)
  (0.370370, 0.641499)
  (0.358025, 0.641499)
  (0.351852, 0.652191)
  (0.339506, 0.652191)
  (0.333333, 0.641499)
  (0.339506, 0.630807)
  (0.333333, 0.620115)
  (0.320988, 0.620116)
  (0.314815, 0.609424)
  (0.320988, 0.598732)
  (0.314815, 0.588041)
  (0.320988, 0.577349)
  (0.333333, 0.577349)
  (0.339506, 0.566657)
  (0.333333, 0.555965)
  (0.320988, 0.555965)
  (0.314815, 0.545274)
  (0.302469, 0.545274)
  (0.296296, 0.555966)
  (0.283951, 0.555965)
  (0.277778, 0.545274)
  (0.265432, 0.545274)
  (0.259259, 0.555966)
  (0.265432, 0.566657)
  (0.259259, 0.577349)
  (0.246913, 0.577349)
  (0.240741, 0.588040)
  (0.228395, 0.588040)
  (0.222222, 0.577349)
  (0.209876, 0.577349)
  (0.203704, 0.588040)
  (0.209876, 0.598732)
  (0.203704, 0.609424)
  (0.209876, 0.620115)
  (0.222222, 0.620115)
  (0.228395, 0.630807)
  (0.222222, 0.641499)
  (0.209876, 0.641499)
  (0.203704, 0.652190)
  (0.209876, 0.662882)
  (0.203704, 0.673574)
  (0.191358, 0.673574)
  (0.185185, 0.684265)
  (0.172839, 0.684265)
  (0.166667, 0.673574)
  (0.172839, 0.662882)
  (0.166667, 0.652190)
  (0.154321, 0.652190)
  (0.148148, 0.641499)
  (0.135802, 0.641499)
  (0.129630, 0.652190)
  (0.117284, 0.652190)
  (0.111111, 0.641499)
  (0.098765, 0.641499)
  (0.092593, 0.652190)
  (0.098765, 0.662882)
  (0.092593, 0.673574)
  (0.080247, 0.673573)
  (0.074074, 0.684265)
  (0.061728, 0.684265)
  (0.055556, 0.673573)
  (0.061728, 0.662882)
  (0.055556, 0.652190)
  (0.043210, 0.652190)
  (0.037037, 0.641498)
  (0.024691, 0.641499)
  (0.018518, 0.652190)
  (0.006173, 0.652190)
  (-0.000000, 0.641498)
  (0.006173, 0.630807)
  (-0.000000, 0.620115)
  (-0.012346, 0.620115)
  (-0.018519, 0.609423)
  (-0.012346, 0.598732)
  (-0.018519, 0.588040)
  (-0.012346, 0.577348)
  (-0.000000, 0.577348)
  (0.006173, 0.588040)
  (0.018518, 0.588040)
  (0.024691, 0.577348)
  (0.037037, 0.577348)
  (0.043210, 0.566657)
  (0.037037, 0.555965)
  (0.043210, 0.545273)
  (0.055555, 0.545273)
  (0.061728, 0.534582)
  (0.055555, 0.523890)
  (0.043210, 0.523890)
  (0.037037, 0.513198)
  (0.043210, 0.502507)
  (0.037037, 0.491815)
  (0.043210, 0.481123)
  (0.055555, 0.481123)
  (0.061728, 0.470432)
  (0.055555, 0.459740)
  (0.043210, 0.459740)
  (0.037037, 0.449048)
  (0.024691, 0.449048)
  (0.018518, 0.459740)
  (0.006173, 0.459740)
  (-0.000000, 0.449048)
  (0.006173, 0.438357)
  (-0.000000, 0.427665)
  (-0.012346, 0.427665)
  (-0.018519, 0.416973)
  (-0.012346, 0.406282)
  (-0.018519, 0.395590)
  (-0.012346, 0.384898)
  (-0.000000, 0.384898)
  (0.006173, 0.374207)
  (-0.000000, 0.363515)
  (-0.012346, 0.363515)
  (-0.018519, 0.352823)
  (-0.030864, 0.352823)
  (-0.037037, 0.363515)
  (-0.049383, 0.363515)
  (-0.055556, 0.352823)
  (-0.067901, 0.352823)
  (-0.074074, 0.363515)
  (-0.067901, 0.374207)
  (-0.074074, 0.384898)
  (-0.086420, 0.384898)
  (-0.092593, 0.395590)
  (-0.104938, 0.395590)
  (-0.111111, 0.384898)
  (-0.104938, 0.374207)
  (-0.111111, 0.363515)
  (-0.123457, 0.363515)
  (-0.129630, 0.352823)
  (-0.141975, 0.352823)
  (-0.148148, 0.363515)
  (-0.160494, 0.363515)
  (-0.166667, 0.352823)
  (-0.160494, 0.342132)
  (-0.166667, 0.331440)
  (-0.179012, 0.331440)
  (-0.185185, 0.320748)
  (-0.179012, 0.310057)
  (-0.185185, 0.299365)
  (-0.179012, 0.288673)
  (-0.166667, 0.288673)
  (-0.160494, 0.299365)
  (-0.148148, 0.299365)
  (-0.141975, 0.288673)
  (-0.129630, 0.288673)
  (-0.123457, 0.277982)
  (-0.129630, 0.267290)
  (-0.123457, 0.256598)
  (-0.111111, 0.256598)
  (-0.104938, 0.245907)
  (-0.111111, 0.235215)
  (-0.123457, 0.235215)
  (-0.129630, 0.224523)
  (-0.123457, 0.213832)
  (-0.129630, 0.203140)
  (-0.123457, 0.192448)
  (-0.111111, 0.192448)
  (-0.104938, 0.181756)
  (-0.111111, 0.171065)
  (-0.123457, 0.171065)
  (-0.129630, 0.160373)
  (-0.141975, 0.160373)
  (-0.148148, 0.171065)
  (-0.160494, 0.171065)
  (-0.166667, 0.160373)
  (-0.160494, 0.149681)
  (-0.166667, 0.138990)
  (-0.179013, 0.138990)
  (-0.185185, 0.128298)
  (-0.179013, 0.117606)
  (-0.185185, 0.106915)
  (-0.179013, 0.096223)
  (-0.166667, 0.096223)
  (-0.160494, 0.106915)
  (-0.148148, 0.106915)
  (-0.141975, 0.096223)
  (-0.129630, 0.096223)
  (-0.123457, 0.085531)
  (-0.129630, 0.074840)
  (-0.123457, 0.064148)
  (-0.111111, 0.064148)
  (-0.104938, 0.053456)
  (-0.111111, 0.042765)
  (-0.123457, 0.042765)
  (-0.129630, 0.032073)
  (-0.123457, 0.021381)
  (-0.129630, 0.010690)
  (-0.123457, -0.000002)
  (-0.111111, -0.000002)
  (-0.104938, 0.010690)
  (-0.092593, 0.010690)
  (-0.086420, -0.000002)
  (-0.074074, -0.000002)
  (-0.067901, -0.010694)
  (-0.074074, -0.021385)
  (-0.067901, -0.032077)
  (-0.055556, -0.032077)
  (-0.049383, -0.021385)
  (-0.037037, -0.021385)
  (-0.030864, -0.032077)
  (-0.018519, -0.032077)
  (-0.012346, -0.021385)
  (-0.000000, -0.021385)
  (0.006173, -0.010694)
  (-0.000000, -0.000002)
}
\end{pspicture}
%
\hskip0.1in\dots
%

\end{zztask}

%%%%%%%%%%%%%%%%%%%%%%%%%%%%%%%%%%%%%%%%%%%%%%%%%%%%%%%%%%%%%%%%%%%%%%%%%%%%%%

