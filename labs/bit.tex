%%%%%%%%%%%%%%%%%%%%%%%%%%%%%%%%%%%%%%%%%%%%%%%%%%%%%%%%%%%%%%%%%%%%%%%%%%%%%%
\zztaskgroup{BIT}{Манипуляции с битами}
%%%%%%%%%%%%%%%%%%%%%%%%%%%%%%%%%%%%%%%%%%%%%%%%%%%%%%%%%%%%%%%%%%%%%%%%%%%%%%
\begin{itemize}
    \item Здесь гораздо подробнее надо остановиться на Integer, как его нужно универсально использовать, пользуясь sizeof и прочими средствами. 
    
    \item Объяснить про считывание с клавиатуры, про тип MaxInt. Рассмотреть приведение типов, а то они не понимают ничерта в этом, тупят ужасно.
    \item Показать на наглядном примерчике с картинкой в памяти как выставить бит, например, или как выставить байт в числе. Как "склеить" числа.
    \item Кроме того, научить их выводить числа в правильном формате. Нужно ли объяснять про двоичную что-то?
\end{itemize}
В следующих задачах предлагается написать ту или иную функцию для
преобразования целых чисел, оптимально используя битовые операции языка Си.
При разработке алгоритма и написании функции не следует заранее делать никаких
предположений о размере целых чисел в битах, пользуясь при необходимости
операцией \texttt{sizeof()}.

Для того чтобы обобщить свой код, абстрагироваться от конкретного целого типа
(который в зависимости от вашего выбора, архитектуры системы, компилятора
может иметь разный размер в битах), \textbf{введём свой собственный тип} с
именем \texttt{Integer}. Определим его как синоним одного из имеющихся
беззнаковых типов языка Си при помощи ключевого слова \texttt{typedef}:

\medskip
\begin{minted}{c}
// typedef unsigned char  Integer;
// typedef unsigned short Integer;
   typedef unsigned int   Integer;
// typedef unsigned long  Integer;
\end{minted}
\medskip

Не только написанная функция, но и вся программа должна быть целиком
универсальной относительно конкретного определения типа \texttt{Integer}.
Сразу следует предостеречь от стандартной ошибки: проверок вида
\texttt{if (sizeof(Integer) == ...)} в коде быть не должно.

Требуется использовать эту функцию в тестовой программе, которая должна читать
с клавиатуры числа и давать ответ до тех пор, пока пользователь не введёт ноль
(перед выходом из программы \textbf{для нуля ответ тоже должен выдаться}).
Введённое значение и полученный результат необходимо выводить на экран
\textbf{в трёх системах счисления}: десятичной, двоичной и шестнадцатеричной,
полностью указывая \textbf{даже ведущие нулевые цифры} в последних двух
системах, дополняя число до максимально возможного количества цифр.\zztodo{Эта задачка мне нравится. Только можно написать сразу про чтение и вывод правильный, чтобы избежать общей ошибки. И можно еще попросить выравнивать инты как-нибудь(можно по максимуму но тогда нужно их изначально обработать, можно еще как-нибудь, например по максимуму типа) }

Прототип функции: \texttt{Integer Process(Integer n);}


%%%%%%%%%%%%%%%%%%%%%%%%%%%%%%%%%%%%%%%%%%%%%%%%%%%%%%%%%%%%%%%%%%%%%%%%%%%%%%
\bigskip
%%%%%%%%%%%%%%%%%%%%%%%%%%%%%%%%%%%%%%%%%%%%%%%%%%%%%%%%%%%%%%%%%%%%%%%%%%%%%%


\begin{zztask}[Реверс битов]
В рамках общего условия задачи написать функцию, которая по заданному
числу типа \texttt{Integer} возвращает другое число, в котором переставлены местами
самый младший бит с самым старшим, второй сверху со вторым снизу и т.д.
Пример работы для 8-битного типа \texttt{char}:
\begin{zzoutput}
  Задание \thezztask: Реверс битов (8 бит)
  Введите число: \zzuser{163}
  Вы ввели  :  163 = 0xA3 = 10100011
  Результат :  197 = 0xC5 = 11000101
  Введите число: \zzuser{ }
\end{zzoutput}
\end{zztask}

%%%%%%%%%%%%%%%%%%%%%%%%%%%%%%%%%%%%%%%%%%%%%%%%%%%%%%%%%%%%%%%%%%%%%%%%%%%%%%

\begin{zztask}[Реверс пар]
В рамках общего условия задачи написать функцию, которая по заданному числу
типа \texttt{Integer} возвращает другое число, в котором переставлены
местами пары битов: самая младшая пара с самой старшей, вторая
сверху со второй снизу и т.д.
Пример работы для 8-битного типа \texttt{char}:
\begin{zzoutput}
  Задание \thezztask: Реверс пар (8 бит)
  Введите число: \zzuser{163}
  Вы ввели  :  163 = 0xA3 = 10100011
  Результат :  202 = 0xСA = 11001010
  Введите число: \zzuser{ }
\end{zzoutput}
\end{zztask}

%%%%%%%%%%%%%%%%%%%%%%%%%%%%%%%%%%%%%%%%%%%%%%%%%%%%%%%%%%%%%%%%%%%%%%%%%%%%%%

\begin{zztask}[Реверс четверок]
В рамках общего условия задачи написать функцию, которая по заданному числу
типа \texttt{Integer} возвращает другое число, в котором переставлены
местами четверки битов: самая младшая четверка с самой старшей, вторая
сверху со второй снизу и т.д.
Пример работы для 8-битного типа \texttt{char}:
\begin{zzoutput}
  Задание \thezztask: Реверс четверок (8 бит)
  Введите число: \zzuser{163}
  Вы ввели  :  163 = 0xA3 = 10100011
  Результат :   58 = 0x3A = 00111010
  Введите число: \zzuser{ }
\end{zzoutput}
\end{zztask}

%%%%%%%%%%%%%%%%%%%%%%%%%%%%%%%%%%%%%%%%%%%%%%%%%%%%%%%%%%%%%%%%%%%%%%%%%%%%%%

\begin{zztask}[Реверс байтов]
В рамках общего условия задачи написать функцию, которая по заданному числу
типа \texttt{Integer} возвращает другое число, в котором переставлены
местами байты: самый младший байт с самым старшим, второй сверху со вторым
снизу и т.д.
Пример работы для 16-битного типа \texttt{short}:
\begin{zzoutput}
  Задание \thezztask: Реверс байтов (16 бит)
  Введите число: \zzuser{41906}
  Вы ввели  : 41906 = 0xA3B2 = 10100011 10110010
  Результат : 45731 = 0xB2A3 = 10110010 10100011
  Введите число: \zzuser{ }
\end{zzoutput}
\end{zztask}

%%%%%%%%%%%%%%%%%%%%%%%%%%%%%%%%%%%%%%%%%%%%%%%%%%%%%%%%%%%%%%%%%%%%%%%%%%%%%%

\begin{zztask}[Обмен битов]
В рамках общего условия задачи написать функцию, которая по заданному числу
типа \texttt{Integer} возвращает другое число, в котором переставлены
местами соседние биты: первый со вторым, третий с четвертым и т.д.
Пример работы для 8-битного типа \texttt{char}:
\begin{zzoutput}
  Задание \thezztask: Обмен битов (8 бит)
  Введите число: \zzuser{163}
  Вы ввели  :  163 = 0xA3 = 10100011
  Результат :   83 = 0x53 = 01010011
  Введите число: \zzuser{ }
\end{zzoutput}
\end{zztask}

%%%%%%%%%%%%%%%%%%%%%%%%%%%%%%%%%%%%%%%%%%%%%%%%%%%%%%%%%%%%%%%%%%%%%%%%%%%%%%

\begin{zztask}[Реверс битов в четверках]
В рамках общего условия задачи написать функцию, которая по заданному числу
типа \texttt{Integer} возвращает другое число, в котором переставлены
местами биты в четверках: самый младший бит четверки с самым старшим битом
той же четверки, второй сверху со вторым снизу и т.д.
Пример работы для 8-битного типа \texttt{char}:
\begin{zzoutput}
  Задание \thezztask: Реверс битов в четверках (8 бит)
  Введите число: \zzuser{163}
  Вы ввели  :  163 = 0xA3 = 10100011
  Результат :   92 = 0x5С = 01011100
  Введите число: \zzuser{ }
\end{zzoutput}
\end{zztask}

%%%%%%%%%%%%%%%%%%%%%%%%%%%%%%%%%%%%%%%%%%%%%%%%%%%%%%%%%%%%%%%%%%%%%%%%%%%%%%

\begin{zztask}[Реверс битов в байтах]
В рамках общего условия задачи написать функцию, которая по заданному числу
типа \texttt{Integer} возвращает другое число, в котором переставлены
местами биты в байтах: самый младший бит байта с самым старшим битом того
же байта, второй сверху со вторым снизу и т.д.
Пример работы для 16-битного типа \texttt{short}:
\begin{zzoutput}
  Задание \thezztask: Реверс битов в байтах (16 бит)
  Введите число: \zzuser{23456}
  Вы ввели  : 23456 = 0x5BA0 = 01011011 10100000
  Результат : 55813 = 0xDA05 = 11011010 00000101
  Введите число: \zzuser{ }
\end{zzoutput}
\end{zztask}

%%%%%%%%%%%%%%%%%%%%%%%%%%%%%%%%%%%%%%%%%%%%%%%%%%%%%%%%%%%%%%%%%%%%%%%%%%%%%%

\begin{zztask}[Циклический сдвиг влево]
В рамках общего условия задачи написать функцию, которая по заданному числу
типа \texttt{Integer} и целому числу возвращает результат циклического
сдвига первого числа влево на количество бит, определяемое вторым числом.
Пример работы для 16-битного типа \texttt{short}:
\begin{zzoutput}
  Задание \thezztask: Циклический сдвиг влево (16 бит)
  Введите числа: \zzuser{23456 3}
  Вы ввели  : 23456 = 0x5BA0 = 01011011 10100000
  Результат : 56578 = 0xDD02 = 11011101 00000010
  Введите числа: \zzuser{ }
\end{zzoutput}
\end{zztask}

%%%%%%%%%%%%%%%%%%%%%%%%%%%%%%%%%%%%%%%%%%%%%%%%%%%%%%%%%%%%%%%%%%%%%%%%%%%%%%

\begin{zztask}[Циклический сдвиг вправо]
В рамках общего условия задачи написать функцию, которая по заданному числу
типа \texttt{Integer} и целому числу возвращает результат циклического
сдвига первого числа вправо на количество бит, определяемое вторым числом.
Пример работы для 16-битного типа \texttt{short}:
\begin{zzoutput}
  Задание \thezztask: Циклический сдвиг вправо (16 бит)
  Введите числа: \zzuser{23456 3}
  Вы ввели  : 23456 = 0x5BA0 = 01011011 10100000
  Результат :  2932 = 0x0B74 = 00001011 01110100
  Введите числа: \zzuser{ }
\end{zzoutput}
\end{zztask}

%%%%%%%%%%%%%%%%%%%%%%%%%%%%%%%%%%%%%%%%%%%%%%%%%%%%%%%%%%%%%%%%%%%%%%%%%%%%%%

\begin{zztask}[Циклический сдвиг влево внутри байтов]
В рамках общего условия задачи написать функцию, которая по заданному числу
типа \texttt{Integer} и целому числу возвращает результат, в котором каждый байт
циклически сдвинут влево на количество бит, определяемое вторым числом.
Пример работы для 16-битного типа \texttt{short}:
\begin{zzoutput}
  Задание \thezztask: Циклический сдвиг влево внутри байтов (16 бит)
  Введите числа: \zzuser{23456 3}
  Вы ввели  : 23456 = 0x5BA0 = 01011011 10100000
  Результат : 55813 = 0xDA05 = 11011010 00000101
  Введите числа: \zzuser{ }
\end{zzoutput}
\end{zztask}

%%%%%%%%%%%%%%%%%%%%%%%%%%%%%%%%%%%%%%%%%%%%%%%%%%%%%%%%%%%%%%%%%%%%%%%%%%%%%%

\begin{zztask}[Циклический сдвиг вправо внутри байтов]
В рамках общего условия задачи написать функцию, которая по заданному числу
типа \texttt{Integer} и целому числу возвращает результат, в котором каждый байт
циклически сдвинут вправо на количество бит, определяемое вторым числом.
Пример работы для 16-битного типа \texttt{short}:
\begin{zzoutput}
  Задание \thezztask: Циклический сдвиг вправо внутри байтов (16 бит)
  Введите числа: \zzuser{23456 3}
  Вы ввели  : 23456 = 0x5BA0 = 01011011 10100000
  Результат : 27412 = 0x6B14 = 01101011 00010100
  Введите числа: \zzuser{ }
\end{zzoutput}
\end{zztask}

%%%%%%%%%%%%%%%%%%%%%%%%%%%%%%%%%%%%%%%%%%%%%%%%%%%%%%%%%%%%%%%%%%%%%%%%%%%%%%

