%%%%%%%%%%%%%%%%%%%%%%%%%%%%%%%%%%%%%%%%%%%%%%%%%%%%%%%%%%%%%%%%%%%%%%%%%%%%%%
\zztaskgroup{SEQ}{Последовательность}
%%%%%%%%%%%%%%%%%%%%%%%%%%%%%%%%%%%%%%%%%%%%%%%%%%%%%%%%%%%%%%%%%%%%%%%%%%%%%%

В следующих задачах требуется написать программу, находящую сумму первых $n$
чисел, отвечающих определенному закону ($n$ вводится в программу
пользователем). Сумму надо посчитать двумя способами, суммируя в цикле и
используя заданную формулу. Результаты сравнить. На экран также вывести
суммируемые числа, разделенные знаками <<плюс>> (или <<минус>>, если число
отрицательное).

Примечание: в некоторых задачах формулу вывести сложно или невозможно, это
указано в их условии.

Примеры диалога программы и пользователя:

\begin{zzoutput}
  Задание \thezztaskgroup-1: Суммирование N натуральных чисел
  Введите N: \zzuser{5}
  Сумма равна 1 + 2 + 3 + 4 + 5 = 15 (и 15 по формуле)
\end{zzoutput}


%%%%%%%%%%%%%%%%%%%%%%%%%%%%%%%%%%%%%%%%%%%%%%%%%%%%%%%%%%%%%%%%%%%%%%%%%%%%%%
\bigskip
%%%%%%%%%%%%%%%%%%%%%%%%%%%%%%%%%%%%%%%%%%%%%%%%%%%%%%%%%%%%%%%%%%%%%%%%%%%%%%


\begin{zztask}[Натуральные числа]
В рамках общего условия задачи найти сумму первых $n$ натуральных чисел
(первые числа: $1$, $2$, $3$, $4$, $5$\dots).
% Формула: $s = n(n+1)/2$.
\end{zztask}

%%%%%%%%%%%%%%%%%%%%%%%%%%%%%%%%%%%%%%%%%%%%%%%%%%%%%%%%%%%%%%%%%%%%%%%%%%%%%%

\begin{zztask}[Нечётные числа]
В рамках общего условия задачи найти сумму первых $n$ нечётных натуральных 
чисел (первые числа: $1$, $3$, $5$, $7$, $9$\dots).
% Формула: $s = n^2$.
\end{zztask}

%%%%%%%%%%%%%%%%%%%%%%%%%%%%%%%%%%%%%%%%%%%%%%%%%%%%%%%%%%%%%%%%%%%%%%%%%%%%%%

\begin{zztask}[Степени двойки]
В рамках общего условия задачи найти сумму первых $n$ чисел, являющихся
степенью двойки. Примечание: $k$-ая степень двойки получается из предыдущей
умножением на 2 (первые числа: $1$, $1\cdot2=2$, $2\cdot2=4$, $4\cdot2=8$, 
$16$\dots).
% Формула: $s = 2^{n+1}-1$.
\end{zztask}

%%%%%%%%%%%%%%%%%%%%%%%%%%%%%%%%%%%%%%%%%%%%%%%%%%%%%%%%%%%%%%%%%%%%%%%%%%%%%%

\begin{zztask}[Треугольные числа]
В рамках общего условия задачи найти сумму первых $n$ треугольных чисел.
Примечание: $k$-ое треугольное число получается из предыдущего прибавлением
к нему $k$ (первые числа: $1$, $1+2=3$, $3+3=6$, $6+4=10$, $15$, $21$\dots).
% Формула: $s = n(n+1)(n+2)/6$.
\end{zztask}

%%%%%%%%%%%%%%%%%%%%%%%%%%%%%%%%%%%%%%%%%%%%%%%%%%%%%%%%%%%%%%%%%%%%%%%%%%%%%%

\begin{zztask}[Прямоугольные числа]
В рамках общего условия задачи найти сумму первых $n$ прямоугольных чисел.
Примечание: прямоугольные числа --- это такие натуральные числа, которые
являются произведением двух последовательных натуральных чисел
(первые числа: $1\cdot2=2$, $2\cdot3=6$, $3\cdot4=12$, $20$, $30$, $42$\dots).
% Формула: $s = n(n+1)(n+2)/3$.
\end{zztask}

%%%%%%%%%%%%%%%%%%%%%%%%%%%%%%%%%%%%%%%%%%%%%%%%%%%%%%%%%%%%%%%%%%%%%%%%%%%%%%

\begin{zztask}[Шестиугольные числа]
В рамках общего условия задачи найти сумму первых $n$ шестиугольных чисел.
Примечание: $k$-ое шестиугольное число получается из предыдущего прибавлением
к нему $4k-3$ (первые числа: $1$, $1+5=6$, $6+9=15$, $15+13=28$, $45$, $66$\dots).
% Формула: $s = n(n+1)(4n-1)/6$.
\end{zztask}

%%%%%%%%%%%%%%%%%%%%%%%%%%%%%%%%%%%%%%%%%%%%%%%%%%%%%%%%%%%%%%%%%%%%%%%%%%%%%%

\begin{zztask}[Числа Фибоначчи]
В рамках общего условия задачи найти сумму первых $n$ чисел Фибоначчи.
Шаг подсчета по формуле опустить ввиду нетривиальности. Примечание: первые
два числа в последовательности чисел Фибоначчи это 0 и 1, а каждое следующее
считается как сумма двух предыдущих: $0$, $1$, $0+1=1$, $1+1=2$, $1+2=3$, $2+3=5$,
$8$, $13$\dots
\end{zztask}

%%%%%%%%%%%%%%%%%%%%%%%%%%%%%%%%%%%%%%%%%%%%%%%%%%%%%%%%%%%%%%%%%%%%%%%%%%%%%%

\begin{zztask}[Числа анти-Фибоначчи]
В рамках общего условия задачи найти сумму первых $n$ чисел анти-Фибоначчи.
Шаг подсчета по формуле опустить ввиду нетривиальности.  Примечание: первые
два числа в последовательности чисел анти-Фибоначчи это 1 и 0, а каждое
следующее считается как разность двух предыдущих: $1$, $0$, $1-0=1$, $0-1=-1$,
$1-(-1)=2$, $-3$, $5$\dots
\end{zztask}

%%%%%%%%%%%%%%%%%%%%%%%%%%%%%%%%%%%%%%%%%%%%%%%%%%%%%%%%%%%%%%%%%%%%%%%%%%%%%%

\begin{zztask}[$^\star$Автоморфные числа]
В рамках общего условия задачи найти сумму первых $n$ автоморфных чисел.
Шаг подсчета по формуле опустить ввиду нетривиальности. Примечание:
автоморфные числа --- это такие натуральные числа, квадрат которых
оканчивается на само число (первые числа: $1$, $5$, $6$, $25$, $76$\dots).
\end{zztask}

%%%%%%%%%%%%%%%%%%%%%%%%%%%%%%%%%%%%%%%%%%%%%%%%%%%%%%%%%%%%%%%%%%%%%%%%%%%%%%

\begin{zztask}[$^\star$Простые числа]
В рамках общего условия задачи найти сумму первых $n$ простых чисел. Шаг
подсчета по формуле опустить ввиду нетривиальности. Примечание: простые
числа --- это такие натуральные числа, которые делятся нацело только на 1 и
на себя (первые числа: $1$, $3$, $5$, $7$, $11$\dots).
\end{zztask}

%%%%%%%%%%%%%%%%%%%%%%%%%%%%%%%%%%%%%%%%%%%%%%%%%%%%%%%%%%%%%%%%%%%%%%%%%%%%%%

\begin{zztask}[$^\star$Нечётные числа в диапазоне]
В рамках общего условия задачи найти сумму всех нечётных целых чисел,
лежащих между вещественными границами $a$ и $b$ включительно. Границы вводятся
пользователем с клавиатуры, не обязательно в порядке возрастания.
\end{zztask}

%%%%%%%%%%%%%%%%%%%%%%%%%%%%%%%%%%%%%%%%%%%%%%%%%%%%%%%%%%%%%%%%%%%%%%%%%%%%%%

