%%%%%%%%%%%%%%%%%%%%%%%%%%%%%%%%%%%%%%%%%%%%%%%%%%%%%%%%%%%%%%%%%%%%%%%%%%%%%%
\zztaskgroup{SEQ}{Последовательность}
%%%%%%%%%%%%%%%%%%%%%%%%%%%%%%%%%%%%%%%%%%%%%%%%%%%%%%%%%%%%%%%%%%%%%%%%%%%%%%

В следующих задачах требуется написать программу, находящую сумму первых $n$
чисел последовательности, которая строится по определенному правилу ($n$ вводится в программу
пользователем). Сумму надо посчитать двумя способами, складывая числа в цикле и
используя аналитическую формулу, выведенную из математических соображений, там, где это возможно.
На экран необходимо вывести
суммируемые члены последовательности, разделенные знаками <<плюс>> (или <<минус>>, если число
отрицательное), итоговый результат суммирования и значение, вычисленное по формуле\zztodo{Надо ли реально сравнивать и выводить разный текст в зависимости от этого?
Поправил старое, потому что показалось, что там как-то всё неявно.}.

Примечание: в некоторых задачах формулу вывести сложно или невозможно, это
указано в их условии.

Примеры диалога программы и пользователя:

\begin{zzoutput}
  Задание \thezztaskgroup-1: Суммирование N натуральных чисел
  Введите N: \zzuser{5}
  Сумма равна 1 + 2 + 3 + 4 + 5 = 15 (и 15 по формуле)
\end{zzoutput}


%%%%%%%%%%%%%%%%%%%%%%%%%%%%%%%%%%%%%%%%%%%%%%%%%%%%%%%%%%%%%%%%%%%%%%%%%%%%%%
\zzsectionCOMMENTS
%%%%%%%%%%%%%%%%%%%%%%%%%%%%%%%%%%%%%%%%%%%%%%%%%%%%%%%%%%%%%%%%%%%%%%%%%%%%%%

\paragraph{Накопитель}
Стандартный подход к суммированию чисел --- это схема с накопителем: изначально
накопитель пуст, а затем, в цикле на каждом шаге к нему прибавляется очередной
элемент последовательности. Этот элемент в большинстве случаев легко получается
из предыдущего: в арифметической прогрессии, например, последовательные элементы
отличаются на константу.

\paragraph{Плюс или минус}
Если вынести первый член последовательности из цикла (он обычно задан, а не
вычисляется по предыдущим), то числа надо не разделять, а предварять знаками,
что гораздо проще. Какой именно знак поставить несложно определить по знаку
самого числа, а выводить уже модуль числа (для целых чисел в языке Си есть
функция \texttt{abs()}).


%%%%%%%%%%%%%%%%%%%%%%%%%%%%%%%%%%%%%%%%%%%%%%%%%%%%%%%%%%%%%%%%%%%%%%%%%%%%%%
\zzsectionPLAN
%%%%%%%%%%%%%%%%%%%%%%%%%%%%%%%%%%%%%%%%%%%%%%%%%%%%%%%%%%%%%%%%%%%%%%%%%%%%%%

Набор шагов, на которые рекомендуется разбить задачу во время решения:
\begin{enumerate}
\item Сначала напишите простейшую часть --- вычисление по формуле. Эту чаcть
будем в дальнейшем использовать для сверки результатов.
%
\item Выразите очередное слагаемое через предыдущее или через его номер,
в зависимости от того, как проще. Вычисляйте значения в цикле и выводите их
для контроля на экран через пробел, убедитесь, что они совпадают с нужными.
%
\item Используйте схему с накопителем для подсчета суммы. Каждый вычисленный
в цикле элемент добавляйте к накопителю, а после цикла выведите результат,
сравните с найденным по формуле.
%
\item Исправьте вывод так, чтобы он соответствовал условию задачи: добавьте
знаки сложения и вычитания, знак равенства.
\end{enumerate}


%%%%%%%%%%%%%%%%%%%%%%%%%%%%%%%%%%%%%%%%%%%%%%%%%%%%%%%%%%%%%%%%%%%%%%%%%%%%%%
\zzsectionVARIATIONS
%%%%%%%%%%%%%%%%%%%%%%%%%%%%%%%%%%%%%%%%%%%%%%%%%%%%%%%%%%%%%%%%%%%%%%%%%%%%%%


\begin{zztask}[Натуральные числа]
В рамках общего условия задачи найти сумму первых $n$ натуральных чисел
(первые числа: $1$, $2$, $3$, $4$, $5$\dots).
Формула: $s = n(n+1)/2$.
\end{zztask}

%%%%%%%%%%%%%%%%%%%%%%%%%%%%%%%%%%%%%%%%%%%%%%%%%%%%%%%%%%%%%%%%%%%%%%%%%%%%%%

\begin{zztask}[Знакопеременный ряд]
В рамках общего условия задачи найти сумму первых $n$ натуральных чисел,
взятых с чередующимися знаками
(первые числа: $1$, $-2$, $3$, $-4$, $5$, $-6$\dots).
Формула: $s = (-1)^{n+1} \lfloor (n+1)/2 \rfloor$.
\end{zztask}

%%%%%%%%%%%%%%%%%%%%%%%%%%%%%%%%%%%%%%%%%%%%%%%%%%%%%%%%%%%%%%%%%%%%%%%%%%%%%%

\begin{zztask}[Нечётные числа]
В рамках общего условия задачи найти сумму первых $n$ нечётных натуральных 
чисел (первые числа: $1$, $3$, $5$, $7$, $9$\dots).
Формула: $s = n^2$.
\end{zztask}

%%%%%%%%%%%%%%%%%%%%%%%%%%%%%%%%%%%%%%%%%%%%%%%%%%%%%%%%%%%%%%%%%%%%%%%%%%%%%%

\begin{zztask}[Степени двойки]
В рамках общего условия задачи найти сумму первых $n$ чисел, являющихся
степенью двойки. Примечание: $k$-ая степень двойки получается из предыдущей
умножением на 2 (первые числа: $1$, $1\cdot2=2$, $2\cdot2=4$, $4\cdot2=8$, 
$16$\dots).
Формула: $s = 2^{n+1}-1$.
\end{zztask}

%%%%%%%%%%%%%%%%%%%%%%%%%%%%%%%%%%%%%%%%%%%%%%%%%%%%%%%%%%%%%%%%%%%%%%%%%%%%%%

\begin{zztask}[Треугольные числа]
В рамках общего условия задачи найти сумму первых $n$ треугольных чисел.
Примечание: $k$-ое треугольное число получается из предыдущего прибавлением
к нему $k$ (первые числа: $1$, $1+2=3$, $3+3=6$, $6+4=10$, $15$, $21$\dots).
Формула: $s = n(n+1)(n+2)/6$.
\end{zztask}

%%%%%%%%%%%%%%%%%%%%%%%%%%%%%%%%%%%%%%%%%%%%%%%%%%%%%%%%%%%%%%%%%%%%%%%%%%%%%%

\begin{zztask}[Прямоугольные числа]
В рамках общего условия задачи найти сумму первых $n$ прямоугольных чисел.
Примечание: прямоугольные числа --- это такие натуральные числа, которые
являются произведением двух последовательных натуральных чисел
(первые числа: $1\cdot2=2$, $2\cdot3=6$, $3\cdot4=12$, $20$, $30$, $42$\dots).
Формула: $s = n(n+1)(n+2)/3$.
\end{zztask}

%%%%%%%%%%%%%%%%%%%%%%%%%%%%%%%%%%%%%%%%%%%%%%%%%%%%%%%%%%%%%%%%%%%%%%%%%%%%%%

\begin{zztask}[Шестиугольные числа]
В рамках общего условия задачи найти сумму первых $n$ шестиугольных чисел.
Примечание: $k$-ое шестиугольное число получается из предыдущего прибавлением
к нему $4k-3$ (первые числа: $1$, $1+5=6$, $6+9=15$, $15+13=28$, $45$, $66$\dots).
Формула: $s = n(n+1)(4n-1)/6$.
\end{zztask}

%%%%%%%%%%%%%%%%%%%%%%%%%%%%%%%%%%%%%%%%%%%%%%%%%%%%%%%%%%%%%%%%%%%%%%%%%%%%%%

\begin{zztask}[Числа Фибоначчи]
В рамках общего условия задачи найти сумму первых $n$ чисел Фибоначчи.
Примечание: первые
два числа в последовательности чисел Фибоначчи это 0 и 1, а каждое следующее
считается как сумма двух предыдущих: $0$, $1$, $0+1=1$, $1+1=2$, $1+2=3$, $2+3=5$,
$8$, $13$, $21$\dots
Формула: $s = (\varphi^{n+2}-\psi^{n+2})/(\varphi-\psi) - 1$, где за $\varphi$
и $\psi$ обозначено $(1\pm\sqrt5)/2$.
\end{zztask}

%%%%%%%%%%%%%%%%%%%%%%%%%%%%%%%%%%%%%%%%%%%%%%%%%%%%%%%%%%%%%%%%%%%%%%%%%%%%%%

\begin{zztask}[Числа анти-Фибоначчи]
В рамках общего условия задачи найти сумму первых $n$ чисел анти-Фибоначчи.
Примечание: первые
два числа в последовательности чисел анти-Фибоначчи это 0 и 1, а каждое
следующее считается как разность двух предыдущих: $0$, $1$, $0-1=-1$,
$1-(-1)=2$, $-3$, $5$, $-8$, $13$, $-21$\dots
Формула: $s = 1+(-1)^{n-1}(\varphi^{n-1}-\psi^{n-1})/(\varphi-\psi)$, где за $\varphi$
и $\psi$ обозначено $(1\pm\sqrt5)/2$.
\end{zztask}

%%%%%%%%%%%%%%%%%%%%%%%%%%%%%%%%%%%%%%%%%%%%%%%%%%%%%%%%%%%%%%%%%%%%%%%%%%%%%%

\begin{zztask}[$^\star$Автоморфные числа]
В рамках общего условия задачи найти сумму первых $n$ автоморфных чисел.
Примечание:
автоморфные числа --- это такие натуральные числа, квадрат которых
оканчивается на само число (первые числа: $1$, $5$, $6$, $25$, $76$\dots).
Шаг подсчета по формуле опустить ввиду нетривиальности.  
\end{zztask}

%%%%%%%%%%%%%%%%%%%%%%%%%%%%%%%%%%%%%%%%%%%%%%%%%%%%%%%%%%%%%%%%%%%%%%%%%%%%%%

\begin{zztask}[$^\star$Простые числа]
В рамках общего условия задачи найти сумму первых $n$ простых чисел.
Примечание: простые
числа --- это такие натуральные числа, которые делятся нацело только на 1 и
на себя (первые числа: $1$, $3$, $5$, $7$, $11$\dots).
Шаг подсчета по формуле опустить ввиду нетривиальности.
\end{zztask}

%%%%%%%%%%%%%%%%%%%%%%%%%%%%%%%%%%%%%%%%%%%%%%%%%%%%%%%%%%%%%%%%%%%%%%%%%%%%%%

\begin{zztask}[$^\star$Нечётные числа в диапазоне]
В рамках общего условия задачи найти сумму всех нечётных целых чисел,
лежащих между вещественными границами $a$ и $b$ включительно. Границы вводятся
пользователем с клавиатуры, не обязательно в порядке возрастания.
\end{zztask}

%%%%%%%%%%%%%%%%%%%%%%%%%%%%%%%%%%%%%%%%%%%%%%%%%%%%%%%%%%%%%%%%%%%%%%%%%%%%%%

