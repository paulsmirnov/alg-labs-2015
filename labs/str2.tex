%%%%%%%%%%%%%%%%%%%%%%%%%%%%%%%%%%%%%%%%%%%%%%%%%%%%%%%%%%%%%%%%%%%%%%%%%%%%%%
\zztaskgroup{STR}{Строки (старые)}
%%%%%%%%%%%%%%%%%%%%%%%%%%%%%%%%%%%%%%%%%%%%%%%%%%%%%%%%%%%%%%%%%%%%%%%%%%%%%%

Задача D-1

В буферы фиксированного размера прочитать с клавиатуры длинную и короткую
строку. Не используя страндартные функции (из string.h), найти все места, где
в длинной строке встречается короткая, вывести номера позиций (считая с нуля).
При поиске не менять исходных буферов. Повторять выполнение программы до тех
пор, пока в качестве длинной строки не введут пустую.

Иванов И.И. (1057/1): Строки. Поиск подстроки.
Введите строку-1 (<40 символов): abcd efab cdabcdef
Введите строку-2 (<40 символов): abcd
Позиции: 0 12
Введите строку-1 (<40 символов):

%%%%%%%%%%%%%%%%%%%%%%%%%%%%%%%%%%%%%%%%%%%%%%%%%%%%%%%%%%%%%%%%%%%%%%%%%%%%%%

Задача D-2

В буферы фиксированного размера прочитать с клавиатуры длинную и короткую
строку. Не используя страндартные функции (из string.h), найти все места, где
в длинной строке встречается короткая, записанная задом наперед, вывести
номера позиций (считая с нуля). При поиске не менять исходных буферов.
Повторять выполнение программы до тех пор, пока в качестве длинной строки не
введут пустую.

Иванов И.И. (1057/1): Строки. Поиск обратной подстроки.
Введите строку-1 (<40 символов): abcd efab cdabcdef
Введите строку-2 (<40 символов): dcba
Позиции: 0 12
Введите строку-1 (<40 символов):

%%%%%%%%%%%%%%%%%%%%%%%%%%%%%%%%%%%%%%%%%%%%%%%%%%%%%%%%%%%%%%%%%%%%%%%%%%%%%%

Задача D-3

В буфер фиксированного размера прочитать с клавиатуры строку. Не используя
страндартные функции (из string.h), развернуть строку задом наперед. Повторять
выполнение программы до тех пор, пока в качестве строки не введут пустую.

Иванов И.И. (1057/1): Строки. Разворот строки.
Введите строку (<40 символов): abcd efab
Строка наоборот: bafe dcba
Введите строку (<40 символов):

%%%%%%%%%%%%%%%%%%%%%%%%%%%%%%%%%%%%%%%%%%%%%%%%%%%%%%%%%%%%%%%%%%%%%%%%%%%%%%

Задача D-4

В буфер фиксированного размера прочитать с клавиатуры строку. Не используя
страндартные функции (из string.h и др.), подсчитать количество заглавных и
строчных букв. Другие символы игнорировать. Повторять выполнение программы до
тех пор, пока в качестве строки не введут пустую.

Иванов И.И. (1057/1): Строки. Подсчет букв.
Введите строку (<40 символов): abCD eFab
В строке 3 заглавные и 5 строчных букв 
Введите строку (<40 символов):

%%%%%%%%%%%%%%%%%%%%%%%%%%%%%%%%%%%%%%%%%%%%%%%%%%%%%%%%%%%%%%%%%%%%%%%%%%%%%%

Задача D-5

В буфер фиксированного размера прочитать с клавиатуры строку. Не используя
страндартные функции (из string.h и др.), заменить в строке заглавные буквы на
пробелы. Повторять выполнение программы до тех пор, пока в качестве строки не
введут пустую.

Иванов И.И. (1057/1): Строки. Забивка заглавных букв.
Введите строку (<40 символов): abCD eFab
Результат: ab   e ab
Введите строку (<40 символов):

%%%%%%%%%%%%%%%%%%%%%%%%%%%%%%%%%%%%%%%%%%%%%%%%%%%%%%%%%%%%%%%%%%%%%%%%%%%%%%

Задача D-6

В буфер фиксированного размера прочитать с клавиатуры строку. Не используя
страндартные функции (из string.h и др.), во всех последовательностях
повторяющихся букв заменить все буквы кроме первой на пробелы. Остальные
символы не трогать. Повторять выполнение программы до тех пор, пока в качестве
строки не введут пустую.

Иванов И.И. (1057/1): Строки. Забивка заглавных букв.
Введите строку (<40 символов): abbbaacdbbaaaaffd
Результат: ab  a cdb a   f d
Введите строку (<40 символов):

%%%%%%%%%%%%%%%%%%%%%%%%%%%%%%%%%%%%%%%%%%%%%%%%%%%%%%%%%%%%%%%%%%%%%%%%%%%%%%

Задача D-7

В буфер фиксированного размера прочитать с клавиатуры строку. Не используя
страндартные функции (из string.h), развернуть слова в строке задом наперед
(слова разделены пробелами). Повторять выполнение программы до тех пор, пока в
качестве строки не введут пустую.

Иванов И.И. (1057/1): Строки. Разворот слов.
Введите строку (<40 символов): abcd efab
Слова наоборот: dcba bafe
Введите строку (<40 символов):

%%%%%%%%%%%%%%%%%%%%%%%%%%%%%%%%%%%%%%%%%%%%%%%%%%%%%%%%%%%%%%%%%%%%%%%%%%%%%%

Задача D-8

В буфер фиксированного размера прочитать с клавиатуры строку. Не используя
страндартные функции (из string.h), найти подстроку максимальной длины, такую
что: строка начинается с нее, а заканчивается ее зеркальным отображением (см.
пример). Повторять выполнение программы до тех пор, пока в качестве строки не
введут пустую.

Иванов И.И. (1057/1): Строки. Поиск зеркального начала.
Введите строку (<40 символов): abcxyzcba
Зеркальное начало: abc
Введите строку (<40 символов):

%%%%%%%%%%%%%%%%%%%%%%%%%%%%%%%%%%%%%%%%%%%%%%%%%%%%%%%%%%%%%%%%%%%%%%%%%%%%%%

Задача D-9

В буфер фиксированного размера прочитать с клавиатуры строку. Не используя
страндартные функции (из string.h и др.), удалить из строки все пробелы,
прижав друг к другу слова. Повторять выполнение программы до тех пор, пока в
качестве строки не введут пустую.

Иванов И.И. (1057/1): Строки. Удаление пробелов.
Введите строку (<40 символов): I am a program
Результат: Iamaprogram
Введите строку (<40 символов):

%%%%%%%%%%%%%%%%%%%%%%%%%%%%%%%%%%%%%%%%%%%%%%%%%%%%%%%%%%%%%%%%%%%%%%%%%%%%%%

Задача D-10

В буфер фиксированного размера прочитать с клавиатуры строку, в которой
записан полное имя файла. Не используя страндартные функции (из string.h),
выделить в три буфера три части этого имени: путь к файлу, само имя файла,
расширение файла. Вывести их на экран. Повторять выполнение программы до тех
пор, пока в качестве строки не введут пустую.

Иванов И.И. (1057/1): Строки. Длинное имя файла.
Введите полное имя файла (<40 символов): \verb|c:\windows\system32\xcopy.exe|
Путь: \verb|с:\windows\system32|
Имя: xcopy
Расширение: exe
Введите полное имя файла (<40 символов):

%%%%%%%%%%%%%%%%%%%%%%%%%%%%%%%%%%%%%%%%%%%%%%%%%%%%%%%%%%%%%%%%%%%%%%%%%%%%%%

Задача D-11

В буфер фиксированного размера прочитать с клавиатуры строку. Не используя
страндартные функции (из string.h и др.), перевести все английские буквы в
заглавные. Если в строке встречаются другие, «неправильные» символы, также
сообщить об этом. Повторять выполнение программы до тех пор, пока в качестве
строки не введут пустую.

Иванов И.И. (1057/1): Строки. Заглавные буквы.
Введите строку (<40 символов): abcbax
Результат: ABCBAX
Введите строку (<40 символов):

%%%%%%%%%%%%%%%%%%%%%%%%%%%%%%%%%%%%%%%%%%%%%%%%%%%%%%%%%%%%%%%%%%%%%%%%%%%%%%

Задача D-12

В буфер фиксированного размера прочитать с клавиатуры строку. Не используя
страндартные функции (из string.h и др.), перевести все английские буквы в
строчные. Если в строке встречаются другие, «неправильные» символы, также
сообщить об этом. Повторять выполнение программы до тех пор, пока в качестве
строки не введут пустую.

Иванов И.И. (1057/1): Строки. Строчные буквы.
Введите строку (<40 символов): ABCBAX
Результат: abcbax
Введите строку (<40 символов):

%%%%%%%%%%%%%%%%%%%%%%%%%%%%%%%%%%%%%%%%%%%%%%%%%%%%%%%%%%%%%%%%%%%%%%%%%%%%%%

Задача D-13

В буфер фиксированного размера прочитать с клавиатуры строку. Не используя
страндартные функции (из string.h и др.), перевести все заглавные английские
буквы в строчные и наоборот. Если в строке встречаются другие, «неправильные»
символы, также сообщить об этом. Повторять выполнение программы до тех пор,
пока в качестве строки не введут пустую.

Иванов И.И. (1057/1): Строки. Инвертирование регистра букв.
Введите строку (<40 символов): ABcBax
Результат: abCbAX
Введите строку (<40 символов):

%%%%%%%%%%%%%%%%%%%%%%%%%%%%%%%%%%%%%%%%%%%%%%%%%%%%%%%%%%%%%%%%%%%%%%%%%%%%%%

Задача D-14

В буфер фиксированного размера прочитать с клавиатуры строку. Не используя
страндартные функции (из string.h и др.), проверить правильность написания
предложений в тексте: первая буква предложения должна быть заглавной,
предложения завершаются точкой, между точкой и следующим предложением должен
быть минимум один пробел. Сообщить все ошибки (и индексы символов с ошибкой).
Повторять выполнение программы до тех пор, пока в качестве строки не введут
пустую.

Иванов И.И. (1057/1): Строки. Проверка предложений.
Введите строку (<40 символов): AbcBax. F dcc. Sa nfmmc dfldk.
Предложение правильное.
Введите строку (<40 символов): Abc. f dcc.Sa nf.
Ошибка в символе 5: должна быть заглавная.
Ошибка в символе 11: должен быть пробел.
Введите строку (<40 символов):

%%%%%%%%%%%%%%%%%%%%%%%%%%%%%%%%%%%%%%%%%%%%%%%%%%%%%%%%%%%%%%%%%%%%%%%%%%%%%%

Задача D-15

В буфер фиксированного размера прочитать с клавиатуры строку. Не используя
страндартные функции (из string.h), найти все подстроки длиной больше одного
символа, такие, что зеркальные их отражения тоже присутствуют в ней. Повторять
выполнение программы до тех пор, пока в качестве строки не введут пустую.

Иванов И.И. (1057/1): Строки. Поиск зеркальных подстрок.
Введите строку (<40 символов): abcbaxxyzcb
Подстрока 1: ab
Подстрока 2: abс
Подстрока 3: bс
Подстрока 4: xx
...
Введите строку (<40 символов):

%%%%%%%%%%%%%%%%%%%%%%%%%%%%%%%%%%%%%%%%%%%%%%%%%%%%%%%%%%%%%%%%%%%%%%%%%%%%%%

Задача D-16

В буфер фиксированного размера на 60 символов прочитать с клавиатуры строку.
Не используя страндартные функции (из string.h и др.) и дополнительный буфер,
выровнять строку по центру этого буфера (вставить в буфер необходимое
количество пробелов перед и после текста, так чтобы суммарная длина строки
была равна 60 — проверить с помощью strlen()). Вывести строку на экран в
квадратных скобках, чтобы была видна ширина текста. Повторять выполнение
программы до тех пор, пока в качестве строки не введут пустую. Пример для 20
символов:

Иванов И.И. (1057/1): Строки. Выравнивание по центру.
Введите строку (<60 символов): hello, world!
Результат: [    hello, world!   ]
Введите строку (<60 символов):

%%%%%%%%%%%%%%%%%%%%%%%%%%%%%%%%%%%%%%%%%%%%%%%%%%%%%%%%%%%%%%%%%%%%%%%%%%%%%%

Задача D-17

В буфер фиксированного размера прочитать с клавиатуры строку. Не используя
страндартные функции (из string.h и др.), выбросить из нее все символы,
начиная с последовательности «/*» и заканчивая «*/». Если указанной
последовательности нет, не делать ничего. Если встречается только открывающая
или закрывающая последовательность, выдать сообщение об ошибке. Повторять
выполнение программы до тех пор, пока в качестве строки не введут пустую.

Иванов И.И. (1057/1): Строки. Удаление коментариев.
Введите строку (<40 символов): hello,/*bc12df fh*/world
Результат: hello,world
Введите строку (<40 символов):

%%%%%%%%%%%%%%%%%%%%%%%%%%%%%%%%%%%%%%%%%%%%%%%%%%%%%%%%%%%%%%%%%%%%%%%%%%%%%%

Задача D-18

В буфер фиксированного размера прочитать с клавиатуры строку. Не используя
страндартные функции (из string.h и др.), найти в ней слово минимальной длины.
Слова — это последовательности букв. Остальные символы — разделители.
Повторять выполнение программы до тех пор, пока в качестве строки не введут
пустую.

Иванов И.И. (1057/1): Строки. Минимальное слово.
Введите строку (<40 символов): read a string
Результат: a
Введите строку (<40 символов):

%%%%%%%%%%%%%%%%%%%%%%%%%%%%%%%%%%%%%%%%%%%%%%%%%%%%%%%%%%%%%%%%%%%%%%%%%%%%%%

Задача D-19

В буфер фиксированного размера прочитать с клавиатуры строку. Не используя
страндартные функции (из string.h и др.), найти в ней слово максимальной
длины. Слова — это последовательности букв. Остальные символы — разделители.
Повторять выполнение программы до тех пор, пока в качестве строки не введут
пустую.

Иванов И.И. (1057/1): Строки. Максимальное слово.
Введите строку (<40 символов): read a string
Результат: string
Введите строку (<40 символов):

%%%%%%%%%%%%%%%%%%%%%%%%%%%%%%%%%%%%%%%%%%%%%%%%%%%%%%%%%%%%%%%%%%%%%%%%%%%%%%

Задача D-20

В буфер фиксированного размера на 60 символов прочитать с клавиатуры строку.
Не используя страндартные функции (из string.h и др.) и дополнительный буфер,
равномерно вставить между словами дополнительные пробелы так, чтобы, чтобы
строка заполнила буфер (проверить с помощью strlen()). Вывести строку на экран
в квадратных скобках, чтобы была видна ширина текста. Повторять выполнение
программы до тех пор, пока в качестве строки не введут пустую. Пример для 20
символов:

Иванов И.И. (1057/1): Строки. Выравнивание по обоим краям.
Введите строку (<60 символов): I am a program
Результат: [I   am   a   program]
Введите строку (<60 символов):

%%%%%%%%%%%%%%%%%%%%%%%%%%%%%%%%%%%%%%%%%%%%%%%%%%%%%%%%%%%%%%%%%%%%%%%%%%%%%%

Задача D-21

В буфер фиксированного размера прочитать с клавиатуры строку. Не используя
страндартные функции (из string.h и др.), найти максимальное по модулю целое
число, встречающееся в этой строке. Повторять выполнение программы до тех пор,
пока в качестве строки не введут пустую.

Иванов И.И. (1057/1): Строки. Поиск чисел.
Введите строку (<40 символов): abc12df fh 84ii-10z
Результат: 84
Введите строку (<40 символов):

%%%%%%%%%%%%%%%%%%%%%%%%%%%%%%%%%%%%%%%%%%%%%%%%%%%%%%%%%%%%%%%%%%%%%%%%%%%%%%
