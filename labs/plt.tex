%%%%%%%%%%%%%%%%%%%%%%%%%%%%%%%%%%%%%%%%%%%%%%%%%%%%%%%%%%%%%%%%%%%%%%%%%%%%%%
\zztaskgroup{PLT}{График функции}
%%%%%%%%%%%%%%%%%%%%%%%%%%%%%%%%%%%%%%%%%%%%%%%%%%%%%%%%%%%%%%%%%%%%%%%%%%%%%%

Написать программу, строящую график $y=f(x)$ --- функции одной переменной в заданной
области $x\in(a,b), y\in(c,d)$. Алгоритм следует оформить в виде отдельной
функции \textbf{в отдельном модуле} со своим заголовочным файлом.
Функция должна принимать в качестве параметров указатель на $f(x)$, границы по осям $x, y$
и прямоугольную область на экране для рисования.

Кроме самого графика (кривой линии, а не набора отдельных точек) функция должна 
рисовать прямоугольник, ограничивающий заданную область, и координатные оси, 
если они попадают в эту область.
И прямоугольник и оси должны быть подписаны и на них должны быть указаны крайние 
значения, а также нанесены риски, соответствующие большому 
шагу $\Delta = 1$ и маленькому шагу $\delta = 0.1$. Если масштаб выбран таким образом,
что риски (большие или маленькие) сливаются, то их показывать не надо. Если масштаб 
позволяет (риски достаточно далеко друг от друга), необходимо под (рядом с) риской 
указывать соответствующее значение координаты.

Программа должна уметь строить графики как непрерывных $(y=\sin(x))$, 
так и \textbf{разрывных} функций $(y=\lfloor x \rfloor)$,
функций, \textbf{обращающихся в бесконечность} $(y=1/x)$,
или \textbf{не полностью определенных} на заданном промежутке $(y=\sqrt x)$.
В качестве тестовых функций также можно взять:
%
\[
y = \sin(x)+2\cdot|\sin(x)|, \qquad
y = \sqrt{x^2-x-1}, \qquad
y = \frac{5(x-2)}{x^2}, \qquad
y = \frac{1+\cos x}{3-\sin x}.
\]

Внимание: алгоритм должен быть написан в общем виде, для построения различных 
функций. Для обработки исключительных ситуаций обратить внимание на: 
\texttt{matherr()}, \texttt{signal()}, \texttt{SIGFPE}.
