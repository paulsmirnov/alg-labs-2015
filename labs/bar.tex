%%%%%%%%%%%%%%%%%%%%%%%%%%%%%%%%%%%%%%%%%%%%%%%%%%%%%%%%%%%%%%%%%%%%%%%%%%%%%%
\zztaskgroup{BAR}{Штрих-код}
%%%%%%%%%%%%%%%%%%%%%%%%%%%%%%%%%%%%%%%%%%%%%%%%%%%%%%%%%%%%%%%%%%%%%%%%%%%%%%

См. вордовский файл (примеры). Текст условия надо писать.
Для общего условия: контрольные цифры они должны считать самостоятельно.

Два возможных варианта вывода: в графическом режиме и в текстовом режиме с
использованием псевдографики (половинных символов для увеличения горизонтального
разрешения). Возможно даже потребовать поддержку обоих, это вставляет мозги на
предмет аккуратного планирования --- кодирование отдельно от отображения.

Раньше в эту же кучу, под предлогом кодирования, я сваливал также азбуку Морзе и
шрифт Брайля, но там мне такое выдают\dots Еще хуже, чем для штрих-кодов исходники.
Хотя, исходно идея была неплохая.

\begin{zztask}[EAN-8]
В рамках общего условия задачи\dots
\end{zztask}

\begin{zztask}[EAN-13]
В рамках общего условия задачи\dots
\end{zztask}

\begin{zztask}[UPC-A]
В рамках общего условия задачи\dots
\end{zztask}

\begin{zztask}[UPC-E]
В рамках общего условия задачи\dots
\end{zztask}

\begin{zztask}[CODE-39]
В рамках общего условия задачи\dots
\end{zztask}

\begin{zztask}[CODE-93]
В рамках общего условия задачи\dots
\end{zztask}

\begin{zztask}[CODE-128]
В рамках общего условия задачи\dots
\end{zztask}

\begin{zztask}[CODABAR]
В рамках общего условия задачи\dots
\end{zztask}

\begin{zztask}[ITF]
В рамках общего условия задачи\dots
\end{zztask}

\begin{zztask}[QR-CODE]
В рамках общего условия задачи\dots
\end{zztask}

\begin{zztask}[Data Matrix]
В рамках общего условия задачи\dots
\end{zztask}
