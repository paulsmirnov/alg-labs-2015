%%%%%%%%%%%%%%%%%%%%%%%%%%%%%%%%%%%%%%%%%%%%%%%%%%%%%%%%%%%%%%%%%%%%%%%%%%%%%%
\zztaskgroup{LIB}{Библиотеки функций}
%%%%%%%%%%%%%%%%%%%%%%%%%%%%%%%%%%%%%%%%%%%%%%%%%%%%%%%%%%%%%%%%%%%%%%%%%%%%%%

В следующих задачах требуется написать библиотеку функций с заданным
интерфейсом для работы с определённым типом данных и программу,
демонстрирующую возможности библиотеки. Таким образом, решение должно состоять
из двух проектов:
%
\begin{enumerate}
\item Библиотека \texttt{xxx}: создаётся \texttt{xxx.lib},
      которая собирается из \texttt{xxx.c} и \texttt{xxx.h}.
\item Консольное приложение \texttt{xxxtest}: создаётся \texttt{xxxtest.exe},
      которое собирается из \texttt{xxxtest.c}
      с использованием \texttt{xxx.h} и \texttt{xxx.lib}.
\end{enumerate}

Исходный текст библиотеки (заголовочный файл) должен содержать комментарии для
каждого экспортируемого объекта (функции, типа, константы), для автоматической
генерации документации в формате HTML/CHM с помощью утилиты Doxygen.%\zztodo{Задача ничего, но только они все разнородные и вроде у них мейн не фиксирован, может заставить их еще какие-то штуки писать типа юнит-тестов? }

Все функции библиотеки традиционно в качестве префикса имён должны иметь
название типа, чтобы отличать от аналогичных функций для другого типа,
например, для типа \texttt{xxx\_t} функции \texttt{XxxCreate()},
\texttt{XxxAdd()} и т.п.


%%%%%%%%%%%%%%%%%%%%%%%%%%%%%%%%%%%%%%%%%%%%%%%%%%%%%%%%%%%%%%%%%%%%%%%%%%%%%%
\zzsectionCOMMENTS
%%%%%%%%%%%%%%%%%%%%%%%%%%%%%%%%%%%%%%%%%%%%%%%%%%%%%%%%%%%%%%%%%%%%%%%%%%%%%%


\paragraph{Интерфейс}
Краткое описание возможной функциональности для различного вида чисел или
аналогичных объектов:
%
\begin{itemize}
%
\item[--] \texttt{XxxCreate}, \texttt{XxxDestroy}: создание по набору параметров, с выделением динамической памяти при необходимости, уничтожение, в первую очередь динамической памяти;
%
\item[--] \texttt{XxxFromYyy}, \texttt{XxxAsYyy}: создание из определенного типа, преобразование к определенному типу;
%
\item[--] \texttt{XxxParse}, \texttt{XxxFormat}: создание по строке, запись в строковый буфер;
%
\item[--] \texttt{XxxRead}, \texttt{XxxWrite}: создание из потока, запись в поток, в текстовом виде;
%
\item[--] \texttt{XxxAdd}, \texttt{XxxSub}, \texttt{XxxMul}, \texttt{XxxDiv}: сложение, вычитание, умножение, деление;
%
\item[--] \texttt{XxxNegate}, \texttt{XxxReciprocal}, \texttt{XxxInverse}: противоположное, обратное;
%
\item[--] \texttt{XxxCompare}: сравнение (функция сравнения традиционно принимает два указателя и возвращает –1, 0 или +1 когда первый аргумент меньше, равен или больше второго соответственно);
%
\end{itemize}


%%%%%%%%%%%%%%%%%%%%%%%%%%%%%%%%%%%%%%%%%%%%%%%%%%%%%%%%%%%%%%%%%%%%%%%%%%%%%%
\zzsectionPLAN
%%%%%%%%%%%%%%%%%%%%%%%%%%%%%%%%%%%%%%%%%%%%%%%%%%%%%%%%%%%%%%%%%%%%%%%%%%%%%%


Предлагаем вам следующие шаги решения.

\begin{enumerate}
\item Объявите\dots
%
\end{enumerate}



%%%%%%%%%%%%%%%%%%%%%%%%%%%%%%%%%%%%%%%%%%%%%%%%%%%%%%%%%%%%%%%%%%%%%%%%%%%%%%
\zzsectionVARIATIONS
%%%%%%%%%%%%%%%%%%%%%%%%%%%%%%%%%%%%%%%%%%%%%%%%%%%%%%%%%%%%%%%%%%%%%%%%%%%%%%


\begin{zztask}[Комплексные числа]
В рамках общего условия задачи ввести новый прозрачный тип <<комплексное число>>
(\texttt{complex\_t}) на основе значений \texttt{re}, \texttt{im} типа
\texttt{double} и реализовать функциональность:
%
\begin{itemize}
\item константа ZERO;
\item Create;
\item Parse, Format, Read, Write, в формате \verb|"(1, -0.5)"|;
\item Add, Sub, Mul, Div, Negate, Reciprocal;
\item Abs, Arg: модуль, аргумент;
\item Conjugate: сопряжение;
\end{itemize}
\end{zztask}

%%%%%%%%%%%%%%%%%%%%%%%%%%%%%%%%%%%%%%%%%%%%%%%%%%%%%%%%%%%%%%%%%%%%%%%%%%%%%%

\begin{zztask}[Нечёткая логика]
В рамках общего условия задачи ввести новый прозрачный тип <<fuzzy bool>>
(\texttt{fool\_t}) на основе значения типа \texttt{double} и реализовать
функциональность:
%
\begin{itemize}
\item FromInt, FromDouble, AsInt, AsDouble;
\item Parse, Format, Read, Write, в формате \verb|"fool[0.25]"|;
\item And, Or, Not: конъюнкция, дизъюнкция, отрицание;
\item Compare;
\end{itemize}
\end{zztask}

%%%%%%%%%%%%%%%%%%%%%%%%%%%%%%%%%%%%%%%%%%%%%%%%%%%%%%%%%%%%%%%%%%%%%%%%%%%%%%

\begin{zztask}[Рациональные дроби]
В рамках общего условия задачи ввести новый прозрачный тип <<несократимая рациональная
дробь>> (\texttt{rational\_t}) на основе значений \texttt{num}, \texttt{den}
типа \texttt{int} и реализовать функциональность (не забывая сокращать дробь):
%
\begin{itemize}
\item константа ZERO;
\item Create; \textit{FromDouble}${}^{(\ast)}$, AsDouble, AsInt;
\item Parse, Format, Read, Write, в формате \verb|"-2\3"|;
\item Add, Sub, Mul, Div, Negate, Reciprocal, Abs;
\item Round, Floor, Ceil: округления; Compare;
\end{itemize}
\end{zztask}

%%%%%%%%%%%%%%%%%%%%%%%%%%%%%%%%%%%%%%%%%%%%%%%%%%%%%%%%%%%%%%%%%%%%%%%%%%%%%%

\begin{zztask}[Вещественные числа с фиксированной точкой]
В рамках общего условия задачи ввести новый прозрачный тип <<с фиксированной точкой>>
(\texttt{fixed\_t}) на основе типа \texttt{int} (напр., 32 бита как 16:16) и
реализовать функциональность:
%
\begin{itemize}
\item константа ZERO;
\item FromInt, FromDouble, AsInt, AsDouble;
\item Parse, Format, Read, Write, в формате \verb|"Fix[-12.75]"|;
\item Add, Sub, Mul, Div, Negate, Reciprocal, Abs;
\item Round, Floor, Ceil: округления; Compare;
\end{itemize}
\end{zztask}

%%%%%%%%%%%%%%%%%%%%%%%%%%%%%%%%%%%%%%%%%%%%%%%%%%%%%%%%%%%%%%%%%%%%%%%%%%%%%%

\begin{zztask}[Длинные целые числа]
В рамках общего условия задачи ввести новый прозрачный 18-значный тип <<очень длинный>>
(\texttt{verylong\_t}) на основе двух значений \texttt{hi}, \texttt{lo} типа
\texttt{long} (в каждом хранить по 9 десятичных цифр, в старшем числе знак) и
реализовать функциональность:
%
\begin{itemize}
\item константа ZERO;
\item FromLong, AsLong;
\item Parse, Format, Read, Write;
\item Add, Sub, Mul, Div, Negate, Abs; Compare;
\end{itemize}
\end{zztask}

%%%%%%%%%%%%%%%%%%%%%%%%%%%%%%%%%%%%%%%%%%%%%%%%%%%%%%%%%%%%%%%%%%%%%%%%%%%%%%

\begin{zztask}[Приближённые вещественные числа]
В рамках общего условия задачи ввести новый прозрачный тип <<приближённое число>>
(\texttt{approx\_t}) на основе двух значений \texttt{lo}, \texttt{hi} типа
\texttt{double} и реализовать функциональность:
%
\begin{itemize}
\item FromDouble: по центру и радиусу; FromRange, AsDouble, GetRadius;
\item Parse, Format, Read, Write, в формате \verb|"<0.75; 1.25>"|;
\item Add, Sub, Mul, Div, Negate, Abs; Compare;
\end{itemize}
\end{zztask}

%%%%%%%%%%%%%%%%%%%%%%%%%%%%%%%%%%%%%%%%%%%%%%%%%%%%%%%%%%%%%%%%%%%%%%%%%%%%%%

\begin{zztask}[Матрицы $2\times 2$]
В рамках общего условия задачи ввести новый прозрачный тип <<матрица $2\times 2$>>
(\texttt{matrix22\_t}) на основе значений \texttt{a}, \texttt{b}, \texttt{c},
\texttt{d} типа \texttt{double} и реализовать функциональность:
%
\begin{itemize}
\item константы ZERO, IDENTITY;
\item SetZero, SetIdentity: заполнение как нулевой и единичной;
\item Parse, Format, Read, Write,\\ в формате \verb|"{{1, 0}, {-3.5, 1}}"|;
\item Add, Sub, Mul, MulDouble;
\item Determinant, Trace, Inverse, Transpose;
\item Element: доступ к элементу $A_{ij}$;
\end{itemize}
\end{zztask}

%%%%%%%%%%%%%%%%%%%%%%%%%%%%%%%%%%%%%%%%%%%%%%%%%%%%%%%%%%%%%%%%%%%%%%%%%%%%%%

\begin{zztask}[Матрицы произвольного размера]
В рамках общего условия задачи ввести новый непрозрачный тип <<матрица>>
(\texttt{matrix\_t}) на основе значений типа \texttt{double} и реализовать
функциональность:
%
\begin{itemize}
\item Create, Destroy: в динамической памяти;
\item SetZero, SetIdentity: заполнение как нулевой и единичной;
\item Parse, Format, Read, Write,\\ в формате \verb|"{{1, 0}, {-3.5, 1}}"|;
\item Add, Sub, Mul, MulDouble;
\item \textit{Determinant}${}^{(\ast)}$, Trace, \textit{Inverse}${}^{(\ast)}$, Transpose;
\item Element: доступ к элементу $A_{ij}$;
\end{itemize}
%
Примечание: в качестве типа рекомендуется использовать структуру с тремя
полями, хранящими число строк, столбцов и указатель на линейную память под
элементы.
\end{zztask}

%%%%%%%%%%%%%%%%%%%%%%%%%%%%%%%%%%%%%%%%%%%%%%%%%%%%%%%%%%%%%%%%%%%%%%%%%%%%%%

\begin{zztask}[Многочлены]
В рамках общего условия задачи ввести новый непрозрачный тип <<многочлен>>
(\texttt{poly\_t}) на основе значений типа \texttt{int} и реализовать
функциональность:
%
\begin{itemize}
\item Create, Destroy: по массиву, в динамической памяти;
\item From: по N коэффициентам, с переменным числом аргументов;
\item Parse, Format, Read, Write, в формате \verb|"2x^3 - 5x + 7"|;
\item Add, Sub, Mul, MulInt; Evaluate: вычислить в точке;
\item Element: доступ к элементу $A_{i}$;
\end{itemize}
%
Примечание: в качестве типа можно использовать структуру с двумя полями,
хранящими степень многочлена и указатель на выделенную память под
коэффициенты.
\end{zztask}

%%%%%%%%%%%%%%%%%%%%%%%%%%%%%%%%%%%%%%%%%%%%%%%%%%%%%%%%%%%%%%%%%%%%%%%%%%%%%%

\begin{zztask}[Векторы в 2D]
В рамках общего условия задачи ввести новый прозрачный тип <<вектор на плоскости>>
(\texttt{vec2\_t}) на основе значений \texttt{x}, \texttt{y}
типа \texttt{double} и реализовать функциональность:
%
\begin{itemize}
\item константы ZERO, UNIT\_X, UNIT\_Y;
\item Parse, Format, Read, Write, в формате \verb|"{1.5, -3.5}"|;
\item Add, Sub, MulDouble, Negate, Length, Normalize;
\item Dot, Cross: скалярное и векторное умножение;
\item TurnLeft, TurnRight, Rotate: повороты на $90^\circ$ и на произвольный угол;
\end{itemize}
\end{zztask}

%%%%%%%%%%%%%%%%%%%%%%%%%%%%%%%%%%%%%%%%%%%%%%%%%%%%%%%%%%%%%%%%%%%%%%%%%%%%%%

\begin{zztask}[Строки]
В рамках общего условия задачи ввести новый непрозрачный тип <<динамическая строка>>
(\texttt{string\_t}) и реализовать функциональность:
%
\begin{itemize}
\item Create, Destroy: по Си-строке, в динамической памяти;
\item Parse, ParseWord, Read, ReadWord, Write: отдельно чтение до перевода строки и до пробельного символа;
\item Length, Copy, Append, Compare,\dots;
\item Element: доступ к элементу $S_{i}$;
\end{itemize}
\end{zztask}

%%%%%%%%%%%%%%%%%%%%%%%%%%%%%%%%%%%%%%%%%%%%%%%%%%%%%%%%%%%%%%%%%%%%%%%%%%%%%%

\begin{zztask}[Вещественные числа с десятичной точкой]
В рамках общего условия задачи ввести новый прозрачный тип <<с фиксированной точкой>>,
но не в двоичной, а в десятичной форме (\texttt{fixed10\_t}), на основе типа \texttt{int}
с двумя десятичными знаками после точки и реализовать функциональность:
%
\begin{itemize}
\item константа ZERO;
\item FromInt, FromDouble, AsInt, AsDouble;
\item Parse, Format, Read, Write, в обычном формате \verb|"Dec[-12.75]"|;
\item Add, Sub, Mul, Div, Negate, Reciprocal, Abs;
\item Round, Floor, Ceil: округления; Compare;
\end{itemize}
\end{zztask}

%%%%%%%%%%%%%%%%%%%%%%%%%%%%%%%%%%%%%%%%%%%%%%%%%%%%%%%%%%%%%%%%%%%%%%%%%%%%%%
