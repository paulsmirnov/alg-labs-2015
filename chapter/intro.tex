\chapter{Предисловие}

Данное учебное пособие содержит ряд задач по программированию разных уровней сложности,
охватывая основные разделы\dots


Некоторые работы (\texttt{XPR}, \texttt{SEQ}, \texttt{INT}) являются достаточно простыми, чтобы
выполнять их прямо в классе под руководством преподавателя. Другие требуют больше времени и могут
быть заданы в качестве домашнего задания на одну-две недели. Задание \texttt{GAM}, как творческое и
большое по объему, может использоваться даже в качестве курсового проекта.

Важно понимать, что выполнения приведённых в книге заданий недостаточно для успешной учёбы.
Программированию, как работе с любым профессиональным инструментом, можно научиться, только
используя его постоянно, в том числе и для решения возникающих бытовых задач. Если не хватает
фантазии придумывать задачи самому, можно обратиться к специализированным интернет-сайтам,
таким как Timus, Codeforces, Topcoder и другие. Эти площадки содержат большой архив заданий
и периодически организуют соревнования, позволяя как набрать опыт, так и проявить себя.

В Приложении А приведены шаги по настройке среды разработки для создания простых консольных приложений.
На момент написания пособия подавляющее большинство
персональных компьютеров работает под управлением операционной системы Microsoft Windows 7 или 10 для
которой естественной средой разработки является свободно распространяемая Microsoft Visual Studio Community 2017.
Тем не менее, в книге также приводятся подсказки по работе в Apple Xcode 8 на компьютерах Mac и
в командной строке при использовании компилятора \texttt{gcc}.

В Приложении B приведён пример формальных требований к стилю написания программ.
По современным представлениям, программы пишутся в первую очередь для людей, и только потом~---
для машин~\cite{mcconnel2015code}. Соблюдение стандарта кодирования --- это важная составляющая
профессиональной разработки. Аккуратный <<почерк>> и стиль изложения способствует ясности мысли,
упрощению понимания кода коллегами-программистами, позволяет избегать типичных ошибок
и в итоге получать более качественный и дешёвый продукт.
