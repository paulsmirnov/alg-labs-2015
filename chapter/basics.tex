%%%%%%%%%%%%%%%%%%%%%%%%%%%%%%%%%%%%%%%%%%%%%%%%%%%%%%%%%%%%%%%%%%%%%%%%%%%%%%
%%%%%%%%%%%%%%%%%%%%%%%%%%%%%%%%%%%%%%%%%%%%%%%%%%%%%%%%%%%%%%%%%%%%%%%%%%%%%%
\chapter{Структурное программирование}
%%%%%%%%%%%%%%%%%%%%%%%%%%%%%%%%%%%%%%%%%%%%%%%%%%%%%%%%%%%%%%%%%%%%%%%%%%%%%%
%%%%%%%%%%%%%%%%%%%%%%%%%%%%%%%%%%%%%%%%%%%%%%%%%%%%%%%%%%%%%%%%%%%%%%%%%%%%%%


Ни одного штриха не мог бы я сделать, а никогда не был таким большим
художником, как в эти минуты~\cite{podbelsky2015kurs}.


%%%%%%%%%%%%%%%%%%%%%%%%%%%%%%%%%%%%%%%%%%%%%%%%%%%%%%%%%%%%%%%%%%%%%%%%%%%%%%
\section{Минимальная программа}
%%%%%%%%%%%%%%%%%%%%%%%%%%%%%%%%%%%%%%%%%%%%%%%%%%%%%%%%%%%%%%%%%%%%%%%%%%%%%%


Программа на языке Си оформляется в виде набора функций, совместно решающих
поставленную задачу. Работа программы начинается с главной функции, которая
выполняет необходимые команды, вызывает другие функции, изучает результаты их
выполнения, поэтому каждая программа должна содержать как минимум одну
функцию~--- главную, которая всегда называется \texttt{main()}. Минимальная
программа на языке Си в соответствии с международным стандартом
1989~года\footnote{ANSI X3.159--1989 ``Programming Language C''} выглядит
следующим образом:

\begin{minted}{c}
int main(void)
{
  return 0;
}
\end{minted}

Ваша программа с точки зрения операционной системы (ОС) является
подпрограммой, вызываемой для решения конкретной задачи. Этой <<подпрограмме>>
могут передаваться исходные данные (в виде <<параметров>>), а от неё ожидается
результат (в виде <<возвращаемого значения>>). В приведённом минимальном
примере наша программа не ожидает никакой исходной информации от ОС
(\texttt{void} в скобках), но возвращает ей целое число (\texttt{int} перед
именем функции), которое системой будет рассматриваться как признак ошибки. По
договорённости \texttt{0} считается успешным завершением программы, а любое
ненулевое значение является кодом ошибки. Не существует никаких
предопределённых кодов, автор каждой программы сам решает для себя, какие
ошибки он будет обозначать какими значениями.


%%%%%%%%%%%%%%%%%%%%%%%%%%%%%%%%%%%%%%%%%%%%%%%%%%%%%%%%%%%%%%%%%%%%%%%%%%%%%%
\section{Вычисление выражений}
%%%%%%%%%%%%%%%%%%%%%%%%%%%%%%%%%%%%%%%%%%%%%%%%%%%%%%%%%%%%%%%%%%%%%%%%%%%%%%

Компьютеры, да и языки программирования, были созданы в первую очередь для
автоматизации вычислений. Исходные данные для вычислений могут поступать из
разных источников от разных периферийных устройств, исторически сложилось так,
что в качестве устройства ввода-вывода выступает консоль --- совокупность
клавиатуры для ввода и экрана для вывода текста. Современные системы позволяют
большее, но для этого приходится прикладывать и больше усилий. В учебных целях
ограничимся пока консольными программами.

\zzneedspace
Вместо надоевшего <<Hello, world!>> рассмотрим следующий пример:
%
\inputminted{c}{samples/feet_bad.c}

Самая первая строчка включает (\verb|#include|) \textit{заголовочный файл} с
именем\linebreak \texttt{stdio.h}, тем самым объясняя компилятору, что программа
пользуется стандартным вводом-выводом. В начале функции идёт определение
переменной \texttt{h} типа \texttt{int}. Затем, со стандартного устройства
ввода (т.е. клавиатуры) считывается (\texttt{scanf}) одно целое число
(\texttt{\%i}) в подготовленную переменную (\verb|&h|). После чего печатается
(\texttt{printf}) одно вещественное число (\texttt{\%lf}) и курсор переводится
на следующую строку (\verb|\n|), при выводе в качестве вещественного числа
подставляется результат деления $h / 30.48$.

Не очень понятно, зачем программа всё это делает. И пользователю так же
непонятно, как с ней работать. При запуске программы он увидит чёрный экран и
мигающий курсор. Если нажимать на кнопки, в месте, куда указывает курсор на
экране будут появляться символы. Зачем? И что дальше? Программа недружелюбна
как по отношению к пользователю, так и по отношению к программисту, который,
возможно, будет читать этот код. Лучше будет, если программу немного
поправить:
%
\inputminted{c}{samples/feet.c}

С говорящими именами, комментариями и дополнительно выводимым текстом стало
гораздо понятнее и пользователю, и программисту. Мы определили
(\verb|#define|) константу \verb|FEET_LENGTH| (<<длина фута>>), которая везде
будет заменяться на значение 30.48. Переменную назвали \texttt{height}
(<<высота>>), добавили для пользователя вывод приглашения перед ожиданием
вводимых данных и выводим не просто число, а понятную фразу. Главная функция
разбита на два осмысленных блока, которые встречаются практически во всех
программах: ввод данных и вывод результата. Блоки предваряются короткими
поясняющими комментариями между \verb|/*| и \verb|*/|. Диалог с программой
может выглядеть следующим образом:
%
\begin{verbatim}
    What is your height in cm? 200
    Your height is 6.561680 feet
\end{verbatim}

Если немного постараться, можно научить программу говорить и по-русски. Для
этого воспользуемся функцией \texttt{setlocale()}, объявление которой
находится в заголовочном файле \texttt{locale.h}. Заодно, немного усложним
вычисления и вынесем их в отдельный блок:
%
\inputminted{c}{samples/feet_rus.c}

В новом варианте программы используются однострочные комментарии, которые не
входили в первый стандарт языка Си, но были введены в более поздних стандартах
Си++ и Си, и сейчас поддерживаются практически всеми компиляторами.
Кроме пары целых переменных через запятую мы завели ещё две вещественные
переменные двойной точности (\texttt{double}). В формуле воспользовались
приведением к целому значению \verb|(int)|. Также в конце программы выводим
целое число (\verb|%i|) футов и вещественное число оставшихся дюймов с
точностью до одного знака после десятичной точки (\verb|%.1lf|). Для этого при
вызове функции \texttt{printf()} нужным нам образом составляем форматную
строку (первый аргумент) и через запятую перечисляем все аргументы, которые
будут поставляться по порядку на подготовленные места.

Кроме четырёх арифметических действий в языке Си присутствует операция взятия
остатка от деления, обозначаемая процентом: \texttt{11 \% 3} равно 2.
Сама операция деления очень опасна для новичков --- при делении целых чисел
друг на друга она действует как целочисленное деление, отбрасывая дробную часть.
Если же один из аргументов вещественный, то и результат будет вещественным, поэтому
при необходимости точного деления целых чисел следует воспользоваться операцией
приведения типа:
%
\inputminted{c}{samples/equation.c}

Также, в формулах можно использовать математические функции: извлечение
квадратного корня (\texttt{sqrt()}), разную тригонометрию (\texttt{sin()}),
натуральный логарифм (\texttt{log()}), экспоненту (\texttt{exp()}). Эти и
другие функции становятся доступными при включении заголовочного файла
\texttt{math.h}.


%%%%%%%%%%%%%%%%%%%%%%%%%%%%%%%%%%%%%%%%%%%%%%%%%%%%%%%%%%%%%%%%%%%%%%%%%%%%%%
\section{Управление программным потоком}
%%%%%%%%%%%%%%%%%%%%%%%%%%%%%%%%%%%%%%%%%%%%%%%%%%%%%%%%%%%%%%%%%%%%%%%%%%%%%%

Не все программы можно записать в виде такой линейной последовательности
команд, выполняющихся друг за другом от начала и до конца, поэтому
императивные языки программирования предлагают возможности по управлению
порядком выполнения команд. В Си поддерживаются все основные блоки
структурного программирования --- <<последовательность>>, <<выбор>>,
<<повторение>> и <<подпрограммы>>.

\zzneedspace
Пример на ветвление потока управления (условное выполнение команд):
%
\inputminted{c}{samples/01_if.c}

Все структурные команды языка Си могут иметь только одну подчинённую команду.
Если необходимо выполнять несколько команд, их надо группировать с помощью
фигурных скобок как в примере выше. Операции сравнения в языке Си: меньше
(\verb|<|), меньше или равно (\verb|<=|), больше (\verb|>|), больше или равно
(\verb|>=|), равно (\verb|==|), не равно (\verb|!=|). Условия можно объединять
логическими связками: и (\verb|&&|), или (\verb/||/). Для отрицания
используется восклицателный знак (\verb|!|). Альтернативное действие,
включая слово \texttt{else}, может отсутствовать.

\zzneedspace
Пример на повторение команд с предварительной проверкой условия (цикл с
предусловием):
%
\inputminted{c}{samples/02_while.c}

Здесь использованы две новые операции: уменьшение на единицу (\verb|--|) и
увеличение на некоторый шаг (\verb|+=|). Вместо них можно было бы написать
\verb|n = n - 1| и \verb|sum = sum + n|. Последняя форма доступна для многих
операций (напр., \verb|a -= b|, \verb|b /= 2|) и имеет аналогичный смысл.

\zzneedspace
Цикл с постусловием:
%
\inputminted{c}{samples/03_do_while.c}

\zzneedspace
Цикл со счётчиком:
%
\inputminted{c}{samples/04_for.c}

На самом деле, цикл \texttt{for} в языке Си более гибок, чем в цикл со
счётчиком в других языках программирования, но для начала такой формы
нам должно хватить.

В качестве допустимого отклонения от структурного программирования в языке
присутствуют ставшие уже традиционными команды преждевременного прерывания
цикла (\texttt{break}) и пропуска итерации (\texttt{continue}):
\zzneedspace
\inputminted{c}{samples/05_break.c}

\zzneedspace
Ну и наконец, следует вернуться к командам выбора и упомянуть команду
\verb|switch|, позволяющую сравнить одну переменную или выражение
с несколькими константами для выбора из нескольких альтернатив:
%
\inputminted{c}{samples/06_switch.c}

Такой подход понятнее, чем последовательность \texttt{if} и для программиста,
и для компилятора, позволяет сгенерировать более оптимальный исполняемый код.


%%%%%%%%%%%%%%%%%%%%%%%%%%%%%%%%%%%%%%%%%%%%%%%%%%%%%%%%%%%%%%%%%%%%%%%%%%%%%%
\section{Пользовательские функции}
%%%%%%%%%%%%%%%%%%%%%%%%%%%%%%%%%%%%%%%%%%%%%%%%%%%%%%%%%%%%%%%%%%%%%%%%%%%%%%

Парадигма процедурного и структурного программирования подразумевает выделение
блоков команд, решающих выделенные задачи, в отдельные подпрограммы --- для
большей ясности и для возможного повторного использования. По аналогии с
функцией \texttt{main()} можно определять собственные функции, которые и
выполняют роль подпрограмм в языке Си. При наличии параметра его тип и имя
указывается в круглых скобках в заголовке функции. В начале тела каждой
функции можно определять локальные переменные, которые существуют и видны
только внутри соответствующих фигурных скобок. Выход из функции с возвратом
значения возможен из любого места функции при помощи команды \texttt{return}.
%
\inputminted{c}{samples/factorial.c}

%\zzneedspace
При наличии нескольких параметров они перечисляются через запятую, тип
указывается для каждого параметра в отдельности.
%
\inputminted{c}{samples/power.c}

Здесь была использована операция постфиксного декремента, которая также
уменьшает переменную на 1, но при этом в выражение для сравнения подставляется
старое значение (ещё до уменьшения).


%%%%%%%%%%%%%%%%%%%%%%%%%%%%%%%%%%%%%%%%%%%%%%%%%%%%%%%%%%%%%%%%%%%%%%%%%%%%%%
%%%%%%%%%%%%%%%%%%%%%%%%%%%%%%%%%%%%%%%%%%%%%%%%%%%%%%%%%%%%%%%%%%%%%%%%%%%%%%
\zztaskgroup{XPR}{Вычисление выражений и взаимодействие с пользователем}
%%%%%%%%%%%%%%%%%%%%%%%%%%%%%%%%%%%%%%%%%%%%%%%%%%%%%%%%%%%%%%%%%%%%%%%%%%%%%%
\begin{itemize}
	\item Объявление переменных?СПОРНО
	\item Функциий вывода текста в консоль printf. Вывести текст приветствия. Объявить и вывести переменную с дефолтным значением. Про форматную строчку и перечисление аргументов.
	\item Функции считывания с клавиатуры. scanf. Такая же форматная строчка, упомянуть амперсанд. Можно сделать ссылку на какую-нибудь работу с указателями.
	\item Ветвление. Немного напомнить синтаксис if. Сделать одношаговую штуку с одним ветвлением. 
	\item Циклы. не знаю, нужно ли все показывать, можно показать, как сделать бесконечные форы, вайлы, дувайлы.
\end{itemize}
В следующих задачах требуется написать программу, автоматизирующую вычисления
по заданной формуле. Формула может включать в себя несколько заранее известных
констант и заранее неизвестных переменных (параметров), которые должен ввести
в программу пользователь после соответствующего приглашения. Запуская
программу несколько раз, вводя разные значения параметров, пользователь сможет
получать соответствующее этим параметрам решение задачи.

Приложение должно быть дружелюбным к пользователю (user friendly), то есть
вести с ним разумный диалог. Пользователь в каждый момент времени должен
знать, чего ожидает от него программа. Программа должна производить проверку
корректности исходных данных, вводимых пользователем.
Обычно вводимые величины имеют минимум одно ограничение --- масса, длина,
скорость должны быть неотрицательны.
В случае недопустимых данных программа должна выдавать осмысленное сообщение
и повторно запрашивать соответствующее значение. Запросы должны повторяться
до тех пор пока пользователь не введет корректное значение. Так должно происходить
для каждого вводимого параметра.

Примеры диалога программы и пользователя:

\begin{zzoutput}
  Задание \thezztaskgroup-1: Диаметр шара заданной массы и плотности
  Введите массу (кг): \zzuser{-1.5}
  Ошибка ввода, масса должна быть больше нуля!
  Введите массу (кг): \zzuser{0}
  Ошибка ввода, масса должна быть больше нуля!
  Введите массу (кг): \zzuser{1.5}
  Введите плотность (кг/м3): \zzuser{0}
  Ошибка ввода, плотность должна быть больше нуля!
  Введите плотность (кг/м3): \zzuser{19320}
  Диаметр получившегося шара (м): d = 0.052929
\end{zzoutput}
 
Рекомендуется проверять программу на реальных значениях параметров, при
которых ответ на задачу заранее известен или может быть легко проверен на
адекватность. В тестировании могут пригодиться некоторые табличные данные.
Для сведения:
%
\begin{itemize}
%
\item плотность~$\rho$ золота, железа, льда и пробки ---
19320, 7870, 916 и 240~кг/м\textsuperscript{3} соответственно;
%
\item ускорение свободного падения $g$ примерно равно 
9.8~м/с\textsuperscript{2};
%
\item человек привык измерять углы в градусах, а компьютер ---
в радианах, это уже учтено в предлагаемых ниже формулах;
полный круг $360^{\circ}$ равен $2\pi$ радиан ($\pi = 3.14159265358979323846\dots$);
%
\item человек привык измерять проценты в диапазоне от 0 до 100\%,
а компьютер -- в долях, от 0 до 1, что тоже учитывают формулы.
\end{itemize}


%%%%%%%%%%%%%%%%%%%%%%%%%%%%%%%%%%%%%%%%%%%%%%%%%%%%%%%%%%%%%%%%%%%%%%%%%%%%%%
\bigskip
%%%%%%%%%%%%%%%%%%%%%%%%%%%%%%%%%%%%%%%%%%%%%%%%%%%%%%%%%%%%%%%%%%%%%%%%%%%%%%


\begin{zztask} \zztodo{Тест примечаний к условиям задач}
Какого диаметра получится шар массы $m$, изготовленный из материала\zztodo{во даёт!} с
плотностью~$\rho$?
%
\[
d = 2\cdot \sqrt[3]{\frac{3m}{4\pi\rho}}
\]
\end{zztask}

%%%%%%%%%%%%%%%%%%%%%%%%%%%%%%%%%%%%%%%%%%%%%%%%%%%%%%%%%%%%%%%%%%%%%%%%%%%%%%

\begin{zztask}
Какой массы получится шар диаметра $d$, изготовленный из материала с
плотностью~$\rho$?
%
\[
m = \frac{1}{6}\pi\rho d^3
\]
\end{zztask}

%%%%%%%%%%%%%%%%%%%%%%%%%%%%%%%%%%%%%%%%%%%%%%%%%%%%%%%%%%%%%%%%%%%%%%%%%%%%%%

\begin{zztask}
Какого диаметра получится стержень массы $m$ и длины $l$, изготовленный из
материала с плотностью~$\rho$?
%
\[
d = 2\cdot \sqrt{\frac{m}{\pi\rho l}}
\]
\end{zztask}

%%%%%%%%%%%%%%%%%%%%%%%%%%%%%%%%%%%%%%%%%%%%%%%%%%%%%%%%%%%%%%%%%%%%%%%%%%%%%%

\begin{zztask}
Какой длины получится стержень массы $m$ диаметром $d$, изготовленный из
материала с плотностью~$\rho$?
%
\[
l = \frac{4m}{\pi\rho d^2}
\]
\end{zztask}

%%%%%%%%%%%%%%%%%%%%%%%%%%%%%%%%%%%%%%%%%%%%%%%%%%%%%%%%%%%%%%%%%%%%%%%%%%%%%%

\begin{zztask}
Какой массы получится стержень диаметром $d$ длиной $l$, изготовленный из
материала с плотностью~$\rho$?
%
\[
m = \frac{1}{4}\pi\rho d^2 l
\]
\end{zztask}

%%%%%%%%%%%%%%%%%%%%%%%%%%%%%%%%%%%%%%%%%%%%%%%%%%%%%%%%%%%%%%%%%%%%%%%%%%%%%%

\begin{zztask}
Какой массы получится пирамида с длиной ребра $a$, изготовленная из
материала с плотностью~$\rho$?
%
\[
m = \frac{\sqrt2}{12}\rho a^3
\]
\end{zztask}

%%%%%%%%%%%%%%%%%%%%%%%%%%%%%%%%%%%%%%%%%%%%%%%%%%%%%%%%%%%%%%%%%%%%%%%%%%%%%%

\begin{zztask}
Какой высоты получится пирамида массы $m$, изготовленная из
материала с плотностью~$\rho$?
%
\[
h = \frac{\sqrt6}{3} \sqrt[3]{\frac{6\sqrt2 m}{\rho}}
\]
\end{zztask}

%%%%%%%%%%%%%%%%%%%%%%%%%%%%%%%%%%%%%%%%%%%%%%%%%%%%%%%%%%%%%%%%%%%%%%%%%%%%%%

\begin{zztask}
Как далеко улетит тело, если его бросить под углом $\alpha$ к горизонту
со скоростью $v$?
%
\[
l = \frac{v^2}{g}\sin\frac{\alpha\pi}{90}
\]
\end{zztask}

%%%%%%%%%%%%%%%%%%%%%%%%%%%%%%%%%%%%%%%%%%%%%%%%%%%%%%%%%%%%%%%%%%%%%%%%%%%%%%

\begin{zztask}
Под каким углом к горизонту нужно бросить тело, чтобы оно улетело на
расстояние $l$ со скоростью $v$?
%
\[
\alpha = \frac{90}{\pi}\arcsin\frac{lg}{v^2}
\]
\end{zztask}

%%%%%%%%%%%%%%%%%%%%%%%%%%%%%%%%%%%%%%%%%%%%%%%%%%%%%%%%%%%%%%%%%%%%%%%%%%%%%%

\begin{zztask}
С какой скоростью нужно бросить тело под углом $\alpha$ к горизонту,
чтобы оно улетело на расстояние $l$?
%
\[
v = \sqrt{\frac{lg}{\sin(\alpha\pi/90)}}
\]
\end{zztask}

%%%%%%%%%%%%%%%%%%%%%%%%%%%%%%%%%%%%%%%%%%%%%%%%%%%%%%%%%%%%%%%%%%%%%%%%%%%%%%

\begin{zztask}
Какой доход за год принесет сумма $m$ на счёте в банке при годовой
процентной ставке $p$, если вклад подразумевает ежемесячную
капитализацию процентов?
%
\[
i = m \left(1 + \frac{p}{1200}\right)^{12} - m
\]
\end{zztask}

%%%%%%%%%%%%%%%%%%%%%%%%%%%%%%%%%%%%%%%%%%%%%%%%%%%%%%%%%%%%%%%%%%%%%%%%%%%%%%

\begin{zztask}
Сколько денег надо положить на счет в банк, чтобы при ежемесячной
капитализации процентов получить годовой доход $i$ при процентной
ставке $p$?
%
\[
m = \frac{i}{\left(1 + p/1200\right)^{12} - 1}
\]
\end{zztask}

%%%%%%%%%%%%%%%%%%%%%%%%%%%%%%%%%%%%%%%%%%%%%%%%%%%%%%%%%%%%%%%%%%%%%%%%%%%%%%

\begin{zztask}
Какой должна быть ежемесячная выплата по кредиту, если сумма $m$ берётся 
под процент $p$ на $k$ месяцев?
%
\[
a = \frac{mp}{1200}\left(1 + \frac{1}{(1 + p/1200)^k - 1}\right)
\]
\end{zztask}

%%%%%%%%%%%%%%%%%%%%%%%%%%%%%%%%%%%%%%%%%%%%%%%%%%%%%%%%%%%%%%%%%%%%%%%%%%%%%%

\begin{zztask}
Какой будет суммарная переплата по кредиту, если сумма $m$ берётся
под процент $p$ на $k$ месяцев?
%
\[
i = \frac{mkp}{1200}\left(1 + \frac{1}{(1 + p/1200)^k - 1}\right) - m
\]
\end{zztask}

%%%%%%%%%%%%%%%%%%%%%%%%%%%%%%%%%%%%%%%%%%%%%%%%%%%%%%%%%%%%%%%%%%%%%%%%%%%%%%

\begin{zztask}
На какую максимальную сумму кредита под процент $p$ на $k$ месяцев
можно расчитывать, если есть возможность отдавать ежемесячно сумму $a$?
%
\[
m = \frac{a}{p/1200}\left(1 - \frac{1}{(1 + p/1200)^k}\right)
\]
\end{zztask}

%%%%%%%%%%%%%%%%%%%%%%%%%%%%%%%%%%%%%%%%%%%%%%%%%%%%%%%%%%%%%%%%%%%%%%%%%%%%%%


%%%%%%%%%%%%%%%%%%%%%%%%%%%%%%%%%%%%%%%%%%%%%%%%%%%%%%%%%%%%%%%%%%%%%%%%%%%%%%
\zztaskgroup{SEQ}{Последовательность}
%%%%%%%%%%%%%%%%%%%%%%%%%%%%%%%%%%%%%%%%%%%%%%%%%%%%%%%%%%%%%%%%%%%%%%%%%%%%%%

В следующих задачах требуется написать программу, находящую сумму первых $n$
чисел последовательности, строящейся по определенному правилу ($n$ вводится в программу
пользователем). Сумму надо посчитать двумя способами, складывая числа в цикле и
используя замкнутую\zztodo{Готовую? Формулу частичной суммы?} формулу, выведенную из математических соображений, если это возможно. % \zztodo{А почему написано "заданную формулу", а сами формулы закомменчены?}
Результаты сравнить. На экран также вывести
суммируемые числа, разделенные знаками <<плюс>> (или <<минус>>, если число
отрицательное).

Примечание: в некоторых задачах формулу вывести сложно или невозможно, это
указано в их условии. % \zztodo{Что изучается в этой лабе? Циклы? Вместо чего ее можно давать? вместо вычисления выражения для сильных или как? Довольно простая задачка, может ли на классную пойти?}

Примеры диалога программы и пользователя:

\begin{zzoutput}
  Задание \thezztaskgroup-1: Суммирование N натуральных чисел
  Введите N: \zzuser{5}
  Сумма равна 1 + 2 + 3 + 4 + 5 = 15 (и 15 по формуле)
\end{zzoutput}


%%%%%%%%%%%%%%%%%%%%%%%%%%%%%%%%%%%%%%%%%%%%%%%%%%%%%%%%%%%%%%%%%%%%%%%%%%%%%%
\zzsectionCOMMENTS
%%%%%%%%%%%%%%%%%%%%%%%%%%%%%%%%%%%%%%%%%%%%%%%%%%%%%%%%%%%%%%%%%%%%%%%%%%%%%%

\paragraph{Накопитель}
Стандартный подход к суммированию чисел --- это схема с накопителем: изначально
накопитель пуст, а затем, в цикле на каждом шаге к нему прибавляется очередной
элемент последовательности. Этот элемент в большинстве случаев легко получается
из предыдущего: в арифметической прогрессии, например, последовательные элементы
отличаются на константу.

\paragraph{Плюс или минус}
Если вынести первый член последовательности из цикла (он обычно задан, а не
вычисляется по предыдущим), то числа надо не разделять, а предварять знаками,
что гораздо проще. Какой именно знак поставить несложно определить по знаку
самого числа, а выводить уже модуль числа (для целых чисел в языке Си есть
функция \texttt{abs()}).


%%%%%%%%%%%%%%%%%%%%%%%%%%%%%%%%%%%%%%%%%%%%%%%%%%%%%%%%%%%%%%%%%%%%%%%%%%%%%%
\zzsectionPLAN
%%%%%%%%%%%%%%%%%%%%%%%%%%%%%%%%%%%%%%%%%%%%%%%%%%%%%%%%%%%%%%%%%%%%%%%%%%%%%%

Набор шагов, на которые рекомендуется разбить задачу во время решения:
\begin{enumerate}
\item Сначала необходимо написать простейшую часть --- вычисление по формуле. Эту чать будем в дальнейшем использовать для сверки результатов.
%
\item Собственно, вычисление частичных сумм. При использовании накопителя состоит из следующих пунктов:
\begin{itemize}
	\item Инициализация накопителя.
	\item Приращение накопителя. Здесь происходит один шаг в продвижении по последовательности.
	\item Условие на количество шагов. Изменение накопителя должно происходить лишь необходимое для решения задачи число раз.
\end{itemize}
\end{enumerate}


%%%%%%%%%%%%%%%%%%%%%%%%%%%%%%%%%%%%%%%%%%%%%%%%%%%%%%%%%%%%%%%%%%%%%%%%%%%%%%
\zzsectionVARIATIONS
%%%%%%%%%%%%%%%%%%%%%%%%%%%%%%%%%%%%%%%%%%%%%%%%%%%%%%%%%%%%%%%%%%%%%%%%%%%%%%


\begin{zztask}[Натуральные числа]
В рамках общего условия задачи найти сумму первых $n$ натуральных чисел
(первые числа: $1$, $2$, $3$, $4$, $5$\dots).
Формула: $s = n(n+1)/2$.
\end{zztask}

%%%%%%%%%%%%%%%%%%%%%%%%%%%%%%%%%%%%%%%%%%%%%%%%%%%%%%%%%%%%%%%%%%%%%%%%%%%%%%

\begin{zztask}[Нечётные числа]
В рамках общего условия задачи найти сумму первых $n$ нечётных натуральных 
чисел (первые числа: $1$, $3$, $5$, $7$, $9$\dots).
Формула: $s = n^2$.
\end{zztask}

%%%%%%%%%%%%%%%%%%%%%%%%%%%%%%%%%%%%%%%%%%%%%%%%%%%%%%%%%%%%%%%%%%%%%%%%%%%%%%

\begin{zztask}[Степени двойки]
В рамках общего условия задачи найти сумму первых $n$ чисел, являющихся
степенью двойки. Примечание: $k$-ая степень двойки получается из предыдущей
умножением на 2 (первые числа: $1$, $1\cdot2=2$, $2\cdot2=4$, $4\cdot2=8$, 
$16$\dots).
Формула: $s = 2^{n+1}-1$.
\end{zztask}

%%%%%%%%%%%%%%%%%%%%%%%%%%%%%%%%%%%%%%%%%%%%%%%%%%%%%%%%%%%%%%%%%%%%%%%%%%%%%%

\begin{zztask}[Треугольные числа]
В рамках общего условия задачи найти сумму первых $n$ треугольных чисел.
Примечание: $k$-ое треугольное число получается из предыдущего прибавлением
к нему $k$ (первые числа: $1$, $1+2=3$, $3+3=6$, $6+4=10$, $15$, $21$\dots).
Формула: $s = n(n+1)(n+2)/6$.
\end{zztask}

%%%%%%%%%%%%%%%%%%%%%%%%%%%%%%%%%%%%%%%%%%%%%%%%%%%%%%%%%%%%%%%%%%%%%%%%%%%%%%

\begin{zztask}[Прямоугольные числа]
В рамках общего условия задачи найти сумму первых $n$ прямоугольных чисел.
Примечание: прямоугольные числа --- это такие натуральные числа, которые
являются произведением двух последовательных натуральных чисел
(первые числа: $1\cdot2=2$, $2\cdot3=6$, $3\cdot4=12$, $20$, $30$, $42$\dots).
Формула: $s = n(n+1)(n+2)/3$.
\end{zztask}

%%%%%%%%%%%%%%%%%%%%%%%%%%%%%%%%%%%%%%%%%%%%%%%%%%%%%%%%%%%%%%%%%%%%%%%%%%%%%%

\begin{zztask}[Шестиугольные числа]
В рамках общего условия задачи найти сумму первых $n$ шестиугольных чисел.
Примечание: $k$-ое шестиугольное число получается из предыдущего прибавлением
к нему $4k-3$ (первые числа: $1$, $1+5=6$, $6+9=15$, $15+13=28$, $45$, $66$\dots).
Формула: $s = n(n+1)(4n-1)/6$.
\end{zztask}

%%%%%%%%%%%%%%%%%%%%%%%%%%%%%%%%%%%%%%%%%%%%%%%%%%%%%%%%%%%%%%%%%%%%%%%%%%%%%%

\begin{zztask}[Числа Фибоначчи]
В рамках общего условия задачи найти сумму первых $n$ чисел Фибоначчи.
Примечание: первые
два числа в последовательности чисел Фибоначчи это 0 и 1, а каждое следующее
считается как сумма двух предыдущих: $0$, $1$, $0+1=1$, $1+1=2$, $1+2=3$, $2+3=5$,
$8$, $13$\dots
Шаг подсчета по формуле опустить ввиду нетривиальности.
\end{zztask}

%%%%%%%%%%%%%%%%%%%%%%%%%%%%%%%%%%%%%%%%%%%%%%%%%%%%%%%%%%%%%%%%%%%%%%%%%%%%%%

\begin{zztask}[Числа анти-Фибоначчи]
В рамках общего условия задачи найти сумму первых $n$ чисел анти-Фибоначчи.
Примечание: первые
два числа в последовательности чисел анти-Фибоначчи это 1 и 0, а каждое
следующее считается как разность двух предыдущих: $1$, $0$, $1-0=1$, $0-1=-1$,
$1-(-1)=2$, $-3$, $5$\dots
Шаг подсчета по формуле опустить ввиду нетривиальности.
\end{zztask}

%%%%%%%%%%%%%%%%%%%%%%%%%%%%%%%%%%%%%%%%%%%%%%%%%%%%%%%%%%%%%%%%%%%%%%%%%%%%%%

\begin{zztask}[$^\star$Автоморфные числа]
В рамках общего условия задачи найти сумму первых $n$ автоморфных чисел.
Примечание:
автоморфные числа --- это такие натуральные числа, квадрат которых
оканчивается на само число (первые числа: $1$, $5$, $6$, $25$, $76$\dots).
Шаг подсчета по формуле опустить ввиду нетривиальности.  
\end{zztask}

%%%%%%%%%%%%%%%%%%%%%%%%%%%%%%%%%%%%%%%%%%%%%%%%%%%%%%%%%%%%%%%%%%%%%%%%%%%%%%

\begin{zztask}[$^\star$Простые числа]
В рамках общего условия задачи найти сумму первых $n$ простых чисел.
Примечание: простые
числа --- это такие натуральные числа, которые делятся нацело только на 1 и
на себя (первые числа: $1$, $3$, $5$, $7$, $11$\dots).
Шаг подсчета по формуле опустить ввиду нетривиальности.
\end{zztask}

%%%%%%%%%%%%%%%%%%%%%%%%%%%%%%%%%%%%%%%%%%%%%%%%%%%%%%%%%%%%%%%%%%%%%%%%%%%%%%

\begin{zztask}[$^\star$Нечётные числа в диапазоне]
В рамках общего условия задачи найти сумму всех нечётных целых чисел,
лежащих между вещественными границами $a$ и $b$ включительно. Границы вводятся
пользователем с клавиатуры, не обязательно в порядке возрастания.
\end{zztask}

%%%%%%%%%%%%%%%%%%%%%%%%%%%%%%%%%%%%%%%%%%%%%%%%%%%%%%%%%%%%%%%%%%%%%%%%%%%%%%


%%%%%%%%%%%%%%%%%%%%%%%%%%%%%%%%%%%%%%%%%%%%%%%%%%%%%%%%%%%%%%%%%%%%%%%%%%%%%%
\zztaskgroup{SER}{Сумма бесконечного ряда}
%%%%%%%%%%%%%%%%%%%%%%%%%%%%%%%%%%%%%%%%%%%%%%%%%%%%%%%%%%%%%%%%%%%%%%%%%%%%%%

В следующих задачах требуется написать программу, выводящую таблицу из $M$
значений некоторой функции $f(x)$ на промежутке от $a$ до $b$ (равномерно распределённых, в том числе и в
граничных точках). Все параметры ($a$, $b$, $M$) задаются пользователем,
функция в каждой задаче своя, причём заданная как сумма бесконечного ряда.
%
\[
  f(x)=S(x)=\sum_{k=0}^\infty u_k(x)=u_0(x)+u_1(x)+\cdots+u_k(x)+\cdots
\]
%
Для сравнения $f(x)$ также задана аналитически, через элементарные функции, и
условие на $x$, при котором это верно. Требуется вывести таблицу из $M$ строк
(по одной строке на значение $x \in [a,b]$), содержащих значение $x$, сумму
$S_n(x)$, точное значением $f(x)$, разницу $\Delta = |S_n(x) - f(x)|$,
количество потребовавшихся членов суммы $n$ и значение
$\varepsilon(S_n)=u_{n+1}$ (описание параметра $\varepsilon$ см. ниже).


%%%%%%%%%%%%%%%%%%%%%%%%%%%%%%%%%%%%%%%%%%%%%%%%%%%%%%%%%%%%%%%%%%%%%%%%%%%%%%
\zzsectionCOMMENTS
%%%%%%%%%%%%%%%%%%%%%%%%%%%%%%%%%%%%%%%%%%%%%%%%%%%%%%%%%%%%%%%%%%%%%%%%%%%%%%


\paragraph{Машинный эпсилон}
В данной задаче бесконечное число слагаемых складывается в течение бесконечного
времени, что делает задачу нерешаемой в такой формулировке за конечное время. 
Поэтому будем вычислять только частичную сумму $S_n$, где $n$ выбирается так,
чтобы получить максимально возможную для типа \texttt{double} точность
суммирования. Для этого будем использовать существование ``машинного
$\varepsilon$'' $\forall y$, то есть такого числа $\varepsilon(y) > 0$, которое при
прибавлении к $y$ не увеличивает его: $y + \varepsilon(y) = y$.
%
\[
S_n(x)=\sum_{k=0}^n u_k(x),\qquad S_n + u_{n+1} = S_n
\]

\paragraph{Вычисление значений слагаемых}
Поскольку слагаемые содержат возведения в степень и факториалы, которые не
могут быть вычислены в лоб с нужной точностью для достаточно больших $n$, а так же в целях увеличения производительности, необходимо составить формулу для вычисления следующего члена суммы с использованием значения предыдущего:
%
\[
u_k(x) = u_{k-1}(x)\cdot v_k(x)
\]
%
В таком случае отпадает необходимость в реализации собственных функции. Кроме того, вычисление в лоб может привести к переполнению, ввиду того, что промежуточные значения могут быть слишком велики (поразмышляйте, например, о вычислении значения слагаемого $\frac{n!}{(n-1)!}$ при $n = 70$).

\paragraph{Форматирование таблицы}
Итоговая таблица должна представлять собой не просто несколько столбцов числен, а аккуратную таблицу с границами, нарисованную с помощью символов.
Вертикальные линии должны обозначать границы каждого столбца, рекомендуется рисовать их с помощью символа ``|''. Горизонтальные линии обозначают границы таблицы и отделяют заголовок от остальных строк, горизонтальные линии нужно рисовать с помощью символа ``-'', а соединительные элементы - символом ``+''. Пример таблицы:
\begin{minted}{c}
+-----------+--------------+------------+-...
|     x     |     f(x)     |    s(x)    |
+-----------+--------------+------------+-...
|      0.10 |      0.00000 |    0.00000 |
|      0.11 |    -10.00001 |  -10.00001 |
|       ... |          ... |        ... |
+-----------+--------------+------------+-...
\end{minted}

\paragraph{Форматирование столбцов}
Для того, чтобы ширина таблицы имела фиксированную величину, необходимо
пользоваться дополнительным модификаторами при использовании функции
\texttt{printf()} (см., например,~\cite{cppref}). Эти модификаторы указываются
после знака $\%$ и перед конструкцией, обозначающей тип параметра, рассмотрим
необходимые нам модификаторы:
%
\begin{itemize}
	\item \textbf{Ширина}. Обозначает минимальное количество символов, которое будет занимать выводимое значение. Если длина значения меньше данной величины, оно будет выровнено по правому краю (для выравнивания по левому краю, используйте символ ``-'', например, \verb|%-8i|). При превышении указанной ширины, значение обрезаться не будет. Например, для вывода числа типа \texttt{double} с шириной поля $10$, необходимо написать \verb|%10lf|.
	\item \textbf{Точность}. Для вывода вещественных чисел может быть полезным использование модификатора точности вывода. Он указывается после модификатора ширины, через точку. Например, конструкция \verb|%10.5f| будет выводить вещественные числа в поле шириной $10$ и с точностью $5$ знаков после запятой.
	\item \textbf{Экспоненциальный формат}. Зачастую, для вывода очень больших или очень маленьких числе с плавающей точкой, вместо модификатора \verb|f| используют \verb|e|, например, при использовании этого модификатора для вывода числа $392.65$, вывод будет следующим: \texttt{3.9265e+2}. Рекомендуется использовать этот модификатор для вывода значения $\varepsilon$ и $\Delta$.
\end{itemize}


\begin{comment}
\begin{itemize}
	\item Здесь нужно оставить описание того, как сделать пошаговое вычисление суммы ряда.
	
	Непонятно только, что тут еще писать, как именно реализовывать работу? В целом тут нужно написать план в стиле:
	\begin{enumerate}
		\item Определитесь с первым членом последовательности
		\item Найдите, как выражается через $x$ и $i$ $v_{k+1}/v_k$. 
		\item Напишите обычный цикл, который вычисляет сумму  первых N членов последовательности и выведите на экран ее и референсное значение функции.
		\item Замените цикл на нормальный.
		 
	\end{enumerate}
	\item Описать, как должен выглядеть заголовок и одна строка таблицы (на живом примере). Описать особенности задания формата в функции printf.
	\item Напишите прототип функции построения таблицы. /*Можно вставить сам прототип*/ Напишите код, выводящий заголовок.
	Затем, напишите цикл, который выводит равномерные значения $x$ от $a$ до $b$. 
	\item Вставьте в тело цикла код вывода одной строчки таблицы.
\end{itemize}
\end{comment}
%%%%%%%%%%%%%%%%%%%%%%%%%%%%%%%%%%%%%%%%%%%%%%%%%%%%%%%%%%%%%%%%%%%%%%%%%%%%%%
\zzsectionPLAN
%%%%%%%%%%%%%%%%%%%%%%%%%%%%%%%%%%%%%%%%%%%%%%%%%%%%%%%%%%%%%%%%%%%%%%%%%%%%%%

\begin{enumerate}
\item Сначала необходимо реализовать функциональность для вычисления слагаемых нашего ряда. Для этого требуется определить, какое значение имеет первое слагаемое и каково отношение между двумя последовательными слагаемыми.
\item После этого можно реализовать вычисление частичных сумм. Для этого оформляем цикл, для начала с фиксированным числом итераций (например, 1000), перед циклом инициализируем значения частичной суммы и слагаемого, после этого в цикле обновляем эти переменные. Для теста можно выводить значения, получаемые на каждом шаге и значения функции, используемой для проверки.\zztodo{Стоит ли рекомендовать вынести вычисление суммы ряда в отдельную функцию, куда дельта и эпсилон передаются по указателю?}
\item Далее необходимо ввести механизм остановки, цикла, подсчет $\varepsilon$ и $\Delta$.
\item На данном этапе подготовлены все данные для вывода таблицы, можно приступать к нему. Комментарии по форматированию можно найти выше. 
\end{enumerate}


%%%%%%%%%%%%%%%%%%%%%%%%%%%%%%%%%%%%%%%%%%%%%%%%%%%%%%%%%%%%%%%%%%%%%%%%%%%%%%
\zzsectionVARIATIONS
%%%%%%%%%%%%%%%%%%%%%%%%%%%%%%%%%%%%%%%%%%%%%%%%%%%%%%%%%%%%%%%%%%%%%%%%%%%%%%


\begin{zztask}
В рамках общего условия задачи вывести таблицу значений функции, заданной рядом:
\[ % 100: "sin z", where z = x/2
	S(x)= \sum_{n=1}^\infty (-1)^{n+1} \frac{x^{2n-1}}{2^{2n-1}(2n-1)!};\quad
	f(x)= \sin \frac{x}{2},
	\quad 0 \leq x \leq 2;
\]
\end{zztask}

%%%%%%%%%%%%%%%%%%%%%%%%%%%%%%%%%%%%%%%%%%%%%%%%%%%%%%%%%%%%%%%%%%%%%%%%%%%%%%

\begin{zztask}
В рамках общего условия задачи вывести таблицу значений функции, заданной рядом:
\[ % 106: "cos z", where z = 2x
  S(x)= 1 + \sum_{n=1}^\infty (-1)^n \frac{2^{2n}}{(2n)!} x^{2n},\quad
  f(x)= \cos 2x,
  \quad |x| \leq 1;
\]
\end{zztask}

%%%%%%%%%%%%%%%%%%%%%%%%%%%%%%%%%%%%%%%%%%%%%%%%%%%%%%%%%%%%%%%%%%%%%%%%%%%%%%

\begin{zztask}
В рамках общего условия задачи вывести таблицу значений функции, заданной рядом:
\[ % 101: "sin^2 x = (1 - cos 2x)/2"
  S(x)= \sum_{n=1}^\infty (-1)^{n-1} \frac{2^{2n-1}}{(2n)!} x^{2n};\quad
  f(x)= \sin^2 x,
  \quad |x| \leq 1;
\]
\end{zztask}

%%%%%%%%%%%%%%%%%%%%%%%%%%%%%%%%%%%%%%%%%%%%%%%%%%%%%%%%%%%%%%%%%%%%%%%%%%%%%%

\begin{zztask}
В рамках общего условия задачи вывести таблицу значений функции, заданной рядом:
\[ % 104: "cos^2 x = (1 + cos 2x)/2"
  S(x)= 1 + \frac{1}{2} \sum_{n=1}^\infty (-1)^n \frac{(2x)^{2n}}{(2n)!};\quad
  f(x)= \cos^2 x,
  \quad |x| \leq 1;
\]
\end{zztask}

%%%%%%%%%%%%%%%%%%%%%%%%%%%%%%%%%%%%%%%%%%%%%%%%%%%%%%%%%%%%%%%%%%%%%%%%%%%%%%

\begin{zztask}
В рамках общего условия задачи вывести таблицу значений функции, заданной рядом:
\[
  S(x)= 2 \sum_{n=0}^\infty \frac{(x-1)^{2n+1}}{(2n+1)(x+1)^{2n+1}};\quad
  f(x)= \ln x,
  \quad 0 < x \leq 1;
\]
\end{zztask}

%%%%%%%%%%%%%%%%%%%%%%%%%%%%%%%%%%%%%%%%%%%%%%%%%%%%%%%%%%%%%%%%%%%%%%%%%%%%%%

\begin{zztask}
В рамках общего условия задачи вывести таблицу значений функции, заданной рядом:
\[
  S(x)= \sum_{n=1}^\infty (-1)^{n+1}\frac{(x-3)^{n-1}}{3^n};\quad
  f(x)= \frac{1}{x},
  \quad 3 < x \leq 4;
\]
\end{zztask}

%%%%%%%%%%%%%%%%%%%%%%%%%%%%%%%%%%%%%%%%%%%%%%%%%%%%%%%%%%%%%%%%%%%%%%%%%%%%%%

\begin{zztask}
В рамках общего условия задачи вывести таблицу значений функции, заданной рядом:
\[
  S(x)= x + \sum_{n=2}^\infty (-1)^{n-1}\frac{2^{n-1}}{(n-1)!} x^n;\quad
  f(x)= x/e^{2x},
  \quad |x| \leq 1;
\]
\end{zztask}

%%%%%%%%%%%%%%%%%%%%%%%%%%%%%%%%%%%%%%%%%%%%%%%%%%%%%%%%%%%%%%%%%%%%%%%%%%%%%%

\begin{zztask}
В рамках общего условия задачи вывести таблицу значений функции, заданной рядом:
\[
  S(x)= \frac{1}{2} \sum_{n=1}^\infty \frac{\big((n-1)!\big)^2}{(2n)!} (2x)^{2n};\quad
  f(x)= \arcsin^2x,
  \quad 0 \leq x \leq \sqrt2/2;
\]
\end{zztask}

%%%%%%%%%%%%%%%%%%%%%%%%%%%%%%%%%%%%%%%%%%%%%%%%%%%%%%%%%%%%%%%%%%%%%%%%%%%%%%

%%%%%%%%%%%%%%%%%%%%%%%%%%%%%%%%%%%%%%%%%%%%%%%%%%%%%%%%%%%%%%%%%%%%%%%%%%%%%%
