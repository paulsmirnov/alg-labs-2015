%%%%%%%%%%%%%%%%%%%%%%%%%%%%%%%%%%%%%%%%%%%%%%%%%%%%%%%%%%%%%%%%%%%%%%%%%%%%%%
\chapter{Краткое введение в язык Си}
%%%%%%%%%%%%%%%%%%%%%%%%%%%%%%%%%%%%%%%%%%%%%%%%%%%%%%%%%%%%%%%%%%%%%%%%%%%%%%

В настоящее время в расположении программиста имеется огромное количество языков программирования на любой вкус. 
Некоторые из них имеют узкие области применения, другие пригодны практически неограниченного использования.
Тем не менее, несмотря на то, что основной стандарт языка был принят около тридцати лет назад, авторы выбрали именно
его, как стартовый язык для изучения в вузах. Причины заключаются в том, что СИ совмещает простоту и глубину. 
Формальная грамматика языка может уместиться на нескольких страницах\zztodo{нужна ли ссылка?}, при этом, он позволяет познакомиться с такими
фундаментальными инструментами программирования, как стек, куча и указатели.

Все языки одинаковые. Поэтому будем изучать Си~\cite{podbelsky2015kurs}, он крут\zztodo{Написать параграфик}.


%%%%%%%%%%%%%%%%%%%%%%%%%%%%%%%%%%%%%%%%%%%%%%%%%%%%%%%%%%%%%%%%%%%%%%%%%%%%%%
\section{Минимальная программа}
%%%%%%%%%%%%%%%%%%%%%%%%%%%%%%%%%%%%%%%%%%%%%%%%%%%%%%%%%%%%%%%%%%%%%%%%%%%%%%

Программа на языке Си оформляется в виде набора функций, совместно решающих
поставленную задачу. Работа программы начинается с главной функции, которая
выполняет необходимые команды, вызывает другие функции, изучает результаты их
выполнения, поэтому каждая программа должна содержать как минимум одну
функцию~--- главную, которая всегда называется \texttt{main()}. Минимальная
программа на языке Си в соответствии со стандартом
1989~года\footnote{ANSI X3.159--1989 ``Programming Language C''} выглядит
следующим образом:

\begin{minted}{c}
int main(void)
{
  return 0;
}
\end{minted}

Ваша программа с точки зрения операционной системы (ОС) является
подпрограммой, вызываемой для решения конкретной задачи. Этой <<подпрограмме>>
могут передаваться исходные данные (в виде <<параметров>>), а от неё ожидается
результат (в виде <<возвращаемого значения>>). В приведённом минимальном
примере наша программа не ожидает никакой исходной информации от ОС
(\texttt{void} в скобках), но возвращает ей целое число (\texttt{int} перед
именем функции), которое системой будет рассматриваться как признак ошибки. По
договорённости \texttt{0} (ноль) считается успешным завершением программы, а любое
ненулевое значение является кодом ошибки. Не существует никаких
предопределённых кодов, автор каждой программы сам решает для себя, какие
ошибки он будет обозначать какими значениями.


%%%%%%%%%%%%%%%%%%%%%%%%%%%%%%%%%%%%%%%%%%%%%%%%%%%%%%%%%%%%%%%%%%%%%%%%%%%%%%
\section{Вычисление выражений}
%%%%%%%%%%%%%%%%%%%%%%%%%%%%%%%%%%%%%%%%%%%%%%%%%%%%%%%%%%%%%%%%%%%%%%%%%%%%%%

Компьютеры, да и языки программирования, были созданы в первую очередь для
автоматизации вычислений. Исходные данные для вычислений могут поступать из
разных источников от разных периферийных устройств, исторически сложилось так,
что в качестве устройства ввода-вывода выступает консоль\zztodo{Мб я наркоман, а это не называется формально терминал? Или мы не будем тут занудствовать?} --- совокупность
клавиатуры для ввода и экрана для вывода текста. Современные системы позволяют
большее, но для этого приходится прикладывать и больше усилий. В учебных целях
ограничимся пока консольными программами.

\zzneedspace
Вместо часто встречающегося <<Hello, world!>> рассмотрим более содержательный пример:
%
\inputminted{c}{samples/feet_bad.c}

Самая первая строчка включает (\texttt{\#include}) \textit{заголовочный файл} с
именем \texttt{stdio.h}, тем самым объясняя компилятору, что программа
пользуется стандартным вводом-выводом (standard input-output). В начале функции идёт определение
переменной с именем \texttt{h} и типом \texttt{int}. Затем, со стандартного устройства
ввода (т.е. клавиатуры) считывается (\texttt{scanf}) одно целое число
(\texttt{\%i}) в подготовленную переменную (\texttt{\&h}). После чего печатается
(\texttt{printf}) одно вещественное число (\texttt{\%lf}) и курсор переводится
на следующую строку (\verb|\n|), при выводе в качестве вещественного числа
подставляется результат деления \texttt{h} на \texttt{30.48}.

Не очень понятно, зачем программа всё это делает. И пользователю
непонятно, как с ней работать. При запуске программы он увидит чёрный экран и
мигающий курсор. Если нажимать на кнопки, в месте, куда указывает курсор на
экране будут появляться символы. Зачем? И что дальше? Программа недружелюбна
как по отношению к пользователю, так и по отношению к программисту, который,
возможно, будет читать этот код. Лучше будет, если программу немного
поправить:
%
\inputminted{c}{samples/feet.c}

С говорящими именами, комментариями и дополнительно выводимым текстом стало
гораздо понятнее и пользователю, и программисту. Мы определили
(\texttt{\#define}) константу \texttt{FEET\_LENGTH} (<<длина фута>>), которая везде
будет заменяться на значение \texttt{30.48}. Переменную назвали \texttt{height}
(<<высота>>), добавили для пользователя вывод приглашения перед ожиданием
вводимых данных и выводим не просто число, а понятную фразу. Главная функция
разбита на два осмысленных блока, которые встречаются практически во всех
программах: ввод данных и вывод результата. Блоки предваряются короткими
поясняющими комментариями между \verb|/*| и \verb|*/|. Диалог с программой
может выглядеть следующим образом:
%
\begin{verbatim}
    What is your height in cm? 200
    Your height is 6.561680 feet
\end{verbatim}

Если немного постараться, можно научить программу говорить и по-русски. Для
этого воспользуемся функцией \texttt{setlocale()}, объявление которой
находится в заголовочном файле \texttt{locale.h}. Заодно, немного усложним
вычисления и вынесем их в отдельный блок:
%
\inputminted{c}{samples/feet_rus.c}

В новом варианте программы используются однострочные комментарии (\texttt{//}), которые не
входили в первый стандарт языка Си, но были введены в более поздних стандартах
Си++ и Си, и сейчас поддерживаются практически всеми компиляторами.
Кроме пары целых переменных через запятую мы завели ещё две вещественные
переменные двойной точности (\texttt{double}). В формуле воспользовались
приведением к целому значению \verb|(int)|. Также в конце программы выводим
целое число (\verb|%i|) футов и вещественное число оставшихся дюймов с
точностью до одного знака после десятичной точки (\verb|%.1lf|). Для этого при
вызове функции \texttt{printf()} нужным нам образом составляем форматную
строку (первый аргумент) и через запятую перечисляем все аргументы, которые
будут поставляться по порядку на подготовленные места.

Кроме четырёх арифметических действий в языке Си присутствует операция взятия
остатка от деления, обозначаемая процентом: \texttt{11\ \%\ 3} равно 2.
Сама операция деления очень опасна для новичков --- при делении целых чисел
друг на друга она действует как целочисленное деление, отбрасывая дробную часть\zztodo{Вот это я бы куда-нибудь выделил, написал бы красным поперёк всей книги, но так нельзя, наверное}.
Если же один из аргументов вещественный, то и результат будет вещественным, поэтому
при необходимости точного деления целых чисел следует делить их как вещественные, т.е. воспользоваться операцией
приведения типа к вещественному (\texttt{double}):
%
\inputminted{c}{samples/equation.c}

Также, в формулах можно использовать математические функции: извлечение
квадратного корня (\texttt{sqrt()}), разную тригонометрию (\texttt{sin()}),
натуральный логарифм (\texttt{log()}), экспоненту (\texttt{exp()}). Эти и
другие функции становятся доступными при включении заголовочного файла
\texttt{math.h}.


%%%%%%%%%%%%%%%%%%%%%%%%%%%%%%%%%%%%%%%%%%%%%%%%%%%%%%%%%%%%%%%%%%%%%%%%%%%%%%
\section{Управление программным потоком}
%%%%%%%%%%%%%%%%%%%%%%%%%%%%%%%%%%%%%%%%%%%%%%%%%%%%%%%%%%%%%%%%%%%%%%%%%%%%%%

Не все программы можно записать в виде такой линейной последовательности
команд, выполняющихся друг за другом от начала и до конца, поэтому
императивные языки программирования предлагают возможности по управлению
порядком выполнения команд. В Си поддерживаются все основные блоки
структурного программирования --- <<последовательность>>, <<выбор>>,
<<повторение>> и <<подпрограммы>>.

\zzneedspace
Пример на ветвление потока управления (условное выполнение команд \texttt{if-else}):
%
\inputminted{c}{samples/01_if.c}

Все структурные команды языка Си могут иметь только одну подчинённую команду.
Если необходимо выполнять несколько команд, их надо группировать с помощью
фигурных скобок как в примере выше. Операции сравнения в языке Си: меньше
(\verb|<|), меньше или равно (\verb|<=|), больше (\verb|>|), больше или равно
(\verb|>=|), равно (\verb|==|), не равно (\verb|!=|). Условия можно объединять
логическими связками: и (\verb|&&|), или (\verb/||/). Для отрицания
используется восклицателный знак (\verb|!|). Альтернативное действие,
включая слово \texttt{else}, может отсутствовать.

\zzneedspace
Пример на повторение команд с предварительной проверкой условия (цикл с
предусловием \texttt{while}):
%
\inputminted{c}{samples/02_while.c}

Здесь использованы две новые операции: уменьшение на единицу (\verb|--|) и
увеличение на некоторый шаг (\verb|+=|). Вместо них можно было бы написать
\verb|n = n - 1| и \verb|sum = sum + n|, соответственно. Последняя форма доступна для многих
операций (напр., \verb|a -= b|, \verb|b /= 2|) и имеет аналогичный смысл.

\zzneedspace
Цикл с постусловием \texttt{do-while}:
%
\inputminted{c}{samples/03_do_while.c}

\zzneedspace
Цикл со счётчиком \texttt{for}:
%
\inputminted{c}{samples/04_for.c}

На самом деле, цикл \texttt{for} в языке Си более гибок, чем в цикл со
счётчиком в других языках программирования, но для начала такой формы
нам должно хватить.

В качестве допустимого отклонения от структурного программирования в языке
присутствуют ставшие уже традиционными команды преждевременного прерывания
цикла (\texttt{break}) и пропуска итерации (\texttt{continue}):
\zzneedspace
\inputminted{c}{samples/05_break.c}

\zzneedspace
Ну и наконец, следует вернуться к командам выбора и упомянуть команду
\verb|switch|, позволяющую сравнить одну переменную или выражение
с несколькими константами для выбора из нескольких альтернатив:
%
\inputminted{c}{samples/06_switch.c}

Такой подход понятнее, чем последовательность \texttt{if} и для программиста,
и для компилятора, и \zztodo{или "что", но без этого предложение как-то ломается.} позволяет сгенерировать более оптимальный исполняемый код.
Обратите внимание как здесь использована команда \texttt{break} для выхода из
тела команды выбора.


%%%%%%%%%%%%%%%%%%%%%%%%%%%%%%%%%%%%%%%%%%%%%%%%%%%%%%%%%%%%%%%%%%%%%%%%%%%%%%
\section{Пользовательские функции}
%%%%%%%%%%%%%%%%%%%%%%%%%%%%%%%%%%%%%%%%%%%%%%%%%%%%%%%%%%%%%%%%%%%%%%%%%%%%%%

Парадигма процедурного и структурного программирования подразумевает выделение
блоков команд, решающих подзадачи исходной задачи, в отдельные подпрограммы --- для
большей ясности и для возможного повторного использования. По аналогии с
функцией \texttt{main()} можно определять собственные функции, которые и
выполняют роль подпрограмм в языке Си. При наличии параметра его тип и имя
указывается в круглых скобках в заголовке функции. В начале тела каждой
функции можно определять локальные переменные, которые существуют и видны
только внутри соответствующих фигурных скобок. Выход из функции с возвратом
значения возможен из любого места функции при помощи команды \texttt{return}.
%
\inputminted{c}{samples/factorial.c}

%\zzneedspace
При наличии нескольких параметров они перечисляются через запятую, тип
указывается для каждого параметра в отдельности.
%
\inputminted{c}{samples/power.c}

Здесь была использована операция постфиксного декремента, которая также
уменьшает переменную на 1, но при этом в выражение для сравнения подставляется
старое значение (ещё до уменьшения).


%%%%%%%%%%%%%%%%%%%%%%%%%%%%%%%%%%%%%%%%%%%%%%%%%%%%%%%%%%%%%%%%%%%%%%%%%%%%%%
\section{Массивы в языке Си}
%%%%%%%%%%%%%%%%%%%%%%%%%%%%%%%%%%%%%%%%%%%%%%%%%%%%%%%%%%%%%%%%%%%%%%%%%%%%%%

Несколько однотипных переменных, связанных по смыслу, объединяют в массивы.
Доступ к отдельным элементам осуществляется по индексу, что упрощает
применение к ним одних и тех же действий в цикле.
%
\inputminted{c}{samples/array.c}

Массив определяется как переменная, после имени которой указывается размер в
квадратных скобках. Элементы массива нумеруются, начиная с нуля, для доступа к
ним используются те же квадратные скобки. Массив можно передавать в функцию и
там использовать (как в \verb|PrintArray()|) или даже менять значения
(как в \verb|ReadArray()|), для этого после имени соответствующего параметра
функции указываются пустые квадратные скобки.


%%%%%%%%%%%%%%%%%%%%%%%%%%%%%%%%%%%%%%%%%%%%%%%%%%%%%%%%%%%%%%%%%%%%%%%%%%%%%%
\section{Строки}
%%%%%%%%%%%%%%%%%%%%%%%%%%%%%%%%%%%%%%%%%%%%%%%%%%%%%%%%%%%%%%%%%%%%%%%%%%%%%%

Для хранения строк в языке Си используются массивы с элементами типа
\texttt{char}, которые хранят коды символов (целые числа от 0 до 127, соответствие кодов и символов называется таблицей ASCII)\zztodo{Поймут ли читатели, что это кодов символов 128, а не диапазон числе в char? Кстати, может тут и введём слово ASCII, как я написал?}. Код 0
означает конец строки, таким образом в большом массиве фиксированного размера
можно будет время от времени хранить как длинные, так и короткие строки. К
сожалению, при обращении к массивам не происходит проверки на выход индекса за
пределы массива, поэтому за всем приходится следить самому и быть аккуратным.
%
\inputminted{c}{samples/string.c}

Для чтения строки можно воспользоваться функцией \texttt{gets()}. Для вывода
с помощью \texttt{printf()} нужен спецификатор \verb|%s|. Чтобы обработать
всю строку необходимо обрабатывать элементы массива друг за другом, пока не
встретится 0 как в функции \texttt{Length()} выше.

Для работы со строками можно использовать стандартные библиотечные функции,
подключив \texttt{string.h}. Например, вместо собственной функции
\texttt{Length()} лучше использовать стандартную \texttt{strlen()}.

С символами в языке Си можно (нужно!) работать и не зная их кодов, а используя
одинарные кавычки (апострофы): запись \verb|'A'| означает код буквы
\textit{A}, т.е. целое число 65 в кодовой таблице ASCII. Также бывает удобно
пользоваться тем, что заглавные буквы \texttt{A}--\texttt{Z}, строчные
\texttt{a}--\texttt{z} и цифры \texttt{0}--\texttt{9} составляют непрерывные
диапазоны, и коды соседних символов этих диапазонов отличаются на 1. Проверка
на то что символ является цифрой может выглядеть следующим образом:
%
\begin{minted}{c}
if (s[i] >= '0' && s[i] <= '9')
{
   ...
}
\end{minted}

К сожалению, это верно только для кодировки ASCII, и на некоторых экзотических
системах такие программы будут работать некорректно. Стандартная библиотека
при подключении заголовочного файла \texttt{ctype.h} предлагает функцию
\texttt{isdigit()} и другие аналогичные, которые будут работать корректно в
любом случае\zztodo{Громкое заявление, но тихое будет занудно длинным, наверное.}.
