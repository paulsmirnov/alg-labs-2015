%%%%%%%%%%%%%%%%%%%%%%%%%%%%%%%%%%%%%%%%%%%%%%%%%%%%%%%%%%%%%%%%%%%%%%%%%%%%%%
%%%%%%%%%%%%%%%%%%%%%%%%%%%%%%%%%%%%%%%%%%%%%%%%%%%%%%%%%%%%%%%%%%%%%%%%%%%%%%
\chapter{Массивы данных}
%%%%%%%%%%%%%%%%%%%%%%%%%%%%%%%%%%%%%%%%%%%%%%%%%%%%%%%%%%%%%%%%%%%%%%%%%%%%%%
%%%%%%%%%%%%%%%%%%%%%%%%%%%%%%%%%%%%%%%%%%%%%%%%%%%%%%%%%%%%%%%%%%%%%%%%%%%%%%


%%%%%%%%%%%%%%%%%%%%%%%%%%%%%%%%%%%%%%%%%%%%%%%%%%%%%%%%%%%%%%%%%%%%%%%%%%%%%%
\section{Массивы в языке Си}
%%%%%%%%%%%%%%%%%%%%%%%%%%%%%%%%%%%%%%%%%%%%%%%%%%%%%%%%%%%%%%%%%%%%%%%%%%%%%%

Несколько однотипных переменных, связанных по смыслу, объединяют в массивы.
Доступ к отдельным элементам осуществляется по индексу, что упрощает
применение к ним одних и тех же действий в цикле.
%
\inputminted{c}{samples/array.c}

Массив определяется как переменная, после имени которой указывается размер в
квадратных скобках. Элементы массива нумеруются, начиная с нуля, для доступа к
ним используются те же квадратные скобки. Массив можно передавать в функцию и
там использовать (как в \verb|PrintArray()|) или даже менять значения
(как в \verb|ReadArray()|), для этого после имени соответствующего параметра
функции указываются пустые квадратные скобки. Более детально все тонкости
работы с массивами мы разберём позднее.


%%%%%%%%%%%%%%%%%%%%%%%%%%%%%%%%%%%%%%%%%%%%%%%%%%%%%%%%%%%%%%%%%%%%%%%%%%%%%%
\section{Строки}
%%%%%%%%%%%%%%%%%%%%%%%%%%%%%%%%%%%%%%%%%%%%%%%%%%%%%%%%%%%%%%%%%%%%%%%%%%%%%%

Для хранения строк в языке Си используются массивы с элементами типа
\texttt{char}, которые хранят коды символов (целые числа от 0 до 127). Код 0
означает конец строки, таким образом в большом массиве фиксированного размера
можно будет время от времени хранить как длинные, так и короткие строки. К
сожалению, при обращении к массивам не происходит проверки на выход индекса за
пределы массива, поэтому за всем приходится следить самому и быть аккуратным.
%
\inputminted{c}{samples/string.c}

Для чтения строки можно воспользоваться функцией \texttt{gets()}. Для вывода
с помощью \texttt{printf()} нужен спецификатор \verb|%s|. Чтобы обработать
всю строку необходимо обрабатывать элементы массива друг за другом, пока не
встретится 0 как в функции \texttt{Length()} выше.

Для работы со строками можно использовать стандартные библиотечные функции,
подключив \texttt{string.h}. Например, вместо собственной функции
\texttt{Length()} лучше использовать стандартную \texttt{strlen()}.

С символами в языке Си можно (нужно!) работать и не зная их кодов, а используя
одинарные кавычки (апострофы): запись \verb|'A'| означает код буквы
\textit{A}, т.е. целое число 65 в кодовой таблице ASCII. Также бывает удобно
пользоваться тем, что заглавные буквы \texttt{A}--\texttt{Z}, строчные
\texttt{a}--\texttt{z} и цифры \texttt{0}--\texttt{9} составляют непрерывные
диапазоны, и коды соседних символов этих диапазонов отличаются на 1. Проверка
на то что символ является цифрой может выглядеть следующим образом:
%
\begin{minted}{c}
if (s[i] >= '0' && s[i] <= '9')
{
   ...
}
\end{minted}

К сожалению, это верно только для кодировки ASCII, и на некоторых экзотических
системах такие программы будут работать некорректно. Стандартная библиотека
предлагает функцию \texttt{isdigit()} и другие аналогичные при подключении
заголовочного файла \texttt{ctype.h}.


%%%%%%%%%%%%%%%%%%%%%%%%%%%%%%%%%%%%%%%%%%%%%%%%%%%%%%%%%%%%%%%%%%%%%%%%%%%%%%
%%%%%%%%%%%%%%%%%%%%%%%%%%%%%%%%%%%%%%%%%%%%%%%%%%%%%%%%%%%%%%%%%%%%%%%%%%%%%%
\zztaskgroup{ARY}{Одномерные массивы}
%%%%%%%%%%%%%%%%%%%%%%%%%%%%%%%%%%%%%%%%%%%%%%%%%%%%%%%%%%%%%%%%%%%%%%%%%%%%%%

Элементы языка: простые массивы, генератор псевдослучайных чисел.

Здесь приводятся упражнения на работу с одномерными массивами и индексацию в
массив. Показаны стандартные приемы нахождения минимального и максимального
элементов массива. Также вводится понятие генератора псевдослучайных чисел,
закрепляются навыки его использования.\zztodo{Не очень понятно, что даст эта лаба, как-то тут кучно все. Все классе тут не успеть нифига,а домашка вроде как для имбецилов получается. }


%%%%%%%%%%%%%%%%%%%%%%%%%%%%%%%%%%%%%%%%%%%%%%%%%%%%%%%%%%%%%%%%%%%%%%%%%%%%%%

Задача C-1

В начало массива из 20 элементов записать N случайных натуральных чисел из
диапазона от 1 до N, воспользовавшись функцией rand(). Число N вводится с
клавиатуры. Сосчитать сумму элементов. В другой массив записать суммы пар
элементов (N сумм): первого и последнего, второго и предпоследнего, третьего и
пред-предпоследнего... Вывести на экран результирующий массив и сумму его
элементов, а также минимальный и максимальный эл-т. Повторять выполнение
программы до тех пор, пока в качестве N не введут 0.\zztodo{Зачем дублировать парные суммы?}

Иванов И.И. (1057/1): Массивы, нат.случайные числа от 1 до N.
Введите размер (или 0 для завершения): 5
Сгенерированный массив A = { 4 3 1 4 5 }, сумма = 17
Результат B = { 9 7 2 7 9 }, сумма = 34
Минимум = 2, максимум = 9
Введите размер (или 0 для завершения): 0

%%%%%%%%%%%%%%%%%%%%%%%%%%%%%%%%%%%%%%%%%%%%%%%%%%%%%%%%%%%%%%%%%%%%%%%%%%%%%%

Задача C-2

В начало массива из 20 элементов записать N случайных натуральных чисел из
диапазона от 1 до N, воспользовавшись функцией rand(). Число N вводится с
клавиатуры. Сосчитать сумму элементов. В другой массив записать суммы пар
элементов (N сумм): первого и второго, второго и третьего, ..., последнего и
первого. Вывести на экран результирующий массив и сумму его элементов, а также
минимальный и максимальный эл-т. Повторять выполнение программы до тех пор,
пока в качестве N не введут 0.

Иванов И.И. (1057/1): Массивы, нат.случайные числа от 1 до N.
Введите размер (или 0 для завершения): 5
Сгенерированный массив A = { 4 3 1 4 5 }, сумма = 17
Результат B = { 7 4 5 9 9 }, сумма = 34
Минимум = 4, максимум = 9
Введите размер (или 0 для завершения): 0

%%%%%%%%%%%%%%%%%%%%%%%%%%%%%%%%%%%%%%%%%%%%%%%%%%%%%%%%%%%%%%%%%%%%%%%%%%%%%%

Задача C-3

В начало массива из 20 элементов записать N случайных целых чисел из диапазона
от A до B, воспользовавшись функцией rand(). Числа N, A, B вводятся с
клавиатуры. Сосчитать сумму элементов. В другой массив записать суммы пар
элементов (N сумм): первого и последнего, второго и предпоследнего, третьего и
пред-предпоследнего... Вывести на экран результирующий массив и сумму его
элементов, а также минимальный и максимальный эл-т. Повторять выполнение
программы до тех пор, пока в качестве N не введут 0.

Иванов И.И. (1057/1): Массивы, целые случайные числа от A до B.
Введите размер (или 0 для завершения): 5
Введите диапазон A B через пробел: 3 7
Сгенерированный массив A = { 4 3 7 6 5 }, сумма = 25
Результат B = { 9 9 14 9 9 }, сумма = 50
Минимум = 9, максимум = 14
Введите размер (или 0 для завершения): 0

%%%%%%%%%%%%%%%%%%%%%%%%%%%%%%%%%%%%%%%%%%%%%%%%%%%%%%%%%%%%%%%%%%%%%%%%%%%%%%

Задача C-4

В начало массива из 20 элементов записать N случайных целых чисел из диапазона
от A до B, воспользовавшись функцией rand(). Числа N, A, B вводятся с
клавиатуры. Сосчитать сумму элементов. В другой массив записать суммы пар
элементов (N сумм): первого и второго, второго и третьего, ..., последнего и
первого. Вывести на экран результирующий массив и сумму его элементов, а также
минимальный и максимальный эл-т. Повторять выполнение программы до тех пор,
пока в качестве N не введут 0.

Иванов И.И. (1057/1): Массивы, целые случайные числа от A до B.
Введите размер (или 0 для завершения): 5
Введите диапазон A B через пробел: 3 7
Сгенерированный массив A = { 4 3 7 6 5 }, сумма = 25
Результат B = { 7 10 13 11 9 }, сумма = 50
Минимум = 7, максимум = 13
Введите размер (или 0 для завершения): 0

%%%%%%%%%%%%%%%%%%%%%%%%%%%%%%%%%%%%%%%%%%%%%%%%%%%%%%%%%%%%%%%%%%%%%%%%%%%%%%

Задача C-5

В начало массива из 20 элементов записать N случайных вещественных чисел из
диапазона от 0 до 1, воспользовавшись функцией rand(). Число N вводится с
клавиатуры. Сосчитать сумму элементов. В другой массив записать суммы пар
элементов (N сумм): первого и последнего, второго и предпоследнего, третьего и
пред-предпоследнего... Вывести на экран результирующий массив и сумму его
элементов, а также минимальный и максимальный эл-т. Повторять выполнение
программы до тех пор, пока в качестве N не введут 0.

Иванов И.И. (1057/1): Массивы, вещ. случайные числа от 0 до 1.
Введите размер (или 0 для завершения): 5
Сгенерированный массив A = { 0.43 0.31 0.74 0.16 0.50 }, сумма = 2.14
Результат B = { 0.93 0.47 1.48 0.47 0.93 }, сумма = 4.28
Минимум = 0.47, максимум = 1.48
Введите размер (или 0 для завершения): 0

%%%%%%%%%%%%%%%%%%%%%%%%%%%%%%%%%%%%%%%%%%%%%%%%%%%%%%%%%%%%%%%%%%%%%%%%%%%%%%

Задача C-6

В начало массива из 20 элементов записать N случайных вещественных чисел из
диапазона от 0 до 1, воспользовавшись функцией rand(). Число N вводится с
клавиатуры. Сосчитать сумму элементов. В другой массив записать суммы пар
элементов (N сумм): первого и второго, второго и третьего, ..., последнего и
первого. Вывести на экран результирующий массив и сумму его элементов, а также
минимальный и максимальный эл-т. Повторять выполнение программы до тех пор,
пока в качестве N не введут 0.

Иванов И.И. (1057/1): Массивы, вещ. случайные числа от 0 до 1.
Введите размер (или 0 для завершения): 5
Сгенерированный массив A = { 0.43 0.31 0.74 0.16 0.50 }, сумма = 2.14
Результат B = { 0.74 1.05 0.90 0.66 0.93 }, сумма = 4.28
Минимум = 0.66, максимум = 1.05
Введите размер (или 0 для завершения): 0

%%%%%%%%%%%%%%%%%%%%%%%%%%%%%%%%%%%%%%%%%%%%%%%%%%%%%%%%%%%%%%%%%%%%%%%%%%%%%%

Задача C-7

В начало массива из 20 элементов записать N случайных вещественных чисел из
диапазона от 0 до N с точностью 0.1, воспользовавшись функцией rand(). Число N
вводится с клавиатуры. Сосчитать сумму элементов. В другой массив записать
суммы пар элементов (N сумм): первого и последнего, второго и предпоследнего,
третьего и пред-предпоследнего... Вывести на экран результирующий массив и
сумму его элементов, а также минимальный и максимальный эл-т. Повторять
выполнение программы до тех пор, пока в качестве N не введут 0.

Иванов И.И. (1057/1): Массивы, вещ. случайные числа от 0 до N.
Введите размер (или 0 для завершения): 5
Сгенерированный массив A = { 3.40 0.30 4.70 1.10 4.50 }, сумма = 14
Результат B = { 7.90 1.40 9.40 1.40 7.90 }, сумма = 28
Минимум = 1.40, максимум = 9.40
Введите размер (или 0 для завершения): 0

%%%%%%%%%%%%%%%%%%%%%%%%%%%%%%%%%%%%%%%%%%%%%%%%%%%%%%%%%%%%%%%%%%%%%%%%%%%%%%

Задача C-8

В начало массива из 20 элементов записать N случайных вещественных чисел из
диапазона от 0 до N с точностью 0.1, воспользовавшись функцией rand(). Число N
вводится с клавиатуры. Сосчитать сумму элементов. В другой массив записать
суммы пар элементов (N сумм): первого и второго, второго и третьего, ...,
последнего и первого. Вывести на экран результирующий массив и сумму его
элементов, а также минимальный и максимальный эл-т. Повторять выполнение
программы до тех пор, пока в качестве N не введут 0.

Иванов И.И. (1057/1): Массивы, вещ. случайные числа от 0 до N.
Введите размер (или 0 для завершения): 5
Сгенерированный массив A = { 3.40 0.30 4.70 1.10 4.50 }, сумма = 14
Результат B = { 3.70 5.00 5.80 5.60 7.90 }, сумма = 28
Минимум = 3.70, максимум = 7.90
Введите размер (или 0 для завершения): 0

%%%%%%%%%%%%%%%%%%%%%%%%%%%%%%%%%%%%%%%%%%%%%%%%%%%%%%%%%%%%%%%%%%%%%%%%%%%%%%

Задача C-9

В начало массива из 20 элементов записать N случайных натуральных чисел из
диапазона от 1 до N, воспользовавшись функцией rand(). Число N вводится с
клавиатуры. Сосчитать сумму элементов. В другой массив записать суммы пар
элементов (N сумм): первого и последнего, второго и предпоследнего, третьего и
пред-предпоследнего... Вывести на экран результирующий массив и сумму его
элементов, а также индексы минимального и максимального эл-та (если таких
несколько, то первый минимальный и последний максимальный). Повторять
выполнение программы до тех пор, пока в качестве N не введут 0.

Иванов И.И. (1057/1): Массивы, нат.случайные числа от 1 до N.
Введите размер (или 0 для завершения): 5
Сгенерированный массив A = { 4 3 1 4 5 }, сумма = 17
Результат B = { 9 7 2 7 9 }, сумма = 34
Минимум = B[2], максимум = B[4]
Введите размер (или 0 для завершения): 0

%%%%%%%%%%%%%%%%%%%%%%%%%%%%%%%%%%%%%%%%%%%%%%%%%%%%%%%%%%%%%%%%%%%%%%%%%%%%%%

Задача C-10

В начало массива из 20 элементов записать N случайных натуральных чисел из
диапазона от 1 до N, воспользовавшись функцией rand(). Число N вводится с
клавиатуры. Сосчитать сумму элементов. В другой массив записать суммы пар
элементов (N сумм): первого и второго, второго и третьего, ..., последнего и
первого. Вывести на экран результирующий массив и сумму его элементов, а также
индексы минимального и максимального эл-та (если таких несколько, то первый
минимальный и последний максимальный). Повторять выполнение программы до тех
пор, пока в качестве N не введут 0.

Иванов И.И. (1057/1): Массивы, нат.случайные числа от 1 до N.
Введите размер (или 0 для завершения): 5
Сгенерированный массив A = { 4 3 1 4 5 }, сумма = 17
Результат B = { 7 4 5 9 9 }, сумма = 34
Минимум = B[1], максимум = B[4]
Введите размер (или 0 для завершения): 0

%%%%%%%%%%%%%%%%%%%%%%%%%%%%%%%%%%%%%%%%%%%%%%%%%%%%%%%%%%%%%%%%%%%%%%%%%%%%%%

Задача C-11

В начало массива из 20 элементов записать N случайных целых чисел из диапазона
от A до B, воспользовавшись функцией rand(). Числа N, A, B вводятся с
клавиатуры. Сосчитать сумму элементов. В другой массив записать суммы пар
элементов (N сумм): первого и последнего, второго и предпоследнего, третьего и
пред-предпоследнего... Вывести на экран результирующий массив и сумму его
элементов, а также индексы минимального и максимального эл-та (если таких
несколько, то первый минимальный и последний максимальный). Повторять
выполнение программы до тех пор, пока в качестве N не введут 0.

Иванов И.И. (1057/1): Массивы, целые случайные числа от A до B.
Введите размер (или 0 для завершения): 5
Введите диапазон A B через пробел: 3 7
Сгенерированный массив A = { 4 3 7 6 5 }, сумма = 25
Результат B = { 9 9 14 9 9 }, сумма = 50
Минимум = B[0], максимум = B[2]
Введите размер (или 0 для завершения): 0

%%%%%%%%%%%%%%%%%%%%%%%%%%%%%%%%%%%%%%%%%%%%%%%%%%%%%%%%%%%%%%%%%%%%%%%%%%%%%%

Задача C-12

В начало массива из 20 элементов записать N случайных целых чисел из диапазона
от A до B, воспользовавшись функцией rand(). Числа N, A, B вводятся с
клавиатуры. Сосчитать сумму элементов. В другой массив записать суммы пар
элементов (N сумм): первого и второго, второго и третьего, ..., последнего и
первого. Вывести на экран результирующий массив и сумму его элементов, а также
индексы минимального и максимального эл-та (если таких несколько, то первый
минимальный и последний максимальный). Повторять выполнение программы до тех
пор, пока в качестве N не введут 0.

Иванов И.И. (1057/1): Массивы, целые случайные числа от A до B.
Введите размер (или 0 для завершения): 5
Введите диапазон A B через пробел: 3 7
Сгенерированный массив A = { 4 3 7 6 5 }, сумма = 25
Результат B = { 7 10 13 11 9 }, сумма = 50
Минимум = B[0], максимум = B[2]
Введите размер (или 0 для завершения): 0

%%%%%%%%%%%%%%%%%%%%%%%%%%%%%%%%%%%%%%%%%%%%%%%%%%%%%%%%%%%%%%%%%%%%%%%%%%%%%%

Задача C-13

В начало массива из 20 элементов записать N случайных вещественных чисел из
диапазона от 0 до 1, воспользовавшись функцией rand(). Число N вводится с
клавиатуры. Сосчитать сумму элементов. В другой массив записать суммы пар
элементов (N сумм): первого и последнего, второго и предпоследнего, третьего и
пред-предпоследнего... Вывести на экран результирующий массив и сумму его
элементов, а также индексы минимального и максимального эл-та (если таких
несколько, то первый минимальный и последний максимальный). Повторять
выполнение программы до тех пор, пока в качестве N не введут 0.

Иванов И.И. (1057/1): Массивы, вещ. случайные числа от 0 до 1.
Введите размер (или 0 для завершения): 5
Сгенерированный массив A = { 0.43 0.31 0.74 0.16 0.50 }, сумма = 2.14
Результат B = { 0.93 0.47 1.48 0.47 0.93 }, сумма = 4.28
Минимум = B[1], максимум = B[2]
Введите размер (или 0 для завершения): 0

%%%%%%%%%%%%%%%%%%%%%%%%%%%%%%%%%%%%%%%%%%%%%%%%%%%%%%%%%%%%%%%%%%%%%%%%%%%%%%

Задача C-14

В начало массива из 20 элементов записать N случайных вещественных чисел из
диапазона от 0 до 1, воспользовавшись функцией rand(). Число N вводится с
клавиатуры. Сосчитать сумму элементов. В другой массив записать суммы пар
элементов (N сумм): первого и второго, второго и третьего, ..., последнего и
первого. Вывести на экран результирующий массив и сумму его элементов, а также
индексы минимального и максимального эл-та (если таких несколько, то первый
минимальный и последний максимальный). Повторять выполнение программы до тех
пор, пока в качестве N не введут 0.

Иванов И.И. (1057/1): Массивы, вещ. случайные числа от 0 до 1.
Введите размер (или 0 для завершения): 5
Сгенерированный массив A = { 0.43 0.31 0.74 0.16 0.50 }, сумма = 2.14
Результат B = { 0.74 1.05 0.90 0.66 0.93 }, сумма = 4.28
Минимум = B[3], максимум = B[1]
Введите размер (или 0 для завершения): 0

%%%%%%%%%%%%%%%%%%%%%%%%%%%%%%%%%%%%%%%%%%%%%%%%%%%%%%%%%%%%%%%%%%%%%%%%%%%%%%

Задача C-15

В начало массива из 20 элементов записать N случайных вещественных чисел из
диапазона от 0 до N с точностью 0.1, воспользовавшись функцией rand(). Число N
вводится с клавиатуры. Сосчитать сумму элементов. В другой массив записать
суммы пар элементов (N сумм): первого и последнего, второго и предпоследнего,
третьего и пред-предпоследнего... Вывести на экран результирующий массив и
сумму его элементов, а также индексы минимального и максимального эл-та (если
таких несколько, то первый минимальный и последний максимальный). Повторять
выполнение программы до тех пор, пока в качестве N не введут 0.

Иванов И.И. (1057/1): Массивы, вещ. случайные числа от 0 до N.
Введите размер (или 0 для завершения): 5
Сгенерированный массив A = { 3.40 0.30 4.70 1.10 4.50 }, сумма = 14
Результат B = { 7.90 1.40 9.40 1.40 7.90 }, сумма = 28
Минимум = B[1], максимум = B[2]
Введите размер (или 0 для завершения): 0

%%%%%%%%%%%%%%%%%%%%%%%%%%%%%%%%%%%%%%%%%%%%%%%%%%%%%%%%%%%%%%%%%%%%%%%%%%%%%%

Задача C-16

В начало массива из 20 элементов записать N случайных вещественных чисел из
диапазона от 0 до N с точностью 0.1, воспользовавшись функцией rand(). Число N
вводится с клавиатуры. Сосчитать сумму элементов. В другой массив записать
суммы пар элементов (N сумм): первого и второго, второго и третьего, ...,
последнего и первого. Вывести на экран результирующий массив и сумму его
элементов, а также индексы минимального и максимального эл-та (если таких
несколько, то первый минимальный и последний максимальный). Повторять
выполнение программы до тех пор, пока в качестве N не введут 0.

Иванов И.И. (1057/1): Массивы, вещ. случайные числа от 0 до N.
Введите размер (или 0 для завершения): 5
Сгенерированный массив A = { 3.40 0.30 4.70 1.10 4.50 }, сумма = 14
Результат B = { 3.70 5.00 5.80 5.60 7.90 }, сумма = 28
Минимум = B[0], максимум = B[4]
Введите размер (или 0 для завершения): 0

%%%%%%%%%%%%%%%%%%%%%%%%%%%%%%%%%%%%%%%%%%%%%%%%%%%%%%%%%%%%%%%%%%%%%%%%%%%%%%

Задача C-17

В начало массива из 20 элементов записать N случайных натуральных чисел из
диапазона от 1 до N, воспользовавшись функцией rand(). Число N вводится с
клавиатуры. Найти индексы минимального и максимального эл-та (если таких
несколько, то первый минимальный и последний максимальный). В другой массив
записать элементы, лежащие между найденными индексами включительно. Вывести на
экран результирующий массив и сумму его элементов. Повторять выполнение
программы до тех пор, пока в качестве N не введут 0.

Иванов И.И. (1057/1): Массивы, нат.случайные числа от 1 до N.
Введите размер (или 0 для завершения): 5
Сгенерированный массив A = { 4 3 1 4 5 }
Минимум = A[2], максимум = A[4]
Результат B = { 1 4 5 }, сумма = 10
Введите размер (или 0 для завершения): 0

%%%%%%%%%%%%%%%%%%%%%%%%%%%%%%%%%%%%%%%%%%%%%%%%%%%%%%%%%%%%%%%%%%%%%%%%%%%%%%

Задача C-18

В начало массива из 20 элементов записать N случайных целых чисел из диапазона
от A до B, воспользовавшись функцией rand(). Числа N, A, B вводятся с
клавиатуры. Найти индексы минимального и максимального эл-та (если таких
несколько, то первый минимальный и последний максимальный). В другой массив
записать элементы, лежащие между найденными индексами включительно. Вывести на
экран результирующий массив и сумму его элементов. Повторять выполнение
программы до тех пор, пока в качестве N не введут 0.

Иванов И.И. (1057/1): Массивы, целые случайные числа от A до B.
Введите размер (или 0 для завершения): 5
Введите диапазон A B через пробел: 3 7
Сгенерированный массив A = { 4 3 7 4 5 }
Минимум = A[1], максимум = A[2]
Результат B = { 3 7 }, сумма = 10
Введите размер (или 0 для завершения): 0

%%%%%%%%%%%%%%%%%%%%%%%%%%%%%%%%%%%%%%%%%%%%%%%%%%%%%%%%%%%%%%%%%%%%%%%%%%%%%%

Задача C-19

В начало массива из 20 элементов записать N случайных вещественных чисел из
диапазона от 0 до 1, воспользовавшись функцией rand(). Число N вводится с
клавиатуры. Найти индексы минимального и максимального эл-та (если таких
несколько, то первый минимальный и последний максимальный). В другой массив
записать элементы, лежащие между найденными индексами включительно. Вывести на
экран результирующий массив и сумму его элементов. Повторять выполнение
программы до тех пор, пока в качестве N не введут 0.

Иванов И.И. (1057/1): Массивы, вещ. случайные числа от 0 до 1.
Введите размер (или 0 для завершения): 5
Сгенерированный массив A = { 0.42 0.11 0.29 0.94 0.75 }
Минимум = A[1], максимум = A[3]
Результат B = { 0.11 0.29 0.94 }, сумма = 1.34
Введите размер (или 0 для завершения): 0

%%%%%%%%%%%%%%%%%%%%%%%%%%%%%%%%%%%%%%%%%%%%%%%%%%%%%%%%%%%%%%%%%%%%%%%%%%%%%%

Задача C-20

В начало массива из 20 элементов записать N случайных вещественных чисел из
диапазона от 0 до N с точностью 0.1, воспользовавшись функцией rand(). Число N
вводится с клавиатуры. Найти индексы минимального и максимального эл-та (если
таких несколько, то первый минимальный и последний максимальный). В другой
массив записать элементы, лежащие между найденными индексами включительно.
Вывести на экран результирующий массив и сумму его элементов. Повторять
выполнение программы до тех пор, пока в качестве N не введут 0.

Иванов И.И. (1057/1): Массивы, вещ. случайные числа от 0 до N.
Введите размер (или 0 для завершения): 5
Сгенерированный массив A = { 0.40 0.90 0.20 0.10 0.70 }
Минимум = A[3], максимум = A[1]
Результат B = { 0.90 0.20 0.10 }, сумма = 1.20
Введите размер (или 0 для завершения): 0

%%%%%%%%%%%%%%%%%%%%%%%%%%%%%%%%%%%%%%%%%%%%%%%%%%%%%%%%%%%%%%%%%%%%%%%%%%%%%%

Задача C-21

В начало массива из 20 элементов записать N случайных вещественных чисел из
диапазона от 0 до N с точностью 0.1, воспользовавшись функцией rand(). Число N
вводится с клавиатуры. Найти индексы минимального и максимального эл-та (если
таких несколько, то первый минимальный и последний максимальный). В другой
массив записать элементы, лежащие между найденными индексами включительно,
начиная с минимального индекса (может оказаться в обратном порядке). Вывести
на экран результирующий массив и сумму его элементов. Повторять выполнение
программы до тех пор, пока в качестве N не введут 0.

Иванов И.И. (1057/1): Массивы, вещ. случайные числа от 0 до N.
Введите размер (или 0 для завершения): 5
Сгенерированный массив A = { 0.40 0.90 0.20 0.10 0.70 }
Минимум = A[3], максимум = A[1]
Результат B = { 0.10 0.20 0.90 }, сумма = 1.20
Введите размер (или 0 для завершения): 0

%%%%%%%%%%%%%%%%%%%%%%%%%%%%%%%%%%%%%%%%%%%%%%%%%%%%%%%%%%%%%%%%%%%%%%%%%%%%%%


%%%%%%%%%%%%%%%%%%%%%%%%%%%%%%%%%%%%%%%%%%%%%%%%%%%%%%%%%%%%%%%%%%%%%%%%%%%%%%
\zztaskgroup{PLY}{Полиномы}
%%%%%%%%%%%%%%%%%%%%%%%%%%%%%%%%%%%%%%%%%%%%%%%%%%%%%%%%%%%%%%%%%%%%%%%%%%%%%%

Как я понимаю, эта завуалированная задачка на массивы.

В следующих задачах требуется написать программу, 

Примеры диалога программы и пользователя:

\begin{zzoutput}
  Задание \thezztaskgroup-1: Значение полинома
  Введите N: \zzuser{5}
  Случайный полином: 3x^5 + 2x^4 + 9x^2 + 7x + 5
  Введите x: \zzuser{0.0}
  Значение полинома при x = 0: 5.00000
\end{zzoutput}


\begin{zztask}[Значение полинома]
В рамках общего условия задачи составить случайный полином и вывести на экран
значение полинома в указанной пользователем точке $x$.
\end{zztask}

\begin{itemize}
\item изменить его коэффициенты на дополнения до 9 ($x^2+5x+3$ $\rightarrow$ $8x^2+4x+6$)
\item перевернуть его коэффициенты задом наперед($x^2+5x+3$ $\rightarrow$ $3x^2+5x+1$)
\item \dots
\end{itemize}

%%%%%%%%%%%%%%%%%%%%%%%%%%%%%%%%%%%%%%%%%%%%%%%%%%%%%%%%%%%%%%%%%%%%%%%%%%%%%%
\zztaskgroup{STR}{Обработка строк}
%%%%%%%%%%%%%%%%%%%%%%%%%%%%%%%%%%%%%%%%%%%%%%%%%%%%%%%%%%%%%%%%%%%%%%%%%%%%%%

В следующих задачах требуется написать указанную в варианте функцию для
обработки строк, \textbf{отвечающую заданному прототипу}. Требуется использовать
эту функцию в тестовой программе, которая должна в цикле читать строки с
клавиатуры и выдавать ответ на экран (желательно точно под исходной строкой)
до тех пор, \textbf{пока пользователь не введет пустую строку} (строку нулевой длины).
В этом задании можно считать, что все вводимые строки ограниченной
длины~($<100$) и использовать буфер (массив) фиксированного размера и
функцию чтения строк \texttt{gets(s)}.

Буквами договоримся считать прописные и строчные латинские буквы 
A--Z и a--z, словами~--- последовательность букв и цифр 0--9. Все остальные
символы в строке будем считать \textbf{разделителями} слов.

В учебных целях при решении задач данного раздела функций стандартной
библиотеки, упрощающих работу со строками и символами (напр. из 
\texttt{<string.h>}), следует \textbf{избегать}, реализуя необходимую
функциональность самостоятельно.\zztodo{К этой нет вопросов. и примерчики и полный пакет.}

Примеры диалога программы и пользователя:

\begin{zzoutput}
  Задание \thezztaskgroup-1: Разворот строк
  Введите строку: \zzuser{Hello, world!}
  Результат     : !dlrow ,olleH
  Введите строку: \zzuser{int main(void);}
  Результат     : ;)diov(niam tni
  Введите строку: \zzuser{ }
\end{zzoutput}


%%%%%%%%%%%%%%%%%%%%%%%%%%%%%%%%%%%%%%%%%%%%%%%%%%%%%%%%%%%%%%%%%%%%%%%%%%%%%%
\bigskip
%%%%%%%%%%%%%%%%%%%%%%%%%%%%%%%%%%%%%%%%%%%%%%%%%%%%%%%%%%%%%%%%%%%%%%%%%%%%%%


\begin{zztask}[Разворот строк]
В рамках общего условия задачи написать функцию, которая в заданном буфере
разворачивает строку задом наперед.
Например, из
<<\texttt{The good and the EVIL ones.}>>
должно получиться
<<\texttt{.seno LIVE eht dna doog ehT}>>.

Прототип: \mintinline{c}|void Reverse(char str[]);|
\end{zztask}

%%%%%%%%%%%%%%%%%%%%%%%%%%%%%%%%%%%%%%%%%%%%%%%%%%%%%%%%%%%%%%%%%%%%%%%%%%%%%%

\begin{zztask}[Разворот слов]
В рамках общего условия задачи написать функцию, которая в заданном буфере
разворачивает каждое слово строки задом наперед. Разделители при этом не
меняются.
Например, из
<<\texttt{The good and the EVIL ones.}>>
должно получиться
<<\texttt{ehT doog dna eht LIVE seno.}>>.

Прототип: \mintinline{c}|void ReverseWords(char str[]);|
\end{zztask}

%%%%%%%%%%%%%%%%%%%%%%%%%%%%%%%%%%%%%%%%%%%%%%%%%%%%%%%%%%%%%%%%%%%%%%%%%%%%%%

\begin{zztask}[Верхний регистр]
В рамках общего условия задачи написать функцию, которая в заданном буфере
все буквы заменяет на заглавные.
Например, из
<<\texttt{The good and the EVIL ones.}>>
должно получиться
<<\texttt{THE GOOD AND THE EVIL ONES.}>>.

Прототип: \mintinline{c}|void UpperCase(char str[]);|
\end{zztask}

%%%%%%%%%%%%%%%%%%%%%%%%%%%%%%%%%%%%%%%%%%%%%%%%%%%%%%%%%%%%%%%%%%%%%%%%%%%%%%

\begin{zztask}[Нижний регистр]
В рамках общего условия задачи написать функцию, которая в заданном буфере
все буквы заменяет на строчные.
Например, из
<<\texttt{The good and the EVIL ones.}>>
должно получиться
<<\texttt{the good and the evil ones.}>>.

Прототип: \mintinline{c}|void LowerCase(char str[]);|
\end{zztask}

%%%%%%%%%%%%%%%%%%%%%%%%%%%%%%%%%%%%%%%%%%%%%%%%%%%%%%%%%%%%%%%%%%%%%%%%%%%%%%

\begin{zztask}[Смена регистра]
В рамках общего условия задачи написать функцию, которая в заданном буфере
все заглавные буквы заменяет на строчные, а строчные~--- на заглавные.
Например, из
<<\texttt{The good and the EVIL ones.}>>
должно получиться
<<\texttt{tHE GOOD AND THE evil ONES.}>>.

Прототип: \mintinline{c}|void SwapCase(char str[]);|
\end{zztask}

%%%%%%%%%%%%%%%%%%%%%%%%%%%%%%%%%%%%%%%%%%%%%%%%%%%%%%%%%%%%%%%%%%%%%%%%%%%%%%

\begin{zztask}[Упрощенный заголовок]
В рамках общего условия задачи написать функцию, которая в заданном буфере
все первые буквы слов заменяет на заглавные, а остальные~--- на строчные.
Например, из
<<\texttt{The good and the EVIL ones.}>>
должно получиться
<<\texttt{The Good And The Evil Ones.}>>.

Прототип: \mintinline{c}|void TitleCase(char str[]);|
\end{zztask}

%%%%%%%%%%%%%%%%%%%%%%%%%%%%%%%%%%%%%%%%%%%%%%%%%%%%%%%%%%%%%%%%%%%%%%%%%%%%%%

\begin{zztask}[Заголовок]
В рамках общего условия задачи написать функцию, которая в заданном буфере
все первые буквы длинных слов ($>3$) заменяет на заглавные, а остальные~--- на
строчные. Короткое слово должно начинаться с заглавной только если это первое
слово строки.
Например, из
<<\texttt{The good and the EVIL ones.}>>
должно получиться
<<\texttt{The Good and the Evil Ones.}>>.

Прототип: \mintinline{c}|void TitleCase(char str[]);|
\end{zztask}

%%%%%%%%%%%%%%%%%%%%%%%%%%%%%%%%%%%%%%%%%%%%%%%%%%%%%%%%%%%%%%%%%%%%%%%%%%%%%%

\begin{zztask}[Удаление разделителей]
В рамках общего условия задачи написать функцию, которая в заданном буфере
удаляет разделители, прижимая слова друг к другу.
Например, из
<<\texttt{The good and the EVIL ones.}>>
должно получиться\linebreak
<<\texttt{ThegoodandtheEVILones}>>.

Прототип: \mintinline{c}|void RemoveSeparators(char str[]);|
\end{zztask}

%%%%%%%%%%%%%%%%%%%%%%%%%%%%%%%%%%%%%%%%%%%%%%%%%%%%%%%%%%%%%%%%%%%%%%%%%%%%%%

\begin{zztask}[Удаление подстрок]
В рамках общего условия задачи написать функцию, которая в заданном буфере
находит и удаляет все подстроки <<\texttt{main}>> (не обязательно слова).
Необходимо обойтись без дополнительного буфера.

Прототип: \mintinline{c}|void RemoveMain(char str[]);|
\end{zztask}

%%%%%%%%%%%%%%%%%%%%%%%%%%%%%%%%%%%%%%%%%%%%%%%%%%%%%%%%%%%%%%%%%%%%%%%%%%%%%%

\begin{zztask}[Удаление слов]
В рамках общего условия задачи написать функцию, которая в заданном буфере
находит и удаляет все слова <<\texttt{main}>> (только слова).
Необходимо обойтись без дополнительного буфера.

Прототип: \mintinline{c}|void RemoveMainWord(char str[]);|
\end{zztask}

%%%%%%%%%%%%%%%%%%%%%%%%%%%%%%%%%%%%%%%%%%%%%%%%%%%%%%%%%%%%%%%%%%%%%%%%%%%%%%

\begin{zztask}[Замена слов]
В рамках общего условия задачи написать функцию, которая в заданном буфере
находит и заменяет все слова <<\texttt{hello}>> на <<\texttt{bye}>> (только слова).
Необходимо обойтись без дополнительного буфера.

Прототип: \mintinline{c}|void ReplaceHelloWord(char str[]);|
\end{zztask}

%%%%%%%%%%%%%%%%%%%%%%%%%%%%%%%%%%%%%%%%%%%%%%%%%%%%%%%%%%%%%%%%%%%%%%%%%%%%%%

\begin{zztask}[Длинные слова]
В рамках общего условия задачи написать функцию, которая в заданном буфере
находит первое из слов максимальной длины и оставляет в буфере только его,
сдвигая к началу.
Например, из
<<\texttt{The good and the EVIL ones.}>>
должно получиться
<<\texttt{good}>>.

Прототип: \mintinline{c}|void FindLongWords(char str[]);|
\end{zztask}

%%%%%%%%%%%%%%%%%%%%%%%%%%%%%%%%%%%%%%%%%%%%%%%%%%%%%%%%%%%%%%%%%%%%%%%%%%%%%%

\begin{zztask}[Забой длинных слов]
В рамках общего условия задачи написать функцию, которая в заданном буфере
находит все слова максимальной длины и забивает их звездочками.
Например, из
<<\texttt{The good and the EVIL ones.}>>
должно получиться
<<\texttt{The **** and the **** ****.}>>.

Прототип: \mintinline{c}|void HideLongWords(char str[]);|
\end{zztask}

%%%%%%%%%%%%%%%%%%%%%%%%%%%%%%%%%%%%%%%%%%%%%%%%%%%%%%%%%%%%%%%%%%%%%%%%%%%%%%

\begin{zztask}[Забой слов-палиндромов]
В рамках общего условия задачи написать функцию, которая в заданном буфере
находит все слова, являющиеся палиндромами (игнорируя регистр букв), и
забивает их звездочками.

Прототип: \mintinline{c}|void HidePalindromes(char str[]);|
\end{zztask}

%%%%%%%%%%%%%%%%%%%%%%%%%%%%%%%%%%%%%%%%%%%%%%%%%%%%%%%%%%%%%%%%%%%%%%%%%%%%%%

\begin{zztask}[Подмена слов]
В рамках общего условия задачи написать функцию, которая в заданном буфере
находит все слова максимальной длины и заменяет их все на первое из
найденных.
Необходимо обойтись без дополнительного буфера.
Например, из
<<\texttt{The good and the EVIL ones.}>>
должно получиться
<<\texttt{The good and the good good.}>>.

Прототип: \mintinline{c}|void SubstituteLongWords(char str[]);|
\end{zztask}

%%%%%%%%%%%%%%%%%%%%%%%%%%%%%%%%%%%%%%%%%%%%%%%%%%%%%%%%%%%%%%%%%%%%%%%%%%%%%%

\begin{zztask}[Обмен слов]
В рамках общего условия задачи написать функцию, которая в заданном буфере
меняет местами соседние пары слов, сдвигая разделители при необходимости, то
есть из <<\texttt{a+=bc*d}>> должно получиться <<\texttt{bc+=a*d}>>.
Необходимо обойтись без дополнительного буфера.

Прототип: \mintinline{c}|void SwapWords(char str[]);|
\end{zztask}

%%%%%%%%%%%%%%%%%%%%%%%%%%%%%%%%%%%%%%%%%%%%%%%%%%%%%%%%%%%%%%%%%%%%%%%%%%%%%%

\begin{zztask}[Циклическая перестановка слов]
В рамках общего условия задачи написать функцию, которая в заданном буфере
циклически переставляет слова влево, сдвигая разделители при необходимости,
то есть из <<\texttt{a+=bc*d}>> должно получиться <<\texttt{bc+=d*a}>>.
Необходимо обойтись без дополнительного буфера.

Прототип: \mintinline{c}|void RotateWords(char str[]);|
\end{zztask}

%%%%%%%%%%%%%%%%%%%%%%%%%%%%%%%%%%%%%%%%%%%%%%%%%%%%%%%%%%%%%%%%%%%%%%%%%%%%%%

\begin{zztask}[Удаление комментариев]
В рамках общего условия задачи написать функцию, которая в заданном буфере
удаляет все между соответствующими парами \texttt{/*} и \texttt{*/}.
Вложенные комментарии поддерживать не надо, то есть от 
<<\texttt{a/*b/*c*/d*/e}>> останется <<\texttt{ad*/e}>>. 
Необходимо обойтись без дополнительного буфера.

Прототип: \mintinline{c}|void RemoveComments(char str[]);|
\end{zztask}

%%%%%%%%%%%%%%%%%%%%%%%%%%%%%%%%%%%%%%%%%%%%%%%%%%%%%%%%%%%%%%%%%%%%%%%%%%%%%%

\begin{zztask}[Удаление вложенных комментариев]
В рамках общего условия задачи написать функцию, которая в заданном буфере
удаляет все между соответствующими парами \texttt{/*} и \texttt{*/}.
Надо поддержать вложенные комментарии, то есть от 
<<\texttt{a/*b/*c*/d*/e}>> останется <<\texttt{ae}>>. 
Необходимо обойтись без дополнительного буфера.

Прототип: \mintinline{c}|void RemoveNestedComments(char str[]);|
\end{zztask}

%%%%%%%%%%%%%%%%%%%%%%%%%%%%%%%%%%%%%%%%%%%%%%%%%%%%%%%%%%%%%%%%%%%%%%%%%%%%%%

%%%%%%%%%%%%%%%%%%%%%%%%%%%%%%%%%%%%%%%%%%%%%%%%%%%%%%%%%%%%%%%%%%%%%%%%%%%%%%
