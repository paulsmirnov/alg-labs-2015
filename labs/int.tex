%%%%%%%%%%%%%%%%%%%%%%%%%%%%%%%%%%%%%%%%%%%%%%%%%%%%%%%%%%%%%%%%%%%%%%%%%%%%%%
\zztaskgroup{INT}{Указатели на функцию в численных методах}
%%%%%%%%%%%%%%%%%%%%%%%%%%%%%%%%%%%%%%%%%%%%%%%%%%%%%%%%%%%%%%%%%%%%%%%%%%%%%%

В следующих задачах предлагается реализовать алгоритм численного 
интегрирования в виде отдельной функции, принимающей подынтегральное выражение
как параметр (в виде указателя на функцию определённого типа). Подынтегральная 
функция вычисляется в равномерно распределенных точках $x_i$, и интеграл от 
функции на отрезке от $a$ до $b$ представляется как сумма нескольких интегралов 
на отрезках $[x_i,\, x_{i+k}]$, которые уже считаются по приближённой формуле.

\[ x_i = a + i h,\qquad h = (b-a)/n, \]

\[ 
\int\limits_a^b f(x)\,dx = 
\sum_{{i=0} \above 0pt {i\div k}}^{n-k} \int\limits_{x_i}^{x_{i+k}} f(x)\,dx \approx
\sum_{{i=0} \above 0pt {i\div k}}^{n-k} I(h, f(x_i),\dots, f(x_{i+k}))
\]

Указатель на исследуемую математическую функцию, границы отрезка и количество
подразбиений (или шаг разбиения) должны приниматься как параметры. Необходимо 
учитывать, что некоторые методы требуют, чтобы количество точек со значениями 
функции было кратно некоторому числу $k$.

В целях достижения максимальной производительности \textbf{запрещается} вычислять
значение подынтегрального выражения несколько раз в одной и той же точке (потому 
что вызов $f(x)$ может быть дорогим в плане вычислений, например, если функция 
задана рядом, другим интегралом и т.п.).

%%%%%%%%%%%%%%%%%%%%%%%%%%%%%%%%%%%%%%%%%%%%%%%%%%%%%%%%%%%%%%%%%%%%%%%%%%%%%%

\begin{zztask}
В рамках общего условия задачи написать функцию находящую приближенное значение 
интеграла от $f(x)$ на отрезке $[a, b]$ (т.е. площади под графиком функции)
с использованием метода левых прямоугольников. 
\[
\int\limits_{x_i}^{x_{i+1}} f(x)\,dx \approx h f(x_i)
\]
\end{zztask}

%%%%%%%%%%%%%%%%%%%%%%%%%%%%%%%%%%%%%%%%%%%%%%%%%%%%%%%%%%%%%%%%%%%%%%%%%%%%%%

\begin{zztask}
В рамках общего условия задачи написать функцию находящую приближенное значение 
интеграла от $f(x)$ на отрезке $[a, b]$ (т.е. площади под графиком функции)
с использованием метода трапеций. 
\[
\int\limits_{x_i}^{x_{i+1}} f(x)\,dx \approx \frac{h}{2} \left(f(x_i)+f(x_{i+1})\right)
\]
\end{zztask}

%%%%%%%%%%%%%%%%%%%%%%%%%%%%%%%%%%%%%%%%%%%%%%%%%%%%%%%%%%%%%%%%%%%%%%%%%%%%%%

\begin{zztask}
В рамках общего условия задачи написать функцию находящую приближенное значение 
интеграла от $f(x)$ на отрезке $[a, b]$ (т.е. площади под графиком функции)
с использованием метода Симпсона (парабол).
\[
\int\limits_{x_i}^{x_{i+2}} f(x)\,dx \approx 
\frac{h}{3} \left(f(x_i) + 4 f(x_{i+1}) + f(x_{i+2})\right)
\]
\end{zztask}

%%%%%%%%%%%%%%%%%%%%%%%%%%%%%%%%%%%%%%%%%%%%%%%%%%%%%%%%%%%%%%%%%%%%%%%%%%%%%%

\begin{zztask}
В рамках общего условия задачи написать функцию находящую приближенное значение 
интеграла от $f(x)$ на отрезке $[a, b]$ (т.е. площади под графиком функции)
с использованием метода $3/8$ Симпсона.
\[
\int\limits_{x_i}^{x_{i+3}} f(x)\,dx \approx 
\frac{3h}{8} \left(f(x_i) + 3 f(x_{i+1}) + 3 f(x_{i+2}) + f(x_{i+3})\right)
\]
\end{zztask}

%%%%%%%%%%%%%%%%%%%%%%%%%%%%%%%%%%%%%%%%%%%%%%%%%%%%%%%%%%%%%%%%%%%%%%%%%%%%%%

\begin{zztask}
В рамках общего условия задачи написать функцию находящую приближенное значение 
интеграла от $f(x)$ на отрезке $[a, b]$ (т.е. площади под графиком функции)
с использованием метода Буля.
\[
\int\limits_{x_i}^{x_{i+4}} f(x)\,dx \approx 
\frac{2h}{45} \left(7 f(x_i) + 32 f(x_{i+1}) + 12 f(x_{i+2}) + 32 f(x_{i+3}) + 7 f(x_{i+4})\right)
\]
\end{zztask}

%%%%%%%%%%%%%%%%%%%%%%%%%%%%%%%%%%%%%%%%%%%%%%%%%%%%%%%%%%%%%%%%%%%%%%%%%%%%%%

\begin{zztask}
В рамках общего условия задачи написать функцию находящую приближенное значение 
интеграла от $f(x)$ на отрезке $[a, b]$ (т.е. площади под графиком функции)
с использованием следующей приближенной формулы (открытого типа). 
\[
\int\limits_{x_i}^{x_{i+2}} f(x)\,dx \approx 2h f(x_{i+1})
\]
\end{zztask}

%%%%%%%%%%%%%%%%%%%%%%%%%%%%%%%%%%%%%%%%%%%%%%%%%%%%%%%%%%%%%%%%%%%%%%%%%%%%%%

\begin{zztask}
В рамках общего условия задачи написать функцию находящую приближенное значение 
интеграла от $f(x)$ на отрезке $[a, b]$ (т.е. площади под графиком функции)
с использованием следующей приближенной формулы (открытого типа). 
\[
\int\limits_{x_i}^{x_{i+3}} f(x)\,dx \approx \frac{3h}{2} \left( f(x_{i+1}) + f(x_{i+2}) \right)
\]
\end{zztask}

%%%%%%%%%%%%%%%%%%%%%%%%%%%%%%%%%%%%%%%%%%%%%%%%%%%%%%%%%%%%%%%%%%%%%%%%%%%%%%

\begin{zztask}
В рамках общего условия задачи написать функцию находящую приближенное значение 
интеграла от $f(x)$ на отрезке $[a, b]$ (т.е. площади под графиком функции)
с использованием следующей приближенной формулы (открытого типа). 
\[
\int\limits_{x_i}^{x_{i+4}} f(x)\,dx \approx \frac{4h}{3} \left( 2f(x_{i+1}) - f(x_{i+2}) + 2f(x_{i+3}) \right)
\]
\end{zztask}

%%%%%%%%%%%%%%%%%%%%%%%%%%%%%%%%%%%%%%%%%%%%%%%%%%%%%%%%%%%%%%%%%%%%%%%%%%%%%%

\begin{zztask}
В рамках общего условия задачи написать функцию находящую приближенное значение 
интеграла от $f(x)$ на отрезке $[a, b]$ (т.е. площади под графиком функции)
с использованием следующей приближенной формулы (открытого типа). 
\[
\int\limits_{x_i}^{x_{i+5}} f(x)\,dx \approx 
\frac{5h}{24} \left( 11 f(x_{i+1}) + f(x_{i+2}) + f(x_{i+3}) + 11f(x_{i+4}) \right)
\]
\end{zztask}

%%%%%%%%%%%%%%%%%%%%%%%%%%%%%%%%%%%%%%%%%%%%%%%%%%%%%%%%%%%%%%%%%%%%%%%%%%%%%%
