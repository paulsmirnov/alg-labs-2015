%%%%%%%%%%%%%%%%%%%%%%%%%%%%%%%%%%%%%%%%%%%%%%%%%%%%%%%%%%%%%%%%%%%%%%%%%%%%%%
\zztaskgroup{GAM}{Мини-игры}
%%%%%%%%%%%%%%%%%%%%%%%%%%%%%%%%%%%%%%%%%%%%%%%%%%%%%%%%%%%%%%%%%%%%%%%%%%%%%%

Программы пишутся в среде Borland C с использованием BGI-графики. Графический
режим обычный, 640х480 (VGAHI) или 640х350 (VGAMED, зато 2 страницы), 16
цветов. Выход из игры по ESC. По F1 должна показываться помощь (список
клавиш). Игра должна быть красиво оформлена (графически), улучшения
приветствуются. Ниже перечислены приблизительные правила игр.\zztodo{{Хорошая задачка, игры всегда хорошо. Только я бы дал все поновее, может винапи там или еще че или граф библу бы вашу выдал(хотя винапи лучше, имхо). Пойдет на конец второго семестра. Не все им там с клавиатуры два числа считывать.}}

%%%%%%%%%%%%%%%%%%%%%%%%%%%%%%%%%%%%%%%%%%%%%%%%%%%%%%%%%%%%%%%%%%%%%%%%%%%%%%
\zzsectionCOMMENTS
%%%%%%%%%%%%%%%%%%%%%%%%%%%%%%%%%%%%%%%%%%%%%%%%%%%%%%%%%%%%%%%%%%%%%%%%%%%%%%

%%%%%%%%%%%%%%%%%%%%%%%%%%%%%%%%%%%%%%%%%%%%%%%%%%%%%%%%%%%%%%%%%%%%%%%%%%%%%%
\zzsectionPLAN
%%%%%%%%%%%%%%%%%%%%%%%%%%%%%%%%%%%%%%%%%%%%%%%%%%%%%%%%%%%%%%%%%%%%%%%%%%%%%%

%%%%%%%%%%%%%%%%%%%%%%%%%%%%%%%%%%%%%%%%%%%%%%%%%%%%%%%%%%%%%%%%%%%%%%%%%%%%%%
\zzsectionVARIATIONS
%%%%%%%%%%%%%%%%%%%%%%%%%%%%%%%%%%%%%%%%%%%%%%%%%%%%%%%%%%%%%%%%%%%%%%%%%%%%%%

\begin{zztask}[Tetris]
Фигуры тетрамино (всевозможные стыковки четырех квадратиков сторона к стороне)
в случайном порядке по очереди появляются наверху стакана и начинают медленно
падать вниз (дискретно). Долетая до дна или другого препятствия, фигура
останавливается, окаменевает и становится частью стакана. Пока фигура падает,
ей можно управлять: вращать или двигать вправо-влево в стакане, если позволяют
препятствия. Если после падения фигуры образуется непрерывный ряд
«окаменелостей» от левой до правой стенки стакана, этот ряд исчезает, все ряды
которые выше, сдвигаются вниз. Игра заканчивается, если очередная фигура не
может появиться наверху из-за препятствий. Очки начисляются пропорционально
количеству уничтоженных линий.
\end{zztask}

%%%%%%%%%%%%%%%%%%%%%%%%%%%%%%%%%%%%%%%%%%%%%%%%%%%%%%%%%%%%%%%%%%%%%%%%%%%%%%

\begin{zztask}[Columns]
Вертикальные столбики из трех квадратиков раскрашенных в случайные цвета (из
ограниченного набора, напр. три-четыре цвета) появляются наверху стакана и
начинают медленно падать вниз (дискретно). Долетая до дна или другого
препятствия, столбик останавливается. Пока столбик падает, им можно управлять:
двигать вправо-влево в стакане, если позволяют препятствия, и циклически
менять цвета. Если после падения столбика образуются непрерывные области из
трех или более квадратиков одного цвета, они исчезают. Все квадратики выше
исчезнувших падают по-отдельности на освободившиеся места. Повторно
производится проверка на области. Игра заканчивается, если очередная фигура не
может появиться наверху из-за препятствий. Очки начисляются пропорционально
количеству уничтоженных квадратиков и площади областей.
\end{zztask}

%%%%%%%%%%%%%%%%%%%%%%%%%%%%%%%%%%%%%%%%%%%%%%%%%%%%%%%%%%%%%%%%%%%%%%%%%%%%%%

\begin{zztask}[Gems]
Прямоугольное клеточное поле заполнено камушками разных цветов (набор из
четырех-пяти цветов). Игроку предоставляется возможность выбрать клетку, после
чего непрерывная область камушков одного цвета с выбранным (и прилегающая к
нему) исчезает (если ее площадь равна трем или больше). Все камушки выше
исчезнувших падают по-отдельности на освободившиеся места. Если образуется
пустой вертикальный ряд, то все ряды правее его сдвигаются влево. Игра (или
уровень игры) заканчивается если больше не удалить ни одну область (нет
смежных камней или они вообще закончились). Очки начисляются пропорционально
площади уничтоженных областей. За полную очистку дают бонусные очки.
\end{zztask}

%%%%%%%%%%%%%%%%%%%%%%%%%%%%%%%%%%%%%%%%%%%%%%%%%%%%%%%%%%%%%%%%%%%%%%%%%%%%%%

\begin{zztask}[Lines]
На квадратном клеточном поле в случайных местах находятся три разноцветных
шарика. Игрок делает ход, переставляя один шарик на другое место. Переместить
шарик можно только если существует путь из начальной в конечную точку (т.е.
шарик можно только «перекатывать»). Цель — выстроить 5 или более шариков
одного цвета в ряд, после чего они исчезают, а игроку дается дополнительный
ход. Если же ход игрока не привел к исчезновению пятерки, то на поле
выбрасываются еще три шарика (если автоматически получается ряд, то шарики
убираются с поля, но очки игроку не даются). Игра заканчивается, когда игрок
не может сделать ход (поле заполнено). Очки начисляются пропорционально
количеству убранных с поля шариков и длине убираемого ряда. За несколько
убранных рядов подряд начисляются бонус-очки.
\end{zztask}

%%%%%%%%%%%%%%%%%%%%%%%%%%%%%%%%%%%%%%%%%%%%%%%%%%%%%%%%%%%%%%%%%%%%%%%%%%%%%%

\begin{zztask}[Arcanoid]
Наверху игрового поля выстроена стенка из нескольких рядов кирпичей. Внизу
поля расположена ракетка, управляемая игроком. Ракетка может двигаться только
влево и вправо, отбивая мячик, летающий по полю и отскакивающий от границ поля
и от кирпичей. При ударе о кирпичи мячик их уничтожает. Если ракетка не
отбивает мячик, он улетает за нижнюю границу поля и теряется. В распоряжении
игрока три мячика. Игра заканчивается при потере всех мячиков. При очистке
всего поля происходит переход на следующий уровень. Можно ввести кирпичи
разной стойкости, которые разрушаются не от одного, а от двух, трех ударов.
Также можно в некоторых кирпичах прятать бонусы, которые после разрушения
выпадают и летят вниз. Если ракеткой поймать такой бонус, то можно заработать
очки, дополнительную жизнь, увеличить или уменьшить размер ракетки, увеличить
скорость мячика, сделать его всепробивающим (уничтожающим простые кирпичи без
отражения от них), размножить мячик...
\end{zztask}

%%%%%%%%%%%%%%%%%%%%%%%%%%%%%%%%%%%%%%%%%%%%%%%%%%%%%%%%%%%%%%%%%%%%%%%%%%%%%%

\begin{zztask}[Snake]
По клеточному игровому полю бегает маленький питон, которой управляет игрок с
помощью стрелок. На поле периодически появляются лягушки, которых должен есть
питон. Через некоторое время несъеденная лягушка исчезает. При съедании
лягушки питон увеличивается на одну клеточку. На поле также могут появляться
камни, съедание которых укорачивает питона. Также питон укорачивается при
длительном голодании. При врезании в стену (в границу поля) или кусании самого
себя питон умирает. Также на поле могут быть дополнительные стены, в
зависимости от уровня сложности. Через некоторое время (через N появившихся
лягушек) происходит переход на следующий уровень. Очки начисляются
пропорционально длине питона на момент конца уровня. Можно вводить
дополнительные денежные бонусы на поле.
\end{zztask}

%%%%%%%%%%%%%%%%%%%%%%%%%%%%%%%%%%%%%%%%%%%%%%%%%%%%%%%%%%%%%%%%%%%%%%%%%%%%%%
