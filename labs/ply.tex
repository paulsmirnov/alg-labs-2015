%%%%%%%%%%%%%%%%%%%%%%%%%%%%%%%%%%%%%%%%%%%%%%%%%%%%%%%%%%%%%%%%%%%%%%%%%%%%%%
\zztaskgroup{PLY}{Полиномы}
%%%%%%%%%%%%%%%%%%%%%%%%%%%%%%%%%%%%%%%%%%%%%%%%%%%%%%%%%%%%%%%%%%%%%%%%%%%%%%

Как я понимаю, эта завуалированная задачка на массивы.

В следующих задачах требуется написать программу, 

Примеры диалога программы и пользователя:

\begin{zzoutput}
  Задание \thezztaskgroup-1: Значение полинома
  Введите N: \zzuser{5}
  Случайный полином: 3x^5 + 2x^4 + 9x^2 + 7x + 5
  Введите x: \zzuser{0.0}
  Значение полинома при x = 0: 5.00000
\end{zzoutput}


%%%%%%%%%%%%%%%%%%%%%%%%%%%%%%%%%%%%%%%%%%%%%%%%%%%%%%%%%%%%%%%%%%%%%%%%%%%%%%
\subsection*{Комментарии к задаче}
%%%%%%%%%%%%%%%%%%%%%%%%%%%%%%%%%%%%%%%%%%%%%%%%%%%%%%%%%%%%%%%%%%%%%%%%%%%%%%

\paragraph{Топик}
Мудрые слова на тему.


%%%%%%%%%%%%%%%%%%%%%%%%%%%%%%%%%%%%%%%%%%%%%%%%%%%%%%%%%%%%%%%%%%%%%%%%%%%%%%
\subsection*{План решения}
%%%%%%%%%%%%%%%%%%%%%%%%%%%%%%%%%%%%%%%%%%%%%%%%%%%%%%%%%%%%%%%%%%%%%%%%%%%%%%


\begin{enumerate}
\item Сначала\dots
\end{enumerate}


%%%%%%%%%%%%%%%%%%%%%%%%%%%%%%%%%%%%%%%%%%%%%%%%%%%%%%%%%%%%%%%%%%%%%%%%%%%%%%
\subsection*{Варианты}
%%%%%%%%%%%%%%%%%%%%%%%%%%%%%%%%%%%%%%%%%%%%%%%%%%%%%%%%%%%%%%%%%%%%%%%%%%%%%%

\begin{zztask}[Значение полинома]
В рамках общего условия задачи составить случайный полином и вывести на экран
значение полинома в указанной пользователем точке $x$.
\end{zztask}

Идеи для остальных вариантов\zztodo{оформить в виде задачек}:

\begin{itemize}
\item изменить его коэффициенты на дополнения до 9 ($x^2+5x+3$ $\rightarrow$ $8x^2+4x+6$)
\item перевернуть его коэффициенты задом наперед($x^2+5x+3$ $\rightarrow$ $3x^2+5x+1$)
\item \dots
\end{itemize}
