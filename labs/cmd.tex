%%%%%%%%%%%%%%%%%%%%%%%%%%%%%%%%%%%%%%%%%%%%%%%%%%%%%%%%%%%%%%%%%%%%%%%%%%%%%%
\zztaskgroup{CMD}{Командная строка}
%%%%%%%%%%%%%%%%%%%%%%%%%%%%%%%%%%%%%%%%%%%%%%%%%%%%%%%%%%%%%%%%%%%%%%%%%%%%%%

Написать консольную программу, принимающую на вход параметры командной строки
и выдающую результаты на экран. Заголовочный файл conio.h и функции из него
(для расширенной работы с экраном и клавиатурой в текстовом режиме, типа
gotoxy или getch) запрещены. Если программа вызвана с меньшим или большим
числом параметров чем нужно, выдать help. Если параметр один и он равен «/?»
или «-?», тоже вывести помощь.


%%%%%%%%%%%%%%%%%%%%%%%%%%%%%%%%%%%%%%%%%%%%%%%%%%%%%%%%%%%%%%%%%%%%%%%%%%%%%%

VIII-1 Поиск по маске

Параметры программы — стартовый каталог и набор масок файлов для поиска. Маска
кроме обычных букв может содержать спецсимволы * и ?. Вопросик обозначает, что
на этом месте может быть один любой символ, а звездочка обозначает любую
последовательность символов произвольной длины (в т.ч. и пустую). Искать надо
только в указанном каталоге (без подкаталогов). Использовать
findfirst()/findnext(). Имена найденных файлов вывести по одному в строчке.


%%%%%%%%%%%%%%%%%%%%%%%%%%%%%%%%%%%%%%%%%%%%%%%%%%%%%%%%%%%%%%%%%%%%%%%%%%%%%%

VIII-2 Поиск по маске в подкаталогах

Параметры программы — стартовый каталог и набор масок файлов для поиска. Маска
кроме обычных букв может содержать спецсимволы * и ?. Вопросик обозначает, что
на этом месте может быть один любой символ, а звездочка обозначает любую
последовательность символов произвольной длины (в т.ч. и пустую). Искать надо
в указанном каталоге и всех подкаталогах. Использовать findfirst()/findnext().
Имена найденных файлов вывести по одному в строчке, с полным путем и буквой
диска.


%%%%%%%%%%%%%%%%%%%%%%%%%%%%%%%%%%%%%%%%%%%%%%%%%%%%%%%%%%%%%%%%%%%%%%%%%%%%%%

VIII-3 Поиск строки в файлах

Параметры программы — стартовый каталог и подстрока для поиска. Искать
подстроку надо во всех файлах с расширением txt в указанном каталоге и всех
подкаталогах. Использовать findfirst()/findnext(). Вывести имена файлов с
относительным путем (от указанного каталога) и после двоеточия номера строк
(через запятую), в которых встретилась подстрока.


%%%%%%%%%%%%%%%%%%%%%%%%%%%%%%%%%%%%%%%%%%%%%%%%%%%%%%%%%%%%%%%%%%%%%%%%%%%%%%

VIII-4 Подсчет слов (wc)

Параметры программы — набор опций-ключей (начинаются с минуса) и одно или
несколько имен файлов. Вместо имени файла можно указывать одинокий знак минус,
или вообще опустить этот параметр, тогда читать надо не из файла, а со
стандартного ввода (stdin). Требуется вывести на одной строчке имя файла и
после двоеточия три числа, разделенных знаками табуляции: количество переводов
строк, количество слов, количество байт в файле. Если указано больше одного
файла, то вывести еще дополнительную строчку с суммарными значениями. Ключи: C
– выводить количество байт, L – количество строк, W – количество слов.


%%%%%%%%%%%%%%%%%%%%%%%%%%%%%%%%%%%%%%%%%%%%%%%%%%%%%%%%%%%%%%%%%%%%%%%%%%%%%%

VIII-5 Статистика кода

Параметры программы — набор стартовых каталогов. Требуется просмотреть все эти
каталоги с подкаталогами, найти файлы *.c, *.cpp *.h и вывести статистику про
каждый файл (по строчке на файл): размер файла в байтах, количество строк в
файле, объем коментариев в тексте в байтах и процентах от размера, средняя
длина строки в файле. Вывести общую статистику: суммарный объем кода в байтах,
суммарный объем коментариев в байтах и процентах, суммарное количество строк.
Вывести максимальное и среднее количество строк в файле по группам (по строчке
для .c, .h, .cpp).


%%%%%%%%%%%%%%%%%%%%%%%%%%%%%%%%%%%%%%%%%%%%%%%%%%%%%%%%%%%%%%%%%%%%%%%%%%%%%%

VIII-6 Dir

Написать программу-аналог команды dir, которая бы выводила информацию о файлах
в таком же виде. Поддержать два ключа: /B (без заголовков, только имена
файлов) и /W (вывод в несколько столбцов).


%%%%%%%%%%%%%%%%%%%%%%%%%%%%%%%%%%%%%%%%%%%%%%%%%%%%%%%%%%%%%%%%%%%%%%%%%%%%%%

VIII-7 Сравнение файлов

Написать программу сравнения двух файлов, имена которых передаются как
параметры. Необходимо проверить и выдать сообщение – файлы одинаковы или
различаются.


%%%%%%%%%%%%%%%%%%%%%%%%%%%%%%%%%%%%%%%%%%%%%%%%%%%%%%%%%%%%%%%%%%%%%%%%%%%%%%

VIII-8 Преобразование регистра

Написать программу преобразования текстового файла в заглавные или строчные
буквы. Программа принимает на входе имя файла и тип преобразования: /U -
преобразовать все буквы в заглавные, /L - преобразовать все буквы в строчные.
Новый файл под тем же самым именем должен быть записать вместо исходного.


%%%%%%%%%%%%%%%%%%%%%%%%%%%%%%%%%%%%%%%%%%%%%%%%%%%%%%%%%%%%%%%%%%%%%%%%%%%%%%

VIII-9 Шифрование

Написать программу шифрования/расшифровки текстового файла. Шифровка
осуществляется путем сдвигания букв на одну (замена ‘a’ на ‘b’, ‘b’ на ‘c’ и
т.д., т.е. увеличение ASCII кода символа на 1). Для шифрования файла
используется ключ /C, для дешифрования файла используется /D. Новый файл под
тем же самым именем должен быть записать вместо исходного.

При написании программы учтите, что файл должен остаться текстовым, то есть
символу 255 будет соответствовать символ 32, а служебные символы с кодами от 0
до 31 должны остаться без изменений.


%%%%%%%%%%%%%%%%%%%%%%%%%%%%%%%%%%%%%%%%%%%%%%%%%%%%%%%%%%%%%%%%%%%%%%%%%%%%%%
