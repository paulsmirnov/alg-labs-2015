%%%%%%%%%%%%%%%%%%%%%%%%%%%%%%%%%%%%%%%%%%%%%%%%%%%%%%%%%%%%%%%%%%%%%%%%%%%%%%
\zztaskgroup{TXT}{Текстовые базы данных}
%%%%%%%%%%%%%%%%%%%%%%%%%%%%%%%%%%%%%%%%%%%%%%%%%%%%%%%%%%%%%%%%%%%%%%%%%%%%%%

В следующих задачах предлагается работать с текстовыми файлами, в цикле
отвечая на запросы пользователя. Информация в файлах располагается в строках,
по строке на одну <<запись>>. Поля в строке разделены запятыми, строки
завершаются переводами строк. Требуется сначала прочитать весь файл целиком в
память, разместив базу данных в динамическом массиве структур с полями
подходящего типа, а затем, отобразив на экране меню с вариантами запросов,
выдавать по ним подходящую информацию.

Имя файла запрашивать у пользователя. Не следует полагаться на то, что все
рабочие файлы будут в правильном формате, ошибки следует вовремя обнаруживать
и выдавать внятные предупреждения. Входные файлы для проверки подготовить
самостоятельно.

%%%%%%%%%%%%%%%%%%%%%%%%%%%%%%%%%%%%%%%%%%%%%%%%%%%%%%%%%%%%%%%%%%%%%%%%%%%%%%

\begin{zztask}
В рамках общего условия задачи написать программу, работающую с текстовым файлом
следующего формата: в строчках файла задана информация 
об успеваемости студентов (по одной строчке на студента).
Для каждого студента указаны через запятую такие данные: ФИО полностью и четыре оценки:
по математическому анализу, линейной алгебре, физике и программированию.

Требуется прочитать файл и вывести (таблицей) на экран список студентов, указав для 
каждого фамилию, инициалы и средний балл, пометив особо тех, у кого средний балл ниже 4.0.
\end{zztask}

%%%%%%%%%%%%%%%%%%%%%%%%%%%%%%%%%%%%%%%%%%%%%%%%%%%%%%%%%%%%%%%%%%%%%%%%%%%%%%

\begin{zztask}
В рамках общего условия задачи написать программу, работающую с текстовым файлом
следующего формата: в строчках файла задана информация 
об абитуриентах и их баллах (0--10) за вступительные экзамены (по одной строчке на абитуриента).
Для каждого абитуриента указаны через запятую такие данные: ФИО полностью и баллы:
по алгебре ($A$), геометрии ($G$), физике ($P$), русскому языку ($R$) и информатике ($I$).

Требуется прочитать файл и вывести (таблицей) на экран список поступивших студентов, то есть 
набравших суммарный балл более 80, который считается по следующей формуле:
$2.5\cdot(A+G) + 1.5 P + 0.5 R + 3 I$. Список должен быть отсортирован в порядке убывания баллов.
\end{zztask}

%%%%%%%%%%%%%%%%%%%%%%%%%%%%%%%%%%%%%%%%%%%%%%%%%%%%%%%%%%%%%%%%%%%%%%%%%%%%%%

\begin{zztask}
В рамках общего условия задачи написать программу, работающую с текстовым файлом
следующего формата: в строчках файла задана информация 
о курсах валют в обменном пункте (по одной строчке на валюту).
Для каждой валюты указаны через запятую такие данные: название, курс покупки и курс продажи
по отношению к рублю.

Требуется прочитать файл и отвечать на запросы пользователя о переводе суммы из одной
валюты в другую через рубли.
\end{zztask}

%%%%%%%%%%%%%%%%%%%%%%%%%%%%%%%%%%%%%%%%%%%%%%%%%%%%%%%%%%%%%%%%%%%%%%%%%%%%%%

\begin{zztask}
В рамках общего условия задачи написать программу, работающую с текстовым файлом
следующего формата: в строчках файла задана информация
об имеющихся в продаже книгах (по одной строчке на книгу).
Для каждой книги указаны через запятую такие данные: автор, название, издательство, год,
отдел магазина (тематика), цена.

Требуется прочитать файл и отвечать на запросы пользователя:
%
\begin{itemize}
\item поиск книги по части названия;
\item поиск книги по автору;
\item вывести все книги определенной тематики (на выбор из имеющихся);
\item вывести все книги в определенном ценовом диапазоне.
\end{itemize}
\end{zztask}

%%%%%%%%%%%%%%%%%%%%%%%%%%%%%%%%%%%%%%%%%%%%%%%%%%%%%%%%%%%%%%%%%%%%%%%%%%%%%%

\begin{zztask}
В рамках общего условия задачи написать программу, работающую с текстовым файлом
следующего формата: в строчках файла задана информация
о совершенных звонках по телефону (по одной строчке на звонок).
Для каждого звонка указаны через запятую такие данные: дата и время звонка, телефонный
номер (в одиннадцатизначном виде, напр., \verb|+7 (812) 123-45-67|), продолжительность 
разговора (с точностью до секунд). 

Требуется прочитать файл и вывести (таблицей) на экран счет за переговоры, расписав
отдельно стоимость по пяти следующим тарифам:
%
\begin{itemize}
\item международный звонок (код отличается от кода России +7) --- тарификация
      блоками по три минуты, 30 руб. за минуту;
\item звонок на городской телефон С.-Петербурга (код +7812) в дневные часы (с
      8:00 до 20:00) --- поминутная тарификация, 7 руб. за минуту;
\item звонок на городской телефон С.-Петербурга (код +7812) в ночные часы ---
      поминутная тарификация, 4 руб. за минуту;
\item звонок на мобильные телефоны дружественных операторов (коды +79xx) ---
      посекундная тарификация начиная со второй минуты, 2 руб. за минуту;
\item остальные звонки --- поминутная тарификация, 15 руб. за минуту.
\end{itemize}
\end{zztask}

%%%%%%%%%%%%%%%%%%%%%%%%%%%%%%%%%%%%%%%%%%%%%%%%%%%%%%%%%%%%%%%%%%%%%%%%%%%%%%

\begin{zztask}
В рамках общего условия задачи написать программу, работающую с текстовым файлом
следующего формата: в строчках файла задана информация 
о погоде в определенное время суток (по одной 
строчке на замер). Для каждого замера указаны через запятую следующие данные:
дата, время, температура в градусах цельсия, давление в мм.ртутного столба, влажность
в процентах.

Требуется прочитать файл и отвечать на вопросы пользователя о средних, минимальных и 
максимальных значений параметров за определенную дату. Необходимо корректно предупреждать
об отсутствии данных.
\end{zztask}

%%%%%%%%%%%%%%%%%%%%%%%%%%%%%%%%%%%%%%%%%%%%%%%%%%%%%%%%%%%%%%%%%%%%%%%%%%%%%%

\begin{zztask}
В рамках общего условия задачи написать программу, работающую с текстовым файлом
следующего формата: в строчках файла задана информация 
о сотрудниках кафедры (по одной строчке на сотрудника). Для каждого сотрудника
указаны через запятую следующие данные:
ФИО полностью, должность, ученая степень.

Требуется прочитать файл и подготовить ведомость на зарплату (таблицу), в которой
указать для каждого в алфавитном порядке фамилию, инициалы, сумму к выдаче (оклад 
за должность, плюс надбавка за степень, минус налог).
\end{zztask}

%%%%%%%%%%%%%%%%%%%%%%%%%%%%%%%%%%%%%%%%%%%%%%%%%%%%%%%%%%%%%%%%%%%%%%%%%%%%%%

\begin{zztask}
В рамках общего условия задачи прочитать файл, содержащий в своих строках
информацию о проходящих через пропускной пункт людях: фамилию, имя, отчество,
дату, время, направление прохода, место назначения. Требуется обрабатывать
следующие запросы пользователя:
%
\begin{itemize}
\item вывести всех находившихся на предприятии в указанный момент времени,
\item вывести всех приходивших в указанное место за все время.
\end{itemize}
\end{zztask}

%%%%%%%%%%%%%%%%%%%%%%%%%%%%%%%%%%%%%%%%%%%%%%%%%%%%%%%%%%%%%%%%%%%%%%%%%%%%%%

\begin{zztask}
В рамках общего условия задачи прочитать файл, содержащий в своих строках
информацию о рейсах самолетов: пункт вылета, дата и время вылета, пункт
прилета и дата и время прилета. Требуется обрабатывать следующие запросы
пользователя:
%
\begin{itemize}
\item вывести все рейсы за указанное число,
\item вывести все рейсы из указанного пункта,
\item вывести все рейсы в указанном временном диапазоне,
\item вывести все рейсы продолжающиеся дольше указанного времени.
\end{itemize}
\end{zztask}

%%%%%%%%%%%%%%%%%%%%%%%%%%%%%%%%%%%%%%%%%%%%%%%%%%%%%%%%%%%%%%%%%%%%%%%%%%%%%%

\begin{zztask}
В рамках общего условия задачи прочитать файл, содержащий в своих строках
информацию об успеваемости студентов: фамилию, имя, отчество, номер группы,
предмет, текущие оценки по этому предмету (одну или несколько). Требуется
обрабатывать следующие запросы пользователя:
%
\begin{itemize}
\item вывести всех студентов, получивших оценки по указанному предмету,
\item вывести студента с наивысшим средним баллом по указанному предмету,
\item вывести всех студентов со средним баллом выше (ниже) указанного.
\end{itemize}
%
Имя и отчество выводить на экран только инициалами.
\end{zztask}

%%%%%%%%%%%%%%%%%%%%%%%%%%%%%%%%%%%%%%%%%%%%%%%%%%%%%%%%%%%%%%%%%%%%%%%%%%%%%%

\begin{zztask}
В рамках общего условия задачи прочитать файл, содержащий в своих строках
информацию о выставленных на продажу машинах: производитель, модель, год,
пробег, стоимость, дата начала продажи. Требуется обрабатывать следующие
запросы пользователя:
%
\begin{itemize}
\item вывести все продающиеся марки машин указанного производителя,
\item вывести все предложения в указанном ценовом диапазоне,
\item вывести все предложения, находящиеся в базе дольше указанного времени,
\item вывести самую старую машину по каждому производителю.
\end{itemize}
\end{zztask}

%%%%%%%%%%%%%%%%%%%%%%%%%%%%%%%%%%%%%%%%%%%%%%%%%%%%%%%%%%%%%%%%%%%%%%%%%%%%%%

