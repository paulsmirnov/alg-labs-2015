\chapter{Предисловие}

Теоретический курс по алгоритмам, алгоритмическим языкам и программированию на них не может быть не
подкреплён практикой. Кроме изучения теории студенту необходимо постоянно решать задачи, начиная с
самых простых алгоритмических конструкций и постепенно переходя к более сложным программам
и свободным формулировкам задач.

Предлагаемое учебное пособие содержит ряд заданий разных уровней сложности, охватывая основные
разделы типичного базового курса по программированию: вычисление по формулам, условные конструкции,
циклы, массивы, строки, файлы, динамические структуры данных. В качестве языка программирования
используется язык Си, краткий обзор которого приводится в начале книги. Тем не менее, при
необходимости для решения задач может использоваться практически любой императивный язык
программирования~--- как изучаемые в школе Бейсик и Паскаль, так и Си++, Java, Python из
университетской программы.

Каждое условие задачи в сборнике разбито на две части: общие для серии требования и набор конкретных вариантов,
которые могут распределяться между студентами (задания, немного отличающиеся по сложности, отмечены
звёздочкой$^\star$). Кроме общей формулировки многие задачи содержат комментарии, поясняющие её и
помогающие выбрать конкретные алгоритмы при решении возникающих подзадач, а также план решения,
который позволит приближаться к финальному результату постепенно, шаг за шагом.

Некоторые работы (\texttt{XPR}, \texttt{SEQ}, \texttt{INT}) являются достаточно простыми, чтобы
выполнять их прямо в классе под руководством преподавателя. Другие требуют больше времени и могут
быть заданы в качестве домашнего задания на одну-две недели.
% Задание \texttt{GAM}, как творческое и
% большое по объему, может использоваться даже в качестве курсового проекта.

Важно понимать, что выполнения приведённых в книге заданий недостаточно для успешной учёбы.
Программированию, как работе с любым профессиональным инструментом, можно научиться, только
используя его постоянно, в том числе и для решения возникающих бытовых задач. Если не хватает
фантазии придумывать их самому, можно обратиться к специализированным интернет-сайтам, таким как
Timus, Codeforces, Topcoder и другим. Эти площадки содержат большой архив заданий и периодически
организуют соревнования, позволяя как набрать опыт, так и проявить себя.

% В Приложении А приведены шаги по настройке среды разработки для создания простых консольных
% приложений. На момент написания пособия подавляющее большинство персональных компьютеров работает
% под управлением операционной системы Microsoft Windows 7 или 10 для которой естественной средой
% разработки является свободно распространяемая Microsoft Visual Studio Community 2017. Тем не менее,
% в книге также приводятся подсказки по работе в Apple Xcode 8 на компьютерах Mac и в командной строке
% при использовании компилятора \texttt{gcc}.

В Приложении A приведён пример формальных требований к стилю написания программ. По современным
представлениям, программы пишутся в первую очередь для людей, и только потом~--- для
машин~\cite{mcconnel2015code}. Аккуратный <<почерк>> и стиль изложения способствует ясности мысли,
упрощению понимания кода коллегами-программистами, позволяет избегать типичных ошибок и в итоге
получать более качественный и менее дорогой продукт. Соблюдение стандарта кодирования --- это важная составляющая
профессиональной разработки.
