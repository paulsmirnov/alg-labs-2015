%%%%%%%%%%%%%%%%%%%%%%%%%%%%%%%%%%%%%%%%%%%%%%%%%%%%%%%%%%%%%%%%%%%%%%%%%%%%%%
\zztaskgroup{SER}{Сумма бесконечного ряда}
%%%%%%%%%%%%%%%%%%%%%%%%%%%%%%%%%%%%%%%%%%%%%%%%%%%%%%%%%%%%%%%%%%%%%%%%%%%%%%

В следующих задачах требуется написать программу, выводящую таблицу из $M$
значений некоторой функции $f(x)$ на промежутке от $a$ до $b$ (равномерно распределённых, в том числе и в
граничных точках). Все параметры ($a$, $b$, $M$) задаются пользователем,
функция в каждой задаче своя, причём заданная как сумма бесконечного ряда.
%
\[
  f(x)=S(x)=\sum_{k=0}^\infty u_k(x)=u_0(x)+u_1(x)+\cdots+u_k(x)+\cdots
\]
%
Для сравнения $f(x)$ также задана аналитически, через элементарные функции, и
условие на $x$, при котором это верно. Требуется вывести таблицу из $M$ строк
(по одной строке на значение $x \in [a,b]$), содержащих значение $x$, сумму
$S_n(x)$, точное значением $f(x)$, разницу $\Delta = |S_n(x) - f(x)|$,
количество потребовавшихся членов суммы $n$ и значение
$\varepsilon(S_n)=u_{n+1}$ (описание параметра $\varepsilon$ см. ниже).


%%%%%%%%%%%%%%%%%%%%%%%%%%%%%%%%%%%%%%%%%%%%%%%%%%%%%%%%%%%%%%%%%%%%%%%%%%%%%%
\zzsectionCOMMENTS
%%%%%%%%%%%%%%%%%%%%%%%%%%%%%%%%%%%%%%%%%%%%%%%%%%%%%%%%%%%%%%%%%%%%%%%%%%%%%%


\paragraph{Машинный эпсилон}
В данной задаче бесконечное число слагаемых складывается в течение бесконечного
времени, что делает задачу нерешаемой в такой формулировке за конечное время. 
Поэтому будем вычислять только частичную сумму $S_n$, где $n$ выбирается так,
чтобы получить максимально возможную для типа \texttt{double} точность
суммирования. Для этого будем использовать существование ``машинного
$\varepsilon$'' $\forall y$, то есть такого числа $\varepsilon(y) > 0$, которое при
прибавлении к $y$ не увеличивает его: $y + \varepsilon(y) = y$.
%
\[
S_n(x)=\sum_{k=0}^n u_k(x),\qquad S_n + u_{n+1} = S_n
\]

\paragraph{Вычисление значений слагаемых}
Поскольку слагаемые содержат возведения в степень и факториалы, которые не
могут быть вычислены в лоб с нужной точностью для достаточно больших $n$, а так же в целях увеличения производительности, необходимо составить формулу для вычисления следующего члена суммы с использованием значения предыдущего:
%
\[
u_k(x) = u_{k-1}(x)\cdot v_k(x)
\]
%
В таком случае отпадает необходимость в реализации собственных функции. Кроме того, вычисление в лоб может привести к переполнению, ввиду того, что промежуточные значения могут быть слишком велики (поразмышляйте, например, о вычислении значения слагаемого $\frac{n!}{(n-1)!}$ при $n = 70$).

\paragraph{Форматирование таблицы}
Итоговая таблица должна представлять собой не просто несколько столбцов числен, а аккуратную таблицу с границами, нарисованную с помощью символов.
Вертикальные линии должны обозначать границы каждого столбца, рекомендуется рисовать их с помощью символа ``|''. Горизонтальные линии обозначают границы таблицы и отделяют заголовок от остальных строк, горизонтальные линии нужно рисовать с помощью символа ``-'', а соединительные элементы - символом ``+''. Пример таблицы:
\begin{minted}{c}
+-----------+--------------+------------+-...
|     x     |     f(x)     |    s(x)    |
+-----------+--------------+------------+-...
|      0.10 |      0.00000 |    0.00000 |
|      0.11 |    -10.00001 |  -10.00001 |
|       ... |          ... |        ... |
+-----------+--------------+------------+-...
\end{minted}

\paragraph{Форматирование столбцов}
Для того, чтобы ширина таблицы имела фиксированную величину, необходимо
пользоваться дополнительным модификаторами при использовании функции
\texttt{printf()} (см., например,~\cite{cppref}). Эти модификаторы указываются
после знака $\%$ и перед конструкцией, обозначающей тип параметра, рассмотрим
необходимые нам модификаторы:
%
\begin{itemize}
	\item \textbf{Ширина}. Обозначает минимальное количество символов, которое будет занимать выводимое значение. Если длина значения меньше данной величины, оно будет выровнено по правому краю (для выравнивания по левому краю, используйте символ ``-'', например, \verb|%-8i|). При превышении указанной ширины, значение обрезаться не будет. Например, для вывода числа типа \texttt{double} с шириной поля $10$, необходимо написать \verb|%10lf|.
	\item \textbf{Точность}. Для вывода вещественных чисел может быть полезным использование модификатора точности вывода. Он указывается после модификатора ширины, через точку. Например, конструкция \verb|%10.5f| будет выводить вещественные числа в поле шириной $10$ и с точностью $5$ знаков после запятой.
	\item \textbf{Экспоненциальный формат}. Зачастую, для вывода очень больших или очень маленьких числе с плавающей точкой, вместо модификатора \verb|f| используют \verb|e|, например, при использовании этого модификатора для вывода числа $392.65$, вывод будет следующим: \texttt{3.9265e+2}. Рекомендуется использовать этот модификатор для вывода значения $\varepsilon$ и $\Delta$.
\end{itemize}


\begin{comment}
\begin{itemize}
	\item Здесь нужно оставить описание того, как сделать пошаговое вычисление суммы ряда.
	
	Непонятно только, что тут еще писать, как именно реализовывать работу? В целом тут нужно написать план в стиле:
	\begin{enumerate}
		\item Определитесь с первым членом последовательности
		\item Найдите, как выражается через $x$ и $i$ $v_{k+1}/v_k$. 
		\item Напишите обычный цикл, который вычисляет сумму  первых N членов последовательности и выведите на экран ее и референсное значение функции.
		\item Замените цикл на нормальный.
		 
	\end{enumerate}
	\item Описать, как должен выглядеть заголовок и одна строка таблицы (на живом примере). Описать особенности задания формата в функции printf.
	\item Напишите прототип функции построения таблицы. /*Можно вставить сам прототип*/ Напишите код, выводящий заголовок.
	Затем, напишите цикл, который выводит равномерные значения $x$ от $a$ до $b$. 
	\item Вставьте в тело цикла код вывода одной строчки таблицы.
\end{itemize}
\end{comment}
%%%%%%%%%%%%%%%%%%%%%%%%%%%%%%%%%%%%%%%%%%%%%%%%%%%%%%%%%%%%%%%%%%%%%%%%%%%%%%
\zzsectionPLAN
%%%%%%%%%%%%%%%%%%%%%%%%%%%%%%%%%%%%%%%%%%%%%%%%%%%%%%%%%%%%%%%%%%%%%%%%%%%%%%

\begin{enumerate}
\item Сначала необходимо реализовать функциональность для вычисления слагаемых нашего ряда. Для этого требуется определить, какое значение имеет первое слагаемое и каково отношение между двумя последовательными слагаемыми.
\item После этого можно реализовать вычисление частичных сумм. Для этого оформляем цикл, для начала с фиксированным числом итераций (например, 1000), перед циклом инициализируем значения частичной суммы и слагаемого, после этого в цикле обновляем эти переменные. Для теста можно выводить значения, получаемые на каждом шаге и значения функции, используемой для проверки.\zztodo{Стоит ли рекомендовать вынести вычисление суммы ряда в отдельную функцию, куда дельта и эпсилон передаются по указателю?}
\item Далее необходимо ввести механизм остановки, цикла, подсчет $\varepsilon$ и $\Delta$.
\item На данном этапе подготовлены все данные для вывода таблицы, можно приступать к нему. Комментарии по форматированию можно найти выше. 
\end{enumerate}


%%%%%%%%%%%%%%%%%%%%%%%%%%%%%%%%%%%%%%%%%%%%%%%%%%%%%%%%%%%%%%%%%%%%%%%%%%%%%%
\zzsectionVARIATIONS
%%%%%%%%%%%%%%%%%%%%%%%%%%%%%%%%%%%%%%%%%%%%%%%%%%%%%%%%%%%%%%%%%%%%%%%%%%%%%%


\begin{zztask}
В рамках общего условия задачи вывести таблицу значений функции, заданной рядом:
\[ % 100: "sin z", where z = x/2
	S(x)= \sum_{n=1}^\infty (-1)^{n+1} \frac{x^{2n-1}}{2^{2n-1}(2n-1)!};\quad
	f(x)= \sin \frac{x}{2},
	\quad 0 \leq x \leq 2;
\]
\end{zztask}

%%%%%%%%%%%%%%%%%%%%%%%%%%%%%%%%%%%%%%%%%%%%%%%%%%%%%%%%%%%%%%%%%%%%%%%%%%%%%%

\begin{zztask}
В рамках общего условия задачи вывести таблицу значений функции, заданной рядом:
\[ % 106: "cos z", where z = 2x
  S(x)= 1 + \sum_{n=1}^\infty (-1)^n \frac{2^{2n}}{(2n)!} x^{2n},\quad
  f(x)= \cos 2x,
  \quad |x| \leq 1;
\]
\end{zztask}

%%%%%%%%%%%%%%%%%%%%%%%%%%%%%%%%%%%%%%%%%%%%%%%%%%%%%%%%%%%%%%%%%%%%%%%%%%%%%%

\begin{zztask}
В рамках общего условия задачи вывести таблицу значений функции, заданной рядом:
\[ % 101: "sin^2 x = (1 - cos 2x)/2"
  S(x)= \sum_{n=1}^\infty (-1)^{n-1} \frac{2^{2n-1}}{(2n)!} x^{2n};\quad
  f(x)= \sin^2 x,
  \quad |x| \leq 1;
\]
\end{zztask}

%%%%%%%%%%%%%%%%%%%%%%%%%%%%%%%%%%%%%%%%%%%%%%%%%%%%%%%%%%%%%%%%%%%%%%%%%%%%%%

\begin{zztask}
В рамках общего условия задачи вывести таблицу значений функции, заданной рядом:
\[ % 104: "cos^2 x = (1 + cos 2x)/2"
  S(x)= 1 + \frac{1}{2} \sum_{n=1}^\infty (-1)^n \frac{(2x)^{2n}}{(2n)!};\quad
  f(x)= \cos^2 x,
  \quad |x| \leq 1;
\]
\end{zztask}

%%%%%%%%%%%%%%%%%%%%%%%%%%%%%%%%%%%%%%%%%%%%%%%%%%%%%%%%%%%%%%%%%%%%%%%%%%%%%%

\begin{zztask}
В рамках общего условия задачи вывести таблицу значений функции, заданной рядом:
\[
  S(x)= 2 \sum_{n=0}^\infty \frac{(x-1)^{2n+1}}{(2n+1)(x+1)^{2n+1}};\quad
  f(x)= \ln x,
  \quad 0 < x \leq 1;
\]
\end{zztask}

%%%%%%%%%%%%%%%%%%%%%%%%%%%%%%%%%%%%%%%%%%%%%%%%%%%%%%%%%%%%%%%%%%%%%%%%%%%%%%

\begin{zztask}
В рамках общего условия задачи вывести таблицу значений функции, заданной рядом:
\[
  S(x)= \sum_{n=1}^\infty (-1)^{n+1}\frac{(x-3)^{n-1}}{3^n};\quad
  f(x)= \frac{1}{x},
  \quad 3 < x \leq 4;
\]
\end{zztask}

%%%%%%%%%%%%%%%%%%%%%%%%%%%%%%%%%%%%%%%%%%%%%%%%%%%%%%%%%%%%%%%%%%%%%%%%%%%%%%

\begin{zztask}
В рамках общего условия задачи вывести таблицу значений функции, заданной рядом:
\[
  S(x)= x + \sum_{n=2}^\infty (-1)^{n-1}\frac{2^{n-1}}{(n-1)!} x^n;\quad
  f(x)= x/e^{2x},
  \quad |x| \leq 1;
\]
\end{zztask}

%%%%%%%%%%%%%%%%%%%%%%%%%%%%%%%%%%%%%%%%%%%%%%%%%%%%%%%%%%%%%%%%%%%%%%%%%%%%%%

\begin{zztask}
В рамках общего условия задачи вывести таблицу значений функции, заданной рядом:
\[
  S(x)= \frac{1}{2} \sum_{n=1}^\infty \frac{\big((n-1)!\big)^2}{(2n)!} (2x)^{2n};\quad
  f(x)= \arcsin^2x,
  \quad 0 \leq x \leq \sqrt2/2;
\]
\end{zztask}

%%%%%%%%%%%%%%%%%%%%%%%%%%%%%%%%%%%%%%%%%%%%%%%%%%%%%%%%%%%%%%%%%%%%%%%%%%%%%%
