%%%%%%%%%%%%%%%%%%%%%%%%%%%%%%%%%%%%%%%%%%%%%%%%%%%%%%%%%%%%%%%%%%%%%%%%%%%%%%
\zztaskgroup{DIC}{Словарь}
%%%%%%%%%%%%%%%%%%%%%%%%%%%%%%%%%%%%%%%%%%%%%%%%%%%%%%%%%%%%%%%%%%%%%%%%%%%%%%
\zztodo{Не вижу особого смысла в этих задачках, вроде занятные, но странные.}
Имеется английский словарь. Пользователь вводит набор цифр, и программа должна
предложить ему набор слов той же длины, буквы в словах соответствуют цифрам
как на кнопках телефона. Например, числовой последовательности 2233
соответствует слово cafe и некоторые другие.

%%%%%%%%%%%%%%%%%%%%%%%%%%%%%%%%%%%%%%%%%%%%%%%%%%%%%%%%%%%%%%%%%%%%%%%%%%%%%%

Имеется английский словарь. Пользователь вводит набор букв, программа должна
выдать набор слов, которые можно составить из этих букв (возможно не используя
всех), отсортированный по суммарному количеству очков. Буквам назначаются
баллы от 1 до 5 в соответствие с частотой встречаемости (определить по словарю
и оценить самостоятельно). Например, набору букв “o a d r a b” соответствует
набор слов (неотсортированный) road, abroad, board, aboard, rad,… Решение
задачи составить из двух частей — построение по словарю промежуточной
структуры данных (придуманной самостоятельно для решения этой задачи), и
собственно осуществление запросов (возможно множественных). Во второй части
полный перебор всех слов словаря запрещен как неэффективный. Запрещена
рекурсия.

%%%%%%%%%%%%%%%%%%%%%%%%%%%%%%%%%%%%%%%%%%%%%%%%%%%%%%%%%%%%%%%%%%%%%%%%%%%%%%

Имеется английский словарь. Пользователь «задает» (можно рандомно) матрицу из
букв (порядка 20 на 20), программа находит в этой матрице слова (по прямой во
всех 8 направлениях) и выводит, сортируя по суммарному количеству очков за
буквы умноженному на длину слова. Буквам назначаются баллы от 1 до 5 в
соответствие с частотой встречаемости (определить по словарю и оценить
самостоятельно). Решение задачи составить из двух частей — построение по
словарю промежуточной структуры данных (придуманной самостоятельно для решения
этой задачи), и собственно осуществление запросов (возможно множественных). Во
второй части полный перебор всех слов словаря запрещен как неэффективный, но
перебор всех клеток неизбежен. Запрещена рекурсия.

%%%%%%%%%%%%%%%%%%%%%%%%%%%%%%%%%%%%%%%%%%%%%%%%%%%%%%%%%%%%%%%%%%%%%%%%%%%%%%

Имеется английский словарь. Пользователь задает матрицу из букв (порядка 6 на
6), программа находит в этой матрице слова (червяком, от буквы к букве
движение возможно во всех 8 направлениях, если там еще не были) и выводит,
сортируя по суммарному количеству очков за буквы, умноженному на длину слова.
Буквам назначаются баллы от 1 до 5 в соответствие\zztodo{ии?} с частотой встречаемости
(определить по словарю и оценить самостоятельно). Решение задачи составить из
двух частей — построение по словарю промежуточной структуры данных
(придуманной самостоятельно для решения этой задачи), и собственно
осуществление запросов (возможно множественных). Во второй части полный
перебор всех слов словаря запрещен как неэффективный, но перебор всех клеток
неизбежен. Запрещена рекурсия.

%%%%%%%%%%%%%%%%%%%%%%%%%%%%%%%%%%%%%%%%%%%%%%%%%%%%%%%%%%%%%%%%%%%%%%%%%%%%%%

Имеется английский словарь. Определить частоту встречаемости букв в словаре.

%%%%%%%%%%%%%%%%%%%%%%%%%%%%%%%%%%%%%%%%%%%%%%%%%%%%%%%%%%%%%%%%%%%%%%%%%%%%%%

Имеется английский словарь. Для пар букв определить частоту встречаемости их
друг за другом в словах. Какие буквосочетания самые «вероятные»?

%%%%%%%%%%%%%%%%%%%%%%%%%%%%%%%%%%%%%%%%%%%%%%%%%%%%%%%%%%%%%%%%%%%%%%%%%%%%%%
