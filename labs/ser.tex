%%%%%%%%%%%%%%%%%%%%%%%%%%%%%%%%%%%%%%%%%%%%%%%%%%%%%%%%%%%%%%%%%%%%%%%%%%%%%%
\zztaskgroup{SER}{Сумма бесконечного ряда}
%%%%%%%%%%%%%%%%%%%%%%%%%%%%%%%%%%%%%%%%%%%%%%%%%%%%%%%%%%%%%%%%%%%%%%%%%%%%%%

В следующих задачах требуется написать программу, выводящую таблицу из $M$
значений некоторой функции $f(x)$ на промежутке от $a$ до $b$ (в том числе и в
граничных точках). Все параметры ($a$, $b$, $M$) задаются пользователем,
функция в каждой задаче своя, причём заданная как сумма бесконечного ряда. Для
сравнения $f(x)$ также задана аналитически, через элементарные функции.\zztodo{Нужно ли быть педантичным немцем и писать $u_1(x)$?}
%
\[
  f(x)=S(x)=\sum_{k=0}^\infty u_k(x)=u_0+u_1+\cdots+u_k+\cdots
\]

Бесконечное число слагаемых складывается в течение бесконечного времени,
поэтому будем вычислять только частичную сумму $S_n$, где $n$ выбирается так,
чтобы получить максимально возможную для типа \texttt{double} точность
суммировния. Для этого будем использовать существование ``машинного
$\varepsilon$'', то есть такого числа $\varepsilon(y) > 0$, которое при
прибавлении к $y$ не увеличивает его: $y + \varepsilon(y) = y$.
%
\[
  S_n(x)=\sum_{k=0}^n u_k(x),\qquad S_n + u_{n+1} = S_n
\]

Поскольку слагаемые содержат возведения в степень и факториалы, которые не
могут быть вычислены в лоб с нужной точностью для достаточно больших $n$,
требуется постараться выразить следующий член суммы через предыдущий:
%
\[
u_k(x) = u_{k-1}(x)\cdot v_k(x)
\]

Для сравнения указана функция, совпадающая по значениям с суммой ряда, и
условие на $x$, при котором это верно. Требуется вывести таблицу из $M$ строк
(по одной строке на значение $x \in [a,b]$), содержащих значение $x$, сумму
$S_n(x)$, точное значением $f(x)$, разницу $\Delta = |S_n(x) - f(x)|$,
количество потребовавшихся членов суммы $n$ и значение
$\varepsilon(S_n)=u_{n+1}$.\zztodo{тут я бы добавил вид заголовка таблицы в аски и затребовал бы форматирования, как у нас было на лабах. а еще, мне кажется, что стоит добавить примерчик с синусом для наглядности. В целом задачу действительно стоит сделать домашкой, потому что сделать как надо и прочувствовать все, что надо в классе могут только единицы. если у нас будет домашка, то можно еще затребовать проверки условия, что отрезок $[a,b]$ входит в область определения.}


%%%%%%%%%%%%%%%%%%%%%%%%%%%%%%%%%%%%%%%%%%%%%%%%%%%%%%%%%%%%%%%%%%%%%%%%%%%%%%
\subsection*{Комментарии к задаче}
%%%%%%%%%%%%%%%%%%%%%%%%%%%%%%%%%%%%%%%%%%%%%%%%%%%%%%%%%%%%%%%%%%%%%%%%%%%%%%


\paragraph{Топик}
Мудрые слова на тему.

\begin{itemize}
	\item Здесь нужно оставить описание того, как сделать пошаговое вычисление суммы ряда.
	
	Непонятно только, что тут еще писать, как именно реализовывать работу? В целом тут нужно написать план в стиле:
	\begin{enumerate}
		\item Определитесь с первым членом последовательности
		\item Найдите, как выражается через $x$ и $i$ $v_{k+1}/v_k$. 
		\item Напишите обычный цикл, который вычисляет сумму  первых N членов последовательности и выведите на экран ее и референсное значение функции.
		\item Замените цикл на нормальный.
		 
	\end{enumerate}
	\item Описать, как должен выглядеть заголовок и одна строка таблицы (на живом примере). Описать особенности задания формата в функции printf.
	\item Напишите прототип функции построения таблицы. /*Можно вставить сам прототип*/ Напишите код, выводящий заголовок.
	Затем, напишите цикл, который выводит равномерные значения $x$ от $a$ до $b$. 
	\item Вставьте в тело цикла код вывода одной строчки таблицы.
\end{itemize}


%%%%%%%%%%%%%%%%%%%%%%%%%%%%%%%%%%%%%%%%%%%%%%%%%%%%%%%%%%%%%%%%%%%%%%%%%%%%%%
\subsection*{План решения}
%%%%%%%%%%%%%%%%%%%%%%%%%%%%%%%%%%%%%%%%%%%%%%%%%%%%%%%%%%%%%%%%%%%%%%%%%%%%%%


\begin{enumerate}
\item Сначала\dots
\end{enumerate}


%%%%%%%%%%%%%%%%%%%%%%%%%%%%%%%%%%%%%%%%%%%%%%%%%%%%%%%%%%%%%%%%%%%%%%%%%%%%%%
\subsection*{Варианты}
%%%%%%%%%%%%%%%%%%%%%%%%%%%%%%%%%%%%%%%%%%%%%%%%%%%%%%%%%%%%%%%%%%%%%%%%%%%%%%


\begin{zztask}
В рамках общего условия задачи вывести таблицу значений функции, заданной рядом:
\[ % 100: "sin z", where z = x/2
	S(x)= \sum_{n=1}^\infty (-1)^{n+1} \frac{x^{2n-1}}{2^{2n-1}(2n-1)!};\quad
	f(x)= \sin \frac{x}{2},
	\quad 0 \leq x \leq 2;
\]
\end{zztask}

%%%%%%%%%%%%%%%%%%%%%%%%%%%%%%%%%%%%%%%%%%%%%%%%%%%%%%%%%%%%%%%%%%%%%%%%%%%%%%

\begin{zztask}
В рамках общего условия задачи вывести таблицу значений функции, заданной рядом:
\[ % 106: "cos z", where z = 2x
  S(x)= 1 + \sum_{n=1}^\infty (-1)^n \frac{2^{2n}}{(2n)!} x^{2n},\quad
  f(x)= \cos 2x,
  \quad |x| \leq 1;
\]
\end{zztask}

%%%%%%%%%%%%%%%%%%%%%%%%%%%%%%%%%%%%%%%%%%%%%%%%%%%%%%%%%%%%%%%%%%%%%%%%%%%%%%

\begin{zztask}
В рамках общего условия задачи вывести таблицу значений функции, заданной рядом:
\[ % 101: "sin^2 x = (1 - cos 2x)/2"
  S(x)= \sum_{n=1}^\infty (-1)^{n-1} \frac{2^{2n-1}}{(2n)!} x^{2n};\quad
  f(x)= \sin^2 x,
  \quad |x| \leq 1;
\]
\end{zztask}

%%%%%%%%%%%%%%%%%%%%%%%%%%%%%%%%%%%%%%%%%%%%%%%%%%%%%%%%%%%%%%%%%%%%%%%%%%%%%%

\begin{zztask}
В рамках общего условия задачи вывести таблицу значений функции, заданной рядом:
\[ % 104: "cos^2 x = (1 + cos 2x)/2"
  S(x)= 1 + \frac{1}{2} \sum_{n=1}^\infty (-1)^n \frac{(2x)^{2n}}{(2n)!};\quad
  f(x)= \cos^2 x,
  \quad |x| \leq 1;
\]
\end{zztask}

%%%%%%%%%%%%%%%%%%%%%%%%%%%%%%%%%%%%%%%%%%%%%%%%%%%%%%%%%%%%%%%%%%%%%%%%%%%%%%

\begin{zztask}
В рамках общего условия задачи вывести таблицу значений функции, заданной рядом:
\[
  S(x)= 2 \sum_{n=0}^\infty \frac{(x-1)^{2n+1}}{(2n+1)(x+1)^{2n+1}};\quad
  f(x)= \ln x,
  \quad 0 < x \leq 1;
\]
\end{zztask}

%%%%%%%%%%%%%%%%%%%%%%%%%%%%%%%%%%%%%%%%%%%%%%%%%%%%%%%%%%%%%%%%%%%%%%%%%%%%%%

\begin{zztask}
В рамках общего условия задачи вывести таблицу значений функции, заданной рядом:
\[
  S(x)= \sum_{n=1}^\infty (-1)^{n+1}\frac{(x-3)^{n-1}}{3^n};\quad
  f(x)= \frac{1}{x},
  \quad 3 < x \leq 4;
\]
\end{zztask}

%%%%%%%%%%%%%%%%%%%%%%%%%%%%%%%%%%%%%%%%%%%%%%%%%%%%%%%%%%%%%%%%%%%%%%%%%%%%%%

\begin{zztask}
В рамках общего условия задачи вывести таблицу значений функции, заданной рядом:
\[
  S(x)= x + \sum_{n=2}^\infty (-1)^{n-1}\frac{2^{n-1}}{(n-1)!} x^n;\quad
  f(x)= x/e^{2x},
  \quad |x| \leq 1;
\]
\end{zztask}

%%%%%%%%%%%%%%%%%%%%%%%%%%%%%%%%%%%%%%%%%%%%%%%%%%%%%%%%%%%%%%%%%%%%%%%%%%%%%%

\begin{zztask}
В рамках общего условия задачи вывести таблицу значений функции, заданной рядом:
\[
  S(x)= \frac{1}{2} \sum_{n=1}^\infty \frac{\big((n-1)!\big)^2}{(2n)!} (2x)^{2n};\quad
  f(x)= \arcsin^2x,
  \quad 0 \leq x \leq \sqrt2/2;
\]
\end{zztask}

%%%%%%%%%%%%%%%%%%%%%%%%%%%%%%%%%%%%%%%%%%%%%%%%%%%%%%%%%%%%%%%%%%%%%%%%%%%%%%
